\documentclass[a4, 11pt]{report}


\pagestyle{myheadings}
\markboth{}{Paper I, 2013
\ \ \ \ \ 
\today 
}               

\RequirePackage{amssymb}
\RequirePackage{amsmath}
\RequirePackage{graphicx}
\RequirePackage{color}
\RequirePackage[flushleft]{paralist}[2013/06/09]



\RequirePackage{geometry}
\geometry{%
  a4paper,
  lmargin=2cm,
  rmargin=2.5cm,
  tmargin=3.5cm,
  bmargin=2.5cm,
  footskip=12pt,
  headheight=24pt}


\newcommand{\comment}[1]{{\bf Comment} {\it #1}}
%\renewcommand{\comment}[1]{}

\newcommand{\bluecomment}[1]{{\color{blue}#1}}
%\renewcommand{\comment}[1]{}
\newcommand{\redcomment}[1]{{\color{red}#1}}



\usepackage{epsfig}
\usepackage{pstricks-add}
\usepackage{tgheros} %% changes sans-serif font to TeX Gyre Heros (tex-gyre)
\renewcommand{\familydefault}{\sfdefault} %% changes font to sans-serif
%\usepackage{sfmath}  %%%% this makes equation sans-serif
%\input RexFigs


\setlength{\parskip}{10pt}
\setlength{\parindent}{0pt}

\newlength{\qspace}
\setlength{\qspace}{20pt}


\newcounter{qnumber}
\setcounter{qnumber}{0}

\newenvironment{question}%
 {\vspace{\qspace}
  \begin{enumerate}[\bfseries 1\quad][10]%
    \setcounter{enumi}{\value{qnumber}}%
    \item%
 }
{
  \end{enumerate}
  \filbreak
  \stepcounter{qnumber}
 }


\newenvironment{questionparts}[1][1]%
 {
  \begin{enumerate}[\bfseries (i)]%
    \setcounter{enumii}{#1}
    \addtocounter{enumii}{-1}
    \setlength{\itemsep}{5mm}
    \setlength{\parskip}{8pt}
 }
 {
  \end{enumerate}
 }



\DeclareMathOperator{\cosec}{cosec}
\DeclareMathOperator{\Var}{Var}

\def\d{{\mathrm d}}
\def\e{{\mathrm e}}
\def\g{{\mathrm g}}
\def\h{{\mathrm h}}
\def\f{{\mathrm f}}
\def\p{{\mathrm p}}
\def\s{{\mathrm s}}
\def\t{{\mathrm t}}


\def\A{{\mathrm A}}
\def\B{{\mathrm B}}
\def\E{{\mathrm E}}
\def\F{{\mathrm F}}
\def\G{{\mathrm G}}
\def\H{{\mathrm H}}
\def\P{{\mathrm P}}


\def\bb{\mathbf b}
\def \bc{\mathbf c}
\def\bx {\mathbf x}
\def\bn {\mathbf n}

\newcommand{\low}{^{\vphantom{()}}}
%%%%% to lower suffices: $X\low_1$ etc


\newcommand{\subone}{ {\vphantom{\dot A}1}}
\newcommand{\subtwo}{ {\vphantom{\dot A}2}}




\def\le{\leqslant}
\def\ge{\geqslant}
\def\arcosh{{\rm arcosh}\,}


\def\var{{\rm Var}\,}

\newcommand{\ds}{\displaystyle}
\newcommand{\ts}{\textstyle}
\def\half{{\textstyle \frac12}}
\def\l{\left(}
\def\r{\right)}



\begin{document}
\setcounter{page}{2}

 
\section*{Section A: \ \ \ Pure Mathematics}

%%%%%%%%%%Q1
\begin{question}
\begin{questionparts}
\item
Use the substitution $\sqrt x = y$ (where $y\ge0$) to find the 
real root of the equation
\[
x + 3\,  \sqrt x - \tfrac12 =0\,.
\]

\item Find all real roots  of the following equations:
\begin{itemize}
\setlength\itemindent {7pt}
\item [\bf (a) \ \ ] $x+10\,\sqrt{x+2\, }\,  -22 =0\,$;
\setlength\itemindent {8pt}
\item [\bf (b) \ \ ] $x^2 -4x  + \sqrt{2x^2 -8x-3 \,}\, -9 =0\,$.
\end{itemize}
\end{questionparts}
\end{question}

%%%%%%%%%%Q2
\begin{question}
In this question, $\lfloor x \rfloor$ denotes the greatest integer
that  is less than or equal to $x$, so that 
$\lfloor 2.9 \rfloor = 2 = \lfloor 2.0 \rfloor$
and
 $\lfloor -1.5 \rfloor = -2$.

The function $\f$ is defined, for $x\ne0$,
 by $\f(x) = \dfrac{\lfloor x \rfloor}{x}\,$.
\begin{questionparts}
\item Sketch the graph of $y=\f(x)$ for $-3\le x \le 3$ (with $x\ne0$).

\item By considering the line $y= \frac7{12}$ on your graph, or otherwise,
solve the equation $\f(x) = \frac7 {12}\,$.

Solve also the equations   $\f(x) =\frac{17}{24}$
 \ \ and \ $\f(x) = \frac{4 }{3 }\,$.

\item
Find the largest root of the equation $\f(x) =\frac9{10}\,$.


\end{questionparts}

Give necessary and sufficient conditions, in the form of inequalities,
for the equation $\f(x) =c$ to have exactly $n$ roots, where $n\ge1$. 
\end{question}

%%%%%%%%% Q3
\begin{question}
For any two points $X$ and $Y$, with position vectors
$\bf x$ and $\bf y$ respectively, $X*Y$ is defined
to be the point with position vector $\lambda {\bf x}+ (1-\lambda){\bf y}$,
where $\lambda$ is a fixed number. 



\begin{questionparts}
\item
If $X$ and $Y$ are distinct,
 show that 
$X*Y$ and  $Y*X$ are distinct
unless $\lambda$ takes a certain value (which you should
state). 

\item Under what conditions 
are $(X*Y)*Z$ and   $X*(Y*Z)\,$ distinct?

\item
Show that, for any points $X$, $Y$ and $Z$,
\[
(X*Y)*Z = (X*Z)*(Y*Z)\,
\]
and 
obtain  the corresponding result for $X*(Y*Z)$.

\item  The points $P_1$, $P_2$, $\ldots$
are defined by
$ 
P_1 = X*Y
$
and, for $n \ge2$,         $P_n= P_{n-1}*Y\,.$
Given that $X$ and $Y$ are distinct and that
$0<\lambda<1$, find the
 ratio in which  $P_n$ divides the line segment $XY$.

\end{questionparts}
\end{question}

%%%%%% Q4 
\begin{question}
\begin{questionparts}
\item
Show that, for $n>  0$, 
\[
\int_0^{\frac14\pi} \tan^n x \,\sec^2 x \, \d x =
\frac 1 {n+1} \; 
\text{ \ \ \ and  \ \ \ }
 \int_0^{\frac14\pi} \!\! \sec ^n\! x \, \tan x \,
\d x = \frac{(\sqrt 2)^n - 1}n \,.
\] 

\item Evaluate the following integrals:
\[
\displaystyle 
\int_0^{\frac14\pi}
\!\! x\, \sec ^4 \! x\, \tan x \,  \d x \,
\text{ \ \ \ and \ \ \ }
\int_0^{\frac14\pi}
\!\! x^2 \sec ^2 \! x\, \tan x \,  \d x \,.
\]

\end{questionparts}
\end{question}

%%%%%%%%% Q5
\begin{question}
The point $P$ has coordinates $(x,y)$ which satisfy
\[
x^2+y^2 + kxy +3x +y =0\,.
\]
\vspace*{-8mm}
\begin{questionparts}
\item
Sketch the locus of $P$ in the case $k=0$, giving the points of 
intersection with the coordinate axes.
\item By factorising $3x^2 +3y^2 +10xy$, or otherwise,
 sketch the locus of $P$ in the case $k=\frac{10}{3}\,$,
 giving the points of 
intersection with the coordinate axes.

\item In the case $k=2$, 
let $Q$ be the point obtained by rotating $P$ 
clockwise about the origin 
by an angle~$\theta$, so that the coordinates $(X,Y)$
of $Q$
are given by 
\[
X=x\cos\theta +y\sin\theta\,, \ \ \  \ Y= -x\sin\theta + y\cos\theta\,.
\]
Show that, for $\theta =45^\circ$, 
the locus of $Q$  is 
$ 
\sqrt2 Y=   (\sqrt2 X+1  )^2 - 1 .
$ 

Hence, or otherwise, sketch the locus of $P$ in the case $k=2$, 
giving the equation of the 
line of symmetry.
\end{questionparts}
\end{question}
	
%%%%%%%%% Q6
\begin{question}
By considering the coefficient of $x^r$ in the series for $(1+x)(1+x)^n$,
or otherwise, obtain the following relation between binomial coefficients:
\[
\binom n r + \binom n {r-1} = \binom {n+1} r
\ \ \ \ \ \ \ \ \ \ \ \ \ (1\le r\le n).
\]

The sequence of numbers $B_0$, $B_1$, $B_2$, $\ldots$ is defined by
\[
B_{2m} = \sum_{j=0}^m \binom{2m-j}j 
\text{ \ \ \ \ \ and \ \ \ \ \ }
B_{2m+1} = \sum_{k=0}^m \binom{2m+1-k}k 
.
\]
Show that $B_{n+2} - B_{n+1} = B_{n}\,$ ($n=0$, $1$, $2$, $\ldots\,$).

What is the relation between the sequence $B_0$,   $B_1$, $B_2$, $\ldots$ 
and the Fibonacci sequence  $F_0$, $F_1$, $F_2$, $\ldots$  defined by $F_0=0$,
$F_1=1$ and $F_n = F_{n-1}+F_{n-2}$ for $n\ge2$?
\end{question}
	
%%%%%%%%% Q7
\begin{question}
\begin{questionparts}
\item
Use the substitution $y=ux$, where $u$ is a function of $x$,
to show that the solution of the differential equation
\[
\frac{\d y}{\d x} = \frac x y + \frac y x
\ \ \ \ \ \ \ \ \ \ (x>  0, \ y>  0)
\] that satisfies $y=2$ when $x=1$
is 
\[
y= x\,  \sqrt{4+2\ln  x  \, }
\ \ \ \ \ \ \ \ \ \ ( x > \e^{-2}).
\]
\item Use a substitution to find the solution of the differential equation
\[
\frac{\d y}{\d x} = \frac x y + \frac {2y} x
\ \ \ \ \ \ \ \ \ \ (x>  0, \ y>  0)
\] that satisfies $y=2$ when $x=1$.
\item
Find the solution of the differential equation
\[
\frac{\d y}{\d x} = \frac {x^2} y + \frac {2y} x
\ \ \ \ \ \ \ \ \ \ (x>  0, \ y>  0)
\] that satisfies $y=2$ when $x=1$.
\end{questionparts}
\end{question}
		
%%%%%%%%% Q8
\begin{question}
\begin{questionparts}
\item
The functions $\mathrm{a, b, c}$ and $\mathrm{d}$ are defined 
by
\begin{itemize}
\item[]
${\rm a}(x) =x^2 \ \ \ \ (-\infty<x<\infty),$
\item[]
${\rm b}(x) = \ln x  \ \  \ \ (x>0),$
\item[]
${\rm c}(x) =2x \ \ \ \     (-\infty<x<\infty),$
\item[]
${\rm d}(x)= \sqrt x  \ \ \ \ (x\ge0) \,.$
\end{itemize}
Write down the following composite
functions, giving  the domain and range of each:
\[
\rm 
cb, \ \ \ \ \ \
ab, \ \ \ \ \ \
da, \ \ \ \ \ \
ad. \ \ \ \ \ \ 
\]

\item The functions $\mathrm{f}$ and $\mathrm{g}$ are defined 
 by
\begin{itemize}
\item[] $ \f(x)= \sqrt{x^2-1\,} \ \ \ \ (\vert x \vert \ge 1),$
\item []        
$ \g(x) = \sqrt{x^2+1\,} \ \ \ \ (-\infty<x<\infty).$
\end{itemize}
Determine the composite functions $\mathrm{fg}$ and $\mathrm{gf}$, giving 
the domain and range of each.

\item Sketch the graphs of the functions h and k defined 
 by
\begin{itemize}
\item[]
$\h(x) = x+\sqrt{x^2-1\,}\, \ \ \ \ ( x \ge1)$,
\item[]
 ${\rm k}(x) = x-\sqrt{x^2-1\,}\, \ \ \ \ (\vert x\vert \ge1),$
\end{itemize}
justifying the main features of the graphs, and giving the equations of 
any asymptotes. Determine
the domain and range of the composite function
$\mathrm{kh}$.

\end{questionparts}
\end{question}	
		

		
	
\newpage
\section*{Section B: \ \ \ Mechanics}


	
%%%%%%%%%% Q9
\begin{question}
Two particles, $A$  and $B$, are 
projected simultaneously towards each other from two points 
which are a distance
$d$ apart in a   
horizontal plane.
Particle $A$ has mass $m$ and is projected
at speed~$u$ at angle $\alpha$ above the horizontal.  
Particle $B$ has mass $M$ and 
is projected at speed $v$ at angle $\beta$
above the horizontal. 
The trajectories of the two particles lie in the same  vertical 
plane.


The particles collide
directly when each is at its point of greatest height above the
plane. 
 Given that 
both  $A$ and $B$  return to their starting points, and that
momentum is conserved in the collision, show that
\[
m\cot \alpha = M \cot \beta\,.
\]

Show further that the collision occurs at a point which is a 
horizontal distance  $b$ 
from the point
of projection of $A$ 
where
\[
b= \frac{Md}{m+M}\, ,
\]
and find, 
in terms of $b$ and $\alpha$, 
the height above the horizontal plane at which the collision occurs.
	\end{question}
	
%%%%%%%%%% Q10 
\begin{question}	
Two parallel vertical barriers are fixed a distance $d$ apart on 
horizontal ice. 
A small ice hockey puck moves on the ice backwards and forwards
between the barriers, 
in the direction perpendicular
to the barriers,
colliding with 
each in turn.
The coefficient of friction 
between the puck and the ice is $\mu$ and the coefficient
of restitution between the puck and each of the barriers is~$r$. 


The puck starts at one of the barriers,  moving with speed
$v$ towards the other barrier. 
Show that
\[
v_{i+1}^2 - r^2 v_i^2 = - 2 r^2 \mu gd\,
\]
where $v_i$ is the speed of the puck just after its $i$th collision.



The puck comes to rest against one of the barriers
after traversing the gap between them $n$ times.
In the case $r\ne1$, express $n$ in terms of 
$r$ and $k$, where 
$k= \dfrac{v^2}{2\mu g d}\,$. \ 
If $r=\e^{-1}$ (where $\e$ is the base of natural logarithms) 
show that
\[
n = \tfrac12 \ln\big(1+k(\e^2-1)\big)\,.
\]

Give an expression for $n$ in the case $r=1$.
\end{question}

%%%%%%%%%% Q11

\begin{question}$\,$
\begin{center}
\psset{xunit=1.2cm,yunit=1.2cm,algebraic=true,dimen=middle,dotstyle=o,dotsize=3pt 0,linewidth=0.3pt,arrowsize=3pt 2,arrowinset=0.25}
\begin{pspicture*}(-3.5,-0.23)(3.5,2.88)
\pspolygon[fillcolor=black,fillstyle=solid,opacity=1.0](-0.5,0)(0,0)(0,0.2)(-0.5,0.2)
\psline(-0.5,0)(0,0)
\psline(0,0)(0,0.2)
\psline(0,0.2)(-0.5,0.2)
\psline(-0.5,0.2)(-0.5,0)
\psline(0,0.1)(3,2.5)
\psline(-0.5,0.08)(-2.5,2.5)
\psline(-0.5,0.08)(-0.43,0)
\parametricplot{2.2619644025662233}{3.141592653589793}{1.4*cos(t)+-0.43|1.4*sin(t)+0}
\psline(0,0.1)(-0.13,0)
\parametricplot{0.0}{0.6744985164193813}{1*cos(t)+-0.13|1*sin(t)+0}
\psline(0,0)(5.5,0)
\psline(-0.5,0)(-7,0)
\rput[tl](0.3,0.25){$\alpha$}
\rput[tl](-1.7,0.43){$\frac{1}{2}\pi-\alpha$}
\rput[tl](-0.35,0.55){$C$}
\rput[tl](-2.7,2.85){$A$}
\rput[tl](3,2.8){$B$}
\end{pspicture*}
\end{center}

The diagram shows a small block $C$ of weight $W$ initially at
rest on a rough horizontal surface. The coefficient of friction
between the block and the surface is $\mu$. Two light strings,
$AC$ and $BC$, are attached to the block, making angles 
$\frac12 \pi -\alpha$ and $\alpha$ to the horizontal, respectively.
The 
tensions in $AC$ and $BC$ are $T\sin\beta$ and $T\cos\beta$ 
respectively, where $0< \alpha+\beta<\frac12\pi$.

\begin{questionparts}
\item In the case $W> T\sin(\alpha+\beta)$, show that
the block will remain at rest provided
\[
W\sin\lambda \ge T\cos(\alpha+\beta- \lambda)\,,
\]
where $\lambda$ is the acute angle such that $\tan\lambda = \mu$. 
\item
In the case $W=T\tan\phi$, where $2\phi =\alpha+\beta$, show
that the block will start to move in a direction that makes an angle
$\phi$ with the horizontal.

\end{questionparts}
\end{question}
	

	
	\newpage
\section*{Section C: \ \ \ Probability and Statistics}


%%%%%%%%%% Q12
\begin{question}
Each day, I have to take $k$ different types of medicine, one
tablet of each. The tablets are identical in appearance. When I go
on holiday for $n$ days, I put $n$ tablets of each type in a
container and on each day of the holiday I select $k$ tablets
at random from the container.

\begin{questionparts}
\item In the case $k=3$, show that the probability that
I will select one tablet of each type on the first day of a
three-day holiday is $\frac9{28}$. 

Write down  the probability
that I will be left with one tablet of each type on the 
last day (irrespective of the tablets I select on the first day).

\item In the case $k=3$, find the probability that
I will select one tablet of each type on the first day of an
$n$-day holiday. 

\item In the case $k=2$,   find the probability that
I will select one tablet of each type on each day of an
$n$-day holiday, and use Stirling's approximation
\[
n!\approx \sqrt{2n\pi} \left(\frac n\e\right)^n
\]
to show that this probability is approximately $2^{-n} \sqrt{n\pi\;}$.

\end{questionparts}
\end{question}

%%%%%%%%%% Q13
\begin{question}
From the integers $1, 2, \ldots , 52$, I choose seven 
(distinct) integers at random, all choices being equally likely. From these
seven, I discard any pair that sum to 53. Let $X$ be the random variable
the value of which is the number of 
discarded pairs. Find the probability distribution of $X$ and show
that $\E   (X) = \frac 7 {17}$.

\noindent {\bf Note:} $7\times 17 \times 47 =5593$.
\end{question}

\end{document}
