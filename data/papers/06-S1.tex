\documentclass[a4, 11pt]{report}


\pagestyle{myheadings}
\markboth{}{Paper I, 2006
\ \ \ \ \ 
\today 
}               

\RequirePackage{amssymb}
\RequirePackage{amsmath}
\RequirePackage{graphicx}
\RequirePackage{color}
\RequirePackage[flushleft]{paralist}[2013/06/09]



\RequirePackage{geometry}
\geometry{%
  a4paper,
  lmargin=2cm,
  rmargin=2.5cm,
  tmargin=3.5cm,
  bmargin=2.5cm,
  footskip=12pt,
  headheight=24pt}


\newcommand{\comment}[1]{{\bf Comment} {\it #1}}
%\renewcommand{\comment}[1]{}

\newcommand{\bluecomment}[1]{{\color{blue}#1}}
%\renewcommand{\comment}[1]{}
\newcommand{\redcomment}[1]{{\color{red}#1}}



\usepackage{epsfig}
\usepackage{pstricks-add}
\usepackage{tgheros} %% changes sans-serif font to TeX Gyre Heros (tex-gyre)
\renewcommand{\familydefault}{\sfdefault} %% changes font to sans-serif
%\usepackage{sfmath}  %%%% this makes equation sans-serif
%\input RexFigs


\setlength{\parskip}{10pt}
\setlength{\parindent}{0pt}

\newlength{\qspace}
\setlength{\qspace}{20pt}


\newcounter{qnumber}
\setcounter{qnumber}{0}

\newenvironment{question}%
 {\vspace{\qspace}
  \begin{enumerate}[\bfseries 1\quad][10]%
    \setcounter{enumi}{\value{qnumber}}%
    \item%
 }
{
  \end{enumerate}
  \filbreak
  \stepcounter{qnumber}
 }


\newenvironment{questionparts}[1][1]%
 {
  \begin{enumerate}[\bfseries (i)]%
    \setcounter{enumii}{#1}
    \addtocounter{enumii}{-1}
    \setlength{\itemsep}{5mm}
    \setlength{\parskip}{8pt}
 }
 {
  \end{enumerate}
 }



\DeclareMathOperator{\cosec}{cosec}
\DeclareMathOperator{\Var}{Var}

\def\d{{\mathrm d}}
\def\e{{\mathrm e}}
\def\g{{\mathrm g}}
\def\h{{\mathrm h}}
\def\f{{\mathrm f}}
\def\p{{\mathrm p}}
\def\s{{\mathrm s}}
\def\t{{\mathrm t}}


\def\A{{\mathrm A}}
\def\B{{\mathrm B}}
\def\E{{\mathrm E}}
\def\F{{\mathrm F}}
\def\G{{\mathrm G}}
\def\H{{\mathrm H}}
\def\P{{\mathrm P}}


\def\bb{\mathbf b}
\def \bc{\mathbf c}
\def\bx {\mathbf x}
\def\bn {\mathbf n}

\newcommand{\low}{^{\vphantom{()}}}
%%%%% to lower suffices: $X\low_1$ etc


\newcommand{\subone}{ {\vphantom{\dot A}1}}
\newcommand{\subtwo}{ {\vphantom{\dot A}2}}




\def\le{\leqslant}
\def\ge{\geqslant}


\def\var{{\rm Var}\,}

\newcommand{\ds}{\displaystyle}
\newcommand{\ts}{\textstyle}
\def\half{{\textstyle \frac12}}
\def\l{\left(}
\def\r{\right)}



\begin{document}
\setcounter{page}{2}

 
\section*{Section A: \ \ \ Pure Mathematics}

%%%%%%%%%%Q1
\begin{question}
Find the integer, $n$, that satisfies  
$n^2 < 33\hspace{1.2pt}127< (n+1)^2$. Find
also a small integer $m$ such that $(n+m)^2 -33\hspace{1.2pt}127$
 is a perfect square.
Hence express $33\hspace{1.2pt}127$ in the form $pq$, where $p$ and $q$ are integers
greater than $1$. 

By considering the possible factorisations of $33\hspace{1.2pt}127$, show that
there are exactly two values of $m$ 
for which  $(n+m)^2 -33\hspace{1.2pt}127$ is a perfect square,
and find the other value.
\end{question}

%%%%%%%%%%Q2
\begin{question}
A small goat is tethered by a rope to a point at ground level on a
side of a 
square barn  which stands in a large horizontal field of grass. 
The sides of the barn are of length $2a$
and  the rope is of length~$4a$. 
Let $A$ be the area of the grass that the goat can
graze. Prove that $A\le14\pi a^2$ and determine the minimum value of
$A$.
\end{question}

%%%%%%%%% Q3
\begin{question}
In this question $b$, $c$, $p$ and $q$ are real numbers.
\begin{questionparts}
\item By considering the graph $y=x^2 + bx + c$ 
show that $c < 0$ is a sufficient condition for the equation 
$\displaystyle x^2 + bx + c = 0$ to have distinct real roots. 
Determine whether $c < 0$ is a necessary condition for the 
equation to have distinct real roots.


\item
Determine necessary and 
sufficient conditions for the equation $\displaystyle x^2 + bx + c = 0$
to have  distinct positive real roots.


\item What can be deduced about the 
number and the nature of the roots of the equation 
$x^3 + px + q = 0$ if $p>0$ and $q<0$?

What can be deduced if $p<0\,$ and $q<0$? You should consider 
the different cases that arise according to the value of
$4p^3+ 27q^2\,$.

\end{questionparts}
\end{question}

%%%%%% Q4 
\begin{question}
By sketching  on the same axes the graphs of  $y=\sin x$ and
$y=x$, show that, for $x>0$:
\begin{questionparts}
\item $x>\sin x\,$;
\item $\dfrac {\sin x} {x} \approx 1$ for small $x$. 
\end{questionparts}

A regular polygon has $n$ sides, and perimeter $P$. 
Show that the area of the polygon is
\[
\displaystyle \frac{P^2} { {4n \tan \l\dfrac{ \pi} { n} \r}} \;.
\]
Show by differentiation (treating $n$ as a continuous variable)
  that the area of the polygon 
increases as $n$ increases with $P$ fixed.

 Show also that, for large $n$, the ratio of the 
area of the polygon to the 
area of the smallest circle which can be drawn around the polygon is approximately $1$.
\end{question}

%%%%%%%%% Q5
\begin{question}
\begin{questionparts}
\item
Use the substitution $u^2=2x+1$ to show that, for $x>4$, 
\[
\int \frac{3} { ( x-4 ) \sqrt {2x+1}} \; 
\d x = \ln \l \frac{\sqrt{2x+1}-3} {\sqrt{2x+1}+3} \r + K\,,
\]
where $K$ is a constant.

 
\item Show that  $ \displaystyle \int_{\ln 3}^{\ln 8}
 \frac{2} { \e^x \sqrt{ \e^x + 1}}\; \mathrm{d}x\, = 
 \frac 7{12} + \ln \frac23 $
.

\end{questionparts}
	\end{question}
	
%%%%%%%%% Q6
\begin{question}
\begin{questionparts}
\item Show that, if $\l a \, , b\r$  is {\bf any} point on the 
curve $x^2 - 2y^2 = 1$, 
then $\l 3a + 4b \, , 2a + 3b \r\,$ also lies on the curve.

\item Determine the smallest positive integers
$M$ and $N$ such that, if $\l a \,,  b\r$ is {\bf any} point on the curve
$Mx^2 - Ny^2 = 1$, then $(5a+6b\,, 4a+5b)$ also lies on the curve.

\item Given that the point $\l a \, , b\r$ lies on 
the curve $x^2 - 3y^2 = 1\,$, 
find positive integers $P$, $Q$, $R$ and $S$ such that
the point  $(P a +Q b\,, R a + Sb)$ also lies on the  curve.


\end{questionparts}
\end{question}
	
%%%%%%%%% Q7
\begin{question}
\begin{questionparts}
\item Sketch on the same axes the functions 
${\rm cosec}\, x$ and $2x/ \pi$, for 
$0<x<\pi\,$. Deduce that the equation $x\sin x = \pi/2 $ has exactly
two roots in the interval  $0<x<\pi\,$.

Show that 
\[
\displaystyle \int_{\pi/2}^{\pi} \left \vert x\sin x  - 
\frac{\pi} { 2} \right \vert \; \mathrm{d}x 
= 
2\sin\alpha   +\frac{3\pi^2} 4 - \alpha \pi -\pi -2\alpha
\cos\alpha -1
\]
where $\alpha$ is the larger of the roots referred to above.


\item Show that the  region bounded by the positive $x$-axis, the $y$-axis
  and the curve
\[y = \Bigl| \vert \e^x - 1 \vert - 1 \Bigr|\] 
has area $\ln 4-1$.
\end{questionparts}
\end{question}
		
%%%%%%%%% Q8
\begin{question}	
{\it Note that the volume of a 
tetrahedron is equal to  $\frac1  3$ $\times$ 
the area of the base $\times$ the height.}

The points $O$, $A$, $B$ and $C$ have coordinates $(0,0,0)$, $(a,0,0)$, $(0,b,0)$
and $(0,0,c)$, respectively, where $a$, $b$ and $c$ are positive.
\begin{questionparts}
\item Find, in terms of $a$, $b$ and $c$,  the volume of the tetrahedron
  $OABC$.
\item
 Let angle $ACB = \theta$. Show that
\[
\cos\theta = 
\frac {c^2}
{ 
{ \sqrt{\vphantom{ \dot b}
(a^2+c^2)(b^2+c^2)} }
^{\vphantom A}    
\  }
\]
and find, in terms of $a$, $b$ and $c$, the  area of triangle $ABC$.
% is 
%$\displaystyle \tfrac12 \sqrt{ \vphantom{\dot A } a^2b^2 +b^2c^2 + c^2 a^2 \;} \;$.
\end{questionparts}

Hence show that $d$,  the perpendicular distance of the origin from
the  triangle $ABC$, satisfies
\[
\frac 1{d^2} = \frac 1 {a^2} + \frac 1 {b^2} + \frac 1 {c^2} \,.
\]
\end{question}	
		

		
	
\newpage
\section*{Section B: \ \ \ Mechanics}


	
%%%%%%%%%% Q9
\begin{question}
A block of mass $4\,$kg is at rest on a smooth, horizontal
table. A smooth pulley $P$ is fixed to one edge 
of the table and a  smooth pulley $Q$ is fixed to the opposite edge. 
The two pulleys and the block lie in a straight line.

Two horizontal strings are attached to the block. 
One string runs over pulley $P$; a particle of mass $x\,$kg hangs  at the end of this 
string. 
The other string runs over pulley $Q$;   a particle of mass $y\,$kg hangs
at the end of this string, where $x > y$ and $x + y = 6\,$.

The system is released from rest with the strings taut. 
When the $4\,$kg block has moved a distance $d$, 
the string connecting it to the particle of mass $x\,$kg is cut. 
Show that the time taken by the block from the 
start of the motion until it first returns to rest 
(assuming that it does not reach the edge of the table)
is $\sqrt{d/(5g)\,} \,\f(y)$, where
\[
\f(y)= \frac{10}{  \sqrt{6-2y}}+  \left(1 + \frac{4}{ y} \right) \sqrt {6 -2y}.
\]
Calculate the value of $y$ for which $\f'(y)=0$.
	\end{question}
	
%%%%%%%%%% Q10 
\begin{question}	
A particle $P$ is projected in the $x$-$y$ plane,  where the $y$-axis
is vertical and the $x$-axis is horizontal. 
The particle is projected with speed $V$ from  the origin at an
angle of
$45 ^\circ$ above the positive $x$-axis.
Determine the equation of the trajectory 
 of $P$.



The point of projection (the origin) is on  the floor of a barn. The roof of the barn is 
given by the equation
$y= x \tan \alpha +b\,$, where $b>0$ and $\alpha$ is an acute angle. 
Show that, if  the particle just touches the roof, then  
$V(-1+ \tan\alpha) =-2 \sqrt{bg}$;  you should justify the choice of the 
negative root. 
 If this condition is satisfied,
find, in terms of $\alpha$, $V$ and $g$,
 the time after 
projection at which touching  takes place.



A particle $Q$ can slide along a 
smooth rail fixed, in the $x$-$y$ plane, to the under-side of the roof.
It is projected from the point $(0,b)$ with speed $U$
at the same time as $P$ is projected from the origin.
 Given that the particles just touch
in the course of their motions, show that
\[
2 \sqrt 2 \, U \cos \alpha  = V \big(2 +  \sin\alpha\cos\alpha -\sin^2\alpha)
.
\]
\end{question}

%%%%%%%%%% Q11

\begin{question}
Particles  $A_1$, $A_2$, $A_3$, 
$\ldots$, $A_n$ (where $n\ge 2$) lie at rest in that order in a smooth straight horizontal
trough. The mass of $A_{n-1}$ is $m$ and the mass of $A_n$ is
$\lambda m$, where $\lambda>1$.
Another particle, $A_0$, of mass $m$, 
slides along the trough with speed $u$ towards the particles and collides with  $A_1$. 
Momentum and energy are conserved in all  collisions.


\begin{questionparts}
\item Show that 
it is not possible for  there to be exactly one particle 
moving after all collisions have taken place.
\item Show that 
it is not possible for  $A_{n-1}$ and $A_n$ to be the only particles
moving after all collisions have taken place.
\item Show that 
it is not possible for  $A_{n-2}$, $A_{n-1}$ and $A_n$ to be the only particles
moving after all collisions have taken place.
\item Given that there are exactly two 
particles 
moving after all collisions have taken place, find the speeds of these
particles in terms of $u$ and $\lambda$.
\end{questionparts}
\end{question}
	

	
	\newpage
\section*{Section C: \ \ \ Probability and Statistics}


%%%%%%%%%% Q12
\begin{question}
Oxtown and Camville are connected by three roads, 
which are at risk of being blocked by flooding. 
On two of the three roads there are two sections 
which may be blocked. On the third road 
there is only one section which may be blocked. 
The probability that each section is blocked is $p$. 
Each section is blocked independently of the other four sections. 
Show that the probability that Oxtown is cut off from 
Camville is $p^3 \l 2-p \r^2$.

I want to travel from Oxtown to Camville. I choose 
one of the three roads at random and find that my road is not blocked. 
Find the probability that I would not have reached Camville
if I had chosen either of the other two roads.
You should factorise your answer as fully as possible. 

Comment briefly on the value of this probability in the limit $p\to1$.
\end{question}

%%%%%%%%%% Q13
\begin{question}
A very generous shop-owner is hiding small diamonds in 
chocolate bars. Each diamond is hidden independently of any other diamond, 
and on average there is one diamond per kilogram of chocolate.

\begin{questionparts}
\item I go to the shop and roll a fair six-sided die once. 
I decide  that if I roll a score of $N$, I will buy $100N$ grams 
of chocolate. 
Show that the probability that I will have no diamonds is
\[
\frac{\e^{-0.1}}{ 6} \l \frac{1 - \e^{-0.6}  }{  1 - \e^{-0.1}} \r
\]
Show also that the expected number of diamonds I find is 0.35.

\item Instead, I decide to roll a fair six-sided 
die repeatedly until I score a 6. If I roll my first 6 on my $T$th throw, 
I will buy $100T$ grams of chocolate. 
Show that the probability that I will have no diamonds is 
\[
\frac{\e^{-0.1}}{ 6 - 5\e^{-0.1}}
\]
Calculate also the expected number of diamonds that I find.
(You may find it useful to consider the
the binomial 
expansion of $\l 1 - x \r^{-2}$.)

\end{questionparts}
\end{question}

%%%%%%%%%% Q14
\begin{question}
\begin{questionparts}
\item A bag of sweets contains one red sweet and $n$ blue sweets. 
I take a sweet from the bag, note  its colour, 
return it  to the bag, then shake the bag. 
I repeat this until the  sweet I take is the red one.
Find an expression for the probability that I take
 the red sweet on the $r$th attempt. 
What value of $n$ maximises this probability? 

\item Instead, I take sweets from the bag, without 
replacing them in the bag, until I take the red sweet.
Find an expression for the probability that I take
 the red sweet on the $r$th attempt. 
What value of $n$ maximises this probability? 

\end{questionparts}
\end{question}
	
\end{document}
