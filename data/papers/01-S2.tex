\documentclass[a4, 11pt]{report}


\pagestyle{myheadings}
\markboth{}{Paper II, 2001
\ \ \ \ \ 
\today 
}               

\RequirePackage{amssymb}
\RequirePackage{amsmath}
\RequirePackage{graphicx}
\RequirePackage{color}
\RequirePackage[flushleft]{paralist}[2013/06/09]



\RequirePackage{geometry}
\geometry{%
  a4paper,
  lmargin=2cm,
  rmargin=2.5cm,
  tmargin=3.5cm,
  bmargin=2.5cm,
  footskip=12pt,
  headheight=24pt}


\newcommand{\comment}[1]{{\bf Comment} {\it #1}}
%\renewcommand{\comment}[1]{}

\newcommand{\bluecomment}[1]{{\color{blue}#1}}
%\renewcommand{\comment}[1]{}
\newcommand{\redcomment}[1]{{\color{red}#1}}



\usepackage{epsfig}
\usepackage{pstricks-add}
\usepackage{tgheros} %% changes sans-serif font to TeX Gyre Heros (tex-gyre)
\renewcommand{\familydefault}{\sfdefault} %% changes font to sans-serif
%\usepackage{sfmath}  %%%% this makes equation sans-serif
%\input RexFigs


\setlength{\parskip}{10pt}
\setlength{\parindent}{0pt}

\newlength{\qspace}
\setlength{\qspace}{20pt}


\newcounter{qnumber}
\setcounter{qnumber}{0}

\newenvironment{question}%
 {\vspace{\qspace}
  \begin{enumerate}[\bfseries 1\quad][10]%
    \setcounter{enumi}{\value{qnumber}}%
    \item%
 }
{
  \end{enumerate}
  \filbreak
  \stepcounter{qnumber}
 }


\newenvironment{questionparts}[1][1]%
 {
  \begin{enumerate}[\bfseries (i)]%
    \setcounter{enumii}{#1}
    \addtocounter{enumii}{-1}
    \setlength{\itemsep}{5mm}
    \setlength{\parskip}{8pt}
 }
 {
  \end{enumerate}
 }



\DeclareMathOperator{\cosec}{cosec}
\DeclareMathOperator{\Var}{Var}

\def\d{{\mathrm d}}
\def\e{{\mathrm e}}
\def\g{{\mathrm g}}
\def\h{{\mathrm h}}
\def\f{{\mathrm f}}
\def\p{{\mathrm p}}
\def\s{{\mathrm s}}
\def\t{{\mathrm t}}


\def\A{{\mathrm A}}
\def\B{{\mathrm B}}
\def\E{{\mathrm E}}
\def\F{{\mathrm F}}
\def\G{{\mathrm G}}
\def\H{{\mathrm H}}
\def\P{{\mathrm P}}


\def\bb{\mathbf b}
\def \bc{\mathbf c}
\def\bx {\mathbf x}
\def\bn {\mathbf n}

\newcommand{\low}{^{\vphantom{()}}}
%%%%% to lower suffices: $X\low_1$ etc


\newcommand{\subone}{ {\vphantom{\dot A}1}}
\newcommand{\subtwo}{ {\vphantom{\dot A}2}}




\def\le{\leqslant}
\def\ge{\geqslant}


\def\var{{\rm Var}\,}

\newcommand{\ds}{\displaystyle}
\newcommand{\ts}{\textstyle}
\def\half{{\textstyle \frac12}}




\begin{document}
\setcounter{page}{2}

 
\section*{Section A: \ \ \ Pure Mathematics}

%%%%%%%%%%Q1
\begin{question}
 Use the binomial expansion to obtain
  a polynomial of degree $2$ which is a good approximation
 to $\sqrt{1-x}$ when $x$ is small.  

 \begin{questionparts}

 \item
 By taking  $x=1/100$, show that $\sqrt{11}\approx79599/24000$,
 and estimate, correct to 1 significant figure, 
 the error in this approximation. (You may assume that the error is given approximately by the 
 first neglected term in the binomial expansion.)


 \item
 Find a rational number which approximates  $\sqrt{1111}$ with an error
 of about  $2 \times {10}^{-12}$.
 \end{questionparts}
\end{question}

%%%%%%%%%%Q2
\begin{question}
Sketch the graph of the function $[x/N]$, for $0<x<2N$, where 
the notation  $[y]$ means the integer part of $y$.
(Thus $[2.9] = 2$, \ $[4]=4$.) 

\begin{questionparts}
\item Prove that 
\[
\sum_{k=1}^{2N} (-1)^{[k/N]} k = 2N-N^2.
\]
\item Let
\[
S_N = \sum_{k=1}^{2N} (-1)^{[k/N]} 2^{-k}.
\]
Find  $S_N$ in terms of $N$ and determine the limit of $S_N$ as  $N\to\infty$.
\end{questionparts}
\end{question}

%%%%%%%%% Q3
\begin{question}
The cuboid $ABCDEFGH$ is such  $AE$, $BF$, $CG$, $DH$ are perpendicular 
to the opposite faces $ABCD$ and $EFGH$, and $AB =2, BC=1, 
AE={\lambda}$. Show that if $\alpha$ is the acute angle between the
diagonals $AG$ and $BH$ then 
$$
\cos {\alpha} 
= \left \vert \frac {3-{\lambda}^2} {5+{\lambda}^2} \right \vert
$$

Let $R$ be the ratio of the volume of the cuboid to its surface area. Show that
$R<\frac{1}{3}$ for all possible
values of $\lambda$.

Prove that, if $R\ge \frac{1}{4}$, then $\alpha \le \arccos \frac{1}{9}$.
\end{question}

%%%%%% Q4 
\begin{question}
Let
$$
\f(x) = P \, {\sin x} + Q\, {\sin 2x} + R\, {\sin 3x} \;. 
$$
Show that if $Q^2 < 4R(P-R)$, 
then the only values of $x$ for which $\f(x) = 0$ are given by $x=m\pi$, where $m$ is
an integer.
\newline
[You may assume that $\sin 3x  =  \sin x(4\cos^2 x -1)$.]

Now let 
$$
\g(x) =  {\sin 2nx} + {\sin 4nx} - {\sin 6nx},
$$
where $n$ is a positive integer and $0 < x < \frac{1}{2}\pi $.
Find an expression for the largest root of the equation
$\g(x)=0$, distinguishing between the
cases where $n$ is even and $n$ is odd.

\end{question}

%%%%%%%%% Q5
\begin{question}
The curve $C_1$ passes through the origin in the $x$--$y$ plane
and its gradient is given by
$$
\frac{\d y}{\d x} =x(1-x^2)\e^{-x^2}.
$$
Show that
$C_1$ has a minimum point at the origin and
a maximum point at 
$\left(1,{\frac12\, \e^{-1}} \right)$. Find the coordinates of the other stationary point.
Give a rough sketch of $C_1$.



The curve $C_2$ passes through the origin and its gradient is given by
$$
\frac{\d y}{\d x}=
x(1-x^2)\e^{-x^3}.
$$
Show that $C_2$ has a minimum point at the origin and a maximum point at $(1,k)$, where \phantom{}
$k >  \frac12 \,\e^{-1}.$ (You need not find $k$.)
	\end{question}
	
%%%%%%%%% Q6
\begin{question}
Show that
\[
\int_0^1 \frac{x^4}{1+x^2} \, \d x = \frac \pi {4} - \frac 23 \;.
\]

Determine the values of
\begin{questionparts}
\item $\displaystyle
\int_0^1 x^3 \; \tan ^{-1} \left(\frac {1-x} {1+x} \right) \,\d x \;,
$

\item
$\displaystyle
\int_0^1 \frac {(1-y)^3} {(1+y)^5} \; {{\tan}^{-1} y}\, \d y \;.$
\end{questionparts}

\end{question}
	
%%%%%%%%% Q7
\begin{question}
In an Argand diagram, $O$ is the origin and $P$ is the point 
$2+0\mathrm{i}$. The points  $Q$, $R$ and  $S$ are such that
the lengths  $OP$, $PQ$, $QR$ and $RS$ are all equal, and 
the angles $OPQ$, $PQR$ and $QRS$ are all equal to
${5{\pi}}/6$, so that the points $O$, $P$, $Q$, $R$ and $S$ are five vertices of 
a regular 12-sided polygon lying in the upper half of the Argand diagram. 
Show that $Q$ is the point 
$2 + \sqrt 3 + \mathrm{i}$ and find $S$.

The point  $C$ is 
 the centre of the circle that  passes through the points
$O$, $P$ and $Q$. Show that, if the polygon
is  rotated anticlockwise about $O$
until  $C$ first lies on the real axis,  the 
new position of $S$ is
$$
- \tfrac{1}{2}  (3\sqrt 2+ \sqrt6)(\sqrt3-\mathrm{i})\;. 
$$
\end{question}
		
%%%%%%%%% Q8
\begin{question}	
The function $\f$ satisfies $\f(x+1)= \f(x)$ and $\f(x)>0$ for all $x$. 
\begin{questionparts}
\item
Give an example of such a function.
\item
The function $\F$ satisfies
\[
\frac{\d \F}{\d x} =\f(x)
\]
and $\F(0)=0$.
Show that $\F(n) = n\F(1)$, for any positive integer $n$.
\item
Let $y$ be the solution of the differential equation
\[
\frac{\d y}{\d x} +\f(x) y=0
\]
that satisfies $y=1$ when $x=0$. Show that $y(n) \to 0$ as $n\to\infty$, where
$n= 1,\,2,\, 3,\, \ldots$ 
\end{questionparts}
\end{question}	
		

		
	
\newpage
\section*{Section B: \ \ \ Mechanics}


	
%%%%%%%%%% Q9
\begin{question}
A particle of unit mass
is projected vertically upwards with speed $u$. At height $x$, while the 
particle is moving upwards,
it is found to experience a total force $F$, due to gravity and air resistance,
 given by $F=\alpha \e^{-\beta x}$, where
$\alpha$ and $\beta$ are positive constants. Calculate the energy expended in
reaching this height. Show that 
\[
F= {\textstyle \frac12} \beta v^2+ \alpha - {\textstyle \frac12} \beta u^2 \;,
\]
where $v$ is the speed of the particle, and explain why $ \alpha = \frac12 \beta u^2 +g$,
where $g$ is the acceleration due to gravity. 

Determine  an expression, in terms of $y$, $g$ and $\beta$,
 for the air resistance  experienced by the particle on its
downward journey when it is at a distance $y$ below its highest point.
	\end{question}
	
%%%%%%%%%% Q10
\begin{question}	
Two particles $A$ and $B$ of masses $m$ and $km$,
respectively, are at rest on a smooth horizontal surface.
The direction of the line passing through $A$ and $B$ is perpendicular 
to a vertical wall which is on the other side of $B$ from $A$.
The particle $A$ is now set in motion towards $B$ with speed $u$.
The coefficient of restitution between $A$ and $B$ is $e_1$
and between $B$ and the wall is $e_2$. Show that
there will be a second collision between $A$ and $B$ provided 
$$
k< \frac {1+e_2(1+e_1)} {e_1}\;.
$$


Show that, if $e_1=\frac13$, $e_2=\frac12$ and $k<5$, then
the kinetic energy of $A$ and $B$ immediately after $B$ rebounds from the 
wall is greater than $mu^2/27$.
\end{question}

%%%%%%%%%% Q11

\begin{question}
A two-stage missile is projected from a point $A$
on the ground with horizontal and vertical velocity components
$u$ and $v$, respectively. When it reaches the highest point of
its trajectory an internal explosion causes it to break up into 
two fragments. Immediately after this explosion one of these 
fragments, $P$, begins to move vertically upwards with speed
$v_e$, but retains the  previous horizontal velocity. Show that
$P$ will hit the ground at a distance $R$ from $A$ given by 
$$
\frac{gR}u = v+v_e + \sqrt{v_e^2 +v^2}\, .
$$

It is required that the range $R$ should be greater than
a certain distance $D$ (where $D> 2uv/g$). Show that this requirement is satisfied
if 
\[
v_e> \frac{gD}{2u}\left( \frac{gD-2uv}{gD-uv}\right).
\] 

\noindent[{\sl The effect of air resistance is to  be neglected.}]
\end{question}
	

	
	\newpage
\section*{Section C: \ \ \ Probability and Statistics}


%%%%%%%%%% Q12
\begin{question}
The national lottery of Ruritania is based on the positive integers from $1$ to $N$,
where $N$ is very large and fixed. Tickets cost $\pounds1$ each.
For each ticket purchased, the punter (i.e. the purchaser)
chooses a number from  $1$ to $N$. The winning number 
 is chosen at random, and the jackpot is shared equally 
amongst those punters who chose the winning number. 

A syndicate decides to buy $N$ tickets, 
choosing every number once to be sure of winning a share of 
the jackpot. The total number of tickets purchased in this draw is $3.8N$ and 
the jackpot is   $\pounds W$. Assuming that the non-syndicate punters
choose their numbers independently and at random,
find the most probable number of 
winning tickets and show that the expected net loss of the syndicate is
approximately 
\[
N\; - \;
%\textstyle{
\frac{5
 \big(1- e^{-2.8}\big)}{14} \;W\;.
\]

\end{question}

%%%%%%%%%% Q13
\begin{question}
The life times of a large batch of electric light bulbs are independently
and identically distributed. The probability that the life time, $T$ hours, of a 
given light bulb is greater than $t$ hours is given by
\[
\P(T>t) \; = \; \frac{1}{(1+kt)^\alpha}\;,
\]
where $\alpha$ and $k$ are constants, and $\alpha >1$. 
Find the median $M$ and the mean $m$ of $T$ in terms of $\alpha$ and $k$.

Nine randomly selected bulbs are switched on simultaneously and are left until all have failed.
The fifth failure occurs at 1000 hours and the mean life time of all the bulbs is found 
to be 2400 hours. Show that $\alpha\approx2$
and find the approximate value of $k$.
Hence estimate the 
probability that, if a randomly selected bulb is found to last $M$ hours, it will
last a further $m-M$ hours.
\end{question}

%%%%%%%%%% Q14
\begin{question}
Two coins $A$ and $B$ are tossed together. $A$ has  
probability $p$ of showing a head, and $B$ has probability $2p$, independent of $A$,
of showing a head,
where $0<p<\frac12$.
The random variable $X$ takes the value 1 if $A$ 
shows a head and it takes the value $0$  if $A$ shows a tail. 
The random variable $Y$ takes the value 1 if $B$ 
shows a head and it takes the value $0$  if $B$ shows a tail.
The random variable $T$ is defined by
\[
T= \lambda X + {\textstyle\frac12} (1-\lambda)Y.
\]
Show that $\E(T)=p$ and find an expression for $\var(T)$ in terms of $p$ and $\lambda$.
Show that as $\lambda$ varies, the minimum of $\var(T)$ occurs when 
\[
\lambda =\frac{1-2p}{3-4p}\;.
\]

The two coins are tossed $n$ times, where $n>30$, and $\overline{T}$ is the mean value of $T$.
Let $b$ be a fixed positive number. Show that the maximum value of 
$\P\big(\vert \overline{T}-p\vert<b\big)$ as $\lambda$ varies is approximately $2\Phi(b/s)-1$,
 where $\Phi$ is the cumulative distribution function of a standard normal variate and 
\[
s^2= \frac{p(1-p)(1-2p)}{(3-4p)n}\;.
\]
\end{question}
	
\end{document}
