\documentclass[a4, 11pt]{report}


\pagestyle{myheadings}
\markboth{}{Paper I, 2001
\ \ \ \ \ 
\today 
}               

\RequirePackage{amssymb}
\RequirePackage{amsmath}
\RequirePackage{graphicx}
\RequirePackage{color}
\RequirePackage[flushleft]{paralist}[2013/06/09]



\RequirePackage{geometry}
\geometry{%
  a4paper,
  lmargin=2cm,
  rmargin=2.5cm,
  tmargin=3.5cm,
  bmargin=2.5cm,
  footskip=12pt,
  headheight=24pt}


\newcommand{\comment}[1]{{\bf Comment} {\it #1}}
%\renewcommand{\comment}[1]{}

\newcommand{\bluecomment}[1]{{\color{blue}#1}}
%\renewcommand{\comment}[1]{}
\newcommand{\redcomment}[1]{{\color{red}#1}}



\usepackage{epsfig}
\usepackage{pstricks-add}
\usepackage{tgheros} %% changes sans-serif font to TeX Gyre Heros (tex-gyre)
\renewcommand{\familydefault}{\sfdefault} %% changes font to sans-serif
%\usepackage{sfmath}  %%%% this makes equation sans-serif
%\input RexFigs


\setlength{\parskip}{10pt}
\setlength{\parindent}{0pt}

\newlength{\qspace}
\setlength{\qspace}{20pt}


\newcounter{qnumber}
\setcounter{qnumber}{0}

\newenvironment{question}%
 {\vspace{\qspace}
  \begin{enumerate}[\bfseries 1\quad][10]%
    \setcounter{enumi}{\value{qnumber}}%
    \item%
 }
{
  \end{enumerate}
  \filbreak
  \stepcounter{qnumber}
 }


\newenvironment{questionparts}[1][1]%
 {
  \begin{enumerate}[\bfseries (i)]%
    \setcounter{enumii}{#1}
    \addtocounter{enumii}{-1}
    \setlength{\itemsep}{5mm}
    \setlength{\parskip}{8pt}
 }
 {
  \end{enumerate}
 }



\DeclareMathOperator{\cosec}{cosec}
\DeclareMathOperator{\Var}{Var}

\def\d{{\mathrm d}}
\def\e{{\mathrm e}}
\def\g{{\mathrm g}}
\def\h{{\mathrm h}}
\def\f{{\mathrm f}}
\def\p{{\mathrm p}}
\def\q{{\mathrm q}}
\def\s{{\mathrm s}}
\def\t{{\mathrm t}}


\def\A{{\mathrm A}}
\def\B{{\mathrm B}}
\def\E{{\mathrm E}}
\def\F{{\mathrm F}}
\def\G{{\mathrm G}}
\def\H{{\mathrm H}}
\def\P{{\mathrm P}}


\def\bb{\mathbf b}
\def \bc{\mathbf c}
\def\bx {\mathbf x}
\def\bn {\mathbf n}

\newcommand{\low}{^{\vphantom{()}}}
%%%%% to lower suffices: $X\low_1$ etc


\newcommand{\subone}{ {\vphantom{\dot A}1}}
\newcommand{\subtwo}{ {\vphantom{\dot A}2}}




\def\le{\leqslant}
\def\ge{\geqslant}


\def\var{{\rm Var}\,}

\newcommand{\ds}{\displaystyle}
\newcommand{\ts}{\textstyle}
\def\half{{\textstyle \frac12}}




\begin{document}
\setcounter{page}{2}

 
\section*{Section A: \ \ \ Pure Mathematics}

%%%%%%%%%%Q1
\begin{question}
The points $A$, $B$ and $C$ lie on the sides of   a square of side 1 cm and 
no two points lie on the same side. 
Show that the length of
at least one side of the triangle $ABC$ must be less than or equal to 
$(\sqrt6 -\sqrt2)$ cm.
\end{question}

%%%%%%%%%%Q2
\begin{question}
Solve  the inequalities
\begin{questionparts}
\item $1+2x-x^2 >2/x$ \ \ \ \ \ \ ($x\ne0$) ,
\item $\sqrt{3x+10} > 2+\sqrt{x+4}$  \ \ \ \ \ \ ($x\ge -10/3$) .
\end{questionparts}
\end{question}

%%%%%%%%% Q3
\begin{question}
Sketch, without calculating the stationary points, the graph of
the function $\f(x)$ given by
\[
\f(x) = (x-p)(x-q)(x-r)\;,
\]
where $p<q<r$. By considering the quadratic  equation $\f'(x)=0$, or otherwise, show that
\[
(p+q+r)^2>3(qr+rp+pq)\;.
\]

By considering $(x^2+gx+h)(x-k)$, or otherwise, show that
$g^2>4h\,$ is a sufficient condition but not a necessary condition for the inequality
\[
(g-k)^2>3(h-gk)
\]
to hold.

\end{question}

%%%%%% Q4 
\begin{question}
Show that $\displaystyle \tan 3\theta = \frac{3\tan\theta -\tan^3\theta}{1-3\tan^2\theta}$ .

Given that $\theta=  \cos^{-1} (2/\sqrt5)$ and
$0<\theta<\pi/2$, show that
$
\tan 3\theta =11/2\;.
$

Hence, or otherwise, find all solutions of the equations
\begin{questionparts}
\item $\displaystyle
\tan(3\cos^{-1} x) =11/2$ , 
\item
$\displaystyle \cos ({\textstyle\frac13}\tan^{-1} y) = 2/\sqrt5$ .
\end{questionparts}
\end{question}

%%%%%%%%% Q5
\begin{question}
Show that (for $t>0$)
\begin{questionparts}
\item
$\displaystyle
\int_0^1 \frac1{(1+tx)^2} \d x = \frac1{(1+t)}\;,
$
\item
$\displaystyle
\int_0^1 \frac{-2x}{(1+tx)^3} \d x = -\frac1{(1+t)^2}\;.
$
\end{questionparts}

Noting that the right hand side of (ii) is the derivative of the right hand side of 
(i), 
conjecture the value of
\[
 \int_0^1 \frac{6x^2}{(1+x)^{4}} \d x \;.
\]
(You need not verify your conjecture.)
	\end{question}
	
%%%%%%%%% Q6
\begin{question}
A spherical loaf of bread is cut into parallel slices of  equal thickness. Show that,
after any number of the  slices have been eaten, 
the area of crust remaining is proportional to the number of slices remaining.

A European ruling decrees that a parallel-sliced
spherical loaf can only be referred to as `crusty' if the ratio of
volume $V$ (in cubic metres) of bread remaining to area $A$ (in square metres)
of  crust remaining after any number of slices have been eaten satisfies
$V/A<1$. Show that the radius of a crusty parallel-sliced
spherical loaf must be less than $2\frac23$ metres.

[{\sl The area $A$ and volume $V$ formed by rotating a curve in the $x$--$y$ plane 
round the $x$-axis from $x=-a$ to $x=-a+t$ 
are given by
\[
A= 2\pi\int_{-a}^{-a+t} 
{ y}\left( 1+ \Big(\frac{\d {y}}{\d x}\Big)^2\right)^{\frac12}
\d x\;,
\ \ \ \ \ \ \ \ \ \ \
V= \pi \int_{-a}^{-a+t} {y}^2 \d x \;. \ \ 
]
\]
}
\end{question}
	
%%%%%%%%% Q7
\begin{question}
In a cosmological model, the radius $\rm R$ of the universe is a function
of the age $t$ of the universe. The function $\rm R$ satisfies
the three conditions:
$$
\mbox{${\rm R}(0)=0$}, \ \ \ \ \ \ \ \ \
\mbox{${\rm R'}(t)>0$ for $t>0$}, \ \ \ \ \ \ \ \ \ \  
\mbox{${\rm R''}(t)<0$ for $t>0$},
\eqno(*)
$$
where ${\rm R''}$ denotes the second derivative of $\rm R$.
 The function ${\rm H}$ is defined by
\[
{\rm H} (t)=  \frac{{\rm R}'(t)}{{\rm R}( t)}\;.
\]



\begin{questionparts}
\item
Sketch a graph of ${\rm R} (t)$.  By considering a tangent
to the graph, show that $t<1/{\rm H}(t)$.

\item
Observations reveal that ${\rm H}(t) = a/t$, where $a$ is constant.
Derive an expression for~${\rm R}(t)$.
What range of values of $a$ is consistent with the three conditions~$(*)$?

\item
Suppose, instead, that observations reveal that ${\rm H}(t)= b t^{-2}$, 
where $b$ is constant.  Show that this is not consistent 
with  conditions $(*)$  for any value of $b$.
\end{questionparts}
\end{question}
		
%%%%%%%%% Q8
\begin{question}	
Given that $y=x$ and $y=1-x^2$ satisfy  the differential equation
$$
\frac{\d^2 {y}}{\d x^2} + \p(x) \frac{\d {y}}{\d x} + \q(x) {y}=0\;,
\eqno(*)
$$
show that $\p(x)= -2x(1+x^2)^{-1}$ and  $\q(x) = 2(1+x^2)^{-1}$. 

Show also that
$ax+b(1-x^2)$ satisfies the differential equation 
for any constants $a$ and $b$.

Given instead that $y=\cos^2(\frac{1}{2}x^2)$ and $y=\sin^2(\frac{1}{2}x^2)$ 
satisfy the  equation $(*)$,
find $\p(x)$ and~$\q(x)$. 
\end{question}	
		

		
	
\newpage
\section*{Section B: \ \ \ Mechanics}


	
%%%%%%%%%% Q9
\begin{question}
A ship sails at $20$ kilometres/hour in a straight line which is, at its closest,
1 kilometre from a~port. A tug-boat with maximum speed 12 kilometres/hour leaves the 
port and intercepts the ship, leaving the port at the latest possible time for which the 
interception is still possible. How far does the tug-boat travel?
	\end{question}
	
%%%%%%%%%% Q10
\begin{question}	
A gun is 
sited on a horizontal plain and can fire 
shells in any direction and at any elevation at speed $v$.
The gun is a distance $d$ from 
a straight railway line which crosses the plain, where
$v^2>gd$. The gunner  
aims to hit the line, choosing the direction and elevation 
so as to maximize the time of flight of the shell. Show that 
 the time of 
flight, $T$, of the shell satisfies 
\[
%\frac{2v}{g} \sin \left( \frac12 \arccos \frac{gd}{v^2}\right)\,.
g^2 T^2 = 2 v^2 +2 \left(v^4 -g^2d^2\right)^{\frac12}\,.
\]

%Find the time of flight if the gun is constrained so that the angle of elevation
%$\alpha $ is not greater than $ 45^\circ$.
\end{question}

%%%%%%%%%% Q11

\begin{question}
A smooth cylinder with circular cross-section of radius $a$ is held
with its axis horizontal. A~light elastic band of unstretched 
length $2\pi a$ and modulus of elasticity
$\lambda$ is wrapped round the circumference of the cylinder, so that it forms a circle 
in a plane perpendicular to the axis of the cylinder. A particle of mass $m$ is then
attached to the  rubber band at its lowest point and released from rest.

\begin{questionparts}
\item Given that the particle falls to a distance $2a$ below the
below the axis of the cylinder, but no further, show that
\[
\lambda = \frac{9\pi m g}{(3\sqrt3-\pi)^2} \;.
\]
\item Given instead that the particle reaches its maximum speed at a
distance $2a$ below the axis of the cylinder, find a similar expression for 
$\lambda$\,.
\end{questionparts} 
\end{question}
	

	
	\newpage
\section*{Section C: \ \ \ Probability and Statistics}


%%%%%%%%%% Q12
\begin{question}
Four students, Arthur, Bertha, Chandra and Delilah, exchange gossip.
When Arthur hears a~rumour, he tells it to one of the other three without 
saying who told it to him. He decides whom to tell by choosing at
random amongst the other three, omitting the ones that  he knows have already heard the 
rumour. When Bertha, Chandra or Delilah hear a rumour, 
they behave in exactly the same  way (even if they have already heard it themselves). 
The rumour stops being passed round
when it is heard by a student who knows that the other three have already heard
it. 

Arthur starts a rumour and tells it to Chandra. By means of a tree diagram, or otherwise,
show that  the probability that Arthur rehears it is $3/4$. 

Find also the probability that
Bertha hears it twice and the probability that Chandra hears it twice.
\end{question}

%%%%%%%%%% Q13
\begin{question}
Four students, one of whom is a mathematician, take turns at washing up over a long
period of time. The number of plates broken by any student in this time
obeys a Poisson distribution, the probability of  any given student breaking 
$n$ plates  being  $\e^{-\lambda} \lambda^n/n!$ for some fixed constant $\lambda$, independent
of the number of breakages by other students. 
Given that five plates are broken, find the probability that 
three or more  were broken by the 
mathematician. 
\end{question}

%%%%%%%%%% Q14
\begin{question}
On the basis of an interview, the $N$ candidates for admission to a college
are ranked in order according to their  mathematical potential. The candidates are interviewed 
in random order (that is, each possible order is equally likely).

\begin{questionparts}
\item Find the probability that the best amongst the first $n$ candidates
interviewed is the best overall.
\item Find the probability that the best amongst the first $n$ candidates
interviewed is  the best or second best overall.
\end{questionparts}
Verify your answers for the case $N=4$, $n=2$ by listing the possibilities.
 
\end{question}
	
\end{document}
