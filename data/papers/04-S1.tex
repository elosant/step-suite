\documentclass[a4, 11pt]{report}


\pagestyle{myheadings}
\markboth{}{Paper I, 2004
\ \ \ \ \ 
\today 
}               

\RequirePackage{amssymb}
\RequirePackage{amsmath}
\RequirePackage{graphicx}
\RequirePackage{color}
\RequirePackage[flushleft]{paralist}[2013/06/09]



\RequirePackage{geometry}
\geometry{%
  a4paper,
  lmargin=2cm,
  rmargin=2.5cm,
  tmargin=3.5cm,
  bmargin=2.5cm,
  footskip=12pt,
  headheight=24pt}


\newcommand{\comment}[1]{{\bf Comment} {\it #1}}
%\renewcommand{\comment}[1]{}

\newcommand{\bluecomment}[1]{{\color{blue}#1}}
%\renewcommand{\comment}[1]{}
\newcommand{\redcomment}[1]{{\color{red}#1}}



\usepackage{epsfig}
\usepackage{pstricks-add}
\usepackage{tgheros} %% changes sans-serif font to TeX Gyre Heros (tex-gyre)
\renewcommand{\familydefault}{\sfdefault} %% changes font to sans-serif
%\usepackage{sfmath}  %%%% this makes equation sans-serif
%\input RexFigs


\setlength{\parskip}{10pt}
\setlength{\parindent}{0pt}

\newlength{\qspace}
\setlength{\qspace}{20pt}


\newcounter{qnumber}
\setcounter{qnumber}{0}

\newenvironment{question}%
 {\vspace{\qspace}
  \begin{enumerate}[\bfseries 1\quad][10]%
    \setcounter{enumi}{\value{qnumber}}%
    \item%
 }
{
  \end{enumerate}
  \filbreak
  \stepcounter{qnumber}
 }


\newenvironment{questionparts}[1][1]%
 {
  \begin{enumerate}[\bfseries (i)]%
    \setcounter{enumii}{#1}
    \addtocounter{enumii}{-1}
    \setlength{\itemsep}{5mm}
    \setlength{\parskip}{8pt}
 }
 {
  \end{enumerate}
 }



\DeclareMathOperator{\cosec}{cosec}
\DeclareMathOperator{\Var}{Var}

\def\d{{\mathrm d}}
\def\e{{\mathrm e}}
\def\g{{\mathrm g}}
\def\h{{\mathrm h}}
\def\f{{\mathrm f}}
\def\p{{\mathrm p}}
\def\s{{\mathrm s}}
\def\t{{\mathrm t}}


\def\A{{\mathrm A}}
\def\B{{\mathrm B}}
\def\E{{\mathrm E}}
\def\F{{\mathrm F}}
\def\G{{\mathrm G}}
\def\H{{\mathrm H}}
\def\P{{\mathrm P}}


\def\bb{\mathbf b}
\def \bc{\mathbf c}
\def\bx {\mathbf x}
\def\bn {\mathbf n}

\newcommand{\low}{^{\vphantom{()}}}
%%%%% to lower suffices: $X\low_1$ etc


\newcommand{\subone}{ {\vphantom{\dot A}1}}
\newcommand{\subtwo}{ {\vphantom{\dot A}2}}




\def\le{\leqslant}
\def\ge{\geqslant}


\def\var{{\rm Var}\,}

\newcommand{\ds}{\displaystyle}
\newcommand{\ts}{\textstyle}
\def\half{{\textstyle \frac12}}
\def\l{\left(}
\def\r{\right)}



\begin{document}
\setcounter{page}{2}

 
\section*{Section A: \ \ \ Pure Mathematics}

%%%%%%%%%%Q1
\begin{question}
\begin{questionparts}
\item
Express $\left(3+2\sqrt{5} \, \right)^3$ 
in the form $a+b\sqrt{5}$ where $a$ and $b$ are integers.

\item Find the positive integers $c$ and $d$ such that 
$\sqrt[3]{99-70\sqrt{2}\;}$ =  $c - d\sqrt{2} \,$.

\item Find the two  real solutions of $x^6 - 198 x^3 + 1 = 0 \,$.
\end{questionparts}
\end{question}

%%%%%%%%%%Q2
\begin{question}
The square bracket notation $\boldsymbol{[}  x\boldsymbol{]}$ means 
the greatest integer less than or equal to $x\,$. 
For example, $\boldsymbol{[}\pi\boldsymbol{]} = 3\,$, 
$\boldsymbol{[}\sqrt{24}\,\boldsymbol{]} = 4\,$ 
and $\boldsymbol{[}5\boldsymbol{]}=5\,$. 

\begin{questionparts}
\item Sketch the graph of $y = \sqrt{\boldsymbol{[}x\boldsymbol{]}}$ 
and show that 
\[
\displaystyle \int^a_0 \sqrt{\boldsymbol{[}x\boldsymbol{]}} \; 
\mathrm{d}x = \sum^{a-1}_{r=0} \sqrt{r}
\] when $a$ is a positive integer.

\item Show that 
$\displaystyle \int^{a}_0 
2_{\vphantom{A}}^{\pmb{\boldsymbol {[} } x \pmb{  \boldsymbol{]}} }\; 
\mathrm{d}x = 2^{a}-1$ 
when $a$ is a positive integer.
\item Determine an expression for 
$\displaystyle \int^{a}_0 2_{\vphantom{\dot A}}^{\pmb{\boldsymbol{[} }x \pmb{ \boldsymbol{]}} } \; 
\mathrm{d}x$ when $a$ is positive but not an  integer.
\end{questionparts}
\end{question}

%%%%%%%%% Q3
\begin{question}
\begin{questionparts}
\item
Show that $x-3$ is a factor of 
\begin{equation}
x^3-5x^2+2x^2y+xy^2-8xy-3y^2+6x+6y \;.
\tag{$*$}
\end{equation}

Express ($ * $) in the form 
$(x-3)(x+ay+b)(x+cy+d)$ where $a$, $b$, $c$ and $d$ 
are integers to be determined. 

\item Factorise 
$6y^3-y^2-21y+2x^2+12x-4xy+x^2y-5xy^2+10$ into three linear factors.
\end{questionparts}
\end{question}

%%%%%% Q4 
\begin{question}
 Differentiate $\sec {t}$ with respect to $t$.

\begin{questionparts}
\item Use the substitution $x=\sec t$ to show that
$\displaystyle \int^2_{\sqrt 2} \frac{1}{ x^3\sqrt {x^2-1} } \; \mathrm{d}x
=\frac{\sqrt 3 - 2}{8} + \frac {\pi}{24} \;.$
\item Determine 
$\displaystyle \int \frac{1}  {( x+2)  \sqrt {(x+1)(x+3)} } \; 
\mathrm{d}x \;$.
\item Determine 
$\displaystyle \int \frac {1} {(x+2)  \sqrt {x^2+4x-5} } \; 
\mathrm{d}x \;$.

\end{questionparts}
\end{question}

%%%%%%%%% Q5
\begin{question}
The positive integers can be split into 
five distinct arithmetic progressions, as shown:
\begin{align*}
A&: \ \ 1, \ 6, \ 11, \ 16, \ ... \\
B&: \ \ 2, \ 7, \ 12, \ 17, \ ...\\
C&: \ \ 3, \ 8, \ 13, \ 18, \ ... \\
D&: \ \ 4, \ 9, \ 14, \ 19, \ ...  \\
E&: \ \ 5,  10, \ 15, \ 20, \ ...
\end{align*}

Write down an expression for the value of the general term 
in each of the five progressions. 
Hence prove that the sum of any term in $B$ 
and any term in $C$ is a term in $E$. 


Prove also that the square of every term in $B$ is a term in $D$. 
State and prove a similar claim about the square of every term in $C$.

\begin{questionparts}
\item Prove that there are no positive integers $x$ and $y$ such that
\[
x^2+5y=243\,723 \,.
\]

\item Prove also that there are no positive integers 
$x$ and $y$ such that
\[
x^4+2y^4=26\,081\,974 \,.
\]
\end{questionparts}
	\end{question}
	
%%%%%%%%% Q6
\begin{question}
The three points $A$, $B$ and $C$ have coordinates 
$\l p_1 \, , \; q_1 \r$, $\l p_2 \, , \; q_2 \r$ and 
$\l p_3 \, , \; q_3 \r\,$, respectively. 
Find the point of intersection of the line joining 
$A$ to the midpoint of $BC$, 
and the line joining~$B$ to the midpoint of $AC$. 
Verify that this point lies 
on the line joining $C$ to the midpoint of~$AB$.


The point $H$ has coordinates 
$\l p_1 + p_2 + p_3 \, , \; q_1 + q_2 + q_3 \r\,$. 
Show that if the line $AH$ intersects the line $BC$ at right angles, 
then $p_2^2 + q_2^2 = p_3^2 + q_3^2\,$, 
and write down  a similar result 
if the line $BH$ intersects the line $AC$ at right angles. 


Deduce that if $AH$ is perpendicular to $BC$ and 
also $BH$ is perpendicular to $AC$, then $CH$ is perpendicular to $AB$.
\end{question}
	
%%%%%%%%% Q7
\begin{question}
\begin{questionparts}
\item
The function $\f(x)$ is defined for $\vert x \vert < \frac15$ by 
\[
\f(x) = \sum_{n=0}^\infty a_n x^n\;,
\]
where  
$a_0=2$, $a_1=7$ and 
$
a_n =7a_{n-1} - 10a_{n-2}
$ 
for $n\ge{2}\,$. 

Simplify $\f(x) - 7x\f(x) + 10x^2\f(x)\,$, 
and hence show that 
$\displaystyle\f(x) = {1\over 1-2x} + {1 \over 1-5x} \;$.


Hence show that $a_n=2^n + 5^n\,$.

\item The function $\g(x)$ is defined for $\vert x \vert < \frac13$ by 
\[
\g(x) = \sum_{n=0}^\infty b_n x^n \;,
\]
where $b_0=5\,$, $b_1 =10 \,$, $b_2=40\,$, $b_3=100$  
and  $b_n = pb_{n-1} + qb_{n-2}$ for $n\ge{2}\,$. 
Obtain an expression for $\g(x)$ as the sum of two algebraic fractions and 
determine $b_n$ in terms of~$n\,$.
\end{questionparts}
\end{question}
		
%%%%%%%%% Q8
\begin{question}	
A sequence $t_0$, $t_1$, $t_2$, $...$ is said to be 
{\sl strictly increasing}  if $t_{n+1} > t_n$ for all $n\ge{0}\,$.

\begin{questionparts}
\item
The terms of the sequence $x_0\,$, $x_1\,$, $x_2\,$, $\ldots$ satisfy
$$
\ds x_{n+1}=\frac{x_n^2 +6}{5}
$$ for $n\ge{0}\,$. 
Prove that if $x_0 > 3$ then the sequence 
is strictly increasing.

\item The terms of the sequence $y_0\,$, $y_1\,$, $y_2\,$, $\ldots$
satisfy
$$
\ds y_{n+1}= 5-\frac 6 {y_n}
$$ 
for  $n\ge{0}\,$. 
Prove that if $2 < y_0 < 3$ 
then the sequence is strictly 
increasing but that $y_n<3$ for all $n\,$.
\end{questionparts}
\end{question}	
		

		
	
\newpage
\section*{Section B: \ \ \ Mechanics}


	
%%%%%%%%%% Q9
\begin{question}
A particle is projected over level ground with a speed  $u$ at 
an angle $\theta$ above the horizontal. 
Derive
 an expression for the greatest height of the particle in terms of $u$, $\theta$ and $g$. 


A  particle is  projected from the floor of a horizontal tunnel of height ${9\over 10}d$. 
Point $P$
 is  ${1\over 2}d$ metres vertically and $d$ metres horizontally along the tunnel
 from the point of 
projection. The particle passes through point $P$ and lands inside the tunnel
without hitting the roof. Show that
\[
\arctan \textstyle {3 \over 5} < \theta < \arctan \, 3 \;.
\]
	\end{question}
	
%%%%%%%%%% Q10 
\begin{question}	
A particle is travelling in a straight line. 
It accelerates from its initial  velocity $u$  to
velocity $v$, where $v > \vert u \vert > 0\,$, travelling a distance $d_1$
with uniform acceleration of magnitude $3a\,$.  
It then comes to rest after travelling
a further distance $d_2\,$ with uniform deceleration of  magnitude $a\,$.
Show that
\begin{questionparts}
\item
if $u>0$ then $3d_1 < d_2\,$;
\item 
if $u<0$ then  $d_2 < 3d_1 < 2d_2\,$. 
\end{questionparts}


Show also that
 the average speed of the particle (that is,  the total distance
travelled divided by the total time)  is greater  in the case $u>0$ than  in the case $u<0\,$.

\noindent
{\bf Note:} In this question $d_1$ and $d_2$ are distances travelled by the particle which
are not the same, in the second case, as displacements from the starting point.
\end{question}

%%%%%%%%%% Q11

\begin{question}
Two uniform ladders $AB$ and $BC$ of equal length are hinged smoothly at $B$.
The weight of $AB$ is $W$ and the weight of $BC$ is $4W $.
The ladders stand on rough horizontal ground with  $\angle ABC=60^\circ\,$. 
 The coefficient of friction between each ladder  and the 
ground is  $\mu$.


A decorator of weight $7W$ begins to climb the ladder $AB$ slowly. 
When she has climbed up $1 \over 3$ of the ladder, 
one of the ladders slips.
Which ladder  slips, and what is the value of $\mu$?
\end{question}
	

	
	\newpage
\section*{Section C: \ \ \ Probability and Statistics}


%%%%%%%%%% Q12
\begin{question}
In a certain factory, microchips are made by two machines.
Machine A makes a 
proportion~$\lambda$ of the chips, where $0 < \lambda < 1$, and machine B makes the rest. 
A proportion $p$ of the chips  made by machine A are perfect, and
a  proportion $q$  of those made by machine B are perfect, 
where $0 < p < 1$ and $0 < q < 1$. The chips are sorted into two groups: group 1
contains those that are
perfect and group 2 contains those that are imperfect.

In a large random sample taken from group 1, it is found that
$\frac 2 5$ 
were made by machine A. Show that $\lambda$ can estimated as
\[
 {2q \over 3p + 2q}\;.
\]

Subsequently, it is discovered that the sorting process 
is faulty: there is a probability of $\frac 14$ that a perfect
 chip is assigned to group 2 and a  probability of $\frac 14$ that an imperfect
 chip is assigned to group 1. Taking into account this additional information,
obtain a new  estimate of $\lambda\,$.
\end{question}

%%%%%%%%%% Q13
\begin{question}
\begin{questionparts}
\item Three real numbers are drawn independently from the continuous 
rectangular distribution on $[ 0, 1 ]\,$. The random variable $X$ is the maximum of the 
three numbers. Show that the probability that $X \le 0.8$ is $0.512\,$, and calculate the
 expectation of $X$. 

\item $N$ real numbers are 
drawn independently from a continuous rectangular distribution on $[ 0, a ]\,$. 
The random variable $X$ is the maximum of the $N$ numbers. 
A hypothesis test  with a significance level of 5\% is carried out using  the value, $x$, of
  $X $. 
The null hypothesis is that $a=1$ and 
the alternative hypothesis is that $a<1 \,$. The form of the test is such that
$H_0$ is rejected 
if $x<c\,$, for some chosen number $c\,$.


Using the approximation 
 $2^{10} \approx 10^3\,$, determine the smallest 
integer value of $N$ such that if $x \le 0.8$ 
the null hypothesis will be rejected.

With this value of $N$,
write down the probability that the null hypothesis is rejected if $a = 0.8\,$, 
and find  the probability that the null hypothesis is rejected if $a = 0.9\,$.


\end{questionparts}
\end{question}

%%%%%%%%%% Q14
\begin{question}
Three pirates are sharing out the contents of a treasure chest containing $n$ 
gold coins and $2$ lead coins. 
The first pirate takes out coins one at a time until 
he takes out one of the lead coins. 
The second pirate then takes out coins one at a time until she draws 
the second lead coin. 
The third pirate takes out all the gold coins remaining in the chest. 


Find:
\begin{questionparts}
\item the probability that the first pirate will have  some gold coins;
\item the probability that the  second pirate will have  some gold coins;
\item the probability that all three pirates will have some gold coins. 
\end{questionparts}
\end{question}
	
\end{document}
