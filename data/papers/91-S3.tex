
\documentclass[a4, 11pt]{report}


\pagestyle{myheadings}
\markboth{}{Paper III, 1991
\ \ \ \ \ 
\today 
}               

\RequirePackage{amssymb}
\RequirePackage{amsmath}
\RequirePackage{graphicx}
\RequirePackage{color}
\RequirePackage[flushleft]{paralist}[2013/06/09]



\RequirePackage{geometry}
\geometry{%
  a4paper,
  lmargin=2cm,
  rmargin=2.5cm,
  tmargin=3.5cm,
  bmargin=2.5cm,
  footskip=12pt,
  headheight=24pt}


\newcommand{\comment}[1]{{\bf Comment} {\it #1}}
%\renewcommand{\comment}[1]{}

\newcommand{\bluecomment}[1]{{\color{blue}#1}}
%\renewcommand{\comment}[1]{}
\newcommand{\redcomment}[1]{{\color{red}#1}}



\usepackage{epsfig}
\usepackage{pstricks-add}
\usepackage{tgheros} %% changes sans-serif font to TeX Gyre Heros (tex-gyre)
\renewcommand{\familydefault}{\sfdefault} %% changes font to sans-serif
%\usepackage{sfmath}  %%%% this makes equation sans-serif
%\input RexFigs


\setlength{\parskip}{10pt}
\setlength{\parindent}{0pt}

\newlength{\qspace}
\setlength{\qspace}{20pt}


\newcounter{qnumber}
\setcounter{qnumber}{0}

\newenvironment{question}%
 {\vspace{\qspace}
  \begin{enumerate}[\bfseries 1\quad][10]%
    \setcounter{enumi}{\value{qnumber}}%
    \item%
 }
{
  \end{enumerate}
  \filbreak
  \stepcounter{qnumber}
 }


\newenvironment{questionparts}[1][1]%
 {
  \begin{enumerate}[\bfseries (i)]%
    \setcounter{enumii}{#1}
    \addtocounter{enumii}{-1}
    \setlength{\itemsep}{5mm}
    \setlength{\parskip}{8pt}
 }
 {
  \end{enumerate}
 }



\DeclareMathOperator{\cosec}{cosec}
\DeclareMathOperator{\Var}{Var}

\def\d{{\rm d}}
\def\e{{\rm e}}
\def\g{{\rm g}}
\def\h{{\rm h}}
\def\f{{\rm f}}
\def\p{{\rm p}}
\def\s{{\rm s}}
\def\t{{\rm t}}


\def\A{{\rm A}}
\def\B{{\rm B}}
\def\E{{\rm E}}
\def\F{{\rm F}}
\def\G{{\rm G}}
\def\H{{\rm H}}
\def\P{{\rm P}}


\def\bb{\mathbf b}
\def \bc{\mathbf c}
\def\bx {\mathbf x}
\def\bn {\mathbf n}

\newcommand{\low}{^{\vphantom{()}}}
%%%%% to lower suffices: $X\low_1$ etc


\newcommand{\subone}{ {\vphantom{\dot A}1}}
\newcommand{\subtwo}{ {\vphantom{\dot A}2}}




\def\le{\leqslant}
\def\ge{\geqslant}


\def\var{{\rm Var}\,}

\newcommand{\ds}{\displaystyle}
\newcommand{\ts}{\textstyle}




\begin{document}
\setcounter{page}{2}

 
\section*{Section A: \ \ \ Pure Mathematics}

%%%%%%%%%%Q1
\begin{question}
\begin{questionparts}
\item Evaluate 
\[
\sum_{r=1}^{n}\frac{6}{r(r+1)(r+3)}.
\]
\item Expand $\ln(1+x+x^{2}+x^{3})$ as a series in powers of $x$,
where $\left|x\right|<1$, giving the first five non-zero terms and
the general term. 
\item Expand $\mathrm{e}^{x\ln(1+x)}$ as a series in powers of $x$,
where $-1<x\leqslant1$, as far as the term in $x^{4}$. 
\end{questionparts}
\end{question}

%%%%%%%%%%Q2
\begin{question}
The distinct points $P_{1},P_{2},P_{3},Q_{1},Q_{2}$ and $Q_{3}$
in the Argand diagram are represented by the complex numbers $z_{1},z_{2},z_{3},w_{1},w_{2}$
and $w_{3}$ respectively. Show that the triangles $P_{1}P_{2}P_{3}$
and $Q_{1}Q_{2}Q_{3}$ are similar, with $P_{i}$ corresponding to
$Q_{i}$ ($i=1,2,3$) and the rotation from $1$ to $2$ to $3$ being
in the same sense for both triangles, if and only if 
\[
\frac{z_{1}-z_{2}}{z_{2}-z_{3}}=\frac{w_{1}-w_{2}}{w_{1}-w_{3}}.
\]
Verify that this condition may be written 
\[
\det\begin{pmatrix}z_{1} & z_{2} & z_{3}\\
w_{1} & w_{2} & w_{3}\\
1 & 1 & 1
\end{pmatrix}=0.
\]
\begin{questionparts}
\item Show that if $w_{i}=z_{i}^{2}$ ($i=1,2,3$) then triangle
$P_{1}P_{2}P_{3}$ is not similar to triangle $Q_{1}Q_{2}Q_{3}.$ 
\item Show that if $w_{i}=z_{i}^{3}$ ($i=1,2,3$) then triangle
$P_{1}P_{2}P_{3}$ is similar to triangle $Q_{1}Q_{2}Q_{3}$ if and
only if the centroid of triangle $P_{1}P_{2}P_{3}$ is the origin. 
{[}The \textit{centroid }of triangle $P_{1}P_{2}P_{3}$ is represented
by the complex number $\frac{1}{3}(z_{1}+z_{2}+z_{3})$.{]} 
\item Show that the triangle $P_{1}P_{2}P_{3}$ is equilateral if
and only if 
\[
z_{2}z_{3}+z_{3}z_{1}+z_{1}z_{2}=z_{1}^{2}+z_{2}^{2}+z_{3}^{2}.
\]
\end{questionparts}
\end{question}

%%%%%%%%% Q3
\begin{question}
The function $\mathrm{f}$ is defined for $x<2$ by 
\[
\mathrm{f}(x)=2\left|x^{2}-x\right|+\left|x^{2}-1\right|-2\left|x^{2}+x\right|.
\]
Find the maximum and minimum points and the points of inflection of
the graph of $\mathrm{f}$ and sketch this graph. Is $\mathrm{f}$
continuous everywhere? Is $\mathrm{f}$ differentiable everywhere? 

Find the inverse of the function $\mathrm{f}$, i.e. expressions for
$\mathrm{f}^{-1}(x),$ defined in the various appropriate intervals. 
\end{question}


%%%%%% Q4 

\begin{question}
The point $P$ moves on a straight line in three-dimensional space.
The position of $P$ is observed from the points $O_{1}(0,0,0)$ and
$O_{2}(8a,0,0).$ At times $t=t_{1}$ and $t=t_{1}'$, the lines of
sight from $O_{1}$ are along the lines 
\[
\frac{x}{2}=\frac{z}{3},y=0\quad\mbox{ and }\quad x=0,\frac{y}{3}=\frac{z}{4}
\]
respectively. At times $t=t_{2}$ and $t=t_{2}'$, the lines of sight
from $O_{2}$ are 
\[
\frac{x-8a}{-3}=\frac{y}{1}=\frac{z}{3}\quad\mbox{ and }\quad\frac{x-8a}{-4}=\frac{y}{2}=\frac{z}{5}
\]
respectively. Find an equation or equations for the path of $P$. 

\end{question}


%%%%%%%%% Q5
\begin{question}
The curve $C$ has the differential equation in polar coordinates
\[
\frac{\mathrm{d}^{2}r}{\mathrm{d}\theta^{2}}+4r=5\sin3\theta,\qquad\text{for }\quad\frac{\pi}{5}\leqslant\theta\leqslant\frac{3\pi}{5},
\]
and, when $\theta=\dfrac{\pi}{2},$ $r=1$ and $\dfrac{\mathrm{d}r}{\mathrm{d}\theta}=-2.$ 


Show that $C$ forms a closed loop and that the area of the region
enclosed by $C$ is 
\[
\frac{\pi}{5}+\frac{25}{48}\left[\sin\left(\frac{\pi}{5}\right)-\sin\left(\frac{2\pi}{5}\right)\right].
\]  
	\end{question}
	
	%%%%%%%%% Q6
	\begin{question}
The transformation $T$ from $\begin{pmatrix}x\\
y
\end{pmatrix}$ to $\begin{pmatrix}x'\\
y'
\end{pmatrix}$ in two-dimensional space is given by 
\[
\begin{pmatrix}x'\\
y'
\end{pmatrix}=\begin{pmatrix}\cosh u & \sinh u\\
\sinh u & \cosh u
\end{pmatrix}\begin{pmatrix}x\\
y
\end{pmatrix},
\]
where $u$ is a positive real constant. Show that the curve with equation
$x^{2}-y^{2}=1$ is transformed into itself. Find the equations of
two straight lines through the origin which transform into themselves. 


A line, not necessary through the origin, which has gradient $\tanh v$
transforms under $T$ into a line with gradient $\tanh v'$. Show
that $v'=v+u$. 


The lines $\ell_{1}$ and $\ell_{2}$ with gradients $\tanh v_{1}$
and $\tanh v_{2}$ transform under $T$ into lines with gradients
$\tanh v_{1}'$ and $\tanh v_{2}'$ respectively. Find the relation
satisfied by $v_{1}$ and $v_{2}$ that is the necessary and sufficient
for $\ell_{1}$ and $\ell_{2}$ to intersect at the same angle as
their transforms. 


In the case when $\ell_{1}$ and $\ell_{2}$ meet at the origin, illustrate
in a diagram the relation between $\ell_{1}$, $\ell_{2}$ and their
transforms.
	 \end{question}
	 
	 %%%%%%%%% Q7
\begin{question}
\begin{questionparts}
\item Prove that 
\[
\int_{0}^{\frac{1}{2}\pi}\ln(\sin x)\,\mathrm{d}x=\int_{0}^{\frac{1}{2}\pi}\ln(\cos x)\,\mathrm{d}x=\tfrac{1}{2}\int_{0}^{\frac{1}{2}\pi}\ln(\sin2x)\,\mathrm{d}x-\tfrac{1}{4}\pi\ln2
\]
and 
\[
\int_{0}^{\frac{1}{2}\pi}\ln(\sin2x)\,\mathrm{d}x=\tfrac{1}{2}\int_{0}^{\pi}\ln(\sin x)\,\mathrm{d}x=\int_{0}^{\frac{1}{2}\pi}\ln(\sin x)\,\mathrm{d}x.
\]
Hence, or otherwise, evaluate ${\displaystyle \int_{0}^{\frac{1}{2}\pi}\ln(\sin x)\,\mathrm{d}x.}$


{[}You may assume that all the integrals converge.{]}


\item Given that $\ln u<u$ for $u\geqslant1$ deduce that 
\[
\tfrac{1}{2}\ln x<\sqrt{x}\qquad\mbox{ for }\quad x\geqslant1.
\]
Deduce that $\dfrac{\ln x}{x}\rightarrow0$ as $x\rightarrow\infty$
and that $x\ln x\rightarrow0$ as $x\rightarrow0$ through positive
values. 


\item Using the results of parts \textbf{(i) }and \textbf{(ii)},
or otherwise, evaluate ${\displaystyle \int_{0}^{\frac{1}{2}\pi}x\cot x\,\mathrm{d}x.}$
\end{questionparts}
	\end{question}
	
	%%%%%%%%% Q8
	\begin{question}
\begin{questionparts}
\item The integral $I_{k}$ is defined by 
\[
I_{k}=\int_{0}^{\theta}\cos^{k}x\,\cos kx\,\mathrm{d}x.
\]
Prove that $2kI_{k}=kI_{k-1}+\cos^{k}\theta\sin k\theta.$
\item Prove that 
\[
1+m\cos2\theta+\binom{m}{2}\cos4\theta+\cdots+\binom{m}{r}\cos2r\theta+\cdots+\cos2m\theta=2^{m}\cos^{m}\theta\cos m\theta.
\]
\item Using the results of \textbf{(i) }and \textbf{(ii)}, show that
\[
m\frac{\sin2\theta}{2}+\binom{m}{2}\frac{\sin4\theta}{4}+\cdots+\binom{m}{r}\frac{\sin2r\theta}{2r}+\cdots+\frac{\sin2m\theta}{2m}
\]
is equal to 
\[
\cos\theta\sin\theta+\cos^{2}\theta\sin2\theta+\cdots+\frac{1}{r}2^{r-1}\cos^{r}\theta\sin r\theta+\cdots+\frac{1}{m}2^{m-1}\cos^{m}\theta\sin m\theta.
\]
\end{questionparts}
		\end{question}
		
		
%%%%%%%%% Q9
		\begin{question}
The parametric equations $E_{1}$ and $E_{2}$ define the same
ellipse, in terms of the parameters $\theta_{1}$ and $\theta_{2}$,
(though not referred to the same coordinate axes). 
\begin{alignat*}{2}
E_{1}:\qquad & x=a\cos\theta_{1}, & \quad & y=b\sin\theta_{1},\\
E_{2}:\qquad & x=\dfrac{k\cos\theta_{2}}{1+e\cos\theta_{2}}, & \quad & y=\dfrac{k\sin\theta_{2}}{1+e\cos\theta_{2}},
\end{alignat*}
where $0<b<a,$ $0<e<1$ and $0<k$. Find the position of the axes
for $E_{2}$ relative to the axes for $E_{1}$ and show that $k=a(1-e^{2})$
and $b^{2}=a^{2}(1-e^{2}).$


{[}The standard polar equation of an ellipse is $r=\dfrac{\ell}{1+e\cos\theta}.]$


By considering expressions for the length of the perimeter of the
ellipse, or otherwise, prove that 
\[
\int_{0}^{\pi}\sqrt{1-e^{2}\cos^{2}\theta}\,\mathrm{d}\theta=\int_{0}^{\pi}\frac{1-e^{2}}{(1+e\cos\theta)^{2}}\sqrt{1+e^{2}+2e\cos\theta}\,\mathrm{d}\theta.
\]
Given that $e$ is so small that $e^{6}$ may be neglected, show that
the value of either integral is 
\[
\tfrac{1}{64}\pi(64-16e^{2}-3e^{4}).
\]
		\end{question}
		
	
%%%%%%%%%% 10
\begin{question}
The equation 
\[
x^{n}-qx^{n-1}+r=0,
\]
 where $n\geqslant5$ and $q$ and $r$ are real constants, has roots
$\alpha_{1},\alpha_{2},\ldots,\alpha_{n}.$ The sum of the products
of $m$ distinct roots is denoted by $\Sigma_{m}$ (so that, for example,
$\Sigma_{3}=\sum\alpha_{i}\alpha_{j}\alpha_{k}$ where the sum runs
over the values of $i,j$ and $k$ with $n\geqslant i>j>k\geqslant1$).
The sum of $m$th powers of the roots is denoted by $S_{m}$ (so that,
for example, $S_{3}=\sum\limits_{i=1}^{n}\alpha_{i}^{3}$). 


Prove that $S_{p}=p^{q}$ for $1\leqslant p\leqslant n-1.$ {[}You
may assume that for any $n$th degree equation and $1\leqslant p\leqslant n$
\[
S_{p}-S_{p-1}\Sigma_{1}+S_{p-2}\Sigma_{2}-\cdots+(-1)^{p-1}S_{1}\Sigma_{p-1}+(-1)^{p}p\Sigma_{p}=0.]
\]
Find expressions for $S_{n},$ $S_{n+1}$ and $S_{n+2}$ in terms
of $q,r$ and $n$. Suggest an expression for $S_{n+m},$ where $m<n$,
and prove its validity by induction. 
\end{question}
			
		
		
		
	
\newpage
\section*{Section B: \ \ \ Mechanics}


	
%%%%%%%%%% Q11
\begin{question}
\noindent \begin{center}
\psset{xunit=1.0cm,yunit=1.0cm,algebraic=true,dimen=middle,dotstyle=o,dotsize=3pt 0,linewidth=0.5pt,arrowsize=3pt 2,arrowinset=0.25} \begin{pspicture*}(-0.93,-0.76)(7.37,6.78) \psline(0,0)(5,0) \psline(0,0)(7,4) \parametricplot{0.0}{0.5191461142465229}{1.31*cos(t)+0|1.31*sin(t)+0} \pscircle(2,4){1} \pscircle(2,4){2.49} \psline(2.6,4.8)(5.18,2.96) \parametricplot{2.52210871855711}{3.6607387678363166}{0.76*cos(t)+5.18|0.76*sin(t)+2.96} \rput[tl](5.15,2.78){$Q$} \rput[tl](2.81,5.07){$P$} \rput[tl](4.56,3.15){$2\alpha$} \rput[tl](0.78,0.33){$\alpha$} \end{pspicture*}
\par\end{center}


\noindent A uniform circular cylinder of radius $2a$ with a groove
of radius $a$ cut in its central cross-section has mass $M$. It
rests, as shown in the diagram, on a rough plane inclined at an acute
angle $\alpha$ to the horizontal. It is supported by a light inextensible
string would round the groove and attached to the cylinder at one
end. The other end of the string is attached to the plane at $Q$,
the free part of the string, $PQ,$ making an angle $2\alpha$ with
the inclined plane. The coefficient of friction at the contact between
the cylinder and the plane is $\mu.$ Show that $\mu\geqslant\frac{1}{3}\tan\alpha.$ 


The string $PQ$ is now detached from the plane and the end $Q$ is
fastened to a particle of mass $3M$ which is placed on the plane,
the position of the string remain unchanged. Given that $\tan\alpha=\frac{1}{2}$
and that the system remains in equilibrium, find the least value of
the coefficient of friction between the particle and the plane.  
	\end{question}
	
%%%%%%%%%% Q12
\begin{question}	
A smooth tube whose axis is horizontal has an elliptic cross-section
in the form of the curve with parametric equations 
\[
x=a\cos\theta\qquad y=b\sin\theta
\]
where the $x$-axis is horizontal and the $y$-axis is vertically
upwards. A particle moves freely under gravity on the inside of the
tube in the plane of this cross-section. By first finding $\ddot{x}$
and $\ddot{y},$ or otherwise, show that the acceleration along the
inward normal at the point with parameter $\theta$ is 
\[
\frac{ab\dot{\theta}^{2}}{\sqrt{a^{2}\sin^{2}\theta+b^{2}\cos^{2}\theta}}.
\]
The particle is projected along the surface in the vertical cross-section
plane, with speed $2\sqrt{bg},$ from the lowest point. Given that
$2a=3b,$ show that it will leave the surface at the point with parameter
$\theta$ where 
\[
5\sin^{3}\theta+12\sin\theta-8=0.
\] 
\end{question}

%%%%%%%%%% Q13

\begin{question}
A smooth particle $P_{1}$ is projected from a point $O$ on the horizontal
floor of a room with has a horizontal ceiling at a height $h$ above
the floor. The speed of projection is $\sqrt{8gh}$ and the direction
of projection makes an acute angle $\alpha$ with the horizontal.
The particle strikes the ceiling and rebounds, the impact being perfectly
elastic. Show that for this to happen $\alpha$ must be at least $\frac{1}{6}\pi$
and that the range on the floor is then 
\[
8h\cos\alpha\left(2\sin\alpha-\sqrt{4\sin^{2}\alpha-1}\right).
\]
Another particle $P_{2}$ is projected from $O$ with the same velocity
as $P_{1}$ but its impact with the ceiling is perfectly inelastic.
Find the difference $D$ between the ranges of $P_{1}$ and $P_{2}$
on the floor and show that, as $\alpha$ varies, $D$ has a maximum
value when $\alpha=\frac{1}{4}\pi.$ 
\end{question}
	
%%%%%%%%%% Q14
\begin{question}
\noindent \begin{center}
\psset{xunit=1.0cm,yunit=1.0cm,algebraic=true,dimen=middle,dotstyle=o,dotsize=3pt 0,linewidth=0.5pt,arrowsize=3pt 2,arrowinset=0.25} \begin{pspicture*}(0.18,0.17)(6.56,7.49) \psline(1,1)(5,7) \psline(5,7)(5,4) \parametricplot{-2.1587989303424644}{-1.570796326794897}{1.19*cos(t)+5|1.19*sin(t)+7} \rput[tl](5.22,6.56){$\sin^{-1}\tfrac{3}{5}$} \rput[tl](5.16,7.36){$O$} \rput[tl](4.88,3.69){$N$} \rput[tl](0.64,0.82){$A$} \rput[tl](2.52,4.27){$B$} \psline(3,4)(3.09,3.8) \psline(3.09,3.8)(2.64,3.78) \psline(2.64,3.78)(2.86,3.51) \psline(2.86,3.51)(2.41,3.45) \psline(2.41,3.45)(2.67,3.2) \psline(2.67,3.2)(2.25,3.15) \psline(2.25,3.15)(2.52,2.88) \psline(2.52,2.88)(2.03,2.85) \psline(2.03,2.85)(2.27,2.5) \psline(2.27,2.5)(1.77,2.49) \psline(1.77,2.49)(1.96,2.12) \psline(1.96,2.12)(1.5,2.09) \psline(1.5,2.09)(1.68,1.76) \psline(1.68,1.76)(1.28,1.71) \psline(1.28,1.71)(1.47,1.42) \psline(1.47,1.42)(1.04,1.41) \psline(1.04,1.41)(1.32,1.07) \psline(1.32,1.07)(1,1) \begin{scriptsize} \psdots[dotstyle=*](1,1) \psdots[dotstyle=*](3,4) \end{scriptsize} \end{pspicture*}
\par\end{center}


\noindent The end $O$ of a smooth light rod $OA$ of length $2a$
is a fixed point. The rod $OA$ makes a fixed angle $\sin^{-1}\frac{3}{5}$
with the downward vertical $ON,$ but is free to rotate about $ON.$
A particle of mass $m$ is attached to the rod at $A$ and a small
ring $B$ of mass $m$ is free to slide on the rod but is joined to
a spring of natural length $a$ and modulus of elasticity $kmg$.
The vertical plane containing the rod $OA$ rotates about $ON$ with
constant angular velocity $\sqrt{5g/2a}$ and $B$ is at rest relative
to the rod. Show that the length of $OB$ is 
\[
\frac{(10k+8)a}{10k-9}.
\]
Given that the reaction of the rod on the particle at $A$ makes an
angle $\tan^{-1}\frac{13}{21}$ with the horizontal, find the value
of $k$. Find also the magnitude of the reaction between the rod and
the ring $B$. 

\end{question}
	
	\newpage
\section*{Section C: \ \ \ Probability and Statistics}


%%%%%%%%%% Q15
\begin{question}
A pack of $2n$ (where $n\geqslant4$) cards consists of two each
of $n$ different sorts. If four cards are drawn from the pack without
replacement show that the probability that no pairs of identical cards
have been drawn is 
\[
\frac{4(n-2)(n-3)}{(2n-1)(2n-3)}.
\]
Find the probability that exactly one pair of identical cards is included
in the four. 


If $k$ cards are drawn without replacement and $2<k<2n,$ find an
expression for the probability that there are exactly $r$ pairs of
identical cards included when $r<\frac{1}{2}k.$ 


For even values of $k$ show that the probability that the drawn cards
consist of $\frac{1}{2}k$ pairs is 
\[
\frac{1\times3\times5\times\cdots\times(k-1)}{(2n-1)(2n-3)\cdots(2n-k+1)}.
\] 
\end{question}

%%%%%%%%%% Q16
\begin{question}
The random variables $X$ and $Y$ take integer values $x$ and $y$
respectively which are restricted by $x\geqslant1,$ $y\geqslant1$
and $2x+y\leqslant2a$ where $a$ is an integer greater than 1. The
joint probability is given by 
\[
\mathrm{P}(X=x,Y=y)=c(2x+y),
\]
where $c$ is a positive constant, within this region and zero elsewhere.
Obtain, in terms of $x,c$ and $a,$ the marginal probability $\mathrm{P}(X=x)$
and show that 
\[
c=\frac{6}{a(a-1)(8a+5)}.
\]
Show that when $y$ is an even number the marginal probability $\mathrm{P}(Y=y)$
is 
\[
\frac{3(2a-y)(2a+2+y)}{2a(a-1)(8a+5)}
\]
and find the corresponding expression when $y$ is off. 


Evaluate $\mathrm{E}(Y)$ in terms of $a$.

\end{question}
\end{document}
