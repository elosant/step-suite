\documentclass[a4, 11pt]{report}


\pagestyle{myheadings}
\markboth{}{Paper I, 1992
\ \ \ \ \ 
\today 
}               

\RequirePackage{amssymb}
\RequirePackage{amsmath}
\RequirePackage{graphicx}
\RequirePackage{color}
\RequirePackage[flushleft]{paralist}[2013/06/09]



\RequirePackage{geometry}
\geometry{%
  a4paper,
  lmargin=2cm,
  rmargin=2.5cm,
  tmargin=3.5cm,
  bmargin=2.5cm,
  footskip=12pt,
  headheight=24pt}


\newcommand{\comment}[1]{{\bf Comment} {\it #1}}
%\renewcommand{\comment}[1]{}

\newcommand{\bluecomment}[1]{{\color{blue}#1}}
%\renewcommand{\comment}[1]{}
\newcommand{\redcomment}[1]{{\color{red}#1}}



\usepackage{epsfig}
\usepackage{pstricks-add}
\usepackage{tgheros} %% changes sans-serif font to TeX Gyre Heros (tex-gyre)
\renewcommand{\familydefault}{\sfdefault} %% changes font to sans-serif
%\usepackage{sfmath}  %%%% this makes equation sans-serif
%\input RexFigs


\setlength{\parskip}{10pt}
\setlength{\parindent}{0pt}

\newlength{\qspace}
\setlength{\qspace}{20pt}


\newcounter{qnumber}
\setcounter{qnumber}{0}

\newenvironment{question}%
 {\vspace{\qspace}
  \begin{enumerate}[\bfseries 1\quad][10]%
    \setcounter{enumi}{\value{qnumber}}%
    \item%
 }
{
  \end{enumerate}
  \filbreak
  \stepcounter{qnumber}
 }


\newenvironment{questionparts}[1][1]%
 {
  \begin{enumerate}[\bfseries (i)]%
    \setcounter{enumii}{#1}
    \addtocounter{enumii}{-1}
    \setlength{\itemsep}{5mm}
    \setlength{\parskip}{8pt}
 }
 {
  \end{enumerate}
 }



\DeclareMathOperator{\cosec}{cosec}
\DeclareMathOperator{\Var}{Var}

\def\d{{\rm d}}
\def\e{{\rm e}}
\def\g{{\rm g}}
\def\h{{\rm h}}
\def\f{{\rm f}}
\def\p{{\rm p}}
\def\s{{\rm s}}
\def\t{{\rm t}}


\def\A{{\rm A}}
\def\B{{\rm B}}
\def\E{{\rm E}}
\def\F{{\rm F}}
\def\G{{\rm G}}
\def\H{{\rm H}}
\def\P{{\rm P}}


\def\bb{\mathbf b}
\def \bc{\mathbf c}
\def\bx {\mathbf x}
\def\bn {\mathbf n}

\newcommand{\low}{^{\vphantom{()}}}
%%%%% to lower suffices: $X\low_1$ etc


\newcommand{\subone}{ {\vphantom{\dot A}1}}
\newcommand{\subtwo}{ {\vphantom{\dot A}2}}




\def\le{\leqslant}
\def\ge{\geqslant}


\def\var{{\rm Var}\,}

\newcommand{\ds}{\displaystyle}
\newcommand{\ts}{\textstyle}




\begin{document}
\setcounter{page}{2}

 
\section*{Section A: \ \ \ Pure Mathematics}

%%%%%%%%%%Q1
\begin{question}
Today's date is June 26th 1992 and the day of the week is Friday.
Find which day of the week was April 3rd 1905, explaining your method
\textbf{carefully. }


{[}30 days hath September, April, June and November. All the rest
have 31, excepting February alone which has 28 days clear and 29 in
each leap year.{]}
\end{question}

%%%%%%%%%%Q2
\begin{question}
A $3\times3$ magic square is a $3\times3$ array 
\[
\begin{array}{ccc}
a & b & c\\
d & e & f\\
g & h & k
\end{array}
\]
whose entries are the nine distinct integers $1,2,3,4,5,6,7,8,9$
and which has the property that all its rows, columns and main diagonals
add up to the same number $n$. (Thus $a+b+c=d+e+f=g+h+k=a+d+g=b+e+h=c+f+k=a+e+k=c+e+g=n.)$
\begin{itemize}
\setlength{\itemsep}{3mm}
\item[\bf (i)]  Show that $n=15.$
\item[\bf (ii)] Show that $e=5.$
\item[\bf (iii)] Show that one of $b,d,h$ or $f$ must have value $9$. 
\item[\bf (iv)] Find all $3\times3$ magic squares with $b=9.$
\item[\bf (v)] How many different $3\times3$ magic squares are there? Why?
\end{itemize}
{[}Two magic squares are different if they have different entries
in any place of the array.{]} 
\end{question}

%%%%%%%%% Q3
\begin{question}
Evaluate 
\begin{itemize}
\setlength{\itemsep}{3mm}
\item[\bf (i)]  ${\displaystyle \int_{-\pi}^{\pi}\left|\sin x\right|\,\mathrm{d}x,}$
\item[\bf (ii)] ${\displaystyle \int_{-\pi}^{\pi}\sin\left|x\right|\,\mathrm{d}x},$ 
\item[\bf (iii)] ${\displaystyle \int_{-\pi}^{\pi}x\sin x\,\mathrm{d}x},$ 
\item[\bf (iv)] ${\displaystyle \int_{-\pi}^{\pi}x^{10}\sin x\,\mathrm{d}x.}$
\end{itemize}
\end{question}

%%%%%% Q4 
\begin{question}
Sketch the following subsets of the complex plane using Argand diagrams.
Give reasons for your answers. 
\begin{questionparts}
\item $\{z:\mathrm{Re}((1+\mathrm{i})z)\geqslant0\}.$ 
\item $\{z:\left|z^{2}\right|\leqslant2,\mathrm{Re}(z^{2})\geqslant0\}.$
\item $\{z=z_{1}+z_{2}:\left|z_{1}\right|=2,\left|z_{2}\right|=1\}.$
\end{questionparts}
\end{question}

%%%%%%%%% Q5
\begin{question}
	Let $\mathrm{p}_{0}(x)=(1-x)(1-x^{2})(1-x^{4}).$ Show that $(1-x)^{3}$
is a factor of $\mathrm{p}_{0}(x).$ If $\mathrm{p}_{1}(x)=x\mathrm{p}_{0}'(x)$
show, by considering factors of the polynomials involved, that $\mathrm{p}_{0}'(1)=0$
and $\mathrm{p}_{1}'(1)=0.$ 


By writing $\mathrm{p}_{0}(x)$ in the form 
\[
\mathrm{p}_{0}(x)=c_{0}+c_{1}x+c_{2}x^{2}+c_{3}x^{3}+c_{4}x^{4}+c_{5}x^{5}+c_{6}x^{6}+c_{7}x^{7},
\]
deduce that 
\begin{alignat*}{2}
1+2+4+7 & \quad=\quad &  & 3+5+6\\
1^{2}+2^{2}+4^{2}+7^{2} & \quad=\quad &  & 3^{2}+5^{2}+6^{2}.
\end{alignat*}
Show that we can write the integers $1,2,\ldots,15$ in some order
as $a_{1},a_{2},\ldots,a_{15}$ in such a way that 
\[
a_{1}^{r}+a_{2}^{r}+\cdots+a_{8}^{r}=a_{9}^{r}+a_{10}^{r}+\cdots+a_{15}^{r}
\]
for $r=1,2,3.$ 
	\end{question}
	
	%%%%%%%%% Q6
	\begin{question}
Explain briefly, by means of a diagram, or otherwise, why 
\[
\mathrm{f}(\theta+\delta\theta)\approx\mathrm{f}(\theta)+\mathrm{f}'(\theta)\delta\theta,
\]
when $\delta\theta$ is small. 


Two powerful telescopes are placed at points $A$ and $B$ which are
a distance $a$ apart. A very distant point $C$ is such that $AC$
makes an angle $\theta$ with $AB$ and $BC$ makes an angle $\theta+\phi$
with $AB$ produced. (A sketch of the arrangement is given in the
diagram.) 


\noindent \begin{center}
\psset{xunit=0.8cm,yunit=0.8cm,algebraic=true,dimen=middle,dotstyle=o,dotsize=3pt 0,linewidth=0.5pt,arrowsize=3pt 2,arrowinset=0.25} \begin{pspicture*}(-4.18,-0.94)(4.4,5.22) \psline(-4,0)(4,0) \psline(-2,0)(2,5) \psline(2,5)(1,0) \rput[tl](-2.3,-0.14){$A$} \rput[tl](1.08,-0.14){$B$} \rput[tl](-1.6,0.46){$\theta$} \rput[tl](1.24,0.52){$\theta+\phi$} \rput[tl](2.14,5.1){$C$} \end{pspicture*}
\par\end{center}


If the perpendicular distance $h$ of $C$ from $AB$ is very large
compared with $a$ show that $h$ is approximately $(a\sin^{2}\theta)/\phi$
and find the approximate value of $AC$ in terms of $a,\theta$ and
$\phi.$ 


It is easy to show (but you are not asked to show it) that errors
in measuring $\phi$ are much more important than errors in measuring
$\theta.$ If we make an error of $\delta\phi$ in measuring $\phi$
(but measure $\theta$ correctly) what is the approximate error in
our estimate of $AC$ and, roughly, in what proportion is it reduced
by doubling the distance between $A$ and $B$? 
	\end{question}
	
	%%%%%%%%% Q7
	\begin{question}
Let $\mathrm{g}(x)=ax+b.$ Show that, if $\mathrm{g}(0)$ and $\mathrm{g}(1)$
are integers, then $\mathrm{g}(n)$ is an integer for all integers
$n$. 


Let $\mathrm{f}(x)=Ax^{2}+Bx+C.$ Show that, if $\mathrm{f}(-1),\mathrm{f}(0)$
and $\mathrm{f}(1)$ are integers, then $\mathrm{f}(n)$ is an integer
for all integers $n$. 


Show also that, if $\alpha$ is any real number and $\mathrm{f}(\alpha-1),$
$\mathrm{f}(\alpha)$ and $\mathrm{f}(\alpha+1)$ are integers, then
$\mathrm{f}(\alpha+n)$ is an integer for all integers $n$. 
\end{question}
		
%%%%%%%%% Q8
	\begin{question}	
Explain diagrammatically, or otherwise, why 
\[
\frac{\mathrm{d}}{\mathrm{d}x}\int_{a}^{x}\mathrm{f}(t)\,\mathrm{d}t=\mathrm{f}(x).
\]
Show that, if 
\[
\mathrm{f}(x)=\int_{0}^{x}\mathrm{f}(t)\,\mathrm{d}t+1,
\]
then $\mathrm{f}(x)=\mathrm{e}^{x}.$


What is the solution of 
\[
\mathrm{f}(x)=\int_{0}^{x}\mathrm{f}(t)\,\mathrm{d}t?
\]
Given that 
\[
\int_{0}^{x}\mathrm{f}(t)\,\mathrm{d}t=\int_{x}^{1}t^{2}\mathrm{f}(t)\,\mathrm{d}t+x-\frac{x^{5}}{5}+C,
\]
find $\mathrm{f}(x)$ and show that $C=-2/15.$ 
\end{question}	
		
%%%%%%%%%% Q9
\begin{question}
The diagram shows a coffee filter consisting of an inverted hollow
right circular cone of height $H$ cm and base radius $a$ cm. 


\noindent \begin{center}
\psset{xunit=1.0cm,yunit=0.8cm,algebraic=true,dimen=middle,dotstyle=o,dotsize=3pt 0,linewidth=0.5pt,arrowsize=3pt 2,arrowinset=0.25} \begin{pspicture*}(-1.67,-2.3)(2.85,3.85) \rput{0}(0,3){\psellipse(0,0)(1.23,0.72)} \rput{0.69}(0,0.01){\psellipse(0,0)(0.49,0.23)} \psline(-1.23,2.95)(0,-2) \psline(0,-2)(1.23,2.96) \psline{->}(0,3)(0.66,3.61) \psline{->}(0.66,3.61)(0,3) \rput[tl](0.35,3.27){$a$} \psline{<->}(1,0)(1,-2) 
 \rput[tl](1.05,-0.86){$x$} \psline{<->}(2,3)(2,-2) \rput[tl](2.09,0.97){$H$} \end{pspicture*}
\par\end{center}


When the water level is $x$ cm above the vertex, water leaves the
cone at a rate $Ax$ $\mathrm{cm}^{3}\mathrm{sec}^{-1},$ where $A$
is a positive constant. Suppose that the cone is initially filled
to a height $h$ cm with $0<h<H.$ Show that it will take $\pi a^{2}h^{2}/(2AH^{2})$
seconds to empty. 


Suppose now that the cone is initially filled to a height $h$ cm,
but that water is poured in at a constant rate $B$ $\mathrm{cm}^{3}\mathrm{sec}^{-1}$
and continues to drain as before. Establish, by considering the sign
of $\mathrm{d}x/\mathrm{d}t$, or otherwise, what will happen subsequently
to the water level in the different cases that arise. (You are not
asked to find an explicit formula for $x$.) 
\end{question}
			
		

		
	
\newpage
\section*{Section B: \ \ \ Mechanics}

			
%%%%%%%%%% Q10
\begin{question}
A projectile of mass $m$ is fired horizontally from a toy cannon
of mass $M$ which slides freely on a horizontal floor. The length
of the barrel is $l$ and the force exerted on the projectile has
the constant value $P$ for so long as the projectile remains in the
barrel. Initially the cannon is at rest. Show that the projectile
emerges from the barrel at a speed relative to the ground of 
\[
\sqrt{\frac{2PMl}{m(M+m)}}.
\]
\end{question}
	
%%%%%%%%%% Q11
\begin{question}
Three light elastic strings $AB,BC$ and $CD$, each of natural length
$a$ and modulus of elasticity $\lambda,$ are joined together as
shown in the diagram. 


\noindent \begin{center}
\psset{xunit=1.0cm,yunit=1.0cm,algebraic=true,dimen=middle,dotstyle=o,dotsize=3pt 0,linewidth=0.5pt,arrowsize=3pt 2,arrowinset=0.25} \begin{pspicture*}(-2.46,-1.7)(5.55,2.55) \psline(-2,2)(-2,-1) \psline(-2,-1)(4,-1) \psline(4,-1)(4,2) \psline(4,2)(-2,2) \psline[linestyle=dashed,dash=2pt 2pt](1,2)(1,-1) \psline{<->}(5,2)(5,-1)  \rput[tl](5.15,0.77){$3d$} \rput[tl](1.08,2.35){$A$} \rput[tl](1.14,0.63){$B$} \rput[tl](1.17,-0.32){$C$} \rput[tl](1.14,-1.1){$D$} \begin{scriptsize} \psdots[dotstyle=*](1,2) \psdots[dotstyle=*](1,-1) \psdots[dotstyle=*](1,0.5) \psdots[dotstyle=*](1,-0.37) \end{scriptsize} \end{pspicture*}
\par\end{center}


$A$ is attached to the ceiling and $D$ to the floor of a room of
height $3d$ in such a way that $A,B,C$ and $D$ are in a vertical
line. Particles of mass $m$ are attached at $B$ and $C$. Find the
heights of $B$ and $C$ above the floor. 


Find the set of values of $d$ for which it is possible, by choosing
$m$ suitably, to have $CD=a$?
	\end{question}
	
%%%%%%%%%% Q12
\begin{question}	
The diagram shows a crude step-ladder constructed by smoothly hinging-together
two light ladders $AB$ and $AC,$ each of length $l,$ at $A$. A
uniform rod of wood, of mass $m$, is pin-jointed to $X$ on $AB$
and to $Y$ on $AC$, where $AX=\frac{3}{4}l=AY.$ The angle $\angle XAY$
is $2\theta.$ 


\noindent \begin{center}
\psset{xunit=1.0cm,yunit=1.0cm,algebraic=true,dimen=middle,dotstyle=o,dotsize=3pt 0,linewidth=0.5pt,arrowsize=3pt 2,arrowinset=0.25} \begin{pspicture*}(-4.3,-1.22)(4.6,6) \psline(-4,0)(4,0) \psline(-2,0)(0,5) \psline(0,5)(2,0) \psline(-1.21,1.97)(1.21,1.97) \parametricplot{-1.9513027039072617}{-1.190289949682532}{1.2*cos(t)+0|1.2*sin(t)+5} \rput[tl](-0.2,4.26){$2\theta$} \rput[tl](-0.1,5.5){$A$} \rput[tl](-1.8,2.1){$X$} \rput[tl](1.5,2.1){$Y$} \rput[tl](-2.36,-0.1){$B$} \rput[tl](2.02,-0.1){$C$} \end{pspicture*}
\par\end{center}


The rod $XY$ will break if the tension in it exceeds $T$. The step-ladder
stands on rough horizontal ground (coefficient of friction $\mu$).
Given that $\tan\theta>\mu,$ find how large a mass $M$ can safely
be placed at $A$ and show that if 
\[
\tan\theta>\frac{6T}{mg}+4\mu
\]
the step-ladder will fail under its own weight. 


{[}You may assume that friction is limiting at the moment of collapse.{]} 
\end{question}

%%%%%%%%%% Q13

\begin{question}
A comet, which may be regarded as a particle of mass $m$, moving
in the sun's gravitational field, at a distance $x$ from the sun,
experiences a force $Gm/x^{2}$ (where $G$ is a constant) directly towards the sun. Show that
if, at some time, $x=h$ and the comet is travelling directly away
from the sun with speed $V$, then $x$ cannot become arbitrarily
large unless $V^{2}\geqslant2G/h$. 


A comet is initially motionless at a great distance from the sun.
If, at some later time, it is at a distance $h$ from the sun, how
long after that will it take to fall into the sun? 
\end{question}

	
	\newpage
\section*{Section C: \ \ \ Probability and Statistics}

%%%%%%%%%% Q14
\begin{question}
The average number of pedestrians killed annually in road accidents
in Poldavia during the period 1974-1989 was 1080 and the average number
killed annually in commercial flight accidents during the same period
was 180. Discuss the following newspaper headlines which appeared
in 1991. (The percentage figures in square brackets give a rough indication
of the weight of marks attached to each discussion.)
\begin{questionparts}
\item {[}10\%{]} Six Times Safer To Fly Than To Walk. 1974-1989 Figures
Prove It.
\item {[}10\%{]} Our Skies Are Safer. Only 125 People Killed In Air
Accidents In 1990. 
\item {[}30\%{]} Road Carnage Increasing. 7 People Killed On Tuesday.
\item {[}50\%{]} Alarming Rise In Pedestrian Casualties. 1350 Pedestrians
Killed In Road Accidents During 1990.
\end{questionparts}
\end{question}



%%%%%%%%%% Q15
\begin{question}
Trains leave Barchester Station for London at 12 minutes past the
hour, taking 60 minutes to complete the journey and at 48 minutes
past the hour taking 75 minutes to complete the journey. The arrival
times of passengers for London at Barchester Station are uniformly
distributed over the day and all passengers take the first available
train. Show that their average journey time from arrival at Barchester
Station to arrival in London is 84.6 minutes. 


Suppose that British Rail decide to retime the fast 60 minute train
so that it leaves at $x$ minutes past the hour. What choice of $x$
will minimise the average journey time? 
\end{question}

%%%%%%%%%% Q16
\begin{question}
The four towns $A,B,C$ and $D$ are linked by roads $AB,AC,CB,BD$
and $CD.$ The probability that any one road will be blocked by snow
on the 1st of January is $p$, independent of what happens to any
other $[0<p<1]$. Show that the probability that any open route from
$A$ to $D$ is $ABCD$ is 
\[
p^{2}(1-p)^{3}.
\]
In order to increase the probability that it is possible to get from
$A$ to $D$ by a sequence of unblocked roads the government proposes
either to snow-proof the road $AB$ (so that it can never be blocked)
or to snow-proof the road $CB.$ Because of the high cost it cannot
do both. Which road should it choose (or are both choices equally
advantageous)? 


In fact, $p=\frac{1}{10}$ and the government decides that it is only
worth going ahead if the present probability of $A$ being cut off
from $D$ is greater than $\frac{1}{100}.$ Will it go ahead?
\end{question}
\end{document}
