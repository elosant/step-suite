\documentclass[a4, 11pt]{report}


\pagestyle{myheadings}
\markboth{}{Paper III, 2006
\ \ \ \ \ 
\today 
}               

\RequirePackage{amssymb}
\RequirePackage{amsmath}
\RequirePackage{graphicx}
\RequirePackage{color}
\RequirePackage[flushleft]{paralist}[2013/06/09]



\RequirePackage{geometry}
\geometry{%
  a4paper,
  lmargin=2cm,
  rmargin=2.5cm,
  tmargin=3.5cm,
  bmargin=2.5cm,
  footskip=12pt,
  headheight=24pt}


\newcommand{\comment}[1]{{\bf Comment} {\it #1}}
%\renewcommand{\comment}[1]{}

\newcommand{\bluecomment}[1]{{\color{blue}#1}}
%\renewcommand{\comment}[1]{}
\newcommand{\redcomment}[1]{{\color{red}#1}}



\usepackage{epsfig}
\usepackage{pstricks-add}
\usepackage{tgheros} %% changes sans-serif font to TeX Gyre Heros (tex-gyre)
\renewcommand{\familydefault}{\sfdefault} %% changes font to sans-serif
%\usepackage{sfmath}  %%%% this makes equation sans-serif
%\input RexFigs


\setlength{\parskip}{10pt}
\setlength{\parindent}{0pt}

\newlength{\qspace}
\setlength{\qspace}{20pt}


\newcounter{qnumber}
\setcounter{qnumber}{0}

\newenvironment{question}%
 {\vspace{\qspace}
  \begin{enumerate}[\bfseries 1\quad][10]%
    \setcounter{enumi}{\value{qnumber}}%
    \item%
 }
{
  \end{enumerate}
  \filbreak
  \stepcounter{qnumber}
 }


\newenvironment{questionparts}[1][1]%
 {
  \begin{enumerate}[\bfseries (i)]%
    \setcounter{enumii}{#1}
    \addtocounter{enumii}{-1}
    \setlength{\itemsep}{5mm}
    \setlength{\parskip}{8pt}
 }
 {
  \end{enumerate}
 }



\DeclareMathOperator{\cosec}{cosec}
\DeclareMathOperator{\Var}{Var}

\def\d{{\mathrm d}}
\def\e{{\mathrm e}}
\def\g{{\mathrm g}}
\def\h{{\mathrm h}}
\def\f{{\mathrm f}}
\def\p{{\mathrm p}}
\def\s{{\mathrm s}}
\def\t{{\mathrm t}}


\def\A{{\mathrm A}}
\def\B{{\mathrm B}}
\def\E{{\mathrm E}}
\def\F{{\mathrm F}}
\def\G{{\mathrm G}}
\def\H{{\mathrm H}}
\def\P{{\mathrm P}}


\def\bb{\mathbf b}
\def \bc{\mathbf c}
\def\bx {\mathbf x}
\def\bn {\mathbf n}

\newcommand{\low}{^{\vphantom{()}}}
%%%%% to lower suffices: $X\low_1$ etc


\newcommand{\subone}{ {\vphantom{\dot A}1}}
\newcommand{\subtwo}{ {\vphantom{\dot A}2}}




\def\le{\leqslant}
\def\ge{\geqslant}


\def\var{{\rm Var}\,}

\newcommand{\ds}{\displaystyle}
\newcommand{\ts}{\textstyle}
\def\half{{\textstyle \frac12}}
\def\l{\left(}
\def\r{\right)}



\begin{document}
\setcounter{page}{2}

 
\section*{Section A: \ \ \ Pure Mathematics}

%%%%%%%%%%Q1
\begin{question}
Sketch the curve with cartesian equation 
\[
y = \frac{2x(x^2-5)}{x^2-4}
\]
and give the equations of the asymptotes and of the tangent to the curve
at the origin.

Hence determine the number of real roots of the following equations:
\begin{questionparts}
\item  $3x(x^2-5)= (x^2-4)(x+3)\,$;
\item $4x(x^2-5)= (x^2-4)(5x-2)\,$;
\item  $4x^2(x^2-5)^2= (x^2-4)^2(x^2+1)\,$.

\end{questionparts}
\end{question}

%%%%%%%%%%Q2
\begin{question}
Let 
\[
I = \int_{-\frac12 \pi}^{\frac12\pi} 
\frac {\cos^2\theta}{1-\sin\theta\sin2\alpha} \, \d\theta  
\text{ \ \  and \ \ }
J = \int_{-\frac12 \pi}^{\frac12\pi} 
\frac {\sec^2\theta}{1+\tan^2\theta\cos^22\alpha} \, \d\theta
\]
where $0<\alpha< \frac14\pi\,$.
\begin{questionparts}
\item
Show that $\displaystyle 
I = \int_{-\frac12 \pi}^{\frac12\pi} 
\frac {\cos^2\theta}{1+\sin\theta\sin2\alpha} \, \d\theta \ $ and hence that
$\displaystyle 
\ 2I = \int_{-\frac12 \pi}^{\frac12\pi} 
\frac {2}{1+\tan^2\theta\cos^22\alpha} \, \d\theta\,$.

\item Find $J$.

\item By considering $I\sin^2 2\alpha +J\cos^2 2\alpha $, or otherwise,
show that  $I =\frac12 \pi \sec^2\alpha$.

\item Evaluate $I$ in the case 
$\frac14\pi < \alpha < \frac12\pi$.
\end{questionparts}
\end{question}

%%%%%%%%% Q3
\begin{question}
\begin{questionparts}
\item
Let 
\[
\tan x = \sum\limits_{n=0}^\infty a_n x^n
\text{ \ \ \ and \ \ \ }
\cot x = \dfrac 1 x +\sum\limits_{n=0}^\infty b_nx^n
\]
for $0<x<\frac12\pi\,$. Explain why $a_n=0$ for even $n$.

Prove  the identity
\[
\cot x - \tan x \equiv 2 \cot 2x\,
\]
and show that \[a_{n} = (1-2^{n+1})b_n\,.\]

\item
Let 
$ \displaystyle {\rm cosec}\, x 
= \frac1x +\sum\limits _{n=0}^\infty c_n x^n\,$ for 
$0<x<\frac12\pi\,$.
By considering $\cot x + \tan x$, or otherwise, show that
\[
c_n = (2^{-n} -1)b_n
\,.
\]

\item Show that 
\[
\left(1+x{ \sum\limits_{n=0}^\infty} b_n x^n \right)^2 +x^2
= \left(1+x{ \sum\limits_{n=0} ^\infty} c_n x^n \right)^2\,.
\]
Deduce from this and the previous results that $a_1=1$, and find $a_3$. 
\end{questionparts}
\end{question}

%%%%%% Q4 
\begin{question}
The function $\f$ satisfies the identity
\begin{equation}
\f(x) +\f(y) \equiv \f(x+y) 
\tag{$*$}
\end{equation}
for all $x$ and $y$. Show that $2\f(x)\equiv \f(2x)$ and deduce that 
$\f''(0)=0$.
By considering the Maclaurin series for $\f(x)$, find the
most general function that satisfies $(*)$.
\newline
 [{\it Do not
consider issues of existence or convergence of Maclaurin series
in this question.}]

\begin{questionparts}
\item
By considering the function $\G$, defined by
$\ln\big(\g(x)\big) =\G(x)$, find the 
most general function that, for all $x$ and $y$, satisfies the identity
\[
\g(x) \g(y) \equiv \g(x+y)\,.
\]


\item
By considering the function $\H$, defined by
$\h(\e^u) =\H(u)$, find the 
most general function that satisfies, for all positive $x$ and $y$, the identity
\[
\h(x) +\h(y) \equiv  \h(xy)
\,.
\]

\item Find the most general function $\t$ that, for all $x$ and $y$,
 satisfies the identity
\begin{equation*}
\t(x) + \t(y) \equiv \t(z)\,,
\end{equation*}
where $z= \dfrac{x+y}{1-xy}\,$.


\end{questionparts}
\end{question}

%%%%%%%%% Q5
\begin{question}
Show that  the distinct
complex numbers $\alpha$, $\beta$ and $\gamma$ represent the
vertices of an equilateral triangle (in clockwise or anti-clockwise order)
if and only if
\[
\alpha^2 + \beta^2 +\gamma^2 -\beta\gamma - \gamma \alpha -\alpha\beta =0\,.
\]

Show that 
the roots of the equation 
\begin{equation*}
z^3 +az^2 +bz +c=0
\tag{$*$}
\end{equation*}
represent the vertices of an equilateral triangle if and only if $a^2=3b$. 


Under the transformation  $z=pw+q$, where $p$ and $q$ are given 
complex numbers with $p\ne0$, the equation ($*$) becomes
\[
w^3 +Aw^2 +Bw +C=0\,.
\tag{$**$}
\]
Show that if the roots of equation $(*)$ represent
 the vertices of an equilateral triangle,
then  the roots of equation $(**)$ also
represent  the vertices of an equilateral triangle.
	\end{question}
	
%%%%%%%%% Q6
\begin{question}
Show that in polar coordinates the gradient of any curve 
at the point $(r,\theta)$ is
\[
\frac{ \ \ \dfrac{\d r }{\d\theta} \tan\theta + r \ \ }
{ \dfrac{\d r }{\d\theta} -r\tan\theta}\,.
\]

\noindent \begin{center}
\psset{xunit=1.0cm,yunit=1.0cm,algebraic=true,dotstyle=o,dotsize=3pt 0,linewidth=0.5pt,arrowsize=3pt 2,arrowinset=0.25} \begin{pspicture*}(-0.6,-3)(6.8,3) \psline(0,0)(6.54,0) \rput[tl](4.13,-0.22){$O$} \rput[tl](-0.47,0.07){$L$} \rput{-270}(5.75,0.08){\psplot[plotpoints=500]{-12}{12}{x^2/2/3}} \psline(2,1.5)(5.42,1.5) \psline(3.73,-0.74)(5.42,1.5) \psline[linewidth=0.4pt]{->}(3,1.5)(4,1.5) \psline[linewidth=0.4pt]{->}(5.42,1.5)(4.99,0.93) \psline(3.84,0.78)(6.62,2.05) \end{pspicture*}
\par\end{center}

A mirror is designed so that if an incident ray of light is parallel
to a fixed line $L$ the  reflected ray passes through a fixed point $O$
on $L$. Prove that the mirror intersects any plane containing $L$ in
a parabola. You should assume that the angle between the incident
ray and 
the normal to the mirror is the same as the 
angle between the reflected ray and the normal.
\end{question}
	
%%%%%%%%% Q7
\begin{question}
\begin{questionparts}
\item Solve the equation $u^2+2u\sinh x -1=0$ giving $u$ in terms
of $x$.

Find the solution of the differential equation
\[
\left( \frac{\d y}{\d x}\right)^{\!2} +2 \frac{\d y}{\d x} \sinh x -1 = 0
\]
that satisfies $y=0$ and $\dfrac {\d y}{\d x} >0$ at $x=0$.

\item
Find the solution, not identically zero,  of the differential equation 
\[
\sinh y  \left( \frac{\d y}{\d x}\right)^{\!2} 
+2 \frac{\d y}{\d x}  -\sinh y = 0
\]
that satisfies $y=0$ at $x=0$,
expressing your solution in the form
$\cosh  y=\f(x)$. Show that the asymptotes to the solution curve are
$y=\pm(-x+\ln 4)$.
\end{questionparts}
\end{question}
		
%%%%%%%%% Q8
\begin{question}	
$\triangle$ is an operation that takes polynomials in $x$ to 
polynomials in $x$; that is, given any polynomial $\h(x)$, there
is a polynomial called $\triangle \h(x)$ which is obtained from 
$\h(x)$ using the rules that define $\triangle$. These rules are as follows:
\begin{questionparts}
\item $\triangle x = 1\,$;
\item
$\triangle \big( \f(x)+\g(x)\big) = \triangle \f(x) + \triangle
\g(x)\,$ for any polynomials $\f(x)$ and $\g(x)$;
\item $\triangle \big( \lambda \f(x)\big) =\lambda \triangle \f(x)$
for any constant $\lambda$ and any polynomial $\f(x)$;
\item $\triangle \big( \f(x)\g(x)\big) = \f(x) \triangle \g(x) +
\g(x)\triangle \f(x)$ for any polynomials 
$\f(x)$ and $\g(x)$.
\end{questionparts}

Using these rules show that, 
if $\f(x)$ is a polynomial of degree zero (that is, a constant),
then $\triangle \f(x) =0$. Calculate $\triangle x^2$ and $\triangle x^3$.

Prove that $\triangle \h(x) \equiv  \dfrac{\d \h(x)}{\d x \ \ \ }$ for 
any polynomial $\h(x)$. You should make it  clear whenever you use
one of the above rules in your proof. $\vphantom{\int}$
\end{question}	
		

		
	
\newpage
\section*{Section B: \ \ \ Mechanics}


	
%%%%%%%%%% Q9
\begin{question}
A long, light, inextensible string  passes through a small, smooth ring
fixed at  the point $O$. One end of the string 
is attached to a particle $P$ of mass $m$ 
which hangs freely below $O$. The other end is attached
to a bead, $B$, also of mass $m$, which is threaded on a smooth rigid wire
fixed in the same vertical plane as $O$.
The distance $OB$ is $r$, the distance $OH$ is $h$ 
and the  height of the bead above the horizontal plane through~$O$ is $y$,
as shown
in the diagram.





\begin{center}
\psset{xunit=0.8cm,yunit=0.8cm,algebraic=true,dimen=middle,dotstyle=o,dotsize=3pt 0,linewidth=0.5pt,arrowsize=3pt 2,arrowinset=0.25}
\begin{pspicture*}(-3.93,-2.62)(3.97,4.92)
\psplot[plotpoints=200]{-3.2}{3.2}{4-0.6*x^(2)}
\rput{0}(0,0){\psellipse(0,0)(0.61,0.16)}
\psline(0,0)(1.73,2.19)
\psline[linestyle=dashed,dash=1pt 1pt](-5,0)(5,0)
\rput[tl](0.46,0.58){$\theta$}
\rput[tl](0.06,-0.27){$O$}
\rput[tl](0.15,-1.35){$P$}
\psline[linestyle=dashed,dash=1pt 1pt](1.73,2.19)(1.73,0)
\rput[tl](1.85,1){$y$}
\rput[tl](0.77,1.7){$r$}
\rput[tl](1.84,2.68){$B$}
\rput[tl](0.06,4.47){$H$}
\rput[tl](-0.36,2.3){$h$}
\psline[linestyle=dashed,dash=1pt 1pt](0,7)(0,0)
\psline(0,0)(0,-2)
\begin{scriptsize}
\psdots[dotsize=5pt 0,dotstyle=*](1.73,2.19)
\psdots[dotsize=5pt 0,dotstyle=*](0,-2)
\psdots[dotstyle=*,linecolor=darkgray](0,0)
\end{scriptsize}
\end{pspicture*}
\end{center}


The shape of the wire
is such that the system can be in static equilibrium for all positions
of the bead.
By considering potential energy, show that the 
equation of the wire is $y+r =2h$.

The bead is initially at $H$. It is then projected along the wire with
initial speed $V$. Show that, in the subsequent motion, 
\[
\dot \theta = -\frac {h \dot r }{r \sqrt{rh -h^2}}\,
\] 
where $\theta$ is given by
$\theta = \arcsin(y/r)$.

Hence show that the speed of the particle $P$ is 
$V \Big(\dfrac{r-h}{2r-h}\Big)^{\!\frac12}\,$.

\noindent[{\it Note that $\arcsin \theta$ is another notation for $\sin^{-1}\theta$.}] 
	\end{question}
	
%%%%%%%%%% Q10 
\begin{question}	
A disc rotates freely  in a horizontal plane about a vertical axis
through its centre. The moment of inertia of the disc about this axis
is $mk^2$ (where $k>0$). Along one diameter is a smooth narrow 
groove in which a particle of 
mass $m$ slides freely. At time $t=0\,$, the disc is rotating with angular 
speed $\Omega$, and the particle is a distance $a$ from the axis and
is moving with speed~$V$ along the groove, towards
the axis, where 
$k^2V^2 = \Omega^2a^2(k^2+a^2)\,$.

Show that, at a later time $t$, while the particle is still moving towards
the axis, the angular speed $\omega$ of the disc and the distance $r$ of 
the particle from the axis are related by
\[
\omega = \frac{\Omega(k^2+a^2)}{k^2+r^2}
\text{ \ \ and \ \ }
\left(\frac{\d r}{\d t}\right)^{\!2} = \frac{\Omega^2r^2(k^2+a^2)^2}{k^2(k^2+r^2)}\;.
\]
Deduce that 
\[
k\frac{\d r}{\d\theta} = -r(k^2+r^2)^{\frac12}\,,
\]
where $\theta $ is the angle through which the disc has turned by time $t$.

By making the substitution $u=k/r$, or otherwise, show that 
$r\sinh (\theta+\alpha) = k$, where $\sinh \alpha = k/a\,$. Deduce that
the particle never reaches the axis.
\end{question}

%%%%%%%%%% Q11

\begin{question}
A lift of mass $M$ and 
its counterweight of mass $M$ are connected by a light 
inextensible cable which passes over a fixed frictionless pulley. The lift is
constrained to move vertically between smooth guides. The distance between
the floor and the ceiling of the lift is $h$. Initially, the lift is at
rest, and the distance between the top of the lift and the pulley is
greater than $h$. A small tile of mass $m$ becomes detached from the
ceiling of the lift and falls to the floor of the lift.
Show that the speed of the tile just before the impact
is
\[
\sqrt{\frac{(2M-m)gh \;}{M}}\;.
\]

The coefficient of restitution between the tile 
and the floor of the lift is $e$.
Given that  the magnitude of the impulsive 
force on the lift due to tension in the cable  
is equal to the magnitude of the impulsive force on the counterweight 
due to tension in the cable,
show that the loss of energy of the system due to the impact is $mgh(1-e^2)$.
Comment on this result.
\end{question}
	

	
	\newpage
\section*{Section C: \ \ \ Probability and Statistics}


%%%%%%%%%% Q12
\begin{question}
Fifty times a year, 1024 tourists disembark from a cruise liner at
a port.
 From there they must travel to the  city centre either by bus or by taxi.
Tourists  are equally likely to be directed to the bus station
or to the taxi rank. Each bus of the bus company holds 32 passengers, and the
company currently runs 15 buses. The company makes a profit of $\pounds$1 
for each passenger carried. It carries as many passengers as it can, with
any excess being (eventually) transported by taxi. Show that  the 
largest annual licence fee, in pounds, that  the company should consider paying to be allowed
to run an extra bus is approximately
\[
1600 \Phi(2) -  \frac{800}{\sqrt{2\pi}}\big(1- \e^{-2}\big)\,,
\]
where  
$\displaystyle \Phi(x) =\dfrac1{\sqrt{2\pi}} \int_{-\infty}^x \e^{-\frac12t^2}\d t\,$.

\noindent[{\it You should not consider continuity corrections.}]
\end{question}

%%%%%%%%%% Q13
\begin{question}
Two points are chosen independently at random on the perimeter
(including the diameter) of a semicircle of unit radius.
The area of the triangle whose vertices are these
 two points  and the midpoint of the diameter is denoted by the random
variable $A$. Show that the expected value of $A$ is $(2+\pi)^{-1}$.
\end{question}

%%%%%%%%%% Q14
\begin{question}
For any random variables $X_1$ and $X_2$, state the relationship between
$\E(aX_1+bX_2)$ and $\E(X_1)$ and $\E(X_2)$, where $a$ and $b$ are constants. 
If $X_1$ and $X_2$ are independent,
state the relationship between $\E(X_1X_2)$ and $\E(X_1)$ and $\E(X_2)$.

An industrial process produces rectangular plates.
The length and the breadth of the plates are modelled by
independent random
variables  $X_1$ and $X_2$
with  non-zero means  $\mu_1$ and  $\mu_2$ 
and   non-zero
standard deviations $\sigma_1$ and $\sigma_2$, respectively.
Using the results in the paragraph above, and without quoting a
formula
for $\var(aX_1+bX_2)$, 
find the means and standard deviations of the
perimeter $P$ and area $A$ of the plates. 
Show that $P$ and $A$ are not independent.

The random variable $Z$ is defined by  $Z=P-\alpha A$, where $\alpha $ is a 
constant. Show that $Z$ and~$A$ are not independent if 
\[
\alpha \ne \dfrac{2(\mu_1^{\vphantom2} \sigma_2^2 +\mu_2^{\vphantom2}\sigma_1^2)}
{ \mu_1^2 \sigma_2^2 +\mu_2^2\sigma_1^2 + \sigma_1^2\sigma_2^2 }
\;.
\]

Given that $X_1$ and $X_2$ can each take values 1 and 3 only, and that
they each take these values with probability $\frac 12$, 
 show that $Z$ and $A$ are not independent for any value of $\alpha$.
\end{question}
	
\end{document}
