\documentclass[a4, 11pt]{report}


\pagestyle{myheadings}
\markboth{}{Paper I, 2002
\ \ \ \ \ 
\today 
}               

\RequirePackage{amssymb}
\RequirePackage{amsmath}
\RequirePackage{graphicx}
\RequirePackage{color}
\RequirePackage[flushleft]{paralist}[2013/06/09]



\RequirePackage{geometry}
\geometry{%
  a4paper,
  lmargin=2cm,
  rmargin=2.5cm,
  tmargin=3.5cm,
  bmargin=2.5cm,
  footskip=12pt,
  headheight=24pt}


\newcommand{\comment}[1]{{\bf Comment} {\it #1}}
%\renewcommand{\comment}[1]{}

\newcommand{\bluecomment}[1]{{\color{blue}#1}}
%\renewcommand{\comment}[1]{}
\newcommand{\redcomment}[1]{{\color{red}#1}}



\usepackage{epsfig}
\usepackage{pstricks-add}
\usepackage{tgheros} %% changes sans-serif font to TeX Gyre Heros (tex-gyre)
\renewcommand{\familydefault}{\sfdefault} %% changes font to sans-serif
%\usepackage{sfmath}  %%%% this makes equation sans-serif
%\input RexFigs


\setlength{\parskip}{10pt}
\setlength{\parindent}{0pt}

\newlength{\qspace}
\setlength{\qspace}{20pt}


\newcounter{qnumber}
\setcounter{qnumber}{0}

\newenvironment{question}%
 {\vspace{\qspace}
  \begin{enumerate}[\bfseries 1\quad][10]%
    \setcounter{enumi}{\value{qnumber}}%
    \item%
 }
{
  \end{enumerate}
  \filbreak
  \stepcounter{qnumber}
 }


\newenvironment{questionparts}[1][1]%
 {
  \begin{enumerate}[\bfseries (i)]%
    \setcounter{enumii}{#1}
    \addtocounter{enumii}{-1}
    \setlength{\itemsep}{5mm}
    \setlength{\parskip}{8pt}
 }
 {
  \end{enumerate}
 }



\DeclareMathOperator{\cosec}{cosec}
\DeclareMathOperator{\Var}{Var}

\def\d{{\mathrm d}}
\def\e{{\mathrm e}}
\def\g{{\mathrm g}}
\def\h{{\mathrm h}}
\def\f{{\mathrm f}}
\def\p{{\mathrm p}}
\def\s{{\mathrm s}}
\def\t{{\mathrm t}}


\def\A{{\mathrm A}}
\def\B{{\mathrm B}}
\def\E{{\mathrm E}}
\def\F{{\mathrm F}}
\def\G{{\mathrm G}}
\def\H{{\mathrm H}}
\def\P{{\mathrm P}}


\def\bb{\mathbf b}
\def \bc{\mathbf c}
\def\bx {\mathbf x}
\def\bn {\mathbf n}

\newcommand{\low}{^{\vphantom{()}}}
%%%%% to lower suffices: $X\low_1$ etc


\newcommand{\subone}{ {\vphantom{\dot A}1}}
\newcommand{\subtwo}{ {\vphantom{\dot A}2}}




\def\le{\leqslant}
\def\ge{\geqslant}


\def\var{{\rm Var}\,}

\newcommand{\ds}{\displaystyle}
\newcommand{\ts}{\textstyle}
\def\half{{\textstyle \frac12}}
\def\l{\left(}
\def\r{\right)}



\begin{document}
\setcounter{page}{2}

 
\section*{Section A: \ \ \ Pure Mathematics}

%%%%%%%%%%Q1
\begin{question}
Show that the equation of any circle passing through the points of 
intersection of the ellipse  $(x+2)^2 +2y^2 =18$ and the ellipse
$9(x-1)^2  +16y^2 =  25$ can be written in the form
\[
x^2-2ax +y^2 =5-4a\;.
\]
\end{question}

%%%%%%%%%%Q2
\begin{question}
Let $\f(x) = x^m(x-1)^n$, where $m$ and $n$ are both integers greater than $1$.
Show that the curve $y=\f(x)$ has a stationary point with $0<x<1$. By 
considering $\f\hspace{0.5pt}''(x)$, show that this
stationary point is a maximum if $n$ is even and a minimum if $n$ is odd.

Sketch the graphs of $\f(x)$
in the four cases that arise according to the values of $m$ and $n$.
\end{question}

%%%%%%%%% Q3
\begin{question}
Show that $(a+b)^2\le 2a^2+2b^2\,$.

Find the stationary points on the curve  
$y=\big(a^2\cos^2\theta +b^2\sin^2\theta\big)^{\frac12}
+ \big(a^2\sin^2\theta +b^2\cos^2\theta\big)^{\frac12}\,$, where 
$a$ and $b$ are constants. State, with brief reasons, which points
are maxima and which are minima. Hence prove that
\[
\vert a\vert +\vert b \vert 
\le \big(a^2\cos^2\theta +b^2\sin^2\theta\big)^{\frac12}
+ \big(a^2\sin^2\theta +b^2\cos^2\theta\big)^{\frac12}
\le \big(2a^2+2b^2\big)^{\frac12} \;.
\]
\end{question}

%%%%%% Q4 
\begin{question}
Give a sketch of  the curve  $ \;\displaystyle y= \frac1 {1+x^2}\;$, 
for $x\ge0$. 

Find the equation of the line that intersects  the curve 
at $x=0$ and is tangent to the curve
at some point with $x>0\,$. Prove 
that there are no further 
intersections between the line and the curve. Draw the line
on your sketch.


By considering the area under the curve for $0\le x\le1$, show that $\pi>3\,$.

Show also, by considering the volume formed by rotating the curve about
the $y$ axis, that $\ln 2 >2/3\,$.

\vspace{2mm}
[\hspace{1pt}
{\bf Note}: $\displaystyle \int_0^ 1 \frac1 {1+x^2}\, \d x = \frac\pi 4\,.\;$]
\end{question}

%%%%%%%%% Q5
\begin{question}
Let 
\[
\f(x) = x^n + a_1 x^{n-1} + \cdots + a_n\;,
\]
where $a_1$, $a_2$, $\ldots$\,, $a_n$ are given numbers.
It is given that $\f(x)$ can be written in the form
\[
\f(x) = (x+k_1)(x+k_2)\cdots(x+k_n)\;.
\]
By considering
$\f(0)$, or otherwise, show that $k_1k_2 \ldots k_n =a_n$.

Show also that $$(k_1+1)(k_2+1)\cdots(k_n+1)= 1+a_1+a_2+\cdots+a_n$$ and give a 
corresponding result for $(k_1-1)(k_2-1)\cdots(k_n-1)$.


Find the roots of the equation
\[
x^4 +22x^3 +172x^2 +552x+576=0\;,
\]
given that they are all integers.
	\end{question}
	
%%%%%%%%% Q6
\begin{question}
A   pyramid stands on horizontal ground. Its base is an equilateral triangle with sides of length~$a$,  the other three 
 sides of the pyramid are of length $b$ and its volume is $V$. Given that the 
formula for the volume of any pyramid is   
$
 \textstyle
\frac13 \times \mbox{area of base} \times \mbox {height} \,,
$
show that 
\[
V= \frac1{12} {a^2(3b^2-a^2)}^{\frac12}\;.
\]

The pyramid is then placed  so that a non-equilateral face  lies on the ground.
Show that the new height, $h$, of the pyramid is given by 
\[
h^2 = \frac{a^2(3b^2-a^2)}{4b^2-a^2}\;.
\]
Find, in terms of $a$ and $b\,$, the angle between the 
equilateral triangle and the horizontal.
\end{question}
	
%%%%%%%%% Q7
\begin{question}
Let
\[
I= \int_0^a \frac {\cos x}{\sin x + \cos x} \; \d x \,
\mbox{ \ \ \ \ and  \ \ \ \ }
J= \int_0^a \frac {\sin  x}{\sin x + \cos x} \; \d x \;,
\]
where $0\le a <\frac{3}{4}\pi\,$.
By considering $I+J$ and $I-J$,  show that 
$
2I= a + \ln (\sin a +\cos a)\;.
$

Find also:
\begin{questionparts}
\item 
$\displaystyle \int_0^{\frac{1}{2}\pi} \frac {\cos x}{p\sin x + q\cos x} \; \d x \,$, where $p$ and $q$ 
are positive numbers;
%\item [(ii)] 
%$\displaystyle \int_0^{\frac{1}{2}\pi/2} \frac {\cos x}{\sin (x+k)} \; \d x \,$, where $0<k<\pi/2\,$;
\item 
$\displaystyle \int_0^{\frac{1}{2}\pi} \frac {\cos x+4}{3\sin x + 4\cos x+ 25} \; \d x \,$.
\end{questionparts}
\end{question}
		
%%%%%%%%% Q8
\begin{question}	
I borrow $C$ pounds at interest rate $100\alpha \,\%$ per year.
The interest is added at the end of each year. Immediately after the 
interest is added, I make 
a repayment. The amount I repay  at the end of the $k$th year is $R_k$
pounds and the amount I owe
at the beginning of   $k$th year is $C_k$ pounds (with $C_1=C$).
Express $C_{n+1}$ in terms of $R_k$ ($k= 1$, $2$, $\ldots$, $n$), $\alpha$ and $C$
and show that,  if I pay off the loan in $N$ years with repayments
given by $R_k= (1+\alpha)^kr\,$, where $r$ is constant, then $r=C/N\,$.



If instead I pay off the loan in  $N$ years with $N$ equal repayments
of $R$ pounds, show that 
\[
\frac R C = \frac{\alpha (1+\alpha)^{N} }{(1+\alpha)^N-1} \;,
\]
and that $R/C\approx 27/103$ in the case $\alpha =1/50$, $N=4\,$.
\end{question}	
		

		
	
\newpage
\section*{Section B: \ \ \ Mechanics}


	
%%%%%%%%%% Q9
\begin{question}$\,$
\vspace*{-0.5in}
\begin{center}
\psset{xunit=1.0cm,yunit=1.0cm,algebraic=true,dimen=middle,dotstyle=o,dotsize=3pt 0,linewidth=0.5pt,arrowsize=3pt 2,arrowinset=0.25}
\begin{pspicture*}(-2.23,-0.31)(9.2,7.46)
\psline(0,0)(8,0)
\psline(0,0)(6.96,2.35)
\psline(1.01,0.34)(-0.61,5.05)
\psline(-0.61,5.05)(5.3,7.1)
\psline(5.3,7.1)(6.96,2.35)
\psline(0.84,0.83)(6.77,2.87)
\psline[linestyle=dashed,dash=2pt 2pt](3,4)(4.16,0.56)
\psline{->}(3,4)(9,4)
\psline[linestyle=dashed,dash=2pt 2pt](3,4)(-1.37,2.42)
\psline[linewidth=0.4pt,linestyle=dashed,dash=2pt 2pt]{->}(-0.43,2.76)(0.43,0.16)
\psline[linestyle=dashed,dash=2pt 2pt]{->}(0.43,0.15)(-0.43,2.75)
\psline[linestyle=dashed,dash=2pt 2pt](6.96,2.35)(7.29,1.39)
\parametricplot{0.0}{0.32693321010422505}{1.28*cos(t)+0|1.28*sin(t)+0}
\psline{->}(4.02,0.97)(7.1,1.93)
\psline{->}(7.1,1.93)(4.02,0.97)
\psline[linewidth=0.4pt,linestyle=dashed,dash=2pt 2pt]{->}(-0.71,5.29)(5.24,7.32)
\psline[linestyle=dashed,dash=2pt 2pt]{->}(5.24,7.32)(-0.71,5.29)
\psline(6.96,2.35)(7.98,2.71)
\rput[tl](2.58,3.67){$G$}
\rput[tl](8.9,3.81){$P$}
\rput[tl](5.64,1.39){$d$}
\rput[tl](1.81,6.79){$2d$}
\rput[tl](-0.59,1.5){$h$}
\rput[tl](1.52,0.32){$\alpha$}
\begin{scriptsize}
\psdots[dotstyle=*](3,4)
\end{scriptsize}
\end{pspicture*}
\end{center}

A lorry of weight $W$ stands on a plane inclined at an angle $\alpha$ to the
horizontal. Its wheels are a distance $2d$ apart, and its centre of
gravity $G$ is at a distance $h$ from the plane, and halfway between the sides
of the lorry. A horizontal  force $P$ acts on the lorry 
through $G\,$, as shown.

\begin{questionparts}
\item If the normal reactions on the lower and 
higher  wheels of the lorry  are equal,
show that the sum of the  frictional forces between the  wheels  and the ground is zero. 

\item If  $P$ is such  that the lorry does not tip
over (but the normal reactions on the lower and 
higher  wheels of the lorry  need not be equal), show that
\[
W\tan(\alpha - \beta)
\le P
\le
W\tan(\alpha+\beta)\;,
\]
where $\tan\beta = d/h\,$.
\end{questionparts}
	\end{question}
	
%%%%%%%%%% Q10 
\begin{question}	
A bicycle pump consists of a cylinder and a piston. The piston is pushed
in with steady speed~$u$. A particle of air moves to and fro between the 
piston and the  end of the cylinder, colliding  perfectly elastically with the piston
and the end of the cylinder, and always moving parallel with the axis of the cylinder.
Initially, the particle is moving towards the piston at speed $v$.
Show that the speed, $v_n$, of the particle just after the
$n$th collision with the piston is given by $v_n=v+2nu$.

Let $d_n$ be the distance between the piston and the end of the cylinder
at the $n$th collision, and let $t_n$ be the time between the 
$n$th and $(n+1)$th collisions. Express $d_n - d_{n+1}$ in terms
of $u$ and $t_n$, and show that
\[
d_{n+1} = \frac{v+(2n-1)u}{v+(2n+1)u} \, d_n \;.
\]
Express $d_n$ in terms of $d_1$, $u$, $v$  and $n$. 

In the case $v=u$, show that $ut_n = \displaystyle \frac {d_1} {n(n+1)}$.
%%%%%Verify that $\sum\limits_1^\infty t_n = d/u$.
\end{question}

%%%%%%%%%% Q11

\begin{question}$\,$
	
\begin{center}
\psset{xunit=1.0cm,yunit=1.0cm,algebraic=true,dimen=middle,dotstyle=o,dotsize=3pt 0,linewidth=0.5pt,arrowsize=3pt 2,arrowinset=0.25}
\begin{pspicture*}(-2.62,-2.1)(6.82,1.68)
\psline{->}(-1.7,0)(-0.22,0)
\psline[linestyle=dashed,dash=1pt 1pt](3,0)(6,0)
\psline{->}(4,0)(5.38,0.88)
\psline{->}(4,0)(5,-1)
\rput[tl](-2.36,0.71){$P_1$}
\rput[tl](0.54,0.71){$P_2$}
\rput[tl](-2.51,-0.16){$m$}
\rput[tl](0.61,-0.16){$km$}
\rput[tl](4.59,0.34){$\theta$}
\rput[tl](4.53,-0.08){$\phi$}
\rput[tl](5.87,1.55){$P_1$}
\rput[tl](5.33,-1.17){$P_2$}
\begin{scriptsize}
\psdots[dotsize=5pt 0,dotstyle=*](-2,0)
\psdots[dotsize=5pt 0,dotstyle=*](0.46,0)
\psdots[dotsize=5pt 0,dotstyle=*](5.62,1)
\psdots[dotsize=5pt 0,dotstyle=*](5.13,-1.18)
\end{scriptsize}
\end{pspicture*}
\end{center}
A particle $P_1$ of mass $m$ collides with  a particle $P_2$
of mass $km$ which is at rest. No energy is lost in the collision.
 The direction of motion 
of $P_1$ and $P_2$ after the collision make 
non-zero
angles of  $\theta$ and $\phi$, respectively, with the direction of motion 
of $P_1$ before the collision, as shown. Show that
\[
\sin^2\theta + k\sin^2\phi = k\sin^2(\theta+\phi) \;.
\] 

Show that, if the angle between the particles after the collision is a right angle,
then $k=1\,$.

\end{question}
	

	
	\newpage
\section*{Section C: \ \ \ Probability and Statistics}


%%%%%%%%%% Q12
\begin{question}
Harry the Calculating Horse will do any mathematical problem I set  him,
providing the answer is 1, 2, 3 or 4. When I set  him a problem,
he   places a  hoof on
a large  grid consisting of unit squares and  his   answer is  the 
number of squares partly covered by his hoof.
Harry has circular hoofs, of radius $1/4$ unit.


After many years of collaboration, I suspect that Harry no longer bothers to 
do the calculations, instead merely placing his hoof on the grid completely at random.
I often ask him to  divide 4 by 4, but only  about $1/4$ of his answers are right; 
I often ask him to add 2 and 2, but disappointingly only about 
$\pi/16$ of his answers are right. Is this consistent with my suspicions?

I decide to investigate  further by setting  Harry many problems,  the answers to which 
are 1, 2, 3, or 4 with equal frequency. If Harry is placing his hoof at random,
find the expected value of his answers. The average of Harry's answers turns out to 
be 2. Should I get a new horse? 

\end{question}

%%%%%%%%%% Q13
\begin{question}
The random variable $U$ takes the values $+1$, $0$ and $-1\,$, each with probability
$\frac13\,$. The random variable $V$ takes the values $+1$ and $-1$ as follows:
\begin{center}

\begin{tabular}{ll}
if $U=1\,$,&then $\P(V=1)= \frac13$ and $\P(V=-1)=\frac23\,$;\\[2mm]
if $U=0\,$,&then $\P(V=1)= \frac12$ and $\P(V=-1)=\frac12\,$;\\[2mm]
if $U=-1\,$,&then $\P(V=1)= \frac23$ and $\P(V=-1)=\frac13\,$.
\end{tabular}

\end{center}


\begin{questionparts}
\item Show that  the probability that
 both roots of the equation $x^2+Ux+V=0$ are real is $\frac12\;$.

\item Find the expected value of the larger root of the equation 
 $x^2+Ux+V=0\,$, given that both roots  are real.

\item
Find the probability that the roots of the equation
$$x^3+(U-2V)x^2+(1-2UV)x + U=0$$ are all positive.

\end{questionparts}
\end{question}

%%%%%%%%%% Q14
\begin{question}
In order to get money from a cash dispenser 
I have to punch in an
identification  number.  I have forgotten my  identification number, 
but I do know that it is equally likely to be any one of the 
integers $1$, $2$, \ldots , $n$.
I plan to punch in integers  in order until I get the right 
one. I can do this at the rate of $r$ integers  per minute. 
As soon as I punch in the first  wrong number, the police will be alerted. 
The probability that  they will arrive within  a time $t$ minutes 
is $1-\e^{-\lambda t}$, where $\lambda$ is a  positive constant.
If I follow my plan, show that the probability of the police arriving
before I get my money is
\[
\sum_{k=1}^n \frac{1-\e^{-\lambda(k-1)/r}}n\;.
\]
Simplify the sum.

On past experience, I know that I will  be so flustered that I will 
just punch in possible integers at random, without noticing which  I have 
already tried. Show that the probability of the police arriving before
I get my money is
\[
1-\frac1{n-(n-1)\e^{-\lambda/r}} \;.
\]
\end{question}
	
\end{document}
