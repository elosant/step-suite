\documentclass[a4, 11pt]{report}


\pagestyle{myheadings}
\markboth{}{Paper III, 2009
\ \ \ \ \ 
\today 
}               

\RequirePackage{amssymb}
\RequirePackage{amsmath}
\RequirePackage{graphicx}
\RequirePackage{color}
\RequirePackage[flushleft]{paralist}[2013/06/09]



\RequirePackage{geometry}
\geometry{%
  a4paper,
  lmargin=2cm,
  rmargin=2.5cm,
  tmargin=3.5cm,
  bmargin=2.5cm,
  footskip=12pt,
  headheight=24pt}


\newcommand{\comment}[1]{{\bf Comment} {\it #1}}
%\renewcommand{\comment}[1]{}

\newcommand{\bluecomment}[1]{{\color{blue}#1}}
%\renewcommand{\comment}[1]{}
\newcommand{\redcomment}[1]{{\color{red}#1}}



\usepackage{epsfig}
\usepackage{pstricks-add}
\usepackage{tgheros} %% changes sans-serif font to TeX Gyre Heros (tex-gyre)
\renewcommand{\familydefault}{\sfdefault} %% changes font to sans-serif
%\usepackage{sfmath}  %%%% this makes equation sans-serif
%\input RexFigs


\setlength{\parskip}{10pt}
\setlength{\parindent}{0pt}

\newlength{\qspace}
\setlength{\qspace}{20pt}


\newcounter{qnumber}
\setcounter{qnumber}{0}

\newenvironment{question}%
 {\vspace{\qspace}
  \begin{enumerate}[\bfseries 1\quad][10]%
    \setcounter{enumi}{\value{qnumber}}%
    \item%
 }
{
  \end{enumerate}
  \filbreak
  \stepcounter{qnumber}
 }


\newenvironment{questionparts}[1][1]%
 {
  \begin{enumerate}[\bfseries (i)]%
    \setcounter{enumii}{#1}
    \addtocounter{enumii}{-1}
    \setlength{\itemsep}{5mm}
    \setlength{\parskip}{8pt}
 }
 {
  \end{enumerate}
 }



\DeclareMathOperator{\cosec}{cosec}
\DeclareMathOperator{\Var}{Var}

\def\d{{\mathrm d}}
\def\e{{\mathrm e}}
\def\g{{\mathrm g}}
\def\h{{\mathrm h}}
\def\i{{\mathrm i}}
\def\f{{\mathrm f}}
\def\p{{\mathrm p}}
\def\q{{\mathrm q}}
\def\s{{\mathrm s}}
\def\t{{\mathrm t}}


\def\A{{\mathrm A}}
\def\B{{\mathrm B}}
\def\E{{\mathrm E}}
\def\F{{\mathrm F}}
\def\G{{\mathrm G}}
\def\H{{\mathrm H}}
\def\P{{\mathrm P}}


\def\bb{\mathbf b}
\def \bc{\mathbf c}
\def\bx {\mathbf x}
\def\bn {\mathbf n}

\newcommand{\low}{^{\vphantom{()}}}
%%%%% to lower suffices: $X\low_1$ etc


\newcommand{\subone}{ {\vphantom{\dot A}1}}
\newcommand{\subtwo}{ {\vphantom{\dot A}2}}




\def\le{\leqslant}
\def\ge{\geqslant}
\def\arcosh{{\rm arcosh}\,}


\def\var{{\rm Var}\,}

\newcommand{\ds}{\displaystyle}
\newcommand{\ts}{\textstyle}
\def\half{{\textstyle \frac12}}
\def\l{\left(}
\def\r{\right)}



\begin{document}
\setcounter{page}{2}

 
\section*{Section A: \ \ \ Pure Mathematics}

%%%%%%%%%%Q1
\begin{question}
The points $S$, $T$, $U$ and $V$ have coordinates
$(s,ms)$, $(t,mt)$, $(u,nu)$ and $(v,nv)$, respectively.
The lines $SV$ and $UT$ meet the line $y=0$ at the points
with coordinates $(p,0)$ and $(q,0)$, respectively.
Show that 
\[
p = \frac{(m-n)sv}{ms-nv}\,,
\]
and write down a similar expression for $q$. 

Given that $S$ and $T$ lie on the
circle $x^2 + (y-c)^2 = r^2$, find a quadratic equation
 satisfied by $s$ and by $t$, and hence determine $st$ and $s+t$ in 
 terms of $m$, $c$ and $r$.

 Given that $S$, $T$, $U$ and $V$ lie on the above circle, show that 
 $p+q=0$.
\end{question}

%%%%%%%%%%Q2
\begin{question}
\begin{questionparts}
\item
Let $\displaystyle y= \sum_{n=0}^\infty a_n x^n\,$, where
the coefficients $a_n$ are independent of $x$ and are such
that this series and all others in this question converge.
Show that 
\[
 \displaystyle y'= \sum_{n=1}^\infty na_n x^{n-1}\,,
\]
and write down a similar expression for $y''$.

Write out explicitly  each of the three
series as far as the term containing $a_3$.

\item
It is given that $y$ satisfies the differential
equation
\[
xy''-y'+4x^3y =0\,.
\]
By substituting the series of part (i) into the differential
equation and comparing coefficients, show that $a_1=0$. 

 Show that, for $n\ge4$, 
\[ a_n =- \frac{4}{n(n-2)}\, a_{n-4}\,,
\]
and  that, if $a_0=1$ and $a_2=0$, then $ y=\cos (x^2)\,$.

Find the corresponding result when $a_0=0$ and $a_2=1$.
\end{questionparts}
\end{question}

%%%%%%%%% Q3
\begin{question}
The function $\f(t)$ is defined, for $t\ne0$, by 
\[
\f(t)  = \frac t {\e^t-1}\,.
\]
 
 \begin{questionparts}

\item 
By expanding $\e^t$, show that 
 $\displaystyle \lim _{t\to0} \f(t) = 1\,$.
 Find $\f'(t)$ and evaluate
 $\displaystyle  \lim _{t\to0} \f'(t)\,$.

\item
 Show that $\f(t) +\frac12 t$ is an even function.
[{\bf Note:} A function $\g(t)$ is said to be {\em even}
if $\g(t) \equiv  \g(-t)$.]

\item 
Show with the aid of a sketch that $ \e^t(  1-t)\le 1\,$
and deduce that $\f'(t)\ne 0$ for $t\ne0$.

\end{questionparts}

Sketch the graph of $\f(t)$.
\end{question}

%%%%%% Q4 
\begin{question}
For any given (suitable) function $\f$, the {\sl Laplace transform}
of $\f$ is the function $\F$ defined by
\[
\F(s) = \int_0^\infty \e^{-st}\f(t)\d t
\ \ \ \ \ \ \ \ \ \ (s>0)
\,.
\]

\begin{questionparts}
\item
Show that the Laplace transform of $\e^{-bt}\f(t)$, where $b>0$,
is $\F(s+b)$. 
\item Show that 
the Laplace transform of $\f(at)$, where $a>0$, is $a^{-1}\F(\frac s a)\,$.
\item   Show that the Laplace
transform of $\f'(t)$ is $s\F(s) -\f(0)\,$.
\item
In the case $\f(t)=\sin t$, show that $\F(s)= \dfrac 1 {s^2+1}\,$.
\end{questionparts}

Using only these four results, find the Laplace transform of 
$\e^{-pt}\cos{qt}\,$, where $p>0$ and $q>0$.
\end{question}

%%%%%%%%% Q5
\begin{question}
The numbers $x$, $y$ and $z$ satisfy
\begin{align*}
x+y+z&= 1\\
x^2+y^2+z^2&=2\\
x^3+y^3+z^3&=3\,.
\end{align*}

Show that

\[
yz+zx+xy=-\frac12 \,.\]

Show also that $x^2y+x^2z+y^2z+y^2x+z^2x+z^2y=-1\,$,
and hence that
 \[
 xyz=\frac16 \,.\]

Let $S_n=x^n+y^n+z^n\,$.
Use the above results to 
find numbers $a$, $b$ and $c$
such that the relation 
\[
S_{n+1}=aS_{n}+bS_{n-1}+cS_{n-2}\,,
\]
holds for all $n$.
	\end{question}
	
%%%%%%%%% Q6
\begin{question}
Show that $\big\vert \e^{\i\beta} -\e^{\i\alpha}\big\vert
= 2\sin\frac12 (\beta-\alpha)\,$ for $0<\alpha<\beta<2\pi\,$.
Hence show that 
\[
\big\vert \e^{\i\alpha} -\e^{\i\beta}\big\vert
\;
\big\vert \e^{\i\gamma} -\e^{\i\delta}\big\vert
+
\big\vert \e^{\i\beta} -\e^{\i\gamma}\big\vert
\;
\big\vert \e^{\i\alpha} -\e^{\i\delta}\big\vert
=
\big\vert \e^{\i\alpha} -\e^{\i\gamma}\big\vert
\;
\big\vert \e^{\i\beta} -\e^{\i\delta}\big\vert
\,,
\]
where $0<\alpha<\beta<\gamma<\delta<2\pi$.

Interpret this result as a theorem about  cyclic quadrilaterals.
\end{question}
	
%%%%%%%%% Q7
\begin{question}
\begin{questionparts}
\item 
 The functions
$\f_n(x)$  are defined for $n=0$, $1$, $2$, $\ldots$\, , by
\[
\f_0(x) = \frac 1 {1+x^2}\, 
\qquad \text{and}\qquad
\f_{n+1}(x) =\frac{\d \f_n(x)}{\d x}\,.
\]
Prove, for $n\ge1$,   that 
\[
(1+x^2)\f_{n+1}(x) + 2(n+1)x\f_n(x) + n(n+1)\f_{n-1}(x)=0\,.
\]

\item 
The functions  $\P_n(x)$ are defined for $n=0$, $1$, $2$, $\ldots$\, , by
\[
\P_n(x) = (1+x^2)^{n+1}\f_n(x)\,.
\]
Find expressions for $\P_0(x)$,  $\P_1(x)$ and   $\P_2(x)$.



Prove, for $n\ge0$,  that 
\[
\P_{n+1}(x) -(1+x^2)\frac {\d \P_n(x)}{\d x}+ 2(n+1)x \P_n(x)=0\,,
\]
and   that $\P_n(x)$ is a polynomial of degree $n$.

\end{questionparts}
\end{question}
		
%%%%%%%%% Q8
\begin{question}
Let $m$ be a positive integer and let $n$ be a non-negative integer.                      
\begin{questionparts}
\item
Use the result  $\displaystyle \lim_{t\to\infty}\e^{-mt} t^n=0$ to show that
\[
\lim_{x\to0} x^m (\ln x)^n =0\,.
\]

By writing $x^x$ as  $\e^{x\ln x}$   show that
\[
\lim _{x\to0} x^x=1\,.
\]

\item Let $\displaystyle I_{n} = \int_0^1 x^m (\ln x)^n \d x\,$.
Show that 
\[
I_{n+1} = - \frac {n+1}{m+1} I_{n}
\]
and hence evaluate $I_{n}$.

\item Show that 
\[
\int_0^1 x^x \d x = 1 -\left(\tfrac12\right)^2 +\left(\tfrac13\right)^3
-\left(\tfrac14\right)^4 + \cdots \,.
\]
\end{questionparts}
\end{question}	
		

		
	
\newpage
\section*{Section B: \ \ \ Mechanics}


	
%%%%%%%%%% Q9
\begin{question}
A particle is projected under gravity from a point $P$ and passes
through a point $Q$. The angles of the trajectory with the positive
horizontal direction at $P$ and at $Q$ are $\theta$ and $\phi$,
respectively. The angle of elevation   of $Q$ from $P$ is $\alpha$.

\begin{questionparts}
\item Show that $\tan\theta +\tan\phi = 2\tan\alpha$.

\item It is given that there is a second trajectory from $P$ to $Q$
with the same speed of projection.
The angles of this trajectory with the positive
 horizontal direction at $P$ and at $Q$ are $\theta'$ and $\phi'$,
 respectively.
By considering a quadratic
equation satisfied by $\tan\theta$,
show that  $\tan(\theta+\theta') = -\cot\alpha$.
Show also that $\theta+\theta'=\pi+\phi+\phi'\,$.
\end{questionparts}
	\end{question}
	
%%%%%%%%%% Q10 
\begin{question}
A light spring is fixed at its lower end and its axis is
vertical. When  a certain particle $P$ rests on the top of the 
spring, the compression is $d$. When, instead, $P$ is dropped onto the 
top of the spring from a
height $h$ above it, the compression at time $t$ after $P$ hits the
top of the spring is $x$. Obtain    a second-order differential
equation relating $x$ and $t$ for $0\le t \le T$, where $T$ is the time
at which $P$ first loses contact with the spring.

Find        the  solution of this equation in    the form
\[
x= A +  B\cos (\omega t) + C\sin(\omega t)\,,
\]
where the constants $A$, $B$, $C$ and $\omega$ are to be given in terms
of $d$, $g$ and $h$ as appropriate.  

Show that 
\[
T 
= \sqrt{d/g\;} \left (2 \pi - 2 \arctan
\sqrt{2h/d\;}\;\right)\,.
\]
\end{question}

%%%%%%%%%% Q11

\begin{question}
A comet in deep space picks up mass as it travels through a
large stationary dust cloud. 
It is subject to a gravitational force of magnitude
$M\!f$ acting in the direction of its motion.
When  it entered the 
cloud, the comet had mass  $M$ and  speed  $V$. 
After  a time $t$,
it has travelled a distance $x$ through the cloud, 
its mass is $M(1+bx)$, where~$b$ is a positive constant, and  its speed
is $v$. 

\begin{questionparts}
\item In the case when $f=0$, 
write down an  equation relating 
$V$, $x$, $v$ and $b$.
Hence find an expression for $x$ in  terms of $b$, $V$ and $t$.

\item  In the case when $f$ is a non-zero constant,
use Newton's second law in the form
\[
\text{force} = \text{rate of change of momentum}
\]
to show that
\[
v = \frac{ft+V}{1+bx}\,.
\]
Hence find an expression for $x$ in  terms of $b$, $V$, $f$ and $t$.

Show that it is possible, if $b$, $V$ and $f$ are suitably chosen,
for the comet to move with constant speed.  Show also 
that, 
if the comet does not
move with constant speed, its speed tends to  a constant as $t\to\infty$. 
\end{questionparts}
\end{question}
	

	
	\newpage
\section*{Section C: \ \ \ Probability and Statistics}


%%%%%%%%%% Q12
\begin{question}
\begin{questionparts}
\item
 Albert tosses a fair coin $k$ times, where $k$ is a given 
positive integer. The number of heads he gets is $X_1$. He
then tosses the coin $X_1$ times, getting $X_2$ heads.
He then tosses the coin $X_2$ times, getting $X_3$ heads. 
The random variables $X_4$, $X_5$, $\ldots$ are defined similarly.
Write down $\E(X_1)$. 

By considering $\E(X_2 \; \big\vert \; X_1 = x_1)$, 
or otherwise, show that $\E(X_2) = \frac14 k$. 

Find $\displaystyle \sum_{i=1}^\infty \E(X_i)$.


\item Bertha has $k$ fair coins. She tosses the first coin 
until she gets a tail. The number of heads she gets before
the first tail is $Y_1$. She then tosses the second coin
until she gets a tail and the number of heads she gets with this coin
before
the first tail is $Y_2$. The random variables $Y_3$, $Y_4$,
$\ldots\;$, $Y_k$ are defined similarly, and 
 $Y= \sum\limits_{i=1}^k Y_i\,$. 

Obtain  the probability generating function of $Y$, and
use it to find $\E(Y)$, $\var(Y)$ and $\P(Y=r)$.

\end{questionparts}
\end{question}

%%%%%%%%%% Q13
\begin{question}
\begin{questionparts}
\item The point $P$ lies on the circumference of a circle of unit radius 
and centre $O$. The angle,~$\theta$,  between $OP$ and the 
positive $x$-axis is a random variable, uniformly 
distributed on  the interval  $0\le\theta<2\pi$.
The cartesian coordinates of $P$ with respect to $O$ are $(X, Y)$.
Find the probability density function for $X$, and calculate $\var (X)$.

Show that $X$ and $Y$ are uncorrelated and discuss briefly whether they
are independent.

\item The points $P_i$ ($i=1$, $2$, $\ldots$ , $n$)
are chosen independently on the circumference of the 
circle, as in part (i), and have cartesian coordinates $(X_i, Y_i)$. 
The point $\overline P$ has coordinates $(\overline X, \overline Y)$,
where 
$\overline X =\dfrac1n  \sum\limits _{i=1}^n X_i$
and
$\overline Y =\dfrac1n  \sum\limits _{i=1}^n Y_i$.
Show that $\overline X$ and $\overline Y$ 
are uncorrelated.

Show that, for large $n$, 
$\displaystyle \P\left(\vert \overline X \vert \le \sqrt{\frac2n}\right)
\approx 0.95\,$.

\end{questionparts}
\end{question}

\end{document}
