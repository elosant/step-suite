\documentclass[a4, 11pt]{report}


\pagestyle{myheadings}
\markboth{}{Paper II, 1997
\ \ \ \ \ 
\today 
}               

\RequirePackage{amssymb}
\RequirePackage{amsmath}
\RequirePackage{graphicx}
\RequirePackage{color}
\RequirePackage[flushleft]{paralist}[2013/06/09]



\RequirePackage{geometry}
\geometry{%
  a4paper,
  lmargin=2cm,
  rmargin=2.5cm,
  tmargin=3.5cm,
  bmargin=2.5cm,
  footskip=12pt,
  headheight=24pt}


\newcommand{\comment}[1]{{\bf Comment} {\it #1}}
%\renewcommand{\comment}[1]{}

\newcommand{\bluecomment}[1]{{\color{blue}#1}}
%\renewcommand{\comment}[1]{}
\newcommand{\redcomment}[1]{{\color{red}#1}}



\usepackage{epsfig}
\usepackage{pstricks-add}
\usepackage{tgheros} %% changes sans-serif font to TeX Gyre Heros (tex-gyre)
\renewcommand{\familydefault}{\sfdefault} %% changes font to sans-serif
%\usepackage{sfmath}  %%%% this makes equation sans-serif
%\input RexFigs


\setlength{\parskip}{10pt}
\setlength{\parindent}{0pt}

\newlength{\qspace}
\setlength{\qspace}{20pt}


\newcounter{qnumber}
\setcounter{qnumber}{0}

\newenvironment{question}%
 {\vspace{\qspace}
  \begin{enumerate}[\bfseries 1\quad][10]%
    \setcounter{enumi}{\value{qnumber}}%
    \item%
 }
{
  \end{enumerate}
  \filbreak
  \stepcounter{qnumber}
 }


\newenvironment{questionparts}[1][1]%
 {
  \begin{enumerate}[\bfseries (i)]%
    \setcounter{enumii}{#1}
    \addtocounter{enumii}{-1}
    \setlength{\itemsep}{5mm}
    \setlength{\parskip}{8pt}
 }
 {
  \end{enumerate}
 }



\DeclareMathOperator{\cosec}{cosec}
\DeclareMathOperator{\Var}{Var}

\def\d{{\rm d}}
\def\e{{\rm e}}
\def\g{{\rm g}}
\def\h{{\rm h}}
\def\f{{\rm f}}
\def\p{{\rm p}}
\def\s{{\rm s}}
\def\t{{\rm t}}


\def\A{{\rm A}}
\def\B{{\rm B}}
\def\E{{\rm E}}
\def\F{{\rm F}}
\def\G{{\rm G}}
\def\H{{\rm H}}
\def\P{{\rm P}}


\def\bb{\mathbf b}
\def \bc{\mathbf c}
\def\bx {\mathbf x}
\def\bn {\mathbf n}

\newcommand{\low}{^{\vphantom{()}}}
%%%%% to lower suffices: $X\low_1$ etc


\newcommand{\subone}{ {\vphantom{\dot A}1}}
\newcommand{\subtwo}{ {\vphantom{\dot A}2}}




\def\le{\leqslant}
\def\ge{\geqslant}


\def\var{{\rm Var}\,}

\newcommand{\ds}{\displaystyle}
\newcommand{\ts}{\textstyle}




\begin{document}
\setcounter{page}{2}

 
\section*{Section A: \ \ \ Pure Mathematics}

%%%%%%%%%%Q1
\begin{question}
Find the sum of those numbers between 1000 and 6000 every one of whose digits
is one of the numbers $0,\,2,\,5$ or 7, giving
 your answer as a product of primes.
\end{question}

%%%%%%%%%%Q2
\begin{question}
Suppose that
$$3=\frac{2}{ x_1}=x_1+\frac{2}{ x_2}
=x_2+\frac{2}{ x_3}=x_3+\frac{2}{ x_4}=\cdots.$$
Guess an expression, in terms of $n$, for $x_n$.
Then, by induction or otherwise,
prove the correctness of your guess.
\end{question}

%%%%%%%%% Q3
\begin{question}
Find constants $a,\,b,\,c$ and $d$ such that
$$\frac{ax+b}{ x^2+2x+2}+\frac{cx+d}{ x^2-2x+2}=
\frac{1}{ x^4+4}.$$

\noindent
Show that
$$\int_0^1\frac {\d x}{ x^4+4}\;= \frac{1}{16} \ln 5
 +\frac{1}{8} \tan^{-1}2 .$$
\end{question}

%%%%%% Q4 
\begin{question}
Show that, when the polynomial
${\rm p} (x)$ is divided by $(x-a)$, 
where $a$ is a real number, the remainder is
${\rm p}(a)$.

\begin{questionparts}
\item When the polynomial ${\rm p}(x)$ 
is divided by $x-1,\,x-2,\,x-3$ the
remainders are
3,1,5 respectively. Given that 
$${\rm p}(x)=(x-1)(x-2)(x-3){\rm
q}(x)+{\rm r} (x),$$
 where ${\rm q}(x)$
and ${\rm r}(x)$ are polynomials with ${\rm r}(x)$ 
having degree less than three,
find~${\rm r}(x)$.

\item Find a polynomial ${\rm P}(x)$ of degree $n+1$,
 where $n$ is a  given positive
integer, such that for each integer $a$ satisfying $0\le a\le n$,
the remainder when ${\rm P}_n(x)$ is divided by $x-a$ is $a$. 
\end{questionparts}
	\end{question}

%%%%%%%%% Q5
\begin{question}
The complex numbers $w=u+\mathrm{i}v$ and $z=x+\mathrm{i}y$ are related by the
equation
$$z=
(\cos v+\mathrm{i}\sin v)\mathrm{e}^u.$$
Find all $w$ which correspond to $z=\mathrm{i\,e}$.

Find the loci in the $x$--$y$ plane corresponding to the lines $u=$ constant
in the $u$--$v$ plane. Find also the loci corresponding to the lines $v=$
constant. Illustrate your answers with clearly labelled sketches.

Identify two subsets $W_1$ and $W_2$ of the $u$--$v$ plane each of
which is in one-to-one correspondence with the first quadrant
$\{(x,\,y):\,x>0,\,y>0\}$ of the $x$--$y$ plane.
Identify also two subsets $W_3$ and $W_4$ each of which
is in one-to-one correspondence with the set $\{z\,:0<\,\vert z\vert\,<1\}$.

\noindent[{\bf NB} `one-to-one' means here that to each value of 
$w$ there is only one corresponding value of $z$, and vice-versa.]
	\end{question}
	
%%%%%%%%% Q6
\begin{question}
Show that, if $\,\tan^2\phi=2\tan\phi+1$, then $\tan2\phi=-1$.

Find all solutions of the equation
$$\tan\theta=2+\tan3\theta$$
which satisfy $0<\theta< 2\pi$,
expressing your answers as rational multiples of $\pi$. 

Find all solutions of the equation
the equation
$$\cot\theta=2+\cot3\theta$$
which satisfy $$-\frac{3\pi}{2}<\theta<\frac{\pi}{2}.$$
\end{question}
	
%%%%%%%%% Q7
\begin{question}
Let
$$y^2=x^2(a^2-x^2),$$
where $a$ is a real constant.
Find, in terms of $a$, the maximum and minimum values of $y$.


Sketch carefully on the same axes the graphs of $y$
in the cases $a=1$ and $a=2$.
\end{question}
		
%%%%%%%%% Q8
\begin{question}	
If   ${\rm f}(t)\ge {\rm g}(t)$ for $a\le t\le b$, explain very
briefly why $\int_a^b {\rm f}(t) \d t \ge \int_a^b {\rm g}(t) \d t$.

Prove that if $p>q>0$ and $x\ge1$ then
$$\frac{x^p-1}{ p}\ge\frac{x^q-1}{ q}.$$
Show that this inequality also holds when $p>q>0$ and $0\le x\le1$.

Prove that, if $p>q>0$ and $x\ge0$, then
$$\frac{1}{ p}\left(\frac{x^p}{ p+1}-1\right)\ge
\frac{1}{q}\left(\frac{x^q}{ q+1}-1\right).$$
\end{question}	
		

		
	
\newpage
\section*{Section B: \ \ \ Mechanics}


	
%%%%%%%%%% Q9
\begin{question}
A uniform solid sphere of diameter $d$ and mass $m$ is drawn
slowly and without slipping from horizontal ground onto a step of
height $d/4$ by a horizontal 
force which is always applied to the highest point of the sphere
and is always perpendicular to the vertical plane which forms
the face of the step. Find the maximum horizontal force throughout
the movement, and prove that the coefficient of friction
between the sphere and the edge of the step must exceed
$1/\sqrt{3}$.
	\end{question}
	
%%%%%%%%%% Q10
\begin{question}	
\noindent{\it In this question the effect of gravity is to be neglected.}

A small body of mass $M$ is moving with velocity $v$ along the axis of
a long, smooth, fixed, circular cylinder of radius $L$. An internal
explosion splits the body into two spherical fragments, with masses
$qM$ and $(1-q)M$, where $q\le\frac{1}{2}$. After bouncing perfectly
elastically
off the cylinder (one bounce each) the fragments collide and coalesce
at a point $\frac{1}{2}L$ from the axis. Show that $q=\frac{3}{ 8}$. 

The collision occurs at a time $5L/v$ after the explosion. Find the
energy imparted to the fragments by the explosion, and find the
velocity after coalescence. 
\end{question}

%%%%%%%%%% Q11

\begin{question}
A tennis player serves from height $H$ above horizontal  ground, hitting
the ball downwards with speed $v$ at an angle $\alpha$ below the 
horizontal. The ball just clears the net of height $h$ at horizontal
distance $a$ from the server and hits the ground a further horizontal
distance $b$ beyond the net. Show that
$$
v^2 = \frac{ g(a+b)^2(1+\tan^2\alpha)}{ 2[H-(a+b)\tan\alpha]}
$$
and
$$
\tan\alpha = \frac{2a+b }{ a(a+b)}H - \frac{a+b }{ ab}h \,.
$$
By considering the signs of $v^2$ and $\tan\alpha$, find upper
and lower bounds on $H$ for such a serve to be possible.
\end{question}
	

	
	\newpage
\section*{Section C: \ \ \ Probability and Statistics}


%%%%%%%%%% Q12
\begin{question}
The game of Cambridge Whispers starts with the first participant Albert
flipping an
un-biased coin and whispering to his neighbour Bertha
whether it fell `heads' or `tails'. Bertha then whispers this information to
her neighbour, and so on. The game ends when the final player Zebedee whispers
to Albert and the game is won, by all players, if what Albert hears is
correct. The acoustics are such that the listeners have,
independently at each stage, only
a probability of 2/3 of hearing correctly what is said. Find the probability
that the game is won when there are just three players.

By considering the binomial expansion of $(a+b)^n+(a-b)^n$, or otherwise,
find a concise expression for the probability $P$ that the game is won
when is it played by $n$ players each having a probability $p$ of hearing
correctly.
% Show in particular that, if $n$ is even,
%$P(n,1/10) = P(n,9/10)$.% How do you explain this apparent anomaly?

To avoid the trauma of a lost game, the rules are now modified to require
Albert to whisper to Bertha what he
hears from Zebedee, and so keep the game going, if what he hears
from Zebedee is not correct. Find the expected total number of times that
Albert whispers to Bertha before the modified game ends.

\noindent
[You may use without proof the fact that $\sum_1^\infty kx^{k-1}=(1-x)^{-2}$
for $\vert x\vert<1$.]
\end{question}

%%%%%%%%%% Q13
\begin{question}
A needle of length two cm is dropped at random onto a large piece of paper
ruled with parallel lines two cm apart.

\begin{questionparts}
	\item By considering the angle
which the needle makes with the lines, find the probability that the needle
crosses the nearest line given that its centre is $x$ cm from it, where
$0<x<1$.

\item Given that the centre of the needle is $x$ cm from the nearest
line and that the needle crosses that line, find the cumulative
distribution function for the length of the
shorter segment of the needle cut off by the line.

\item Find the probability that the needle misses all the lines.
\end{questionparts}
\end{question}

%%%%%%%%%% Q14
\begin{question}
Traffic enters a tunnel which is 9600 metres long, and in which overtaking
is impossible. The number of vehicles 
which enter in any given time is governed by the Poisson distribution with 
mean 6 cars per minute. All vehicles travel
at a constant speed until forced to slow down on catching up with a
slower vehicle ahead. I enter the tunnel travelling at 30
m$\,$s$^{-1}$
and all the other traffic is travelling at 32 m$\,$s$^{-1}$. What is
the expected number of vehicles in
the queue behind me when I leave the tunnel? 


Assuming again that I travel at 30 m$\,$s$^{-1}$, but that all the other
vehicles are independently equally likely to be travelling at 30 m$\,$s$^{-1}$
or 32 m$\,$s$^{-1}$, find the probability that 
 exactly two
vehicles enter the tunnel within 20 seconds of my doing so
and catch me up before I leave it.
Find also the probability that there are exactly two vehicles
queuing behind me when I leave the tunnel.


\noindent [Ignore the lengths of the vehicles.]
\end{question}
	
\end{document}
