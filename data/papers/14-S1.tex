\documentclass[a4, 11pt]{report}


\pagestyle{myheadings}
\markboth{}{Paper I, 2014
\ \ \ \ \ 
\today 
}               

\RequirePackage{amssymb}
\RequirePackage{amsmath}
\RequirePackage{graphicx}
\RequirePackage{color}
\RequirePackage[flushleft]{paralist}[2013/06/09]
\RequirePackage{asymptote}



\RequirePackage{geometry}
\geometry{%
  a4paper,
  lmargin=2cm,
  rmargin=2.5cm,
  tmargin=3.5cm,
  bmargin=2.5cm,
  footskip=12pt,
  headheight=24pt}
 

\newcommand{\comment}[1]{{\bf Comment} {\it #1}}
%\renewcommand{\comment}[1]{}

\newcommand{\bluecomment}[1]{{\color{blue}#1}}
%\renewcommand{\comment}[1]{}
\newcommand{\redcomment}[1]{{\color{red}#1}}



\usepackage{epsfig}
\usepackage{pstricks-add}
\usepackage{tgheros} %% changes sans-serif font to TeX Gyre Heros (tex-gyre)
\renewcommand{\familydefault}{\sfdefault} %% changes font to sans-serif
%\usepackage{sfmath}  %%%% this makes equation sans-serif
%\input RexFigs


\setlength{\parskip}{10pt}
\setlength{\parindent}{0pt}

\newlength{\qspace}
\setlength{\qspace}{20pt}


\newcounter{qnumber}
\setcounter{qnumber}{0}

\newenvironment{question}%
 {\vspace{\qspace}
  \begin{enumerate}[\bfseries 1\quad][10]%
    \setcounter{enumi}{\value{qnumber}}%
    \item%
 }
{
  \end{enumerate}
  \filbreak
  \stepcounter{qnumber}
 }


\newenvironment{questionparts}[1][1]%
 {
  \begin{enumerate}[\bfseries (i)]%
    \setcounter{enumii}{#1}
    \addtocounter{enumii}{-1}
    \setlength{\itemsep}{5mm}
    \setlength{\parskip}{8pt}
 }
 {
  \end{enumerate}
 }



\DeclareMathOperator{\cosec}{cosec}
\DeclareMathOperator{\Var}{Var}

\def\d{{\mathrm d}}
\def\e{{\mathrm e}}
\def\g{{\mathrm g}}
\def\h{{\mathrm h}}
\def\f{{\mathrm f}}
\def\p{{\mathrm p}}
\def\s{{\mathrm s}}
\def\t{{\mathrm t}}
\def\i{{\mathrm i}}

\def\A{{\mathrm A}}
\def\B{{\mathrm B}}
\def\E{{\mathrm E}}
\def\F{{\mathrm F}}
\def\G{{\mathrm G}}
\def\H{{\mathrm H}}
\def\P{{\mathrm P}}


\def\bb{\mathbf b}
\def \bc{\mathbf c}
\def\bx {\mathbf x}
\def\bn {\mathbf n}

\newcommand{\low}{^{\vphantom{()}}}
%%%%% to lower suffices: $X\low_1$ etc


\newcommand{\subone}{ {\vphantom{\dot A}1}}
\newcommand{\subtwo}{ {\vphantom{\dot A}2}}

\begin{asydef}
  import markers;
  import geometry;
  import graph;
  usepackage("amsmath");
\end{asydef}


\def\le{\leqslant}
\def\ge{\geqslant}
\def\arcosh{{\rm arcosh}\,}


\def\var{{\rm Var}\,}

\newcommand{\ds}{\displaystyle}
\newcommand{\ts}{\textstyle}
\def\half{{\textstyle \frac12}}
\def\l{\left(}
\def\r{\right)}
\renewcommand{\.}[1]{\ensuremath{\mathrm{#1}}}
\newcommand{\+}[1]{\ensuremath{\mathbf{#1}}}
\newcommand{\ud}{\mathop{}\!\mathrm{d}}



\begin{document}
\setcounter{page}{2}

 
\section*{Section A: \ \ \ Pure Mathematics}

%%%%%%%%%%Q1
\begin{question}
 \emph{All numbers referred to in this question are non-negative
    integers.}

\begin{questionparts}
\item Express each of the numbers 3, 5, 8, 12 and 16 as the
    difference of two non-zero squares.
\item Prove that any odd number can be written as the difference of
    two squares.
\item Prove that all numbers of the form $4k$, where $k$ is a non-negative
integer, can be written as the
    difference of two squares.
\item Prove that no number of the form $4k+2$, where $k$ is a non-negative
integer,  can be written as the
    difference of two squares. 
	
\item Prove that any number of the form $pq$, where $p$ and $q$ are prime
numbers greater than~2, can be written as the difference of two
squares in exactly two distinct ways. Does this result hold if 
$p$ is a prime greater than 2 and $q=2$?

\item Determine the number of  distinct ways  in which 
675 can be written as the difference of 
two squares.

\end{questionparts}
\end{question}

%%%%%%%%%%Q2
\begin{question}
\begin{questionparts}
\item
Show that $\int \ln (2-x) \ud x = -(2-x)\ln (2-x) + (2-x) + c \,,\  $ 
where $x<2$.

\item
  Sketch the curve  $A$ given by
$y= \ln \vert x^2-4\vert$.

  \item Show that the area of the finite region enclosed by the 
positive $x$-axis, the
    $y$-axis and the curve $A$ is $4\ln(2+\sqrt3)-2\sqrt3\,$.

  \item The curve $B$ is given by 
$y= \big\vert \ln \vert x^2-4\vert \big\vert\,$.
Find the area between the curve $B$
and the $x$-axis with $| x| <2$. 
 
[\textit{Note:
    you may assume that $t \ln t \to 0$ as $t\to 0$.}]

  \end{questionparts}
\end{question}

%%%%%%%%% Q3
\begin{question}
 The numbers $a$ and $b$, where $b>a\ge0$,  are such that 
  \[
  \int_a^b x^2 \ud x = \left ( \int_a^b x \ud x\right)^{\!\!2}\,.
  \]

  \begin{questionparts}
  \item In the case $a=0$ and $b>0$, find the  value of~$b$.

  \item In the case $a=1$, show that $b$ satisfies
    \[
    3b^3 -b^2-7b -7 =0\,.
    \]
    Show further, with the help of a sketch, that there is only one
    (real) value of~$b$ that satisfies this equation and that it lies
    between $2$ and~$3$.

  \item Show that  $3p^2 + q^2 = 3p^2q$, 
    where $p=b+a$ and $q=b-a$, and express $p^2$ in terms of~$q$. 
   Deduce that $1<b-a\le\frac43$.

  \end{questionparts}
\end{question}

%%%%%% Q4 
\begin{question}
  An accurate clock has an hour hand of length $a$ and a minute hand
  of length $b$ (where $b>a$), both measured from the pivot at the
  centre of the clock face. Let $x$ be the  distance between the ends
  of the hands when the angle between the hands is $\theta$, where
$0\le\theta<\pi$. 

Show that the rate of increase of $x$ is  greatest
  when $x=(b^2-a^2)^\frac12$.

  In the case when $b=2a$ and the clock starts at mid-day (with both
  hands pointing vertically upwards), show that this occurs for the
  first time a little less than 11 minutes later.
\end{question}

%%%%%%%%% Q5
\begin{question}
  \begin{questionparts}
\item
Let $\.f(x) = (x+2a)^3 -27 a^2 x$, where $a\ge 0$. 
 By sketching  $\.f(x)$, show that $\.f(x)\ge 0$ for~$x \ge0$. 


    \item Use part (i) to 
find the greatest value of $xy^2$ in the region of the 
$x$-$y$ plane given by $x\ge0$, $y\ge0$ and $x+2y\le 3\,$.
For what values of $x$ and $y$ is this greatest value achieved?

\item Use part (i) to show that 
   $(p+q+r)^3 \ge 27pqr$
    for any non-negative numbers $p$, $q$ and~$r$.
 If 
   $(p+q+r)^3 =   27pqr$,
 what relationship must $p$, $q$ and $r$
    satisfy? 

  \end{questionparts}

\end{question}
	
%%%%%%%%% Q6
\begin{question}
  \begin{questionparts}
  \item The sequence of numbers $u_0, u_1, \ldots $ is given by
    $u_0=u$ and, for $n\ge 0$,
    \begin{equation}
      \label{eq:6*}
      u_{n+1} =4u_n(1- u_n)\,.
      \tag{$*$}
    \end{equation}
    In the case $u= \sin^2\theta$ for some given angle $\theta$, write
    down and simplify expressions for $u_1$ and $u_2$ in terms of
    $\theta$. Conjecture an expression for $u_n$ and prove your
    conjecture.

  \item The sequence of numbers $v_0, v_1, \ldots $ is given by $v_0=
    v$ and, for $n\ge 0$,
    \[
    v_{n+1} = -pv_n^2 +qv_n +r\,,
    \]
    where $p$, $q$ and $r$ are given numbers, with $p\ne0$. Show that
    a substitution of the form $v_n =\alpha u_n +\beta$, where
    $\alpha$ and $\beta$ are suitably chosen, results in the
    sequence~\eqref{eq:6*} provided that
    \[
    4pr = 8 +2q -q^2 \,.
    \]
    Hence obtain the sequence satisfying
    $v_0=1$ and, for $n\ge0$, $v_{n+1} = -v_n^2 +2 v_n +2 \,$.
  \end{questionparts}
\end{question}
	
%%%%%%%%% Q7
\begin{question}
  In the triangle $OAB$, the point $D$ divides the side $BO$ in the
  ratio $r:1$ (so that $BD = rDO$), and the point $E$ divides the
  side $OA$ in the ratio $s:1$ (so that $OE =s EA$), 
where $r$ and $s$ are both positive.  
  \begin{questionparts}
  \item 
The lines $AD$ and $BE$ intersect at
 $G$. 
Show that  
    \[
\+g=    \frac{rs}{1+r+rs} \, \+a + \frac 1 {1+r+rs} \, \+b \,,
    \]
    where \+a, \+b and \+g are the position vectors with respect to
$O$ of $A$, $B$ and $G$,
    respectively.                       
  \item 
The line through $G$ and~$O$ meets $AB$ at
 $F$.
Given that $F$ divides $AB$ in the ratio $t:1$, find an
    expression for~$t$ in terms of $r$ and~$s$. 
    
  \end{questionparts}  
\end{question}
		
%%%%%%%%% Q8
\begin{question}
  Let $L_a$ denote the line joining the points $(a,0)$ and $(0, 1-a)$,
  where $0<   a <   1$.  The line~$L_b$ is defined similarly.
  \begin{questionparts}
  \item Determine the point of intersection of $L_a$ and $L_b$, where
    $a\ne b$.
  \item Show that this point of intersection, in the limit as $b\to
    a$, lies on the curve~$C$ given by
    \[
    y=(1-\sqrt x)^2\, \ \ \ \ (0<   x <   1)\,.
    \]
  \item Show that every tangent
    to~$C$ is of the form~$L_a$ for some~$a$.
  \end{questionparts}
\end{question}	
		

		
	
\newpage
\section*{Section B: \ \ \ Mechanics}


	
%%%%%%%%%% Q9
\begin{question}

  A particle of mass $m$
  is projected due east at speed $U$ from a point on horizontal
  ground at an angle $\theta$ above the horizontal, where $0<\theta<
  90^\circ$.  In addition to the  gravitational force $mg$, it
  experiences a horizontal  force of magnitude $mkg$, where $k$ is a positive
constant, acting due west
  in the plane of motion of the particle.  Determine expressions in
  terms of $U$, $\theta$ and~$g$ for the time, $T_H$, at which the
  particle reaches its greatest height and the time, $T_L   $, 
at which it
  lands.

  Let $T = U\cos\theta /(kg)$.  By considering the relative
  magnitudes of $T_H$, $T_L   $ and $T$, or otherwise, sketch the
  trajectory of the particle in the cases
$k\tan\theta<\frac12$,  \ \ $\frac12 < k\tan\theta<1$, and $k\tan\theta>1$.
What happens when $k\tan\theta =1$?
	\end{question}
	
%%%%%%%%%% Q10 
\begin{question}	
  \begin{questionparts}
  \item A uniform spherical ball of mass $M$ and radius $R$ is
    released from rest with its centre a distance $H+R$ above
    horizontal ground.  The coefficient of restitution between the
    ball and the ground is $e$.  Show that, after bouncing, the centre
    of the ball reaches a height $R+He^2$ above the ground.
    
  \item A second uniform spherical ball, of mass $m$ and radius $r$,
    is now released from rest together with the first ball (whose
    centre is again a distance $H+R$ above the ground when it is
    released).  The two balls are initially one on top of the other, with
    the second ball (of mass~$m$) above the first.  The two balls
    separate slightly during their fall, with their centres remaining
    in the same vertical line, so that they collide immediately after the
    first ball has bounced on the ground. The coefficient of 
    restitution between the balls is also $e$.  The centre of the
    second ball attains a height $h$ above the ground.

    Given that $R=0.2$, $r=0.05$, $H=1.8$, $h=4.5$ and $e=\frac23$,
    determine the value of~$M/m$.
  \end{questionparts}
\end{question}

%%%%%%%%%% Q11

\begin{question}
  The diagrams below show two separate systems of particles, strings
  and pulleys.  
In both systems, 
the pulleys  are smooth and
  light, the strings are light and inextensible, the particles
  move vertically and   
the pulleys labelled  with $P$ are fixed.
The masses of the particles are as indicated on the diagrams.

  \begin{center}
\psset{xunit=0.7cm,yunit=0.7cm,algebraic=true,dimen=middle,dotstyle=o,dotsize=3pt 0,linewidth=0.3pt,arrowsize=3pt 2,arrowinset=0.25}
\begin{pspicture*}(-0.6,-5.31)(12.33,7.94)
\pspolygon(1,-3)(1,-2)(0,-2)(0,-3)
\pspolygon(4.01,-1)(4.01,0)(3.01,0)(3.01,-1)
\pspolygon(10.3,-1)(10.3,0)(9.3,0)(9.3,-1)
\pspolygon(11.76,-3)(11.76,-2)(10.76,-2)(10.76,-3)
\pspolygon(8,0)(8,1)(7,1)(7,0)
\pscircle(2,6){1.05}
\pscircle(9,6){1.05}
\psline(3.5,6)(3.49,0)
\psline(0.5,6.01)(0.5,-2)
\psline(1,-3)(1,-2)
\psline(1,-2)(0,-2)
\psline(0,-2)(0,-3)
\psline(0,-3)(1,-3)
\psline(4.01,-1)(4.01,0)
\psline(4.01,0)(3.01,0)
\psline(3.01,0)(3.01,-1)
\psline(3.01,-1)(4.01,-1)
\psline(7.5,5.99)(7.51,1.01)
\psline(10.5,6)(10.48,3.98)
\pscircle(10.5,3.25){0.51}
\psline(9.77,3.26)(9.77,-0.01)
\psline(11.23,3.26)(11.23,-2)
\psline(10.3,-1)(10.3,0)
\psline(10.3,0)(9.3,0)
\psline(9.3,0)(9.3,-1)
\psline(9.3,-1)(10.3,-1)
\psline(11.76,-3)(11.76,-2)
\psline(11.76,-2)(10.76,-2)
\psline(10.76,-2)(10.76,-3)
\psline(10.76,-3)(11.76,-3)
\psline(8,0)(8,1)
\psline(8,1)(7,1)
\psline(7,1)(7,0)
\psline(7,0)(8,0)
\rput[tl](0.16,-2.33){$M$}
\rput[tl](3.2,-0.4){$m$}
\rput[tl](7.18,0.67){$M$}
\rput[tl](9.46,-0.4){$m_1$}
\rput[tl](10.91,-2.4){$m_2$}
\rput[tl](1.82,6.14){$P$}
\rput[tl](8.84,6.14){$P$}
\rput[tl](10.22,3.48){$P_1$}
\rput[tl](1.06,-3.72){$\text{System I}$}
\rput[tl](8.11,-3.68){$\text{System II}$}
\end{pspicture*}
  \end{center}

  \begin{questionparts}
  \item For system I show that                              
    the acceleration,  $a_1$, of the particle of
    mass~$M$, measured in the downwards direction,  is  given by
\[
a_1= \frac{M-m}{M+m} \, g
\,,
\]
where $g$ is the acceleration due to gravity.
Give an expression for the  force on the pulley due to the tension
in the string.


  \item For system II
 show that the  acceleration, $a_2$, of the particle of
    mass $M$,  measured in the downwards direction, is given by
    \[
    a_2= \frac{ M - 4\mu}{M+4\mu}\,g \,,
    \]
where $\mu = \dfrac{m_1m_2}{m_1+m_2}$.

In the case $m= m_1+m_2$, show that $a_1=   a_2$ if and only if 
 $m_1=m_2$.
  \end{questionparts}
\end{question}
	

	
	\newpage
\section*{Section C: \ \ \ Probability and Statistics}


%%%%%%%%%% Q12
\begin{question}
  A game in a casino is played with a fair coin and an unbiased
  cubical die whose faces are labelled $1, 1, 1, 2, 2$ and $3.$  In each
  round of the game, the die is rolled once and the coin is tossed
  once. The outcome of the round is a random variable $X$. The value,
  $x$, of $X$ is determined as follows. If the result of the toss is
  heads then $x= \vert ks -1\vert$, and if the result of the toss is
  tails then $x=\vert k-s\vert$, where $s$ is the number on the die
  and $k$~is a given number.  Show that $\.E(X^2) = k +13(k-1)^2 /6$.

  Given that both $\.E(X^2)$ and $\.E(X)$ are positive integers, and
  that $k$ is a single-digit positive integer, determine the value
  of~$k$, and write down the probability distribution of~$X$.

  A gambler pays  $\pounds 1$ to play the game, which
  consists of two rounds. The gambler is paid:
\newline\hspace*{1cm}
 $\pounds w$, where $w$~is an integer, if 
the sum of the outcomes of the two rounds
  exceeds $25$;
\newline\hspace*{1cm}
 $\pounds 1$ if the sum of the outcomes equals $25$;
\newline\hspace*{1cm}
nothing if the sum of the outcomes is less that $25$.
\newline   
Find, in terms of~$w$, an expression
  for the amount the gambler expects to be paid in a game, and deduce
  the maximum possible value of~$w$, given that the casino's owners
 choose~$w$
  so that the game is in their favour.
\end{question}

%%%%%%%%%% Q13
\begin{question}
  A continuous random variable $X$ has a \emph{triangular}
  distribution, which means that it has a   probability density function 
  of the form
  \[
  \.f(x) =
  \begin{cases}
    \.g(x) & \text{for $a<   x \le c$}  \\
    \.h(x) & \text{for   $c\le x <   b$} \\
    0 & \text{otherwise,}
  \end{cases}
  \]
  where $\.g(x)$ is an increasing linear function with $\.g(a)=0$,
  $\.h(x)$ is a decreasing linear function with $\.h(b) =0$, and
  $\.g(c)=\.h(c)$.

  Show that $\.g(x) = \dfrac{2(x-a)}{(b-a)(c-a)}$ and find a similar
  expression for $\.h(x)$.

  \begin{questionparts}
  \item Show that the mean of the distribution is $\frac13(a+b+c)$.
  \item Find the median of the distribution in the different cases that arise.
  \end{questionparts}
\end{question}
\end{document}
