\documentclass[a4, 11pt]{report}


%\pagestyle{myheadings}
%\markboth{}{Paper I, 2017 final draft  
%\ \ \ \ \ \
%\today
%}               
\pagestyle{empty}




\RequirePackage{amssymb}
\RequirePackage{amsmath}
\RequirePackage{graphicx}
\RequirePackage{color}
\RequirePackage[flushleft]{paralist}[2013/06/09]
\usepackage[utf8]{inputenc}
\usepackage{pstricks-add}



\RequirePackage{geometry}
\geometry{%
  a4paper,
  lmargin=2cm,
  rmargin=2.5cm,
  tmargin=3.5cm,
  bmargin=2.5cm,
  footskip=12pt,
  headheight=24pt}


\newcommand{\comment}[1]{{\bf Comment} {\it #1}}
%\renewcommand{\comment}[1]{}

\newcommand{\bct}[1]{{\color{blue}#1}}
%\renewcommand{\comment}[1]{}
\newcommand{\rct}[1]{{\color{red}#1}}



\usepackage{epsfig}

%\input RexFigs


\setlength{\parskip}{10pt}
\setlength{\parindent}{0pt}

\newlength{\qspace}
\setlength{\qspace}{20pt}


\newcounter{qnumber}
\setcounter{qnumber}{0}

\newenvironment{question}%
 {\vspace{\qspace}
  \begin{enumerate}[\bfseries 1\quad][10]%
    \setcounter{enumi}{\value{qnumber}}%
    \item%
 }
{
  \end{enumerate}
  \filbreak
  \stepcounter{qnumber}
 }


\newenvironment{questionparts}[1][1]%
 {
  \begin{enumerate}[\bfseries (i)]%
    \setcounter{enumii}{#1}
    \addtocounter{enumii}{-1}
    \setlength{\itemsep}{5mm}
    \setlength{\parskip}{8pt}
 }
 {
  \end{enumerate}
 }



\DeclareMathOperator{\cosec}{cosec}
\DeclareMathOperator{\Var}{Var}

\def\d{{\rm d}}
\def\e{{\rm e}}
\def\g{{\rm g}}
\def\h{{\rm h}}
\def\f{{\rm f}}
\def\p{{\rm p}}
\def\s{{\rm s}}
\def\t{{\rm t}}


\def\A{{\rm A}}
\def\B{{\rm B}}
\def\E{{\rm E}}
\def\F{{\rm F}}
\def\G{{\rm G}}
\def\H{{\rm H}}
\def\P{{\rm P}}


\def\bb{\mathbf b}
\def \bc{\mathbf c}
\def\bx {\mathbf x}
\def\bn {\mathbf n}




\def\le{\leqslant}
\def\ge{\geqslant}


\def\var{{\rm Var}\,}

\newcommand{\ds}{\displaystyle}
\newcommand{\ts}{\textstyle}




\begin{document}

\setcounter{page}{2}

 
\section*{Section A: \ \ \ Pure Mathematics}

%%%%%%%%%%Q1
\begin{question}
\begin{questionparts}
\item Use   the substitution $u= x\sin x +\cos x$ to find
\[
\int \frac{x }{x\tan x +1 } \, \d x
\,.
\]

Find by means of a similar substitution, or otherwise,   
\[
\int \frac{x }{x\cot x -1 } \, \d x
\,.
\]
\item Use a substitution  to find
\vspace{2mm}
\[
\int \frac{x\sec^2 x \, \tan x}{x\sec^2 x -\tan x} \,\d x
\,
\]
\vspace{2mm}
and 
\vspace{2mm}
\[
\int \frac{x\sin x \cos x}{(x-\sin x \cos x)^2} \, \d x
\,.
\]
\end{questionparts}




\end{question}


%%%%%%%%%%%%Q2
\begin{question}           
\begin{questionparts}
\item The inequality $\dfrac 1 t \le 1$ holds for $t\ge1$.
By integrating both sides of this  inequality 
over
\\[3mm]
 the interval $1\le t \le x$,
show that 
\[ 
\ln x \le x-1
\tag{$*$}
\]
 for $x \ge 1$. Show similarly 
that $(*)$ also holds for $0 < x \le 1$.

\item
Starting from the inequality
$\dfrac{1}{t^2} \le \dfrac1 t $ for $t \ge 1$,
show that 
\[ 
\ln x \ge 1-\frac{1}{x}
\tag{$**$}
\]
for $x > 0$.

\item
Show, by integrating ($*$) and ($**$), that
\[
\frac{2}{ y+1} \le \frac{\ln  y}{ y-1} \le \frac{ y+1}{2 y}
\] 
for $ y > 0$ and $ y\ne1$.




\end{questionparts}
\end{question}



%%%%%%%%%%%%%%%%Q3
\begin{question}
The points $P(ap^2, 2ap)$ and $Q(aq^2, 2aq)$, where $p>0$ and $q<0$,
 lie on the curve $C$ with equation 
$$y^2= 4ax\,,
$$ 
where $a>0\,$. Show that the equation of the tangent to $C$ at $P$ is
$$y= \frac 1 p \, x +ap\,.$$

The tangents to the curve at $P$ and at $Q $ meet at $R$.      
These tangents meet the 
$y$-axis at $S$ and $T$ respectively, and $O$ is the origin.
Prove that the area of triangle $OPQ$ is  twice the area of triangle $RST$.

\end{question}

%%%%%%%%%%%%%%%%%%Q4
\begin{question}
\begin{questionparts}
\item
Let $r$ be a real number with $\vert r \vert<1$ and let 
\[
S = \sum_{n=0}^\infty r^n\,.
\]
 You may assume  without proof 
that $S = \ds \frac{1}{1-r}\, $.



Let $p= 1 + r +r^2$.
Sketch the graph of the function $1+r+r^2$ 
and deduce 
that $\frac{3}{4} \le  p < 3\,$.

Show that,
if $1 <  p < 3$, 
 then the value of $ p$  determines $r$, and hence $S$,
uniquely. 

Show also
that, if $\frac{3}{4} <  p < 1$, then there are two possible values of 
$S$ and these values  satisfy the equation $(3- p)S^2-3S+1=0$.

\item 
Let $r$ be a real number with $\vert r \vert<1$ and let 
\[
T =\sum_{n=1}^\infty nr^{n-1}
\,.
\]
 You may assume without proof that 
$
T = \ds \dfrac{1}{(1-r)^2}\,.
$


Let $ q= 1+2r+3r^2$.
Find the set of values of $ q$ that  determine $T$ uniquely.

Find the set of values of $ q$ 
for which $T$ has two possible values. 
Find also   a quadratic equation,
with coefficients 
depending on $ q$, that is satisfied by these two values.


\end{questionparts}

\end{question}




%%%%%%%%%%%%%%Q5
\begin{question}
A circle of radius $a$
is centred at the origin $O$. A rectangle $PQRS$ lies in the 
minor sector $OMN$ of this circle 
where $M$ is $(a,0)$ and $N$ is $(a \cos \beta, a \sin \beta)$, 
and $\beta$ is a constant with $0 < \beta < \frac{\pi}{2}\,$. 
Vertex $P$ lies on the positive $x$-axis at $(x,0)$; 
vertex $Q$ lies on $ON$; vertex $R$ lies on the arc of the circle 
between $M$ and $N$; and vertex $S$ lies on the positive $x$-axis at $(s,0)$.

Show that the area $A$ of the rectangle can be written in the form
\[
A= x(s-x)\tan\beta
\,.
\]
Obtain an expression for $s$ in terms of $a$, $x$ and $\beta$, and use it to
show  that
\[
\frac{\d A}{\d x} =
(s-2x) \tan \beta  - \frac {x^2} s \tan^3\beta
\,.
\]
Deduce that  the greatest possible area of 
rectangle $PQRS$ occurs when
$s= x(1+\sec\beta)$ and show that this greatest area is $\tfrac12 a^2 \tan \frac12 \beta\,$.

Show also that this greatest area occurs when $\angle ROS = \frac12\beta\,$.

\end{question}

\vspace{-3mm}
%%%%%%Q6

\begin{question}

In this question, you may assume that, if a continuous  function 
takes both positive and negative values in an interval, then
it takes the value $0$ at some point in that interval. 


\begin{questionparts}
\item
The function  $\f$ is continuous and $\f(x)$ is non-zero for some value 
of $x$ in the interval $0\le x \le 1$. Prove by contradiction, or otherwise,
 that if
\[
\int_0^1 \f(x) \d x = 0\,,
\]
then 
 $\f(x)$ takes both positive and negative values 
in the interval $0\le x\le 1$.


\item The function  $\g$ is continuous and
\[
\int_0^1 \g(x) \, \d x = 1\,,
 \quad 
\int_0^1 x\g(x) \, \d x = \alpha\, , 
\quad 
\int_0^1 x^2\g(x) \, \d x = \alpha^2\,.
\tag{$*$}
\] 
 Show, by considering
\[
\int_0^1 (x - \alpha)^2 \g(x) \, \d x
\,,
\]
 that
$\g(x)=0$ for some value of $x$ in the interval $0\le x\le 1$.

Find a function of the  form $\g(x) = a+bx$ that satisfies the conditions
$(*)$ and verify that $\g(x)=0$ for some value of $x$ in the interval 
$0\le x \le 1$.
\item
The function $\h$ has a continuous derivative $\h'$ and
\[
\h(0) = 0\,,
\quad 
\h(1) = 1\,,
\quad 
\int_0^1 \h(x) \, \d x = \beta\,, 
\quad 
\int_0^1 x \h(x) \, \d x = \ts \frac{1}{2} \ds \beta (2 - \beta)
\,.
\] 
Use the result in part \bf (ii) \rm to show that $\h^\prime(x)=0$ 
for some value of $x$ in the interval $0\le x\le 1$.

\end{questionparts}


\end{question}




%%%%%%%%%%%%%%%%%Q7
\begin{question}


The triangle $ABC$ has side lengths $\left| BC \right| = a$, 
$\left| CA \right| = b$ and $\left| AB \right| = c$.
Equilateral triangles 
$BXC$, \; $CY\!A$  \hspace{0.0mm} and  $AZB$ are erected on the sides of the  triangle $ABC$, 
with~$X$ on the other side of $BC$ from $A$, and similarly for $Y$ and $Z$. 
Points $L$, $M$ and $N$ are the centres 
of rotational symmetry of triangles $BXC$, $CY\!A$ and $AZB$ respectively. 





\begin{questionparts}
\item
Show that $| CM| =  \dfrac {\ b} {\sqrt3} \,$
and write down the corresponding expression for $| CL|$.

\item Use the cosine rule to show that
\[
6 \left| LM \right|^2 = a^2+b^2+c^2 + 4\sqrt3 \, \Delta \,,
\]
where 
$\Delta$ is the area of triangle $ABC$. 
Deduce
 that $LMN$ is an equilateral triangle.

Show further that the areas of triangles $LMN$ and $ABC$ are equal
if and only if
\[
a^2+b^2 +c^2 = 4\sqrt3 \, \Delta
\,.
\]
\item
Show that 
the conditions
\[
(a -b)^2 = -2ab \big( 1 -\cos(C-60^\circ)\big) 
\,\]
and 
\[
a^2+b^2 +c^2 = 4\sqrt3 \, \Delta
\]
are equivalent.


Deduce
that the areas of triangles $LMN$ and $ABC$ are equal
if and only if
$ABC$ is equilateral.






\end{questionparts}
\end{question}

  
%%%%%%%%%%%%%%%Q8
\begin{question}           
 Two sequences are defined by $a_1 = 1$ and $b_1 = 2$ and, for $n \ge 1$,
\begin{equation*}
\begin{split}
a_{n+1} & = a_n+ 2b_n \,,
\\
b_{n+1} & = 2a_n + 5b_n \,. 
\end{split}
\end{equation*}
Prove by induction that, for all $n \ge 1$, 
\[
a_n^2+2a_nb_n  - b_n^2 = 1 
\,.
\tag{$*$}\]
\begin{questionparts}
\item
Let $c_n = \dfrac{a_n}{b_n}$\,.
Show that $b_n  \ge  2 \times 5^{n-1}$
and use $(*)$ to show  that  
\[
c_n \to \sqrt 2 -1 
\text{ as }  n\to\infty\,.
 \]
\item
Show also that
$c_n > \sqrt2 -1$ and hence that $\dfrac2 {c_n+1}<\sqrt2<c_n+1
$\,.

Deduce that  $\dfrac{140}{99}<   \sqrt{2} < \dfrac{99}{70 }\,$.

\end{questionparts}
\end{question}
\newpage

\section*{Section B: \ \ \ Mechanics}


%%%%%%%%%%%%%%%%%%Q9 

\begin{question}

A particle is projected at speed $u$ from a point $O$ on a horizontal plane.
It
passes through a fixed point $P$ which is at a horizontal distance $d$
from $O$ and at   a height  
$d \tan \beta$ above the plane, where $d>0$ and
 $\beta $ is an acute angle.
The  
 angle of projection $\alpha$ is chosen so that $u$ is as small as
possible. 

\begin{questionparts}
\item
Show 
that 
 $u^2 = gd \tan \alpha$ and 
$2\alpha = \beta + 90^\circ\,$.
\item
At what angle to the horizontal is the particle travelling when it 
passes through~$P$? Express your answer in terms of $\alpha$ in its 
simplest form.
\end{questionparts}
\end{question}




%%%%%%%%%%%%%%%%%%%Q10

\begin{question}

Particles $P_1$, $P_2$, $\ldots$
are at rest on the $x$-axis, and the 
$x$-coordinate of $P_n$ is $n$. 
The  mass of~$P_n$ is $\lambda^nm$.
Particle $P$, of mass $m$, is projected 
from the origin at speed $u$ towards $P_1$.
A~series of collisions takes place, and the coefficient of 
restitution at each collision is $e$, where $0<e<1$. The speed of 
$P_n$ immediately after its first collision is $u_n$ and the 
speed of $P_n$ immediately after its second collision is  $v_n$.
No external forces act on the particles.


\begin{questionparts}

\item Show that $u_1=\dfrac{1+e}{1+\lambda}\, u$ and find  expressions for 
$u_n$ and $v_n$  in terms of $e$, $\lambda$, $u$ and $n$.

\item Show that, if $e > \lambda$, then each particle (except $P$)
 is involved 
in exactly two collisions.

\item Describe what happens if $e=\lambda$ and show that, in this case, 
the fraction of the initial kinetic energy lost approaches $e$ as the number of 
collisions increases.

\item Describe what happens if $\lambda e=1$. What fraction of the initial
kinetic energy is \mbox{eventually} lost in this case?

\end{questionparts}

\end{question}



%%%%%%%%%%%%%%%%%%Q11

\begin{question}

A plane makes 
an acute angle $\alpha$ 
with the horizontal.
A box in the shape of a cube is fixed  onto  the plane 
 in such a way that four of its edges are horizontal and two of its sides
are vertical.


A uniform rod of length $2L$ and weight $W$ rests  with 
its lower end at $A$ on the bottom of the box and 
its upper end at $B$ on a side of the box, as shown in the diagram below.
 The vertical plane containing the 
rod  is parallel to the vertical sides of the box   
and cuts the lowest edge of the box
at $O$. The rod makes an acute angle~$\beta$ with the side of the box at $B$.

The coefficients of friction between the rod 
and the box at the two points of contact 
are both $\tan \gamma$, where $0<   \gamma<\frac12\pi$.
%The  frictional force on the rod at $A$ acts toward $O$, 
%and the frictional force on the rod at~$B$
%acts away from $O$.

The rod is in limiting equilibrium, with the end at $A$
 on the point of slipping in the direction away from $O$ and the end at $B$
on the point of slipping towards $O$. Given that
$\alpha < \beta$,
show that $\beta = \alpha + 2\gamma$. 

[{\bf Hint}: You may find it helpful to take moments about the midpoint of 
the rod.]
\vspace{-1.5cm}
\begin{center}
\newrgbcolor{zzttqq}{0.6 0.2 0}
\newrgbcolor{xdxdff}{0.49 0.49 1}
\psset{xunit=6mm  ,yunit=6mm  }
\begin{pspicture*}(-8.3,-6.72)(15.56,8.3)
\psset{xunit=0.8cm,yunit=0.8cm,algebraic=true,dotstyle=o,dotsize=3pt 0,linewidth=0.8pt,arrowsize=3pt 2,arrowinset=0.25}
\pspolygon[linewidth=1.2pt,
%linecolor=zzttqq,
](0,-3)(-1.55,2.8)(4.24,4.35)(5.87,-1.41)
\psline(-2.54,-3   )(12,-3)
\psline(-2.46,-3.62)(11.59,0.11)
%\psline[linewidth=1.2pt,linecolor=zzttqq](0,-3)(-1.55,2.8)
%\psline[linewidth=1.2pt,linecolor=zzttqq](-1.55,2.8)(4.24,4.35)
%\psline[linewidth=1.2pt,linecolor=zzttqq](4.24,4.35)(5.87,-1.41)
%\psline[linewidth=1.2pt,linecolor=zzttqq](5.87,-1.41)(0,-3)
\rput[tl](2.6 ,-2.4 ){$ A $}
\rput[tl](-1.8 ,1.6){$ B $}
\rput[tl](-0.2,-3.15 ){$ O $}
\rput[tl](1.4 ,-2.7){$ \alpha $}
\rput[tl](-0.8,0.8){$ \beta $}
\psline[linewidth=2.3pt](-1.20,1.52)(2.8,-2.22)
\pscustom[linecolor=black,fillcolor=zzttqq,fillstyle=none,opacity=0.0]{\parametricplot{-0.0127510246967992202}{0.2471697861109875}{1*1.84*cos(t)+0*1.84*sin(t)+-0.0|0*1.84*cos(t)+1*1.84*sin(t)+-2.98}\lineto(0,-2.99)\closepath}
\pscustom[linecolor=black,fillcolor=zzttqq,fillstyle=none,opacity=0.0]{\parametricplot{4.974188368183839}{5.533030852440859}{1*1.45*cos(t)+0*1.45*sin(t)+-1.21|0*1.45*cos(t)+1*1.45*sin(t)+1.52}\lineto(-1.21,1.52)\closepath}
\end{pspicture*}
\end{center}


\end{question}

\newpage

\section*{Section C: \ \ \ Probability and Statistics}

\vspace{-3mm}

%%%%%%%Q12

\begin{question}


In a lottery, each of the $N$ participants pays $\pounds c$ 
to the organiser and picks 
a number from $1$ to $N$. 
The organiser picks at random the winning number from $1$ to $N$  and 
all those participants who picked this number receive an equal 
share of the prize, $\pounds J$.



\begin{questionparts}

\item 
The participants pick their numbers 
independently and with equal probability. 
Obtain  an expression for the 
probability that no participant picks the 
winning number, and hence determine the organiser's expected profit.

Use  the approximation
\[
\left( 1 - \frac{a}{N} \right)^N \approx \e^{-a}
\tag{$*$}
\]
 to 
show that if $2Nc = J$ then the organiser will expect to make a loss.

{\bf Note}: $\e > 2$.  
 
\item Instead of the numbers being equally popular, 
a fraction $\gamma$ of the numbers are popular and the rest are unpopular.
For each participant,  
the probability of picking any given popular number is 
$\dfrac{a}{N}$ and the probability of picking
   any given unpopular number is~$\dfrac{b}{N}\,$. 

Find a relationship between $a$, $b$ and $\gamma$.

Show that,
using the 
approximation $(*)$,  
 the organiser's expected profit can be \mbox{expressed} 
in the form
\[
A\e^{-a} + B\e^{-b} +C
\,,
\]
where $A$, $B$ and $C$ can be written in terms of  $J$, $c$, $N$ and $\gamma$. 



In the case $\gamma = \frac18$ and $a=9b$, find $a$ and $b$. Show that,
 if  
 $2Nc = J$, then
 the organiser will expect to make a profit.

{\bf Note}: $\e < 3$.  


\end{questionparts}
\end{question}

\vspace{-2mm}
%%%%%%%%Q13


\begin{question}
I have a sliced loaf which initially contains $n$ slices of bread. 
Each time I finish setting a STEP question, I make myself a snack:
either toast, using one slice of bread;
or a sandwich, using two slices of bread. 
I make 
 toast with probability~$p$ and I make a sandwich  
 with probability $q$, where $p+q=1$, 
unless there is only one slice left in which case I must, of course, 
make toast.
 
Let $s_r$ ($1 \le r \le n$) be the probability that 
the $r${th} slice of bread is the second of two slices 
used to make a sandwich 
and  let $t_r$ ($1 \le r \le n$) be the probability that the 
$r${th} slice of bread is used to make toast. What is the value of $s_1$? 

Explain why the following equations hold:
\begin{align*}
\phantom{\hspace{2cm} (2\le r \le n-1)}   
t_r &= (s_{r-1}+ t_{r-1})\,p  
\hspace{2cm}
(2\le r \le n-1)\,;
\\
 \phantom{\hspace{1.53cm} (2\le r \le n) }           
s_r &= 1- (s_{r-1} + t_{r-1})
 \hspace{1.53cm}
( 2\le r \le n )\,.           
\end{align*}
Hence, or otherwise, show that $s_{r} = q(1-s_{r-1})$ for
$2\le r\le n-1$\,.

Show further that 
\[
\phantom{\hspace{2.7cm} (1\le r\le n)\,,}
s_r = \frac{q+(-q)^r}{1+q}
\hspace{2.7cm}
(1\le r\le n-1)\,,
\,
\hspace{0.14cm}
\]
and find the corresponding expression for $t_r$.

Find also expressions for $s_n$ and $t_n$ in terms of $q$.
%\item
%\end{questionparts}




\end{question}

\end{document}



%%%%%%%%%%%%%% spare Q1
\begin{question}
If $\f$ is a function defined on the interval $0 \le x \le y$, and $\f(x) > 0$ for $0 \le x \le y$, then the geometric mean of $\f$ over this interval is
\[
\G(\f) = e^{\frac{1}{y}\int_0^y \ln \f(x) \, \d x}.
\]
\begin{questionparts}
\item  Prove that, for all $a > 1$,
\[
a^{\frac{1}{y}\int_0^y \log_a \f(x) \, \d x}= \G(\f).
\]

\item Prove that, if $h(x) = f(x)g(x)$, where $\f(x) > 0$ and $\g(x) > 0$ for $0 \le x \le y$, then $\G(\h)= \G(\f) \G(\g)$.

\item Prove that, for all $a > 0$, if $\f(x) = a^x$, then $\ds \G(\f)= \sqrt{f(y)}$.

\item Prove that if $f(0) = 1$, and $\ds \G(\f)= \sqrt{f(y)}$ for all $y \ge 0$, then $\f(x) = a^x$ for some $a >0$.

\end{questionparts}

\bluecomment{I liked this, but thought it probably too sophisticated for 
Paper I. I suggest moving it to another paper rather than butchering it.
Would you be happy with that?

}
\end{question}


%%%%%%%%%%%%%%% spare Q2
\begin{question}

Let $\ds \f(x) = \frac{1}{\sqrt{x}}$
           
\begin{questionparts}
\item Estimate the area under the curve $y=\f(x)$ from $x=1$ to $x=n+1$ using the trapezium rule with strips of width 1.
 
Hence show that
\[
\sum_{r=1}^n{\frac{1}{\sqrt{r}}} \ge 2 \sqrt{n +1} -\frac{3}{2} -\frac{1}{2 \sqrt{n+1}}.
\]

\item Find the area of the trapezium formed by the lines $x= r - \frac{1}{2}$ and $x = r + \frac{1}{2}$, the x-axis, and the tangent at $(r \, , \; \f(r))$ to the curve $y=\f(x)$.

Hence show that 
\[
\sum_{r=1}^n{\frac{1}{\sqrt{r}}} \le 2 \sqrt{n + \tfrac{1}{2}} -\sqrt{2}.
\]

\item Show that when $n=24$, the average of the bounds in \bf (i) \rm and \bf (ii) \rm will certainly differ from the true value of $\ds \sum_{r=1}^n{\frac{1}{\sqrt{r}}}$ by less than one percent.

\end{questionparts}

\bluecomment{This seems good (though I haven't worked through it carefully).
But I don't think we can use it this year: Paper II
has
\[
2\sqrt{n} + \frac1{2\sqrt n}+C\ge \sum \frac {1}{\sqrt r} \ge 2\sqrt n 
\]
The Paper II  proof uses induction, so it is not the same question;
maybe they are sufficiently different to use both if we are stuck,
but it is better to put it in the bag for next year.
}
\end{question}




%%%%%%%%%%%%%%% removed parts of Q7
\item  
 Show that, for $x>1$,
\[
\ln x <  \tfrac{1}{2} \left( x-1/x \right)
\,.
\]
\item Show that, for $x>0$,
\[
\ln x \le \frac{(x+5)(x-1)}{2(2x+1)}
\,.
\]



\begin{question}

Two particles are able to move without friction on the $x$-axis. 
When the particles are a distance $x \le d$ apart, 
the repulsive force between them is given 
by $\ds \Phi \left( 1-\frac{x}{d} \right)$, and is zero otherwise.



Particle 1, with mass $m_1$, is at $x=x_1$, 
and particle 2, with mass $m_2$, is at $x=x_2$; 
initially, $x_1 = 0$ and $x_2=d$ and both particles are at rest. 
For $t \ge 0$, a constant force 
$\alpha \Phi$, with $\alpha  \le 1$, is exerted on particle 1, 
in the direction of the positive $x$-axis.

\begin{questionparts}

\item Show that, while $0 \le x_2 - x_1 \le d$,

\[
\frac{\d^2 (x_2 - x_1)}{\d t^2} = \frac{m_1+m_2}{m_1m_2}\Phi 
\left( 1-\frac{x_2-x_1}{d} \right) -\frac{\alpha \Phi}{m_1}
\]

\item Show that this equation, and the initial 
conditions on $x_1$ and $x_2$, are 
satisfied by $x_2 - x_1 = A \cos^2(\omega t) +c$, 
for suitable values of $A$ and $c$, 
which you should determine in terms of $\Phi$, $d$ and $\alpha$.

\item Find expressions for $x_1$ and $x_2$ as functions of time.

\item Evaluate $x_2 - x_1$ and $\ds \frac{\d (x_2 - x_1)}{\d t}$ 
at $\ds t = \frac{\pi}{\omega}$. 
If the constant force $\alpha \Phi$ acts only for 
$0 \le t \le \ds \frac{\pi}{\omega}$,  
explain how the particles will move for $\ds t > \frac{\pi}{\omega}$.

\end{questionparts}

\end{question}




%%%%%%%%Q13

\begin{question}



Errors occur in a process at unpredictable times. 
The number of errors that occur in any particular time 
interval is modelled by a random variable $N$. 
It is assumed that the distribution of $N$ 
depends only on the length of the time interval, and 
that the numbers of errors that occur in any two independent 
time intervals are independent of each other. 
Let $\f_r(t)$ be the probability that exactly $r$ errors 
occur in a time interval of length $t$.



\begin{questionparts}

\item Explain why $\f_0(t+s) = \f_0(t) \f_0(s)$. 
Assuming that the function $\f$ is sufficiently 
smooth, differentiate this equation with respect to $s$ and 
then set $s = 0$. Deduce that $\f_0(t) = \e^{- \lambda t}$ 
for some positive value of $\lambda$.

\item Write $\f_1(t+s)$ in terms of  
$\f_0(t)$, $\f_0(s)$,$\f_1(t)$ and $\f_1(s)$ and show that

$\f'_1(t) = k\e^{- \lambda t}-\lambda \f_1(t)$, 
where $k = \f'_1(0)$. By considering the differential 
equation satisfied by $\g(t) = \f_1(t) \e^{\lambda t}$, 
show that $\f_1(t) = \lambda t \e^{-\lambda t}$. 

\item Show that 
$\f_2(t) = \left(c t + \frac{1}{2}\lambda^2 t^2 \right) \e^{-\lambda t}$, 
where $c$ is a constant, and explain why you would expect $c$ to be zero.

\end{questionparts}
\bluecomment{I like this, but prefer to put it on the shelf for two
reasons. 

One is that when in the past
I have tried to introduce partial differentiation
to obtain a functional equation, I have always been told off --- so 
probably Paper I is not the place to try it.

Second is that I ditched one of Neil's Paper II questions
and suggested exactly this idea as a replacement
(well before I saw your question). 
I don't know
yet whether he has adopted my suggestion.
}
\end{question}

%%%%%%%%%%%%%%% removed parts of Q10
\begin{questionparts}
\item Show 
the midpoint of the rod is a distance
 $L \sin(\alpha+\beta)$ from the vertical through $B$
and find the distance of $A$ from this vertical.

Deduce that, 
if the wall and the floor are smooth, the base of the rod 
will begin to slide towards $O$ if $\beta < \alpha$ and away 
from $O$ if $\beta > \alpha$. What will happen if $\beta = \alpha$?

\item 

%%%%%%%%%%%%%%% your version of Q13

\begin{question}
A frog is climbing out of a dry well that is $n$ bricks deep. 
From the $r$th brick ($1\le r \le  n-1$), counting from the bottom of
the well,
he  either:
\begin{itemize} 
\item
hops onto the $(r+1)$th brick; or 
\item
jumps that over that brick onto the $(r+2)$th brick (or out of the 
well from the $(n-1)$th brick). 
\end{itemize}
If he
lands on the $n$th brick, he just hops out of the well;
otherwise, he hops with probability~$p$ 
and jumps with probability~$q$, where $p>0$, $q>0$ and $p+q=1$. 
He starts from the first brick.

Let $u_r$ ($2\le r \le n$)
be the probability that he lands on the $r$th brick during his climb.
Express $u_r$, for $4\le r \le n$,
 in terms of $u_{r-1}$, $u_{r-2}$, $p$ and $q$. 
Deduce that
\[
u_r- u_{r-1} = -q (u_{r-1}-u_{r-2})
\]
and hence show that, for $2\le r\le n$,
\[
u_r = \frac{1-(-q)^r}{1+q}
\,.\]
