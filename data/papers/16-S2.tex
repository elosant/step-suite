\documentclass[a4, 11pt]{report}


%\pagestyle{myheadings}
%\markboth{}{Paper II, 2006  second vetter draft 
%\ \ \ \ \ 
%\today 
%}               



\RequirePackage{amssymb}
\RequirePackage{amsmath}
\RequirePackage{graphicx}
\RequirePackage{color}
\RequirePackage[flushleft]{paralist}[2013/06/09]



\RequirePackage{geometry}
\geometry{%
  a4paper,
  lmargin=2cm,
  rmargin=2.5cm,
  tmargin=3.5cm,
  bmargin=2.5cm,
  footskip=12pt,
  headheight=24pt}


\newcommand{\comment}[1]{{\bf Comment} {\it #1}}
%\renewcommand{\comment}[1]{}

\newcommand{\bluecomment}[1]{{\color{blue}#1}}
%\renewcommand{\comment}[1]{}
\newcommand{\redcomment}[1]{{\color{red}#1}}



\usepackage{epsfig}


\setlength{\parskip}{10pt}
\setlength{\parindent}{0pt}

\newlength{\qspace}
\setlength{\qspace}{20pt}


\newcounter{qnumber}
\setcounter{qnumber}{0}

\newenvironment{question}%
 {\vspace{\qspace}
  \begin{enumerate}[\bfseries 1\quad][10]%
    \setcounter{enumi}{\value{qnumber}}%
    \item%
 }
{
  \end{enumerate}
  \filbreak
  \stepcounter{qnumber}
 }


\newenvironment{questionparts}[1][1]%
 {
  \begin{enumerate}[\bfseries (i)]%
    \setcounter{enumii}{#1}
    \addtocounter{enumii}{-1}
    \setlength{\itemsep}{5mm}
    \setlength{\parskip}{8pt}
 }
 {
  \end{enumerate}
 }



\DeclareMathOperator{\cosec}{cosec}
\DeclareMathOperator{\Var}{Var}

\def\d{{\rm d}}
\def\e{{\rm e}}
\def\g{{\rm g}}
\def\h{{\rm h}}
\def\f{{\rm f}}
\def\k{{\rm k}}

\def\p{{\rm p}}
\def\s{{\rm s}}
\def\t{{\rm t}}


\def\A{{\rm A}}
\def\B{{\rm B}}
\def\E{{\rm E}}
\def\F{{\rm F}}
\def\G{{\rm G}}
\def\H{{\rm H}}
\def\P{{\rm P}}


\def\bb{\mathbf b}
\def \bc{\mathbf c}
\def\bx {\mathbf x}
\def\bn {\mathbf n}

\makeatletter
\newcommand{\raisemath}[1]{\mathpalette{\raisem@th{#1}}}
\newcommand{\raisem@th}[3]{\raisebox{#1}{$#2#3$}}
\makeatother
%%%To raise suffices: e.g.  $\Pi_{\raisemath{2pt}{-}}$.


\def\le{\leqslant}
\def\ge{\geqslant}


\def\var{{\rm Var}\,}

\newcommand{\ds}{\displaystyle}
\newcommand{\ts}{\textstyle}




\begin{document}

\setcounter{page}{2}

 
\section*{Section A: \ \ \ Pure Mathematics}

%%%%%%%%%%%%%% Q1

\begin{question}
The curve $C_1$ has parametric equations $x=t^2$, $y= t^3$, where 
$-\infty < t < \infty\,$. 
Let $O$ denote the point $(0,0)$.
The points $P$ and $Q$ on $C_1$  are such that $\angle POQ$ is
a right angle. Show that the tangents to $C_1$  at $P$ and $Q$ 
intersect on the curve $C_2$  with equation $4y^2=3x-1$.

Determine whether  $C_1$ and $C_2$ meet, and sketch the two
curves on the same axes.

\end{question}


%%%%%%%%%%%%%% Q2
\begin{question}
Use the factor theorem to show that $a+b-c$ is a factor of 
\[
(a+b+c)^3 -6(a+b+c)(a^2+b^2+c^2) +8(a^3+b^3+c^3)
\,.
\tag{$*$}
\]
Hence factorise ($*$) completely.

\begin{questionparts}
\item 
Use the result above to solve the equation
\[
(x+1)^3 -3  (x+1)(2x^2 +5) +2(4x^3+13)=0\,.
\]


\item
By setting  $d+e=c$, or otherwise, show that 
$(a+b-d-e)$ is a factor of 
\[
(a+b+d+e)^3 -6(a+b+d+e)(a^2+b^2+d^2+e^2) +8(a^3+b^3+d^3+e^3)
\,
\]
and factorise this expression completely.
  
Hence solve the equation
\[
(x+6)^3 - 6(x+6)(x^2+14) +8(x^3+36)=0\,.
\]
\end{questionparts}

\end{question}

%%%%%%%%%%%%%% Q4
\begin{question}
For each non-negative integer $n$, the polynomial $\f_n$ is defined by
\[
\f_n(x) = 1 + x + \frac{x^2}{2!} + \frac {x^3}{3!} + \cdots + \frac{x^n}{n!}
\,.
\]
\begin{questionparts}
\item
Show that  $\f'_{n}(x) = \f_{n-1}(x)\,$ (for $n\ge1$).
\item 
Show that, if $a$  is a real root 
of  the equation \[\f_n(x)=0\,,\tag{$*$}\] then $a<0$.
\item
Let $a$ and $b$ be  distinct real 
roots of ($*$), for $n\ge2$. Show that  $\f_n'(a)\, \f_n'(b)>0\,$
and use a sketch to deduce
that  $\f_n(c)=0$ for some number $c$ between $a$ and $b$.

Deduce that ($*$ )  has at most one real root.
How many real roots does ($*$)  have 
if $n$ is odd?
How many real roots does ($*$) have 
if $n$ is even?

\end{questionparts}

\end{question}

%%%%%%%%%%%%%% Q4
\begin{question}
Let 
\[
y=\dfrac{x^2+x\sin\theta+1}{x^2+x\cos\theta+1}
\,.\]
\begin{questionparts}
\item
Given that
$x$ is real, show that
\[
(y\cos\theta -\sin\theta)^2 \ge 4 (y-1)^2
\,.
\]
Deduce that 
\[
y^2+1
\ge 
4(y-1)^2 
\,,
\]
and hence that 
\[
\dfrac {4-\sqrt7}3 
\le y \le
\dfrac {4+\sqrt7}3 \,.
\]
\item
In the case $y=   
\dfrac {4+\sqrt7}3 \,$,  show that \[\sqrt{y^2+1}=2(y-1)\]
and find the corresponding values of $x$ and
 $\tan\theta$.
\end{questionparts}


\end{question} 

%%%%%%%%%%%%%% Q5
\begin{question}
In this question, the definition of $\displaystyle\binom pq$ 
is taken to be
\[
\binom pq =
\begin{cases}
\dfrac{p!}{q!(p-q)!} & \text{ if } p\ge q\ge0 \,,\\[4mm]
0 & \text{ otherwise } .
\end{cases}
\]
\begin{questionparts}
\item
Write down the coefficient of $x^n$ in the binomial expansion
for $(1-x)^{-N}$, where $N$ is a positive integer, and write
down the expansion
using the $\Sigma$ summation notation. 

By considering  
$ 
(1-x)^{-1} (1-x)^{-N}
\,
,$ 
where $N$ is a positive integer, show that
\[
\sum_{j=0}^n \binom { N+j -1}{j} = \binom{N+n}{n}\,.
\]
\item
Show that,
for any positive integers $m$, $n$ and $r$ with $r\le m+n$,
\[
\binom{m+n} r = \sum _{j=0}^r \binom m j \binom n {r-j}
\,.
\]

\item
Show that, for any positive integers $m$ and $N$, 
\[
\sum_{j=0}^n(-1)^{j}   \binom {N+m} {n-j} \binom {m+j-1}{j  } =
\displaystyle \binom N n 
.
\]

\end{questionparts}


\end{question}


%%%%%%%%%%%%%% Q6
\begin{question}
This question concerns solutions of 
 the differential equation
\[
(1-x^2) \left(\frac{\d y}{\d x}\right)^2 + k^2 y^2 = k^2\,
\tag{$*$}
\]
where $k$ is a positive integer. 

For each value of $k$, let $y_k(x)$  be
the solution of $(*)$ that satisfies $y_k(1)=1$; 
you may 
assume that
there is only one such solution for each value of $k$.



\begin{questionparts}
\item
Write down the differential equation satisfied by $y_1(x)$ and
verify that $y_1(x) = x\,$. 
\item 
Write down the differential equation satisfied by $y_2(x)$ and
verify that  $y_2(x) = 2x^2-1\,$.
\item 
Let $z(x) = 2\big(y_n(x)\big)^2 -1$. Show that
\[
 (1-x^2) \left(\frac{\d z}{\d x}\right)^2 +4n^2 z^2 = 4n^2\,
\]
and hence obtain an expression for $y_{2n}(x)$ in terms of $y_n(x)$.
\item 
Let $v(x) = y_n\big(y_m(x)\big)\,$. 
Show that $v(x) = y_{mn}(x)\,$.
\end{questionparts}
\end{question}


%%%%%%%%%%%%%% Q7
\begin{question}
Show that 
\[
\int_0^a \f(x) \d x= \int _0^a \f(a-x) \d x\,,
\tag{$*$}
\]
where f is any function for which the integrals exist.
\begin{questionparts}
\item Use ($*$)  to evaluate
\[
\int_0^{\frac12\pi} \frac{\sin x}{\cos x + \sin x} \, \d x
\,.
\]
\item Evaluate
\[
\int_0^{\frac14\pi} \frac{\sin x}{\cos x + \sin x} \, \d x
\,.
\]
\item Evaluate
\[
\int_0^{\frac14\pi} \ln (1+\tan x) \, \d x
\,.
\]
\item Evaluate
\[
\int_0^{\frac14 \pi}
 \frac x {\cos x \, (\cos x + \sin x)}\, \d x
\,.
\]
\end{questionparts}

\end{question} 

%%%%%%%%%%%%%% Q8
\begin{question}
Evaluate  the integral
\[
\hphantom{ \ \ \ \ \ \ \ \ \ 
(m> \tfrac12)\,.}
\int_{m-\frac12} ^\infty \frac 1{x^2}\,  \d x 
{ \ \ \ \ \ \ \ \ \ 
(m> \tfrac12)\,.}
\]
Show by means 
of a sketch that 
\[
\sum_{r=m}^n \frac 1 {r^2} 
\approx \int_{m-\frac12}^{n+\frac12} \frac1 {x^2} \, \d x
\,,
\tag{$*$}
\]
where $m$ and $n$ are positive integers with $m<n$.

\begin{questionparts}
\item You are given that the infinite series $\displaystyle
\sum_{r=1}^\infty \frac 1 {r^2}$
converges to a value denoted by $E$.
Use~$(*)$ to obtain the following approximations for $E$:
\[
E\approx 2\,; \ \ \ \
E\approx \frac53\,; \ \ \ \
E\approx \frac{33}{20}
\,.\]
\item Show that, when $r$ is large,  the error 
in approximating 
$\dfrac 1{r^2}$ 
 by 
$\displaystyle \int_{r-\frac12}^{r+\frac12} \frac 1 {x^2} \, \d x$
  is approximately~$\dfrac 1{4r^4}\,$.

Given that $E \approx 1.645$, show that 
$\displaystyle \sum_{r=1}^\infty \frac1{r^4} \approx 1.08\, $.  

\end{questionparts}

\end{question}


\newpage
\section*{Section B: \ \ \ Mechanics}
%%%%%%%%%%%%%% Q9
\begin{question}
A small bullet of mass $m$ is fired into a block of wood of mass $M$ which is
at rest. 
The speed
of the bullet on entering the block is $u$.  Its trajectory 
within the block is a horizontal
straight line and  the resistance to the 
bullet's motion is $R$, which is constant.

\begin{questionparts}
\item The block is fixed. The bullet travels a distance
$a$ inside the block before coming to rest.
Find an expression for $a$ in terms of $m$, $u$ and $R$.

\item Instead, the block is free to move on a smooth horizontal 
table. The bullet travels a distance $b$ inside the block
before coming to rest relative to the block, at which time the 
block has moved a distance $c$ on the table. 
Find expressions for $b$ and $c$ in terms of $M$, $m$ and $a$.
 \end{questionparts} 

\end{question}


%%%%%%%%%%%%%% Q10
\begin{question}
A thin uniform wire is bent into the shape of an isosceles 
triangle $ABC$,
where 
$AB$ and~$AC$ are of equal length  and the angle
at $A$ is $2\theta$.
%Show that the centre of mass of the wire is a distance
%\[
%\dfrac{b\cos\theta}{2(1+\sin\theta)}\]
% from the side $BC$.
%
The triangle $ABC$  hangs
on a small rough horizontal peg
with the side $BC$ resting on the 
peg. 
 The coefficient of friction between the wire and the peg
is $\mu$.
The  plane containing $ABC$ is vertical.  
Show that the triangle can rest 
in equilibrium with the peg in contact with any point on $BC$ provided 
\[
\mu \ge 2\tan\theta(1+\sin\theta)
\,.
\]
\end{question}


%%%%%%%%%%%%%% Q11
\begin{question}
\begin{questionparts}
\item
Two particles move on a smooth horizontal surface.
The positions, in Cartesian coordinates, of the 
particles at time $t$
are $(a+ut\cos\alpha \,,\, ut\sin\alpha)$  and 
$(vt\cos\beta\,,\, b+vt\sin\beta )$, where $a$, $b$, $u$ and $v$ are positive
constants, $\alpha$ and $\beta$ are constant acute angles, and $t\ge0$.

Given that the two particles collide, show  that 
\[
u \sin(\theta+\alpha)  = v\sin(\theta +\beta)\,,
\]
where $\theta $ is the acute angle satisfying $\tan\theta = \dfrac b a$. 

\item A gun is placed on
 the top of a vertical tower of height $b$ 
which stands
on  horizontal ground. 
The gun fires a  bullet with speed  $v$ and  (acute) angle of 
elevation $\beta$.
Simultaneously, a target
is projected 
from a point on the ground
 a horizontal distance $a$ from the foot of the tower.
The target is projected with speed $u$ and (acute) angle of elevation 
$\alpha$, in a direction
directly away from the tower.
 
Given that  the target is hit before it reaches the ground,
show that
\[
2u\sin\alpha (u\sin\alpha - v\sin\beta)>bg\,.
\]

Explain, with reference to part (i), why the target can
only be hit if  $\alpha>\beta$. 
\end{questionparts}



\end{question}


\newpage
\section*{Section C: \ \ \ Probability and Statistics}
%%%%%%%%%%%%%% Q12
\begin{question}
Starting with the result $\P(A\cup B) = \P(A)+P(B) - \P(A\cap B)$, prove
that 
\[
\P(A\cup B\cup C) = \P(A)+\P(B)+\P(C)  
- \P(A\cap B) - \P(B\cap C) - \P(C \cap A)
+ \P(A\cap B\cap C)
\,.
\] 
Write down, without proof, the  corresponding result for four events
$A$, $B$, $C$ and $D$.

A pack of $n$ cards, numbered $1$, $2$, \ldots , $n$, is shuffled 
and laid out in a row. The result of the shuffle is that each card is 
equally likely to be in any position in the row. Let $E_i$ be
the event that the card bearing the number $i$  is in the $i$th position in the row.
Write down the following probabilities:
\begin{questionparts}
\item $\P(E_i)$\,;
\item $\P(E_i\cap E_j)$, where $i\ne j$;
\item  $\P(E_i\cap E_j\cap E_k)$, where $i\ne j$, $j\ne k$ and $k\ne i$.
\end{questionparts} 
Hence show that the probability that at least one card is in the same position
as the number it bears is
\[
1 - \frac 1 {2!} + \frac 1{3!} - \cdots + (-1)^{n+1} \frac 1 {n!}\,.
\]

Find the probability that exactly one card is in the same position
as the number it bears.

\end{question}




%%%%%%%%%

%%%%% Q13
\begin{question}
\begin{questionparts}
\item
  The random variable $X$ has a binomial distribution with parameters 
$n$ and
$p$, where $n=16$ and 
$p=\frac12$. 
Show, using an approximation in terms of the 
 standard normal
density function 
$\displaystyle
\tfrac{1}{\sqrt{2\pi}} \, \e ^{-\frac12 x^2}
$\,, that 
\[
\P(X=8) \approx \frac 1{2\sqrt{2\pi}}
\,.
\]
\item By considering a binomial 
distribution with parameters $2n$ and~$\frac12$, show that 
\[
(2n)! \approx \frac {2^{2n} (n!)^2}{\sqrt{n\pi}} \,.
\]
\item By considering a Poisson  distribution with parameter $n$, 
 show that  
\[
n! \approx \sqrt{2\pi n\, } \, \e^{-n} \, n^n \,.
\]

\end{questionparts}

\end{question}


\end{document}



\end{document}
\begin{question}
A particle is projected with speed $u$ at an angle $\theta$ 
above the horizontal over level ground. The particle just clears
three thin walls which are perpendicular to the plane of the particle's
trajectory. Two of the walls are of height $h$ and are a distance
$2a$ apart. The third wall is of height $h+k$, where $k>0$,
and is half way between the other two walls. 

Show that $\sec^2\theta = \dfrac{2ku^2}{ga^2}$. 

Express $\tan\theta$
and $u$ in terms of $a$, $h$, $k$ and $g$. 

\bluecomment{This was thought too easy, I recall. But I'm not sure at 
the moment what, if anything, can be done to increase the level of difficulty.

There was some discussion about whether they would quote the geometric 
equation of the trajectory, prove it or not use it. I can't remember
what the conclusion was (if there was a conclusion).
Maybe we should start with a level playing field by
asking them
to derive the $y= x\tan\theta \cdots$ equation.

The result for $\sec^2\alpha$ came out quite quickly from the geometric 
equation, so if we get them to derive the geometric equation, we could just
ask them to find
$\tan\theta $ interms of  $a$, $h$, $k$ (unless there is some quick way
of getting this).

What do you think?

}
\end{question}

\nq
For $k>m>0$, express $\dfrac 1 {(k+m)k(k-m)}$ in the form $\dfrac{A}{k(k-m)} +
\dfrac{B} {k(k+m)} $, where $A$ and $B$ are independent of $k$.

By taking $m=1$,  show that 
\[
\frac 1 {1^3}+\frac 1 {2^3} + \frac 1 {3^3}  + \cdots 
\ < \frac 54
\, .
\]

Show also that
\[
\frac 1{1^3} + \frac 1 {2^3}+ \frac 1 {3^3}  + \cdots \ < \frac {119}{96} \,.
\]
\nq
For $k>m>0$, express $\dfrac 1 {(k+m)k(k-m)}$ in the form $\dfrac{A}{k(k-m)} +
\dfrac{B} {k(k+m)} $, where $A$ and $B$ are independent of $k$.

By taking $m=1$,  show that 
\[
\frac 1 {1^3}+\frac 1 {2^3} + \frac 1 {3^3}  + \cdots 
\ < \frac 54
\, .
\]

Show also that
\[
\frac 1{1^3} + \frac 1 {2^3}+ \frac 1 {3^3}  + \cdots \ < \frac {119}{96} \,.
\]

