
\documentclass[a4, 11pt]{report}


\pagestyle{myheadings}
\markboth{}{Paper II, 1989
\ \ \ \ \ 
\today 
}               

\RequirePackage{amssymb}
\RequirePackage{amsmath}
\RequirePackage{graphicx}
\RequirePackage{color}
\RequirePackage[flushleft]{paralist}[2013/06/09]



\RequirePackage{geometry}
\geometry{%
  a4paper,
  lmargin=2cm,
  rmargin=2.5cm,
  tmargin=3.5cm,
  bmargin=2.5cm,
  footskip=12pt,
  headheight=24pt}


\newcommand{\comment}[1]{{\bf Comment} {\it #1}}
%\renewcommand{\comment}[1]{}

\newcommand{\bluecomment}[1]{{\color{blue}#1}}
%\renewcommand{\comment}[1]{}
\newcommand{\redcomment}[1]{{\color{red}#1}}



\usepackage{epsfig}
\usepackage{pstricks-add}
\usepackage{tgheros} %% changes sans-serif font to TeX Gyre Heros (tex-gyre)
\renewcommand{\familydefault}{\sfdefault} %% changes font to sans-serif
%\usepackage{sfmath}  %%%% this makes equation sans-serif
%\input RexFigs


\setlength{\parskip}{10pt}
\setlength{\parindent}{0pt}

\newlength{\qspace}
\setlength{\qspace}{20pt}


\newcounter{qnumber}
\setcounter{qnumber}{0}

\newenvironment{question}%
 {\vspace{\qspace}
  \begin{enumerate}[\bfseries 1\quad][10]%
    \setcounter{enumi}{\value{qnumber}}%
    \item%
 }
{
  \end{enumerate}
  \filbreak
  \stepcounter{qnumber}
 }


\newenvironment{questionparts}[1][1]%
 {
  \begin{enumerate}[\bfseries (i)]%
    \setcounter{enumii}{#1}
    \addtocounter{enumii}{-1}
    \setlength{\itemsep}{5mm}
    \setlength{\parskip}{8pt}
 }
 {
  \end{enumerate}
 }



\DeclareMathOperator{\cosec}{cosec}
\DeclareMathOperator{\Var}{Var}

\def\d{{\rm d}}
\def\e{{\rm e}}
\def\g{{\rm g}}
\def\h{{\rm h}}
\def\f{{\rm f}}
\def\p{{\rm p}}
\def\s{{\rm s}}
\def\t{{\rm t}}


\def\A{{\rm A}}
\def\B{{\rm B}}
\def\E{{\rm E}}
\def\F{{\rm F}}
\def\G{{\rm G}}
\def\H{{\rm H}}
\def\P{{\rm P}}


\def\bb{\mathbf b}
\def \bc{\mathbf c}
\def\bx {\mathbf x}
\def\bn {\mathbf n}

\newcommand{\low}{^{\vphantom{()}}}
%%%%% to lower suffices: $X\low_1$ etc


\newcommand{\subone}{ {\vphantom{\dot A}1}}
\newcommand{\subtwo}{ {\vphantom{\dot A}2}}




\def\le{\leqslant}
\def\ge{\geqslant}


\def\var{{\rm Var}\,}

\newcommand{\ds}{\displaystyle}
\newcommand{\ts}{\textstyle}




\begin{document}
\setcounter{page}{2}

 
\section*{Section A: \ \ \ Pure Mathematics}

%%%%%%%%%%Q1
\begin{question}
Prove that $\cos3\theta=4\cos^{3}\theta-3\cos\theta$. 


Show how the cubic equation 
\[
24x^{3}-72x^{2}+66x-19=0\tag{\ensuremath{*}}
\]
can be reduced to the form 
\[
4z^{3}-3z=k
\]
by means of the substitution $y=x+a$ and $z=by$, for suitable values
of the constants $a$ and $b$. Hence find the three roots of the
equation $(*)$, to three significant figures.


Show, by means of a counterexample, or otherwise, that not all cubic
equations of the form 
\[
x^{3}+\alpha x^{2}+\beta x+\gamma=0
\]
can be solved by this method.
\end{question}

%%%%%%%%%%Q2
\begin{question}
Let 
\begin{alignat*}{2}
\tan x & =\ \ \, \quad{\displaystyle \sum_{n=0}^{\infty}a_{n}x^{n}} &  & \text{ for small }x,\\
x\cot x & =1+\sum_{n=1}^{\infty}b_{n}x^{n}\quad &  & \text{ for small }x\text{ and not zero}.
\end{alignat*}
Using the relation 
\[
\cot x-\tan x=2\cot2x,\tag{\ensuremath{*}}
\]
 or otherwise, prove that $a_{n-1}=(1-2^{n})b_{n}$, for $n\geqslant1$. 


Let 
\[
x\mathrm{cosec}x=1+{\displaystyle \sum_{n=1}^{\infty}c_{n}x^{n}\quad\text{ for small }x\neq0. \qquad \qquad \, }
\]
Using a relation similar to $(*)$ involving $2\mathrm{cosec}2x$, or
otherwise, prove that 
\[
c_{n}=\frac{2^{n-1}-1}{2^{n}-1}\frac{1}{2^{n-1}}a_{n-1}\qquad(n\geqslant1).
\]
\end{question}

%%%%%%%%% Q3
\begin{question}
	The real numbers $x$ and $y$ are related to the real numbers $u$
	and $v$ by 
	\[
	2(u+\mathrm{i}v)=\mathrm{e}^{x+\mathrm{i}y}-\mathrm{e}^{-x-\mathrm{i}y}.
	\]
	Show that the line in the $x$-$y$ plane given by $x=a$, where $a$
	is a positive constant, corresponds to the ellipse 
	\[
	\left(\frac{u}{\sinh a}\right)^{2}+\left(\frac{v}{\cosh a}\right)^{2}=1
	\]
	in the $u$-$v$ plane. Show also that the line given by $y=b$, where
	$b$ is a constant and $0<\sin b<1,$ corresponds to one branch of
	a hyperbola in the $u$-$v$ plane. Write down the $u$ and $v$ coordinates
	of one point of intersection of the ellipse and hyperbola branch,
	and show that the curves intersect at right-angles at this point. 


	Make a sketch of the $u$-$v$ plane showing the ellipse, the hyperbola
	branch and the line segments corresponding to: 
\begin{itemize}
\setlength{\itemsep}{3mm}
\item[\bf (i)] $x=0$; 
\item[\bf (ii)] $y=\frac{1}{2}\pi,$ $\quad 0\leqslant x\leqslant a.$ 
\end{itemize}
\end{question}

%%%%%% Q4 

\begin{question}
The function $\mathrm{f}$ is defined by 
\[
\mathrm{f}(x)=\frac{\left(x-a\right)\left(x-b\right)}{\left(x-c\right)\left(x-d\right)}\qquad\left(x\neq c,\ x\neq d\right),
\]
where $a,b,c$ and $d$ are real and distinct, and $a+d\neq c+d$.
Show that 
\[
\frac{x\mathrm{f}'(x)}{\mathrm{f}(x)}=\left(1-\frac{a}{x}\right)^{-1}+\left(1-\frac{b}{x}\right)^{-1}-\left(1-\frac{c}{x}\right)^{-1}-\left(1-\frac{d}{x}\right)^{-1},
\]
$(x\neq0,x\neq a,x\neq b)$ and deduce that when $\left|x\right|$
is much larger than each of $\left|a\right|,\left|b\right|,\left|c\right|$
and $\left|d\right|,$ the gradient of $\mathrm{f}(x)$ has the same
sign as $(a+b-c-d).$ 


It is given that there is a real value of real value of $x$ for which
$\mathrm{f}(x)$ takes the real value $z$ if and only if 
\[
[\left(c-d\right)^{2}z+\left(a-c\right)\left(b-d\right)+\left(a-d\right)\left(b-c\right)]^{2}\geqslant4\left(a-c\right)\left(b-d\right)\left(a-d\right)\left(b-c\right).
\]
Describe briefly a method by which this result could be proved, but
do not attempt to prove \nolinebreak it. 


Given that $a<b$ and $a<c<d$, make sketches of the graph of $\mathrm{f}$
in the four distinct cases which arise, indicating the cases for which
the range of $\mathrm{f}$ is not the whole of $\mathbb{R}.$ 
\end{question}


%%%%%%%%% Q5
\begin{question}
	\begin{questionparts}
		\item 
Show that in polar coordinates, the gradient of any curve at the point
$(r,\theta)$ is 
\[
\left.\left(\frac{\mathrm{d}r}{\mathrm{d}\theta}\tan\theta+r\right)\right/\left(\frac{\mathrm{d}r}{\mathrm{d}\theta}-r\tan\theta\right).
\]



\noindent \begin{center}
\psset{xunit=1.0cm,yunit=1.0cm,algebraic=true,dotstyle=o,dotsize=3pt 0,linewidth=0.5pt,arrowsize=3pt 2,arrowinset=0.25} \begin{pspicture*}(-0.6,-3)(6.8,3) \psline(0,0)(6.54,0) \rput[tl](4.13,-0.22){$O$} \rput[tl](-0.47,0.07){$L$} \rput{-270}(5.75,0.08){\psplot[plotpoints=500]{-12}{12}{x^2/2/3}} \psline(2,1.5)(5.42,1.5) \psline(3.73,-0.74)(5.42,1.5) \psline[linewidth=0.4pt]{->}(3,1.5)(4,1.5) \psline[linewidth=0.4pt]{->}(5.42,1.5)(4.99,0.93) \psline(3.84,0.78)(6.62,2.05) \end{pspicture*}
\par\end{center}


\item A mirror is designed so that any ray of light which hits one side
of the mirror and which is parallel to a certain fixed line $L$ is
reflected through a fixed point $O$ on $L$. For any ray hitting
the mirror, the normal to the mirror at the point of reflection bisects
the angle between the incident ray and the reflected ray, as shown
in the figure. Prove that the mirror intersects any plane containing
$L$ in a parabola.
\end{questionparts}
	\end{question}
	
	%%%%%%%%% Q6
	\begin{question}
	The function $\mathrm{f}$ satisfies the condition $\mathrm{f}'(x)>0$
	for $a\leqslant x\leqslant b$, and $\mathrm{g}$ is the inverse of
	$\mathrm{f}.$ By making a suitable change of variable, prove that
	\[
	\int_{a}^{b}\mathrm{f}(x)\,\mathrm{d}x=b\beta-a\alpha-\int_{\alpha}^{\beta}\mathrm{g}(y)\,\mathrm{d}y,
	\]
	where $\alpha=\mathrm{f}(a)$ and $\beta=\mathrm{f}(b)$. Interpret
	this formula geometrically, in the case where $\alpha$ and $a$ are
	both positive. 


	Prove similarly and interpret (for $\alpha>0$ and $a>0$) the formula
	\[
	2\pi\int_{a}^{b}x\mathrm{f}(x)\,\mathrm{d}x=\pi(b^{2}\beta-a^{2}\alpha)-\pi\int_{\alpha}^{\beta}\left[\mathrm{g}(y)\right]^{2}\,\mathrm{d}y.
	\]
	
	 \end{question}
	 
	 %%%%%%%%% Q7
\begin{question}
	By means of the substitution $x^{\alpha},$ where $\alpha$ is a suitably
	chosen constant, find the general solution for $x>0$ of the differential
	equation 
	\[
	x\frac{\mathrm{d}^{2}y}{\mathrm{d}x^{2}}-b\frac{\mathrm{d}y}{\mathrm{d}x}+x^{2b+1}y=0,
	\]
	where $b$ is a constant and $b>-1$. 


	Show that, if $b>0$, there exist solutions which satisfy $y\rightarrow1$
	and $\mathrm{d}y/\mathrm{d}x\rightarrow0$ as $x\rightarrow0$, but
	that these conditions do not determine a unique solution. For what
	values of $b$ do these conditions determine a unique solution?
	
	
	\end{question}
	
	%%%%%%%%% Q8
	\begin{question}
		Let $\Omega=\exp(\mathrm{i}\pi/3).$ Prove that $\Omega^{2}-\Omega+1=0.$


		Two transformations, $R$ and $T$, of the complex plane are defined
		by 
		\[
		R:z\longmapsto\Omega^{2}z\qquad\mbox{ and }\qquad T:z\longmapsto\dfrac{\Omega z+\Omega^{2}}{2\Omega^{2}z+1}.
		\]
		 Verify that each of $R$ and $T$ permute the four point $z_{0}=0,$
		$z_{1}=1,$ $z_{2}=\Omega^{2}$ and $z_{3}=-\Omega.$ Explain, without
		explicitly producing a group multiplication table, why the smallest
		group of transformations which contains elements $R$ and $T$ has
		order at least 12. 


		Are there any permutations of these points which cannot be produced
		by repeated combinations of $R$ and $T$?
		\end{question}
		
		
%%%%%%%%% Q9
		\begin{question}
The matrix $\mathbf{F}$ is defined by 
\[
\mathbf{F}=\mathbf{I}+\sum_{n=1}^{\infty}\frac{1}{n!}t^{n}\mathbf{A}^{n},
\]
where $\mathbf{A}=\begin{pmatrix}-3 & -1\\
8 & 3
\end{pmatrix},$ and $t$ is a variable scalar. Evaluate $\mathbf{A}^{2},$ and show
that 
\[
\mathbf{F}=\mathbf{I}\cosh t+\mathbf{A}\sinh t.
\]
Show also that $\mathbf{F}^{-1}=\mathbf{I}\cosh t-\mathbf{A}\sinh t,$
and that $\dfrac{\mathrm{d}\mathbf{F}}{\mathrm{d}t}=\mathbf{FA}.$


The vector $\mathbf{r}=\begin{pmatrix}x(t)\\
y(t)
\end{pmatrix}$ satisfies the differential equation 
\[
\frac{\mathrm{d}\mathbf{r}}{\mathrm{d}t}+\mathbf{A}\mathbf{r}=\mathbf{0},
\]
with $x=\alpha$ and $y=\beta$ at $t=0.$ Solve this equation by
means of a suitable matrix integrating factor, and hence show that
\begin{alignat*}{1}
x(t) & =\alpha\cosh t+(3\alpha+\beta)\sinh t\\
y(t) & =\beta\cosh t-(8\alpha+3\beta)\sinh t.
\end{alignat*}
		\end{question}
		
	
%%%%%%%%%% 10
\begin{question}
State carefully the conditions which the fixed vectors $\mathbf{a,b,u}$
and $\mathbf{v}$ must satisfy in order to ensure that the line $\mathbf{r=a+}\lambda\mathbf{u}$
intersects the line $\mathbf{r=b+\mu}\mathbf{v}$ in exactly one point. 


Find the two values of the fixed scalar $b$ for which the planes
with equations 
\[
\left.\begin{array}{c}
x+y+bz=b+2\\
bx+by+z=2b+1
\end{array}\right\} \tag{\ensuremath{*}}
\]
do not intersect in a line. For other values of $b$, express the
line of intersection of the two planes in the form $\mathbf{r=a}+\lambda\mathbf{u},$
where $\mathbf{a\cdot u}=0$. 


Find the conditions which $b$ and the fixed scalars $c$ and $d$
must satisfy to ensure that there is exactly one point on the line
\[
\mathbf{r=}\left(\begin{array}{c}
0\\
0\\
c
\end{array}\right)+\mu\left(\begin{array}{c}
1\\
d\\
0
\end{array}\right)
\]
whose coordinates satisfy both equations $(*)$. 
			\end{question}
			
		
		
		
	
\newpage
\section*{Section B: \ \ \ Mechanics}


	
%%%%%%%%%% Q11
\begin{question}
	A lift of mass $M$ and its counterweight of mass $M$ are connected
	by a light inextensible cable which passes over a light frictionless
	pulley. The lift is constrained to move vertically between smooth
	guides. The distance between the floor and the ceiling of the lift
	is $h$. Initially, the lift is at rest, and the distance between
	the top of the lift and the pulley is greater than $h$. A small tile
	of mass $m$ becomes detached from the ceiling of the lift. Show that
	the time taken for it to fall to the floor is 
	\[
	t=\sqrt{\frac{\left(2M-m\right)h}{Mg}}.
	\]
	The collision between the tile and the lift floor is perfectly inelastic.
	Show that the lift is reduced to rest by the collision, and that the
	loss of energy of the system is $mgh$.

	\end{question}
	
%%%%%%%%%% Q12
\begin{question}	
A uniform rectangular lamina of sides $2a$ and $2b$ rests in a vertical
plane. It is supported in equilibrium by two smooth pegs fixed in
the same horizontal plane, a distance $d$ apart, so that one corner
of the lamina is below the level of the pegs. Show that if the distance
between this (lowest) corner and the peg upon which the side of length
$2a$ rests is less than $a$, then the distance between this corner
and the other peg is less than $b$. 


Show also that 
\[
b\cos\theta-a\sin\theta=d\cos2\theta,
\]
where $\theta$ is the acute angle which the sides of length $2b$
make with the horizontal.
\end{question}

%%%%%%%%%% Q13

\begin{question}
A body of mass $m$ and centre of mass $O$ is said to be \textit{dynamically
equivalent }to a system of particles of total mass $m$ and centre
of mass $O$ if the moment of inertia of the system of particles is
the same as the moment of inertia of the body, about any axis through
$O$. Show that this implies that the moment of inertia of the system
of particles is the same as that of the body about \textit{any }axis. 


Show that a uniform rod of length $2a$ and mass $m$ is dynamically
equivalent to a suitable system of three particles, one at each end
of the rod, and one at the midpoint. 


Use this result to deduce that a uniform rectangular lamina of mass
$M$ is dynamically equivalent to a system consisting of particles
each of mass $\frac{1}{36}M$ at the corners, particles each of mass
$\frac{1}{9}M$ at the midpoint of each side, and a particle of mass
$\frac{4}{9}M$ at the centre. Hence find the moment of inertia of
a square lamina, of side $2a$ and mass $M,$ about one of its diagonals. 


The mass per unit length of a thin rod of mass $m$ is proportional
to the distance from one end of the rod, and a dynamically equivalent
system consists of one particle at each end of the rod and one at
the midpoint. Write down a set of equations which determines these
masses, and show that, in fact, only two particles are required. 
\end{question}
	
%%%%%%%%%% Q14
\begin{question}
One end of a light inextrnsible string of length $l$ is fixed to
a point on the upper surface of a thin, smooth, horizontal table-top,
at a distance $(l-a)$ from one edge of the table-top. A particle
of mass $m$ is fixed to the other end of the string, and held a distance
$a$ away from this edge of the table-top, so that the string is horizontal
and taut. The particle is then released. Find the tension in the string
after the string has rotated through an angle $\theta,$ and show
that the largest magnitude of the force on the edge of the table top
is $8mg/\sqrt{3}.$  
\end{question}
	
	\newpage
\section*{Section C: \ \ \ Probability and Statistics}


%%%%%%%%%% Q15
\begin{question}
Two points are chosen independently at random on the perimeter (including
the diameter) of a semicircle of unit radius. What is the probability
that exactly one of them lies on the diameter?


Let the area of the triangle formed by the two points and the midpoint
of the diameter be denoted by the random variable $A$. 
\begin{questionparts}
\item Given that exactly one point lies on the diameter, show that the expected
value of $A$ is \nolinebreak $\left(2\pi\right)^{-1}$. 
\item Given that neither point lies on the diameter, show that the expected
value of $A$ is $\pi^{-1}$. {[}You may assume that if two points
are chosen at random on a line of length $\pi$ units, the probability
density function for the distance $X$ between the two points is $2\left(\pi-x\right)/\pi^{2}$
for $0\leqslant x\leqslant\pi.${]}
\end{questionparts}

Using these results, or otherwise, show that the expected value of
$A$ is $\left(2+\pi\right)^{-1}$. 
\end{question}

%%%%%%%%%% Q16
\begin{question}
Widgets are manufactured in batches of size $(n+N)$. Any widget has
a probability $p$ of being faulty, independent of faults in other
widgets. The batches go through a quality control procedure in which
a sample of size $n$, where $n\geqslant2$, is taken from each batch
and tested. If two or more widgets in the sample are found to be faulty,
all widgets in the batch are tested and all faults corrected. If fewer
than two widgets in the sample are found to be faulty, the sample
is replaced in the batch and no faults are corrected. Show that the
probability that the batch contains exactly $k$, where $k\leqslant N$,
faulty widgets after quality control is 
\[
\frac{\left[N+1+k\left(n-1\right)\right]N!}{\left(N-k+1\right)!k!}p^{k}\left(1-p\right)^{N+n-k},
\]
and verify that this formula also gives the correct answer for $k=N+1$. 


Show that the expected number of faulty widgets in a batch after quality
control is 
\[
\left[N+n+pN(n-1)\right]p(1-p)^{n-1}.
\]
\end{question}
\end{document}
