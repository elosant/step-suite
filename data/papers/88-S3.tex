
\documentclass[a4, 11pt]{report}


\pagestyle{myheadings}
\markboth{}{Paper III, 1988
\ \ \ \ \ 
\today 
}               

\RequirePackage{amssymb}
\RequirePackage{amsmath}
\RequirePackage{graphicx}
\RequirePackage{color}
\RequirePackage[flushleft]{paralist}[2013/06/09]



\RequirePackage{geometry}
\geometry{%
  a4paper,
  lmargin=2cm,
  rmargin=2.5cm,
  tmargin=3.5cm,
  bmargin=2.5cm,
  footskip=12pt,
  headheight=24pt}


\newcommand{\comment}[1]{{\bf Comment} {\it #1}}
%\renewcommand{\comment}[1]{}

\newcommand{\bluecomment}[1]{{\color{blue}#1}}
%\renewcommand{\comment}[1]{}
\newcommand{\redcomment}[1]{{\color{red}#1}}



\usepackage{epsfig}
\usepackage{pstricks-add}
\usepackage{tgheros} %% changes sans-serif font to TeX Gyre Heros (tex-gyre)
\renewcommand{\familydefault}{\sfdefault} %% changes font to sans-serif
%\usepackage{sfmath}  %%%% this makes equation sans-serif
%\input RexFigs


\setlength{\parskip}{10pt}
\setlength{\parindent}{0pt}

\newlength{\qspace}
\setlength{\qspace}{20pt}


\newcounter{qnumber}
\setcounter{qnumber}{0}

\newenvironment{question}%
 {\vspace{\qspace}
  \begin{enumerate}[\bfseries 1\quad][10]%
    \setcounter{enumi}{\value{qnumber}}%
    \item%
 }
{
  \end{enumerate}
  \filbreak
  \stepcounter{qnumber}
 }


\newenvironment{questionparts}[1][1]%
 {
  \begin{enumerate}[\bfseries (i)]%
    \setcounter{enumii}{#1}
    \addtocounter{enumii}{-1}
    \setlength{\itemsep}{5mm}
    \setlength{\parskip}{8pt}
 }
 {
  \end{enumerate}
 }



\DeclareMathOperator{\cosec}{cosec}
\DeclareMathOperator{\Var}{Var}

\def\d{{\rm d}}
\def\e{{\rm e}}
\def\g{{\rm g}}
\def\h{{\rm h}}
\def\f{{\rm f}}
\def\p{{\rm p}}
\def\s{{\rm s}}
\def\t{{\rm t}}


\def\A{{\rm A}}
\def\B{{\rm B}}
\def\E{{\rm E}}
\def\F{{\rm F}}
\def\G{{\rm G}}
\def\H{{\rm H}}
\def\P{{\rm P}}


\def\bb{\mathbf b}
\def \bc{\mathbf c}
\def\bx {\mathbf x}
\def\bn {\mathbf n}

\newcommand{\low}{^{\vphantom{()}}}
%%%%% to lower suffices: $X\low_1$ etc


\newcommand{\subone}{ {\vphantom{\dot A}1}}
\newcommand{\subtwo}{ {\vphantom{\dot A}2}}




\def\le{\leqslant}
\def\ge{\geqslant}


\def\var{{\rm Var}\,}

\newcommand{\ds}{\displaystyle}
\newcommand{\ts}{\textstyle}




\begin{document}
\setcounter{page}{2}

 
\section*{Section A: \ \ \ Pure Mathematics}

%%%%%%%%%%Q1
\begin{question}
Sketch the graph of 
\[
y=\frac{x^{2}\mathrm{e}^{-x}}{1+x},
\]
for $-\infty<x<\infty.$ 


Show that the value of 
\[
\int_{0}^{\infty}\frac{x^{2}\mathrm{e}^{-x}}{1+x}\,\mathrm{d}x
\]
lies between $0$ and $1$. 
\end{question}

%%%%%%%%%%Q2
\begin{question}
The real numbers $u_{0},u_{1},u_{2},\ldots$ satisfy the difference
equation 
\[
\alpha u_{n+2}+bu_{n+1}+cu_{n}=0\qquad(n=0,1,2,\ldots),
\]
where $a,b$ and $c$ are real numbers such that the quadratic equation
\[
ax^{2}+bx+c=0
\]
has two distinct real roots $\alpha$ and $\beta.$ Show that the
above difference equation is satisfied by the numbers $u_{n}$ defined
by 
\[
u_{n}=A\alpha^{n}+B\beta^{n},
\]
where 
\[
A=\frac{u_{1}-\beta u_{0}}{\alpha-\beta}\qquad\mbox{ and }\qquad B=\frac{u_{1}-\alpha u_{0}}{\beta-\alpha}.
\]
Show also, by induction, that these numbers provide the only solution. 


Find the numbers $v_{n}$ $(n=0,1,2,\ldots)$ which satisfy 
\[
8(n+2)(n+1)v_{n+2}-2(n+3)(n+1)v_{n+1}-(n+3)(n+2)v_{n}=0
\]
with $v_{0}=0$ and $v_{1}=1.$ 

\end{question}

%%%%%%%%% Q3
\begin{question}
	Give a parametric form for the curve in the Argand diagram determined
	by $\left|z-\mathrm{i}\right|=2.$ 


	Let $w=(z+\mathrm{i})/(z-\mathrm{i}).$ Find and sketch the locus,
	in the Argand diagram, of the point which represents the complex number
	$w$ when
\begin{itemize}
\setlength{\itemsep}{3mm}
\item[\bf (i)]  $\left|z-\mathrm{i}\right|=2;$ 
\item[\bf (ii)]  $z$ is real; 
\item[\bf (iii)]  $z$ is imaginary. 
\end{itemize}
\end{question}


%%%%%% Q4 

\begin{question}
	A kingdom consists of a vast plane with a central parabolic hill.
	In a vertical cross-section through the centre of the hill, with the
	$x$-axis horizontal and the $z$-axis vertical, the surface of the
	plane and hill is given by 
	\[
	x=\begin{cases}
	\dfrac{1}{2a}(a^{2}-x^{2}) & \mbox{ for }\left|x\right|\leqslant a,\\
	0 & \mbox{ for }\left|x\right|>a.
	\end{cases}
	\]
	The whole surface is formed by rotating this cross-section about the
	$z$-axis. In the $(x,z)$ plane through the centre of the hill, the
	king has a summer residence at $(-R,0)$ and a winter residence at
	$(R,0)$, where $R>a.$ He wishes to connect them by a road, consisting
	of the following segments: 

\begin{itemize}
\setlength{\itemsep}{3mm}
\item[\bf (i)] a path in the $(x,z)$ plane joining $(-R,0)$ to $(-b,(a^{2}-b^{2})/2a),$
	where $0\leqslant b\leqslant a.$
\item[\bf (ii)] a horizontal semicircular path joining the two points $(\pm b,(a^{2}-b^{2})/2a),$
	if $b\neq0;$
\item[\bf (iii)] a path in the $(x,z)$ plane joining $(b,(a^{2}-b^{2})/2a)$ to $(R,0).$
	\end{itemize}

	The king wants the road to be as short as possible. Advise him on
	his choice of $b.$
\end{question}


%%%%%%%%% Q5
\begin{question}
A firm of engineers obtains the right to dig and exploit an undersea
tunnel. Each day the firm borrows enough money to pay for the day's
digging, which costs $\pounds c,$ and to pay the daily interest of
$100k\%$ on the sum already borrowed. The tunnel takes $T$ days
to build, and, once finished, earns $\pounds d$ a day, all of which
goes to pay the daily interest and repay the debt until it is fully
paid. The financial transactions take place at the end of each day's
work. Show that $S_{n},$ the total amount borrowed by the end of
day $n$, is given by 
\[
S_{n}=\frac{c[(1+k)^{n}-1]}{k}
\]
for $n\leqslant T$. 


Given that $S_{T+m}>0,$ where $m>0,$ express $S_{T+m}$ in terms
of $c,d,k,T$ and $m.$ 


Show that, if $d/c>(1+k)^{T}-1,$ the firm will eventually pay off
the debt. 
	\end{question}
	
	%%%%%%%%% Q6
	\begin{question}
Let $\mathrm{f}(x)=\sin2x\cos x.$ Find the 1988th derivative of $\mathrm{f}(x).$


Show that the smallest positive value of $x$ for which this derivative
is zero is $\frac{1}{3}\pi+\epsilon,$ where $\epsilon$ is approximately
equal to 
\[
\frac{3^{-1988}\sqrt{3}}{2}.
\]
	 \end{question}
	 
	 %%%%%%%%% Q7
\begin{question}
For $n=0,1,2,\ldots,$ the functions $y_{n}$ satisfy the differential
equation 
\[
\frac{\mathrm{d}^{2}y_{n}}{\mathrm{d}x^{2}}-\omega^{2}x^{2}y_{n}=-(2n+1)\omega y_{n},
\]
where $\omega$ is a positive constant, and $y_{n}\rightarrow0$ and
$\mathrm{d}y_{n}/\mathrm{d}x\rightarrow0$ as $x\rightarrow+\infty$
and as $x\rightarrow-\infty.$ Verify that these conditions are satisfied,
for $n=0$ and $n=1,$ by 
\[
y_{0}(x)=\mathrm{e}^{-\lambda x^{2}}\qquad\mbox{ and }\qquad y_{1}(x)=x\mathrm{e}^{-\lambda x^{2}}
\]
for some constant $\lambda,$ to be determined. 


Show that 
\[
\frac{\mathrm{d}}{\mathrm{d}x}\left(y_{m}\frac{\mathrm{d}y_{n}}{\mathrm{d}x}-y_{n}\frac{\mathrm{d}y_{m}}{\mathrm{d}x}\right)=2(m-n)\omega y_{m}y_{n},
\]
and deduce that, if $m\neq n,$ 
\[
\int_{-\infty}^{\infty}y_{m}(x)y_{n}(x)\,\mathrm{d}x=0.
\] 
	\end{question}
	
	%%%%%%%%% Q8
	\begin{question}
Find the equations of the tangent and normal to the parabola $y^{2}=4ax$
at the point $(at^{2},2at).$


For $i=1,2,$ and 3, let $P_{i}$ be the point $(at_{i}^{2},2at_{i}),$
where $t_{1},t_{2}$ and $t_{3}$ are all distinct. Let $A_{1}$ be
the area of the triangle formed by the tangents at $P_{1},P_{2}$
and $P_{3},$ and let $A_{2}$ be the area of the triangle formed
by the normals at $P_{1},P_{2}$ and $P_{3}.$ Using the fact that
the area of the triangle with vertices at $(x_{1},y_{1}),(x_{2},y_{2})$
and $(x_{3},y_{3})$ is the absolute value of 
\[
\tfrac{1}{2}\det\begin{pmatrix}x_{1} & y_{1} & 1\\
x_{2} & y_{2} & 1\\
x_{3} & y_{3} & 1
\end{pmatrix},
\]
show that $A_{3}=(t_{1}+t_{2}+t_{3})^{2}A_{1}.$


Deduce a necessary and sufficient condition in terms of $t_{1},t_{2}$
and $t_{3}$ for the normals at $P_{1},P_{2}$ and $P_{3}$ to be
concurrent.
		
		\end{question}
		
		
%%%%%%%%% Q9
		\begin{question}
Let $G$ be a finite group with identity $e.$ For each element $g\in G,$
the order of $g$, $o(g),$ is defined to be the smallest positive
integer $n$ for which $g^{n}=e.$
\begin{questionparts}
\item Show that, if $o(g)=n$ and $g^{N}=e,$ then $n$ divides $N.$
\item Let $g$ and $h$ be elements of $G$. Prove that, for any integer
$m,$ 
\[
gh^{m}g^{-1}=(ghg^{-1})^{m}.
\]

\item Let $g$ and $h$ be elements of $G$, such that $g^{5}=e,h\neq e$
and $ghg^{-1}=h^{2}.$ Prove that $g^{2}hg^{-2}=h^{4}$ and find $o(h).$ 
\end{questionparts}

		\end{question}
		
	
%%%%%%%%%% 10
\begin{question}
Four greyhounds $A,B,C$ and $D$ are held at positions such that
$ABCD$ is a large square. At a given instant, the dogs are released
and $A$ runs directly towards $B$ at constant speed $v$, $B$ runs
directly towards $C$ at constant speed $v$, and so on. Show that
$A$'s path is given in polar coordinates (referred to an origin at
the centre of the field and a suitable initial line) by $r=\lambda\mathrm{e}^{-\theta},$
where $\lambda$ is a constant. 


Generalise this result to the case of $n$ dogs held at the vertices
of a regular $n$-gon ($n\geqslant3$). 
			\end{question}
			
		
		
		
	
\newpage
\section*{Section B: \ \ \ Mechanics}


	
%%%%%%%%%% Q11
\begin{question}
A uniform ladder of length $l$ and mass $m$ rests with one end in
contact with a smooth ramp inclined at an angle of $\pi/6$ to the
vertical. The foot of the ladder rests, on horizontal ground, at a
distance $l/\sqrt{3}$ from the foot of the ramp, and the coefficient
of friction between the ladder and the ground is $\mu.$ The ladder
is inclined at an angle $\pi/6$ to the horizontal, in the vertical
plane containing a line of greatest slope of the ramp. A labourer
of mass $m$ intends to climb slowly to the top of the ladder. 


\noindent \begin{center}
\psset{xunit=1.2cm,yunit=1.2cm,algebraic=true,dotstyle=o,dotsize=3pt 0,linewidth=0.5pt,arrowsize=3pt 2,arrowinset=0.25} \begin{pspicture*}(-1.32,-1.38)(8.14,5.24) \pspolygon[linewidth=0pt,fillcolor=black,fillstyle=solid,opacity=0.1,linecolor=white](0,5)(0,0)(3,0) \psline(0,0)(8,0) \psline(0,5)(3,0) \psline(1.81,1.98)(7,0) \psline(3,0)(3,1) \pscustom{\parametricplot{1.5707963267948966}{2.1112158270654806}{0.7*cos(t)+3|0.7*sin(t)+0}\lineto(3,0)\closepath} \pscustom{\parametricplot{2.7769413914856056}{3.141592653589793}{1*cos(t)+7|1*sin(t)+0}\lineto(7,0)\closepath} \rput[tl](2.55,1.15){$\tfrac{1}{6}\pi$} \rput[tl](5.4,0.5){$\tfrac{1}{2}\pi$} \rput[tl](0.5,2.32){$\text{ramp}$} \psline{->}(2.98,-0.46)(7,-0.48) \psline{->}(7,-0.48)(2.98,-0.46) \rput[tl](4.68,-0.6){$l\sqrt{3}$} \end{pspicture*}
\par\end{center}
\begin{questionparts}
\item Find the value of $\mu$ if the ladder slips as soon as the labourer
reaches the midpoint. 
\item Find the minimum value of $\mu$ which will ensure that the labourer
can reach the top of the ladder. 
\end{questionparts}

	\end{question}
	
%%%%%%%%%% Q12
\begin{question}	
A smooth billiard ball moving on a smooth horizontal table strikes
another identical ball which is at rest. The coefficient of restitution
between the balls is $e(<1)$. Show that after the collision the angle
between the velocities of the balls is less than $\frac{1}{2}\pi.$


Show also that the maximum angle of deflection of the first ball is
\[
\sin^{-1}\left(\frac{1+e}{3-e}\right).
\]
\end{question}

%%%%%%%%%% Q13

\begin{question}
A goalkeeper stands on the goal-line and kicks the football directly
into the wind, at an angle $\alpha$ to the horizontal. The ball has
mass $m$ and is kicked with velocity $\mathbf{v}_{0}.$ The wind
blows horizontally with constant velocity $\mathbf{w}$ and the air
resistance on the ball is $mk$ times its velocity relative to the
wind velocity, where $k$ is a positive constant. Show that the equation
of motion of the ball can be written in the form 
\[
\frac{\mathrm{d}\mathbf{v}}{\mathrm{d}t}+k\mathbf{v}=\mathbf{g}+k\mathbf{w},
\]
where $\mathbf{v}$ is the ball's velocity relative to the ground,
and $\mathbf{g}$ is the acceleration due to gravity. 


By writing down horizontal and vertical equations of motion for the
ball, or otherwise, find its position at time $t$ after it was kicked. 


On the assumption that the goalkeeper moves out of the way, show that
if $\tan\alpha=\left|\mathbf{g}\right|/(k\left|\mathbf{w}\right|),$
then the goalkeeper scores an own goal. 
\end{question}
	
%%%%%%%%%% Q14
\begin{question}
A small heavy bead can slide smoothly in a vertical plane on a fixed
wire with equation 
\[
y=x-\frac{x^{2}}{4a},
\]
where the $y$-axis points vertically upwards and $a$ is a positive
constant. The bead is projected from the origin with initial speed
$V$ along the wire. 

\begin{questionparts}
\item Show that for a suitable value of $V$, to be determined, a motion
is possible throughout which the bead exerts no pressure on the wire. 
\item Show that $\theta,$ the angle between the particle's velocity at
time $t$ and the $x$-axis, satisfies 
\[
\frac{4a^{2}\dot{\theta}^{2}}{\cos^{6}\theta}+2ga(1-\tan^{2}\theta)=V^{2}.
\]

\end{questionparts}
\end{question}
	
	\newpage
\section*{Section C: \ \ \ Probability and Statistics}


%%%%%%%%%% Q15
\begin{question}
Each day, books returned to a library are placed on a shelf in order
of arrival, and left there. When a book arrives for which there is
no room on the shelf, that book and all books subsequently returned
are put on a trolley. At the end of each day, the shelf and trolley
are cleared. There are just two-sizes of book: thick, requiring two
units of shelf space; and thin, requiring one unit. The probability
that a returned book is thick is $p$, and the probability that it
is thin is $q=1-p.$ Let $M(n)$ be the expected number of books that
will be put on the shelf, when the length of the shelf is $n$ units
and $n$ is an integer, on the assumption that more books will be
returned each day than can be placed on the shelf. Show, giving reasoning,
that 
\begin{itemize}
\setlength{\itemsep}{3mm}
\item[\bf (i)] $M(0)=0;$
\item[\bf (ii)] $M(1)=q;$
\item[\bf (iii)] $M(n)-qM(n-1)-pM(n-2)=1,$ for $n\geqslant2.$
\end{itemize}
Verify that a possible solution to these equations is 
\[
M(n)=A(-p)^{n}+B+Cn,
\]
where $A,B$ and $C$ are numbers independent of $n$ which you should
express in terms of $p$. 
\end{question}

%%%%%%%%%% Q16
\begin{question}
Balls are chosen at random without replacement from an urn originally
containing $m$ red balls and $M-m$ green balls. Find the probability
that exactly $k$ red balls will be chosen in $n$ choices $(0\leqslant k\leqslant m,0\leqslant n\leqslant M).$ 


The random variables $X_{i}$ $(i=1,2,\ldots,n)$ are defined for
$n\leqslant M$ by 
\[
X_{i}=\begin{cases}
0 & \mbox{ if the \ensuremath{i}th ball chosen is green}\\
1 & \mbox{ if the \ensuremath{i}th ball chosen is red. }
\end{cases}
\]
Show that 
\begin{itemize}
\setlength{\itemsep}{2mm}
\item[\bf (i)] $\mathrm{P}(X_{i}=1)=\dfrac{m}{M}.$
\item[\bf (ii)] $\mathrm{P}(X_{i}=1\mbox{ and }X_{j}=1)=\dfrac{m(m-1)}{M(M-1)}$,
for $i\neq j$. 
\end{itemize}

Find the mean and variance of the random variable $X$ defined by
\[
X=\sum_{i=1}^{n}X_{i}.
\]
\end{question}
\end{document}
