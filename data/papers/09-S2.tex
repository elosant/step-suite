\documentclass[a4, 11pt]{report}


\pagestyle{myheadings}
\markboth{}{Paper II, 2009
\ \ \ \ \ 
\today 
}               

\RequirePackage{amssymb}
\RequirePackage{amsmath}
\RequirePackage{graphicx}
\RequirePackage{color}
\RequirePackage[flushleft]{paralist}[2013/06/09]



\RequirePackage{geometry}
\geometry{%
  a4paper,
  lmargin=2cm,
  rmargin=2.5cm,
  tmargin=3.5cm,
  bmargin=2.5cm,
  footskip=12pt,
  headheight=24pt}


\newcommand{\comment}[1]{{\bf Comment} {\it #1}}
%\renewcommand{\comment}[1]{}

\newcommand{\bluecomment}[1]{{\color{blue}#1}}
%\renewcommand{\comment}[1]{}
\newcommand{\redcomment}[1]{{\color{red}#1}}



\usepackage{epsfig}
\usepackage{pstricks-add}
\usepackage{tgheros} %% changes sans-serif font to TeX Gyre Heros (tex-gyre)
\renewcommand{\familydefault}{\sfdefault} %% changes font to sans-serif
%\usepackage{sfmath}  %%%% this makes equation sans-serif
%\input RexFigs


\setlength{\parskip}{10pt}
\setlength{\parindent}{0pt}

\newlength{\qspace}
\setlength{\qspace}{20pt}


\newcounter{qnumber}
\setcounter{qnumber}{0}

\newenvironment{question}%
 {\vspace{\qspace}
  \begin{enumerate}[\bfseries 1\quad][10]%
    \setcounter{enumi}{\value{qnumber}}%
    \item%
 }
{
  \end{enumerate}
  \filbreak
  \stepcounter{qnumber}
 }


\newenvironment{questionparts}[1][1]%
 {
  \begin{enumerate}[\bfseries (i)]%
    \setcounter{enumii}{#1}
    \addtocounter{enumii}{-1}
    \setlength{\itemsep}{5mm}
    \setlength{\parskip}{8pt}
 }
 {
  \end{enumerate}
 }



\DeclareMathOperator{\cosec}{cosec}
\DeclareMathOperator{\Var}{Var}

\def\d{{\mathrm d}}
\def\e{{\mathrm e}}
\def\g{{\mathrm g}}
\def\h{{\mathrm h}}
\def\f{{\mathrm f}}
\def\p{{\mathrm p}}
\def\q{{\mathrm q}}
\def\s{{\mathrm s}}
\def\t{{\mathrm t}}


\def\A{{\mathrm A}}
\def\B{{\mathrm B}}
\def\E{{\mathrm E}}
\def\F{{\mathrm F}}
\def\G{{\mathrm G}}
\def\H{{\mathrm H}}
\def\P{{\mathrm P}}


\def\bb{\mathbf b}
\def \bc{\mathbf c}
\def\bx {\mathbf x}
\def\bn {\mathbf n}

\newcommand{\low}{^{\vphantom{()}}}
%%%%% to lower suffices: $X\low_1$ etc


\newcommand{\subone}{ {\vphantom{\dot A}1}}
\newcommand{\subtwo}{ {\vphantom{\dot A}2}}




\def\le{\leqslant}
\def\ge{\geqslant}
\def\arcosh{{\rm arcosh}\,}


\def\var{{\rm Var}\,}

\newcommand{\ds}{\displaystyle}
\newcommand{\ts}{\textstyle}
\def\half{{\textstyle \frac12}}
\def\l{\left(}
\def\r{\right)}



\begin{document}
\setcounter{page}{2}

 
\section*{Section A: \ \ \ Pure Mathematics}

%%%%%%%%%%Q1
\begin{question}
Two curves have equations 
$\; x^4+y^4=u\;$ and $\; xy = v\;$, where $u$ and $v$ are positive
constants. State the equations of the lines of symmetry of each curve.

 The curves 
intersect at the distinct points $A$, $B$, $C$ and $D$
(taken anticlockwise from $A$).
The coordinates of $A$ are $(\alpha,\beta)$,
where $\alpha>\beta>0$. Write down, in terms of 
$\alpha$ and $\beta$, the coordinates of
$B$, $C$ and~$D$. 


Show that the quadrilateral $ABCD$ is a rectangle and find its area
in terms of $u$ and $v$ only. Verify that, for the case
$u=81$ and $v=4$, the area is $14$.
\end{question}

%%%%%%%%%%Q2
\begin{question}
The curve $C$ has equation
\[
y= a^{\sin (\pi \e^ x)}\,,
\]
where $a>1$.

\begin{questionparts}
\item Find the coordinates of the stationary points on $C$.

\item Use the approximations $\e^t \approx 1+t$ and $\sin t \approx t$ 
(both valid for small values of $t$) 
 to show that 
\[
y\approx 1-\pi x \ln a \;
\]
for small values of $x$.

\item Sketch $C$.

\item By approximating $C$ by means of straight lines joining consecutive
stationary points, show that the area between $C$ and the $x$-axis
between the $k$th and $(k+1)$th maxima is approximately
\[
\Big( \frac {a^2+1}{2a} \Big)
\ln \Big ( 1+ \big( k-\tfrac34)^{-1} \Big)\,.
\]
\end{questionparts}
\end{question}

%%%%%%%%% Q3
\begin{question}
Prove that          
\[
\tan \left ( \tfrac14 \pi -\tfrac12 x \right)\equiv \sec x -\tan x\,.
\tag{$*$}
\]

\begin{questionparts}
\item Use $(*)$ to find the value of        
$\tan\frac18\pi\,$. Hence show that 
\[
\tan \tfrac{11}{24} \pi = \frac{\sqrt3 + \sqrt2 -1}{\sqrt3
  -\sqrt6+1}\;.
\]

\item Show that 
\[
 \frac{\sqrt3 + \sqrt2 -1}{\sqrt3
  -\sqrt6+1}= 2+\sqrt2+\sqrt3+\sqrt6\,.
\]


\item Use $(*)$ to show that
\[
\tan \tfrac1{48}\pi = 
\sqrt{16+10\sqrt2+8\sqrt3 +6\sqrt6 \ }-2-\sqrt2-\sqrt3-\sqrt6\,.
\]
\end{questionparts}
\end{question}

%%%%%% Q4 
\begin{question}
The polynomial  $\p(x)$ is  of degree 9 and  
$\p(x)-1$ is  exactly divisible by $(x-1)^5$.

\begin{questionparts}
\item Find the value of $\p(1)$.

\item Show that $\p'(x)$ is exactly divisible by $(x-1)^4$.

\item 
 Given also 
 that $\p(x)+1$ is exactly divisible by
  $(x+1)^5$, find $\p(x)$.

\end{questionparts}
\end{question}

%%%%%%%%% Q5
\begin{question}
Expand and simplify $(\sqrt{x-1}+1)^2\,$.
\begin{questionparts}
\item Evaluate
\[
\int_{5}^{10}
 \frac{ \sqrt{x+2\sqrt{x-1} \;} + \sqrt{x-2\sqrt{x-1} \;} }
{\sqrt{x-1}} \,\d x\;.
\] 
\item
Find the total area between the curve
\[
y= \frac{\sqrt{x-2\sqrt{x-1}\;}}{\sqrt{x-1}\;}
\] and the $x$-axis
between the points $x=\frac54$ and $x=10$.
\item Evaluate
\[
\int_{\frac54}^{10}
 \frac{ \sqrt{x+2\sqrt{x-1}\;} + \sqrt{x-2\sqrt{x+1}+2 \;} }
{\sqrt{x^2-1}  } \;\d x\;.
\] 

\end{questionparts}
	\end{question}
	
%%%%%%%%% Q6
\begin{question}
The Fibonacci sequence $F_1$, $F_2$, $F_3$, $\ldots$ is defined by
$F_1=1$, $F_2= 1$  and 
\[
F_{n+1} = F_n+F_{n-1} \qquad\qquad (n\ge 2).
\]
Write down the  values of $F_3$, $F_4$, $\ldots$, $F_{10}$.

Let $\displaystyle   S=\sum_{i=1}^\infty \dfrac1 {F_i}\,$.


\begin{questionparts}
\item 
Show that $\displaystyle \frac 1{F_i}>\frac1{2F_{i-1}}\,$ for $i\ge4$
  and deduce that $S>3\,$.

Show also that  $S<3\frac23\,$.

\item 
Show further that $3.2 <S<3.5\,$.
\end{questionparts}
\end{question}
	
%%%%%%%%% Q7
\begin{question}
Let $y= (x-a)^n \e^{bx} \sqrt{1+x^2}\,$, where $n$ and  $a$ are 
constants and $b$ is a non-zero
constant. Show that
\[
\frac{\d y}{\d x} = \frac{(x-a)^{n-1} \e^{bx} \q(x)}{\sqrt{1+x^2}}\,,
\]
where $\q(x)$ is a cubic polynomial. 

Using this result, determine:
\begin{questionparts}
\item $\displaystyle \int \frac {(x-4)^{14} \e^{4x}(4x^3-1)}
{\sqrt{1+x^2\;}} \, \d x\,;$
\item $\displaystyle \int \frac{(x-1)^{21}\e^{12x}(12x^4-x^2-11)}
{\sqrt{1+x^2\;}}\,\d x\,;$
\item $\displaystyle \int \frac{(x-2)^{6}\e^{4x}(4x^4+x^3-2)}
{\sqrt{1+x^2\;}}\,\d x\,.$

\end{questionparts}
\end{question}
		
%%%%%%%%% Q8
\begin{question}
The non-collinear points $A$, $B$ and $C$ have position vectors 
$\bf a$, $\bf b$ and $\bf c$, respectively. The points $P$ and
$Q$ have position vectors $\bf p$ and $\bf q$, respectively, given by 
\[
{\bf p}= \lambda {\bf a} +(1-\lambda){\bf b}
\text{ \ \ \ and \ \ \ }
{\bf q}= \mu {\bf a} +(1-\mu){\bf c}
\]
where $0<\lambda<1$ and $\mu>1$. Draw a diagram showing $A$, $B$, $C$,
$P$ and $Q$.

Given that $CQ\times BP = AB\times AC$, find $\mu$ in terms of
$\lambda$, and show that, for all values of $\lambda$,  the 
the line  $PQ$  passes through the fixed 
point $D$, with position vector ${\bf d}$ given by
${\bf d= -a +b +c}\,$.
 What 
can be said about the quadrilateral $ABDC$?
\end{question}	
		

		
	
\newpage
\section*{Section B: \ \ \ Mechanics}


	
%%%%%%%%%% Q9
\begin{question}
\begin{questionparts}

\item
A uniform lamina $OXYZ$ is in the shape of the trapezium 
 shown in the diagram.
It is right-angled at $O$ and
$Z$, and $OX$ is parallel to $YZ$. The lengths of the sides are
given by $OX=9\,$cm, $XY=41\,$cm, $YZ=18\,$cm and $ZO=40\,$cm.
Show that its centre of mass is a distance $7\,$cm from the edge $OZ$.
\begin{center}
\psset{xunit=1.0cm,yunit=1.0cm,algebraic=true,dotstyle=o,dotsize=3pt 0,linewidth=0.3pt,arrowsize=3pt 2,arrowinset=0.25} \begin{pspicture*}(2.34,0.37)(6.59,5.33) \psaxes[labelFontSize=\scriptstyle,xAxis=true,yAxis=true,labels=none,Dx=1,Dy=1,ticksize=-2pt 0,subticks=2]{->}(0,0)(2.34,0.37)(6.59,5.33)[$x$,140] [$y$,-40] \psline(3.1,4.85)(6.04,4.85) \psline(6.04,4.85)(4.77,0.87) \psline(3.1,4.85)(3.1,0.87) \psline(3.1,0.87)(4.77,0.87) \rput[tl](4.39,5.2){$18$} \rput[tl](6.2,5.2){$Y$} \rput[tl](2.8,5.2){$Z$} \rput[tl](2.67,0.92){$O$} \rput[tl](4.86,0.95){$X$} \rput[tl](5.61,3.1){$41$} \rput[tl](3.88,0.8){$9$} \rput[tl](2.6,3.15){$40$} \end{pspicture*}
\end{center}
\item
The diagram shows a tank with no lid made of thin sheet metal. The base 
$OXUT$, the back $OTWZ$ and the front $XUVY$ are rectangular, 
and each end is a trapezium as in part \textbf{(i)} . The width of 
the tank is $d\,$cm.
\begin{center}
\psset{xunit=1cm,yunit=1.1cm,algebraic=true,dotstyle=o,dotsize=3pt 0,linewidth=0.3pt,arrowsize=3pt 2,arrowinset=0.25} \begin{pspicture*}(2.42,0.4)(10.98,6.5) \psaxes[labelFontSize=\scriptstyle,xAxis=true,yAxis=true,labels=none,Dx=1,Dy=1,ticksize=-2pt 0,subticks=2]{->}(0,0)(2.42,0.4)(10.93,6.45)[$x$,140] [$y$,-40] \psline(3.1,4.85)(6.04,4.85) \psline(6.04,4.85)(4.77,0.87) \psline(3.1,4.85)(3.1,0.87) \psline(3.1,0.87)(4.77,0.87) \rput[tl](4.4,4.76){$18$} \rput[tl](6,5.25){$Y$} \rput[tl](2.73,5.15){$Z$} \rput[tl](2.67,0.92){$O$} \rput[tl](4.82,0.89){$X$} \rput[tl](5.61,3.1){$41$} \rput[tl](3.88,0.8){$9$} \rput[tl](2.6,3.1){$40$} \psline(3.1,4.85)(8,6) \psline(8,6)(10.71,6) \psline(10.71,6)(6.04,4.85) \psline(10.71,6)(9.45,1.99) \psline(9.45,1.99)(4.77,0.87) \psline[linestyle=dashed,dash=3pt 1pt](8,6)(8,2) \psline[linestyle=dashed,dash=3pt 1pt](8,2)(3.1,0.87) \psline[linestyle=dashed,dash=3pt 1pt](8,2)(9.45,1.99) \psline(8,6)(8,5.33) \rput[tl](7.88,6.35){$W$} \rput[tl](10.67,6.35){$V$} \rput[tl](9.64,2.16){$U$} \rput[tl](8.05,2.35){$T$} \rput[tl](7.43,1.45){$d$} \rput[tl](5.91,5.9){$d$} \end{pspicture*}
\end{center}

\vspace*{3mm}
 Show that the centre of mass of the tank, when empty,
is a distance 
\[
\frac {3(140+11d)}{5(12+d)}\,\text{cm}
\]
from the back of the tank.


The tank is then filled with a liquid.  
The mass per unit volume of this liquid  is $k$ times the 
mass per unit area of the sheet metal.
 In the case $d=20$, find an expression for the
distance of the centre of mass of the filled tank from the back of the tank.


\end{questionparts}
	\end{question}
	
%%%%%%%%%% Q10 
\begin{question}$\,$
\begin{center}
\psset{xunit=1.5cm,yunit=1.5cm,algebraic=true,dotstyle=o,dotsize=3pt 0,linewidth=0.3pt,arrowsize=3pt 2,arrowinset=0.25} \begin{pspicture*}(-2.73,-2.6)(3.4,1.82) 
\psline{->}(-2.73,0)(2.5,0)
\psline{->}(0,-2.2)(0,1.5)
\rput[tl](2.55,0.05){$x$}
\rput[tl](-0.05,1.75){$y$}
\rput[tl](-2.09,-0.3){$P_1$} \rput[tl](-1.11,-0.3){$P_2$} \rput[tl](-0.55,-0.6){$P_3$} \rput[tl](-0.55,-1.6){$P_4$} \rput[tl](0.07,0.3){$O$} \psline{->}(-2.1,0.4)(-1.6,0.4) \psline{->}(0.4,-1.9)(0.4,-1.4) \rput[tl](-1.55,0.45){$u$} \rput[tl](0.32,-1.2){$u$} \begin{scriptsize} \psdots[dotsize=18pt 0,dotstyle=*](-1,0) \psdots[dotsize=18pt 0,dotstyle=*](-2,0) \psdots[dotsize=18pt 0,dotstyle=*](0,-0.7) \psdots[dotsize=18pt 0,dotstyle=*](0,-1.7) \end{scriptsize} \end{pspicture*}
\end{center}

Four particles $P_1$, $P_2$, $P_3$ and
$P_4$, of masses $m_1$, $m_2$, $m_3$ and $m_4$, respectively, are arranged
on smooth horizontal axes as shown in the diagram. 

Initially, $P_2$ and $P_3$ are stationary, and 
both $P_1$ and $P_4$ are moving towards $O$ with speed $u$.
Then $P_1$ and $P_2$ collide, at the same moment as
$P_4$ and $P_3$ collide. Subsequently, $P_2$ and 
$P_3$ collide at $O$, as do $P_1$ and $P_4$ some time later. 
The coefficient of
restitution between each pair of particles is~$e$, and $e>0$.

Show that initially $P_2$ and $P_3$ are equidistant from $O$.
\end{question}

%%%%%%%%%% Q11

\begin{question}
A train consists of an engine and $n$ trucks.
 It is travelling along a straight
horizontal section of track. The mass of the engine and of each 
 truck is $M$. The resistance to motion
 of the engine and of each 
 truck is $R$, which is constant. 
The maximum power at which the engine can work
is  $P$. 

Obtain an expression for the acceleration  of the train
when its speed is $v$ and  the engine is working at maximum power.


The train starts from rest with the engine working at
maximum power. Obtain an expression for the
time $T$ taken to reach a given speed $V$, and 
show that this speed is only achievable  if
\[
P>(n+1)RV\,.
\]
\begin{questionparts}

\item In the case when $(n+1) RV/P$ is small, use the approximation
$\ln (1-x) \approx -x -\frac12 x^2$ (valid for small $ x $)
to obtain the  approximation
\[
PT\approx \tfrac 12 (n+1) MV^2\,
\]
and interpret this result.

\item In the general case, the distance moved from rest in time $T$ is $X$.
{\em Write down}, with explanation,
an equation relating $P$, $T$, $X$, $M$, $V$, $R$ and $n$ and hence
show that
\[
X= \frac{2PT - (n+1)MV^2}{2(n+1)R}
\,.
\]


\end{questionparts}
\end{question}
	

	
	\newpage
\section*{Section C: \ \ \ Probability and Statistics}


%%%%%%%%%% Q12
\begin{question}
A continuous random variable $X$ has probability density function
given by
\[
\f(x) = 
\begin{cases}
0 & \mbox{for } x<0 \\
k\e^{-2 x^2} & \mbox{for } 0\le x< \infty \;,\\
\end{cases}
\] 
where $k$ is a constant.

\begin{questionparts}
\item Sketch the graph of $\f(x)$.
\item Find the value of  $k$.
\item Determine $\E(X)$ and  $\var(X)$.
\item Use statistical tables  to find,
to three significant figures, the median value of $X$.
\end{questionparts}
\end{question}

%%%%%%%%%% Q13
\begin{question}
Satellites are launched using two different types of rocket: the
Andover and the Basingstoke. The Andover has four engines and the 
Basingstoke has six. Each engine has a probability~$p$ of
failing during any given launch. 
After the launch, the rockets are retrieved
and repaired by replacing some or all of the engines. The cost
of replacing each engine is $K$.

For the Andover, 
if  more than one engine fails, all four engines are replaced.
Otherwise, only the  failed engine (if there is one) is replaced.
Show that the expected repair cost for a single launch
using the Andover is 
\[
4Kp(1+q+q^2-2q^3) \ \ \ \ \ \ \ \ \ \ \ \ \ (q=1-p)
\tag{*}
\]

For the Basingstoke, if more than two engines fail, all six engines
are replaced. Otherwise only the failed engines (if there are any)
are replaced.
Find, in a form similar to $(*)$, the expected repair cost for 
a single launch using the  Basingstoke.

Find the values of $p$ for which the expected repair cost for the Andover
is  $\frac23$  of the expected repair cost for the Basingstoke.
\end{question}

\end{document}
