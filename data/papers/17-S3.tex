\documentclass[a4, 11pt]{report}


%\pagestyle{myheadings}
%\markboth{}{Paper III, 2007  second vetter draft 
%\ \ \ \ \ 
%\today 
%}               
\pagestyle{empty}

\usepackage{pstricks-add}
\usepackage{epsfig}

%\usepackage{relsize}

\RequirePackage{amssymb}
\RequirePackage{amsmath}
\RequirePackage{graphicx}
\RequirePackage{color}
\RequirePackage{xcolor}


\RequirePackage[flushleft]{paralist}[2013/06/09]



\RequirePackage{geometry}
\geometry{%
  a4paper,
  lmargin=2cm,
  rmargin=2.5cm,
  tmargin=3.5cm,
  bmargin=2.5cm,
  footskip=12pt,
  headheight=24pt}


\newcommand{\comment}[1]{{\bf Comment} {\it #1}}
%\renewcommand{\comment}[1]{}

\newcommand{\bct}[1]{{\color{blue}#1}}
%\renewcommand{\comment}[1]{}
\newcommand{\rct}[1]{{\color{red}#1}}






\setlength{\parskip}{10pt}
\setlength{\parindent}{0pt}

\newlength{\qspace}
\setlength{\qspace}{15pt}


\newcounter{qnumber}
\setcounter{qnumber}{0}

\newenvironment{question}%
 {\vspace{\qspace}
  \begin{enumerate}[\bfseries 1\quad][10]%
    \setcounter{enumi}{\value{qnumber}}%
    \item%
 }
{
  \end{enumerate}
  \filbreak
  \stepcounter{qnumber}
 }


\newenvironment{questionparts}[1][1]%
 {
  \begin{enumerate}[\bfseries (i)]%
    \setcounter{enumii}{#1}
    \addtocounter{enumii}{-1}
    \setlength{\itemsep}{5mm}
    \setlength{\parskip}{3pt}
 }
 {
  \end{enumerate}
 }



\DeclareMathOperator{\cosec}{cosec}
\DeclareMathOperator{\Var}{Var}

\def\d{{\rm d}}
\def\e{{\rm e}}
\def\g{{\rm g}}
\def\h{{\rm h}}
\def\f{{\rm f}}
\def\p{{\rm p}}
\def\s{{\rm s}}
\def\t{{\rm t}}


\def\A{{\rm A}}
\def\B{{\rm B}}
\def\C{{\rm C}}
\def\E{{\rm E}}
\def\F{{\rm F}}
\def\G{{\rm G}}
\def\H{{\rm H}}
\def\P{{\rm P}}
\def\T{{\rm T}}

\def\bb{\mathbf b}
\def \bc{\mathbf c}
\def\bx {\mathbf x}
\def\bn {\mathbf n}

\makeatletter
\newcommand{\raisemath}[1]{\mathpalette{\raisem@th{#1}}}
\newcommand{\raisem@th}[3]{\raisebox{#1}{$#2#3$}}
\makeatother
%%%To raise suffices: e.g.  $\Pi_{\raisemath{2pt}{-}}$.




\def\le{\leqslant}
\def\ge{\geqslant}


\def\var{{\rm Var}\,}

\newcommand{\ds}{\displaystyle}
\newcommand{\ts}{\textstyle}




\begin{document}

\setcounter{page}{2}

 
\section*{Section A: \ \ \ Pure Mathematics}

%%%%%%%%%%%%%% Q1

\begin{question}
\begin{questionparts}
\item
Prove that, for any positive integers $n$ and $r$,
\[
\frac{1}{^{n+r}\C_{r+1}}
=\frac{r+1}{r} \left(\frac{1}{^{n+r-1}\C_{r}}-\frac{1}{^{n+r}\C_{r}}\right).
\]
Hence determine
\[
\sum_{n=1}^{\infty}{\frac{1}{^{n+r}\C_{r+1}}} 
\,,
\]
and deduce that \  
$\displaystyle \sum_{n=2}^\infty \frac 1 {^{n+2}\C_3} = \frac12\,$.
\item Show that, 
for $n \ge 3\,$, 
\[
\frac{3!}{n^3}  <  \frac{1}{^{n+1}\C_{3}}
\ \ \ \ \ 
\text{and}
\ \ \ \ \ 
\frac{20}{^{n+1}\C_3} - \frac{1}{^{n+2}\C_{5}}
< \frac{5!}{n^3} 
\,.
\]



By summing these inequalities for $n \ge 3\,$, show that
\[
\frac{115}{96} < \sum_{n=1}^{\infty}{\frac{1}{n^3}} <
 \frac{116}{96} \, .
\]
\end{questionparts}
{\bf Note: } $^n\C_r$ is another notation for $\displaystyle \binom n r $.


\end{question}

%%%%%%%%%%%%%% Q2
\begin{question}
The transformation $R$ in the complex plane is a rotation (anticlockwise) 
by an angle $\theta$ 
about the point represented by the complex number $a$.
The transformation $S$ in the complex plane is a rotation 
(anticlockwise)
by an angle $\phi$ 
about the point represented by the complex number $b$.

\begin{questionparts}
\item
The point $P$ is represented by the complex number~$z$.
Show that the image of $P$ under $R$ is represented by 
\[
\e^{{\mathrm i} \theta}z  + a(1-\e^{{\rm i} \theta})\,.
\]


\item Show that 
the transformation $SR$ (equivalent to $R$ followed by $S$)
is a rotation about the point represented by
$c$, where
\[ 
%\textstyle
c\,\sin
 \tfrac12 (\theta+\phi)
= a\,\e^{ {\mathrm i}\phi/2}\sin  \tfrac12\theta
+  b\,\e^{-{\mathrm i} \theta/2}\sin  \tfrac12 \phi 
\,,
\]
provided $\theta+\phi \ne 2n\pi$ for any integer $n$.


What is the transformation $SR$ if $\theta +\phi = 2\pi$?
\item
Under what circumstances is $RS =SR$?
\end{questionparts}

\end{question}


%%%%%%%%%%% Q3
\begin{question}
Let $\alpha$, $\beta$, $\gamma$ and $\delta$ be the roots of the 
quartic equation
\[
x^4 +px^3 +qx^2 +r x +s =0
\,.
\]

You are given that,  
for any such equation, 
  $\,\alpha \beta + \gamma\delta\,$, \, $\alpha\gamma+\beta\delta\,$
and $\,\alpha \delta + \beta\gamma\,$ 
satisfy a 
cubic equation of the form 
\[
y^3+Ay^2+ 
(pr-4s)y+ (4qs-p^2s -r^2)
=0
\,.
\]
Determine  $A\,$.

\vspace{3mm}
Now consider  the quartic equation given by $p=0\,$, $q= 3\,$, $r=-6\,$
and $s=10\,$.
\begin{questionparts}
\item


Find the value of $\alpha\beta + \gamma \delta$,
given that it is the largest root of the corresponding cubic equation.

\item
Hence, using the values of $q$ and $s$, find  
the value of $(\alpha +\beta)(\gamma+\delta)\,$
and
 the value of~$\alpha\beta$ given that $\alpha\beta >\gamma\delta\,$.

\item
Using these results, and the values of $p$ and $r$,
solve the quartic equation.  
\end{questionparts}

\end{question}


%%%%%%%%%Q4
\begin{question}
For any 
 function $\f$  satisfying 
$\f(x) > 0$,  we define the  {\em geometric mean}, F,
 by   
\[
\F(y)
\;
= 
\mbox{
\fontsize{12}{15.6}\selectfont
$\e$}
\mbox{ 
\fontsize{14}{15.6}\selectfont
$
^{\!
\raisemath {3pt}
{\frac{1}{y}
\!
\int_{\raisemath{-1pt}{0}}^{\raisemath{1pt}{y}}
 \ln \f(x) \, \d x}
}
$ } 
\ \ \ \ \ \ (y>0)\,.
\]

\begin{questionparts}
\item The function f satisfies 
$\f(x) > 0$ and  
$a$ is a positive number with $a\ne1$. Prove that
\[
\F(y) =
\mbox{
\fontsize{12}{15.6}\selectfont
$a$}
\mbox{ 
\fontsize{14}{15.6}\selectfont
$
^{
\!
\raisemath {3pt}
{\frac{1}{y}
\!
\int_{\raisemath{-1pt}{0}}^{\raisemath{1pt}{y}} 
\log_a  \f(x) \, \d x}
}
$ }
.
\]

\item The functions f and  g satisfy 
$\f(x) > 0$ and $\g(x) > 0$, and  the function $\h$ is 
defined by $\h(x) = \f(x)\g(x)$. Their
 geometric means are F, G and H, respectively.
Show that  $\H(y)= \F(y) \G(y)\,$.

\item Prove that, for any positive number  $b$, 
the geometric mean of  $b^x$ 
is  
$\sqrt{b^y}\,$.

\item Prove that, if $\f(x)>0$ and
 the geometric mean of $\f(x)$ 
is $\sqrt{\f(y)}\,$, 
then $\f(x) = b^x$ for some positive number $b$.

\end{questionparts}
\end{question}


%%%%%%%%%%%%%% Q5
\begin{question}
The  point with cartesian coordinates $(x,y)$ lies on a
curve with polar equation $r=\f(\theta)\,$. 
Find an  expression for $\dfrac{\d y}{\d x}$ in terms of $\f(\theta)$,
$\f'(\theta)$ and $\tan\theta\,$.

Two curves, with polar equations $r=\f(\theta)$ and 
$r=\g(\theta)$, meet at right angles.
Show
that where they meet 
\[
\f'(\theta) \g'(\theta) +\f(\theta)\g(\theta) = 0 \,.
\]

The curve $C$ has polar equation $r=\f(\theta)$ and passes through
the point given by $r=4$, $\theta = - \frac12\pi$. 
For each positive value of $a$,   
the curve with polar equation
$r= a(1+\sin\theta)$ 
meets~$C$ at right angles. Find $\f(\theta)\,$.

Sketch on a single diagram the three curves 
with polar equations $r= 1+\sin\theta\,$, \   
$r= 4(1+\sin\theta)$  and $r=\f(\theta)\,$.


\end{question}

\vspace{-3mm}
%%%%%%%%%%%%%% Q6


\begin{question}
{\em In this question, you are not permitted to use any properties of 
trigonometric functions or inverse trigonometric functions.}

The function $\T$ is defined for  $x>0$ by
\[
\T(x) = \int_0^x \! \frac 1 {1+u^2} \, \d u\,,
\]
and $\displaystyle T_\infty = 
 \int_0^\infty \!\! \frac 1 {1+u^2} \, \d u\,$ (which has a finite value).

\begin{questionparts}
\item 
By making an appropriate substitution in the integral for $\T(x)$,
 show that \[\T(x) = \T_\infty - \T(x^{-1})\,.\]


\item
Let  $v= \dfrac{u+a}{1-au}$, where $a$ is a constant. Verify that, for
$u\ne a^{-1}$, 
\[
\frac{\d v}{\d u} = \frac{1+v^2}{1+u^2}
\,.
\]

Hence show that, for $a>0$ and $x< \dfrac1a\,$, 
\[
\T(x)  = \T\left(\frac{x+a}{1-ax}\right) -\T(a) \,.
\]
Deduce that
\[
\T(x^{-1}) 
 = 2\T_\infty -\T\left(\frac{x+a}{1-ax}\right) 
-\T(a^{-1})  
\]
and hence that, for
 $b>0$ and $y>\dfrac1b\,$, 
\[
\T(y)   =2\T_\infty - \T\left(\frac{y+b}{by-1}\right) - \T(b) \,.
\]

\item Use the above results to show that 
$\T(\sqrt3)= \tfrac23 \T_\infty \,$
and 
$\T(\sqrt2 -1)= \frac14 \T_\infty\,$.
\end{questionparts}
\end{question}


%%%%%%%%%%%%%  Q7
\begin{question}
Show that the point $T$ with coordinates
\[
\left( \frac{a(1-t^2)}{1+t^2} \; , \; \frac{2bt}{1+t^2}\right)
\tag{$*$}
\]
(where $a$ and $b$ are non-zero) lies on the ellipse
\[
\frac{x^2}{a^2} + \frac{y^2}{b^2} =1
\,.
\]
\begin{questionparts}
\item
The line $L$ is the  
tangent to the ellipse at $T$.
The point $(X,Y)$ lies on $L$,
 and $X^2\ne a^2$. Show that
\[
(a+X)bt^2 -2aYt +b(a-X) =0
\,.
\]
Deduce that
if $a^2Y^2>(a^2-X^2)b^2$, then
there are
 two distinct
lines through $(X,Y)$ that are tangents to the ellipse. Interpret
this
result geometrically. Show, by means of a sketch,
that the result  holds also  if $X^2=a^2\,$.

\item
The distinct points  $P$ and $Q$ are given by 
$(*)$, with $t=p$ and $t=q$,
respectively.
The tangents to the ellipse at 
$P$
and 
$Q$
meet at the point   with coordinates $(X,Y)$, where $X^2\ne a^2\,$.
Show that \[
(a+X)pq = a-X\]
 and find an expression for $p+q$ in terms
of $a$, $b$, $X$ and $Y$.



Given that the tangents meet the $y$-axis
at points $(0,y_1)$ and $(0,y_2)$, where $y_1+y_2 = 2b\,$, show that
\[
\frac{X^2}{a^2} +\frac{Y}{b}= 1  
\,.
\]
\end{questionparts}
\end{question}
%%%%%%%%%%%%%% Q8
\begin{question}

Prove that, for any numbers $a_1$, $a_2$, $\ldots$\,,
and $b_1$, $b_2$, $\ldots$\,, and for $n\ge1$,
\[
\sum_{m=1}^n a_m(b_{m+1} -b_m) = a_{n+1}b_{n+1} -a_1b_1
-\sum_{m=1}^n b_{m+1}(a_{m+1} -a_m)
\,.
\]

\begin{questionparts}
\item
By setting $b_m = \sin mx$, show that
\[
\sum_{m=1}^n \cos (m+\tfrac12)x 
= \tfrac12 
\big(\sin (n+1)x - \sin x \big)
\cosec \tfrac12 x
\,.
\]
{\bf Note:} 
$\sin A - \sin B = 
\displaystyle 
2 
\cos \big( \tfrac{{\displaystyle A+B\vphantom{_1}}}
{\displaystyle 2\vphantom{^1}} \big)\,
\sin\big( \tfrac{{\displaystyle A-B\vphantom{_1}}}{\displaystyle 2\vphantom{^1}} \big)\,
$.

\item
Show that
\[
\sum_{m=1}^n m\sin mx 
= 
\big (p \sin(n+1)x +q \sin nx\big)
\cosec^2 \tfrac12 x 
\,,
\]
where $p$ and $q$ are to be determined in terms of $n$.

\vspace{3mm}
{\bf Note:} 
$2\sin A \sin B =  \cos (A-B) - \cos (A+B)\,$;
\\[2mm] 
\phantom
{\bf Note:} 
$2\cos A \sin B =  \sin (A+B) - \sin (A-B)\,$.

\end{questionparts}
\end{question}

\newpage
\section*{Section B: \ \ \ Mechanics}

%%%%%%%%%%%%%% Q9
\begin{question}

Two  particles  $A$  and  $B$ of masses	 $m$ and $2 m$,  respectively,
are connected  by a light spring	 of natural	 
length	  $a$   and  modulus of	 elasticity $\lambda$.	They 
are placed on a	  smooth horizontal table with	$AB$ 
perpendicular to the edge of the table,	and $A$ is  held on
the edge of the table.	  Initially  the  
spring	 is at  its  natural  length.  

Particle  $A$ is released. At a	 time	 $t$  later, 
particle $A$  has dropped a distance $y$  and particle $ B$ 
has moved a distance $x$ from its initial position (where $x<a$).
 Show  that $ y + 2x= \frac12 gt^2$. 

The value of $\lambda$ is such  that 
particle $B$  reaches the edge of the table at a time $T$ given by
$T= \sqrt{6a/g\,}\,$.  
By considering the total energy of the system (without solving any
differential equations), 
show that the speed of particle $B$ at this
time is $\sqrt{2ag/3\,}\,$.

\end{question}
%%%%%%%%%%%%%% Q10
\begin{question}
A uniform rod $PQ$ of mass $m$ and length $3a$ is freely  hinged at $P$.

The rod is held horizontally and a particle of mass $m$ is placed on
top of the rod at a distance~$\ell$ from $P$, where $\ell <2a$. 
The coefficient of friction between the rod and the particle is $\mu$.

The rod is then released. Show that, while the particle does not 
slip along the rod,
\[
(3a^2+\ell^2)\dot \theta^2 = g(3a+2\ell)\sin\theta \,,
\]
where $\theta$ is the angle through which the rod 
has turned,
and the dot denotes the time derivative.

Hence, or otherwise, find an expression for $\ddot \theta$ and
 show 
that
the normal reaction of the rod on the 
particle is non-zero
 when~$\theta$ is acute. 

Show further that, when the particle is on the point of slipping,
\[
\tan\theta = \frac{\mu a (2a-\ell)}{2(\ell^2 + a\ell +a^2)}
\,.
\]  

What happens at the moment the rod is released  if, instead,  $\ell>2a$?


\end{question}






\begin{question}
A railway truck, initially at rest, can move forwards
 without friction on a long straight \mbox{horizontal} track. 
On the truck, $n$ guns are mounted parallel to the track and 
facing backwards, where $n>1$. 
Each of the guns is loaded with a single projectile of mass $m$. 
The mass of the truck and guns (but not including the 
projectiles) is $M$. 

When a gun is fired, the projectile leaves its 
muzzle horizontally with a speed  $v-V$ relative to the ground, 
where~$V$ is the
speed of the truck immediately before the gun is fired.

\begin{questionparts}

\item All $n$ guns are fired simultaneously. 
Find the speed, $u$, with which the truck moves, 
and show that the  kinetic energy, $K$, which is gained by the system
(truck, guns and projectiles) is given by
\[
K=  \tfrac{1}{2}nmv^2\left(1   +\frac{nm}{M} \right)
.
\]

\item Instead, the guns are fired one at a time. 
Let $u_r$ be the speed  of the truck
when $r$ guns have been fired, so that $u_0= 0$. 
Show that, for $1\le r \le n\,$,  
\[
u_r - u_{r-1} = \frac{mv}{M+(n-r)m}
\tag{$*$}
\]
and hence that $u_n < u\,$.


\item 
Let $K_r$ 
be the total kinetic energy of the system 
when $r$ guns have been fired (one at a time), so that $K_0 = 0$. 
Using $(*)$, show that, for $1\le r\le n\,$,
\[
K_r -K_{r-1} = \tfrac 12 mv^2 + \tfrac12 mv (u_r-u_{r-1})
\]
and hence 
show that
\[
K_n = \tfrac{1}{2}nmv^2 +\tfrac{1}{2}mvu_n
\,.
\] 
Deduce that $K_n<K$.

\end{questionparts}
\end{question}



\newpage


\section*{Section C: \ \ \ Probability and Statistics}
%%%%%%%%%%%%%% Q12
\begin{question}
The discrete random variables $X$ and $Y$ can each take the values 
$1$, $\ldots\,$, $n$ (where $n\ge2$). Their joint probability distribution is given by
\[
\P(X=x, \  Y=y) = k(x+y) \,,
\] 
where $k$ is a  constant.
\begin{questionparts}

\item Show that \[
\P(X=x) = \dfrac{n+1+2x}{2n(n+1)}\,. 
\]
Hence determine whether $X$ and $Y$
are independent.


\item Show that the covariance of $X$ and $Y$ is negative.

\end{questionparts}
\end{question}
 
%%%%%%%%%%%%%% Q13

\begin{question}
The random variable $X$ has mean $\mu$ and 
variance $\sigma^2$, and  the function ${\rm V}$
is defined, for $-\infty<x<\infty$,   by
\[
{\rm V}(x) = \E \big( (X-x)^2\big)
.
\]
Express ${\rm V}(x)$ in terms of $x$, $ \mu$ and $\sigma$.


The random variable $Y$ is defined by $Y={\rm V}(X)$.
Show that 
\[
\E(Y) = 2 \sigma^2
%\text{ \ \ and \ \ }
%\Var(Y) = \E(X-\mu)^4 -\sigma^4
.
\tag{$*$}
\]

Now suppose that $X$ is uniformly distributed on the interval $0\le x \le1\,$.
Find ${\rm V}(x)\,$. 
Find also the probability density function of $Y\!$ and use it to
verify that $(*)$ holds in this case.
\end{question}
\end{document}


