\documentclass[a4, 11pt]{report}


\pagestyle{myheadings}
\markboth{}{Paper I, 2011
\ \ \ \ \ 
\today 
}               

\RequirePackage{amssymb}
\RequirePackage{amsmath}
\RequirePackage{graphicx}
\RequirePackage{color}
\RequirePackage[flushleft]{paralist}[2013/06/09]



\RequirePackage{geometry}
\geometry{%
  a4paper,
  lmargin=2cm,
  rmargin=2.5cm,
  tmargin=3.5cm,
  bmargin=2.5cm,
  footskip=12pt,
  headheight=24pt}


\newcommand{\comment}[1]{{\bf Comment} {\it #1}}
%\renewcommand{\comment}[1]{}

\newcommand{\bluecomment}[1]{{\color{blue}#1}}
%\renewcommand{\comment}[1]{}
\newcommand{\redcomment}[1]{{\color{red}#1}}



\usepackage{epsfig}
\usepackage{pstricks-add}
\usepackage{tgheros} %% changes sans-serif font to TeX Gyre Heros (tex-gyre)
\renewcommand{\familydefault}{\sfdefault} %% changes font to sans-serif
%\usepackage{sfmath}  %%%% this makes equation sans-serif
%\input RexFigs


\setlength{\parskip}{10pt}
\setlength{\parindent}{0pt}

\newlength{\qspace}
\setlength{\qspace}{20pt}


\newcounter{qnumber}
\setcounter{qnumber}{0}

\newenvironment{question}%
 {\vspace{\qspace}
  \begin{enumerate}[\bfseries 1\quad][10]%
    \setcounter{enumi}{\value{qnumber}}%
    \item%
 }
{
  \end{enumerate}
  \filbreak
  \stepcounter{qnumber}
 }


\newenvironment{questionparts}[1][1]%
 {
  \begin{enumerate}[\bfseries (i)]%
    \setcounter{enumii}{#1}
    \addtocounter{enumii}{-1}
    \setlength{\itemsep}{5mm}
    \setlength{\parskip}{8pt}
 }
 {
  \end{enumerate}
 }



\DeclareMathOperator{\cosec}{cosec}
\DeclareMathOperator{\Var}{Var}

\def\d{{\mathrm d}}
\def\e{{\mathrm e}}
\def\g{{\mathrm g}}
\def\h{{\mathrm h}}
\def\f{{\mathrm f}}
\def\p{{\mathrm p}}
\def\s{{\mathrm s}}
\def\t{{\mathrm t}}


\def\A{{\mathrm A}}
\def\B{{\mathrm B}}
\def\E{{\mathrm E}}
\def\F{{\mathrm F}}
\def\G{{\mathrm G}}
\def\H{{\mathrm H}}
\def\P{{\mathrm P}}


\def\bb{\mathbf b}
\def \bc{\mathbf c}
\def\bx {\mathbf x}
\def\bn {\mathbf n}

\newcommand{\low}{^{\vphantom{()}}}
%%%%% to lower suffices: $X\low_1$ etc


\newcommand{\subone}{ {\vphantom{\dot A}1}}
\newcommand{\subtwo}{ {\vphantom{\dot A}2}}




\def\le{\leqslant}
\def\ge{\geqslant}
\def\arcosh{{\rm arcosh}\,}


\def\var{{\rm Var}\,}

\newcommand{\ds}{\displaystyle}
\newcommand{\ts}{\textstyle}
\def\half{{\textstyle \frac12}}
\def\l{\left(}
\def\r{\right)}



\begin{document}
\setcounter{page}{2}

 
\section*{Section A: \ \ \ Pure Mathematics}

%%%%%%%%%%Q1
\begin{question}
\begin{questionparts}
\item Show that the gradient of the curve
 $\; \dfrac a x + \dfrac by =1$, where $b\ne0$, 
 is $\; -\dfrac{ay^2}{bx^2}\,$.

The point $(p,q)$ lies on both the straight line 
$ax+by=1$ and the curve $\dfrac a x + \dfrac by =1\,$, 
where $ab\ne0$.
Given that, at this point, the line and the curve have the
same gradient, show that $ p=\pm q\,$. 

Show further that
either $
(a-b)^2 =1\,$ or $(a+b)^2 =1\,$.

\item Show that if the straight line 
$ax+by=1$, where $ab\ne0$,
is a normal to 
the curve $\dfrac a x - \dfrac by =1$, then 
$a^2-b^2 = \frac12\,$.

\end{questionparts}
\end{question}

%%%%%%%%%%Q2
\begin{question}
The number $E$ is defined by
$\displaystyle 
E= \int_0^1 \frac{\e^x}{1+x} \, \d x\,.$

Show that 
\[
\int_0^1 \frac{x \e^x}{1+x} \, \d x = \e -1 -E\,
,\]
and evaluate $\displaystyle \int_0^1 \frac{x^2\e^x}{1+x} \, \d x$
in terms of $\e$ and $E$.

Evaluate also, in terms of $E$ and $\rm e$ as appropriate:
\begin{questionparts}
\item 
$\displaystyle
\int_0^1
\frac{\e^{\frac{1-x}{1+x}}}{1+x}\, \d x\,;
$
\item
$\displaystyle 
\int_1^{\sqrt2} \frac {\e^{x^2}}x \, \d x \, $.
\end{questionparts}

\end{question}

%%%%%%%%% Q3
\begin{question}
Prove the identity
\[
4\sin\theta \sin(\tfrac13\pi-\theta) \sin (\tfrac13\pi+\theta)= 
\sin 3\theta\,
.
\tag{$*$}\]

\begin{questionparts}
\item By differentiating $(*)$, or otherwise, show that
\[
 \cot \tfrac19\pi  - \cot \tfrac29\pi + \cot \tfrac49\pi = \sqrt3\,.
\]
\item
By setting $\theta = \frac16\pi-\phi$ in $(*)$, or otherwise,
obtain  a similar identity for $\cos3\theta$ and deduce that
\[
\cot \theta \cot (\tfrac13\pi-\theta) \cot (\tfrac13\pi+\theta) =\cot3\theta\,.
\]
Show that
\[
\cosec \tfrac19\pi  -\cosec \tfrac59\pi +\cosec \tfrac79\pi = 2\sqrt3\,.
\]

\end{questionparts}
\end{question}

%%%%%% Q4 
\begin{question}
The distinct points $P$ and $Q$, with coordinates $(ap^2,2ap)$
and $(aq^2,2aq)$ respectively, lie on the curve $y^2=4ax$. 
The tangents to the curve at $P$ and $Q$ meet at the point $T$.
Show that $T$ has coordinates $\big(apq, a(p+q)\big)$.
You may assume that $p\ne0$ and $q\ne0$.

The point $F$ has coordinates $(a,0)$ and $\phi$ is
the angle $TFP$. Show that 
\[
\cos\phi = \frac{pq+1}{\sqrt{(p^2+1)(q^2+1)}\ }
\]
and deduce that the line $FT$ bisects the angle $PFQ$. 
\end{question}

%%%%%%%%% Q5
\begin{question}
Given that $0<k<1$, show with the help of a sketch that the equation
\[
\sin x = k x
\tag{$*$}\]
has a unique solution in the range $0<x<   \pi$. 

Let  
\[
I= \int_0^\pi \big\vert \sin x -kx\big\vert \, \d x\,.
\]
Show that
\[
I= \frac{\pi^2 \sin\alpha }{2\alpha} -2\cos\alpha - \alpha \sin\alpha\,,
\]
where 
$\alpha$ is the unique solution of $(*)$.

Show that $I$, regarded as a function of $\alpha$, has
a unique stationary value and that this stationary value
 is a minimum. Deduce that
the smallest value of $I$ is
\[
 -2 \cos \frac{\pi}{\sqrt2}\,
.\]
	\end{question}
	
%%%%%%%%% Q6
\begin{question}
Use the binomial expansion to show that  the 
coefficient of $x^r$ in the expansion  of 
$(1-x)^{-3}$ is~$\frac12 (r+1)(r+2)\,$.

\begin{questionparts}
\item
Show that the coefficient of $x^r$ in the expansion of 
\[
\frac{1-x+2x^2}{(1-x)^3}
\]
is $r^2+1$ and hence find the sum of the series
\[
1+\frac22+\frac54+\frac{10}8+\frac{17}{16}+\frac{26}{32}+\frac{37}{64} 
+\frac{50}{128}+ \cdots \,.
\]
\item Find the sum of the series
\[
1+2+\frac94+2+\frac{25}{16}+\frac{9}{8}+\frac{49}{64} 
+ \cdots \,. 
\]

\end{questionparts} 
\end{question}
	
%%%%%%%%% Q7
\begin{question}
In this question, you may assume that $\ln (1+x) \approx x -\frac12 x^2$
when $\vert x \vert $ is small.

The height of the water in a tank at time $t$  is $h$. 
The initial height of the water 
is $H$ and
water flows   into the tank  at a constant rate.
The cross-sectional area of the 
tank is constant. 


\begin{questionparts}
\item 
Suppose
that water leaks out 
at a rate proportional to the height of the water in the tank,
and that when the height
 reaches $\alpha^2 H$, where $\alpha$ is a constant greater than 1,
the height remains constant. 
Show that
\[
\frac {\d h}{\d t } = k( \alpha^2 H -h)\,,
\]
for some positive constant $k$. Deduce
that
the time $T$ taken for the water to reach height $\alpha H$
is given by 
\[
kT = \ln \left(1+\frac1\alpha\right)\,,
\]
and that $kT\approx \alpha^{-1}$ for large values of $\alpha$.  

\item Suppose  that  the rate at which water leaks out of  the 
tank is proportional to $\sqrt h$ (instead of $h$),
and that when the height
 reaches $\alpha^2 H$, where $\alpha$ is a constant greater than 1,
the height remains constant. 
 Show that 
the time $T'$ taken for the water to reach height $\alpha H$ is
given by
\[
cT'=2\sqrt H \left( 1 - \sqrt\alpha +\alpha \ln
\left(1+\frac1 {\sqrt\alpha} \right)\right)\,
\]
 for some positive constant $c$,
and that $ cT'\approx \sqrt H$ for large values of $\alpha$. 
\end{questionparts} 
\end{question}
		
%%%%%%%%% Q8
\begin{question}
\begin{questionparts}
\item
The numbers $m$ and $n$ satisfy
\[
m^3=n^3+n^2+1\,.
\tag{$*$}
\]

\begin{itemize}\setlength{\labelsep}{3.5 mm}
\item[{\bf (a)}]
Show that $m>n$. Show also that 
 $m<n+1$ if and only if $2n^2+3n>0\,$. 
Deduce that $n<m<n+1$ unless $-\frac32 \le n \le 0\,$.

\item[{\bf (b)}] 
Hence show that the only solutions of $(*)$ for which both $m$ and 
$n$ are integers are $(m,n) = (1,0)$ and $(m,n)= (1,-1)$.

\end{itemize}        
\item
Find all integer solutions of the equation
\[
p^3=q^3+2q^2-1\,.
\]
\end{questionparts}
\end{question}	
		

		
	
\newpage
\section*{Section B: \ \ \ Mechanics}


	
%%%%%%%%%% Q9
\begin{question}
A particle is projected at an angle $\theta$ above the horizontal
from a point on a horizontal plane. The particle
just passes over two walls that are
 at horizontal distances $d_1$ and $d_2$ from the point
of projection  and are of heights $d_2$ and $d_1$, respectively.
Show that
\[
\tan\theta = \frac{d_1^2+d_\subone d_\subtwo +d_2^2}{d_\subone d_\subtwo}\,.
\]

Find (and simplify)
an  expression 
 in terms of $d_1$ and $d_2$ only 
for the range of the particle.
	\end{question}
	
%%%%%%%%%% Q10 
\begin{question}	
A particle, $A$, is dropped from a point $P$ which is
at a height $h$ above a horizontal 
plane. 
A~second particle, $B$, is dropped from $P$ and first collides
with $A$ after $A$ has bounced on the plane and before $A$
reaches $P$ again. The bounce and the collision  are both
perfectly elastic. Explain why the speeds of $A$ and $B$ immediately
before the first collision are the same.

The masses of $A$ and $B$ are $M$ and $m$, respectively, where $M>3m$,
and the speed of the particles immediately before
the first collision is $u$. 
Show that both particles move upwards after their 
first collision and that the maximum height of $B$ above the 
plane after the first collision
 and before the second  collision is 
\[
 h+ \frac{4M(M-m)u^2}{(M+m)^2g}\,.
\]
\end{question}

%%%%%%%%%% Q11

\begin{question}
A thin non-uniform bar $AB$ of length $7d$
 has centre of mass 
at a point $G$, where $AG=3d$.
A light inextensible string has one end attached to $A$ and
the other end attached to $B$. The string is hung over a smooth
peg $P$ and the bar hangs freely in equilibrium  with $B$ lower than~$A$.
Show that 
\[
3\sin\alpha = 4\sin\beta\,,
\]
where $\alpha$ and $\beta$ are the angles $PAB$ and $PBA$, respectively.

Given that $\cos\beta=\frac45$ and that $\alpha$ is acute, 
find in terms of $d$
the length of the string and
show that the angle of inclination of the bar to the horizontal
is $\arctan \frac17\,$.
\end{question}
	

	
	\newpage
\section*{Section C: \ \ \ Probability and Statistics}


%%%%%%%%%% Q12
\begin{question}
I am selling raffle tickets for $\pounds1$ per ticket.
In the  queue for tickets,
there are $m$ people each with a single $\pounds1$ coin and $n$
people each with a single $\pounds2$ coin. Each person in the 
queue wants to buy a single raffle ticket and each arrangement
of people in the queue is equally likely to occur.
Initially, I have no coins and
a large supply of tickets. I stop selling tickets if 
I cannot give the required change.

\begin{questionparts}
\item In the case $n=1$ and $m\ge1$, find the probability
that I am able to sell one ticket to each person in the queue.
\item By considering the first three people in the queue, 
show that the probability that I am able to sell one ticket
to each person in the queue in the case $n=2$ and $m\ge2$ is
$\dfrac{m-1}{m+1}\,$.
\item 
Show that the probability that I am able to sell one ticket
to each person in the queue in the case $n=3$ and $m\ge3$ is
$\dfrac{m-2}{m+1}\,$.
\end{questionparts}
\end{question}

%%%%%%%%%% Q13
\begin{question}
In this question, you may use without proof
the following
result:
\[
\int \sqrt{4-x^2}\, \d x 
= 2 \arcsin (\tfrac12 x ) + \tfrac 12 x \sqrt{4-x^2} +c\,.
\] 

A random variable $X$ has probability
density function $\f$ given by
\[
\f(x) = 
\begin{cases}
2k & -a\le x <0     \\[3mm]
k\sqrt{4-x^2} & \phantom{-} 0\le x \le 2 \\[3mm]
0 & \phantom{-}\text{otherwise},
\end{cases}
\]
where $k$ and $a$ are positive constants.

\begin{questionparts}
\item Find, in terms of $a$, the mean of $X$. 
\item Let $d$ be the value of $X$ such that $\P(X> d)=\frac1 {10}\,$. Show
that $d<0$ if $2a>   9\pi$ and find an expression for $d$ in terms of $a$
in this case.
\item Given that $d=\sqrt 2$, find $a$.
\end{questionparts}
\end{question}

\end{document}
