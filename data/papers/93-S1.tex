\documentclass[a4, 11pt]{report}


\pagestyle{myheadings}
\markboth{}{Paper I, 1993
\ \ \ \ \ 
\today 
}               

\RequirePackage{amssymb}
\RequirePackage{amsmath}
\RequirePackage{graphicx}
\RequirePackage{color}
\RequirePackage[flushleft]{paralist}[2013/06/09]



\RequirePackage{geometry}
\geometry{%
  a4paper,
  lmargin=2cm,
  rmargin=2.5cm,
  tmargin=3.5cm,
  bmargin=2.5cm,
  footskip=12pt,
  headheight=24pt}


\newcommand{\comment}[1]{{\bf Comment} {\it #1}}
%\renewcommand{\comment}[1]{}

\newcommand{\bluecomment}[1]{{\color{blue}#1}}
%\renewcommand{\comment}[1]{}
\newcommand{\redcomment}[1]{{\color{red}#1}}



\usepackage{epsfig}
\usepackage{pstricks-add}
\usepackage{pst-coil}
\usepackage{tgheros} %% changes sans-serif font to TeX Gyre Heros (tex-gyre)
\renewcommand{\familydefault}{\sfdefault} %% changes font to sans-serif
%\usepackage{sfmath}  %%%% this makes equation sans-serif
%\input RexFigs


\setlength{\parskip}{10pt}
\setlength{\parindent}{0pt}

\newlength{\qspace}
\setlength{\qspace}{20pt}


\newcounter{qnumber}
\setcounter{qnumber}{0}

\newenvironment{question}%
 {\vspace{\qspace}
  \begin{enumerate}[\bfseries 1\quad][10]%
    \setcounter{enumi}{\value{qnumber}}%
    \item%
 }
{
  \end{enumerate}
  \filbreak
  \stepcounter{qnumber}
 }


\newenvironment{questionparts}[1][1]%
 {
  \begin{enumerate}[\bfseries (i)]%
    \setcounter{enumii}{#1}
    \addtocounter{enumii}{-1}
    \setlength{\itemsep}{5mm}
    \setlength{\parskip}{8pt}
 }
 {
  \end{enumerate}
 }



\DeclareMathOperator{\cosec}{cosec}
\DeclareMathOperator{\Var}{Var}

\def\d{{\rm d}}
\def\e{{\rm e}}
\def\g{{\rm g}}
\def\h{{\rm h}}
\def\f{{\rm f}}
\def\p{{\rm p}}
\def\s{{\rm s}}
\def\t{{\rm t}}


\def\A{{\rm A}}
\def\B{{\rm B}}
\def\E{{\rm E}}
\def\F{{\rm F}}
\def\G{{\rm G}}
\def\H{{\rm H}}
\def\P{{\rm P}}


\def\bb{\mathbf b}
\def \bc{\mathbf c}
\def\bx {\mathbf x}
\def\bn {\mathbf n}

\newcommand{\low}{^{\vphantom{()}}}
%%%%% to lower suffices: $X\low_1$ etc


\newcommand{\subone}{ {\vphantom{\dot A}1}}
\newcommand{\subtwo}{ {\vphantom{\dot A}2}}




\def\le{\leqslant}
\def\ge{\geqslant}


\def\var{{\rm Var}\,}

\newcommand{\ds}{\displaystyle}
\newcommand{\ts}{\textstyle}




\begin{document}
\setcounter{page}{2}

 
\section*{Section A: \ \ \ Pure Mathematics}

%%%%%%%%%%Q1
\begin{question}
I have two dice whose faces are all painted different colours. I number
the faces of one of them $1,2,2,3,3,6$ and the other $1,3,3,4,5,6.$
I can now throw a total of 3 in two different ways using the two number
$2$'s on the first die once each. Show that there are seven different
ways of throwing a total of 6. 


I now renumber the dice (again only using integers in the range 1
to 6) with the results shown in the following table 


\noindent \begin{center}
\begin{tabular}{llllllllllll}
Total shown by the two dice & 2 & 3 & 4 & 5 & 6 & 7 & 8 & 9 & 10 & 11 & 12\tabularnewline
Different ways of obtaining the total & 0 & 2 & 1 & 1 & 4 & 3 & 8 & 6 & 5 & 6 & 0\tabularnewline
\end{tabular}
\par\end{center}


Find how I have numbered the dice explaining your reasoning. 


{[}You will only get high marks if the examiner can follow your argument.{]} 
\end{question}

%%%%%%%%%%Q2
\begin{question}
If $\left|r\right|\neq1,$ show that 
\[
1+r^{2}+r^{4}+\cdots+r^{2n}=\frac{1-r^{2n+2}}{1-r^{2}}\,.
\]
If $r\neq1,$ find an expression for $\mathrm{S}_{n}(r),$ where 
\[
\mathrm{S}_{n}(r)=r+r^{2}+r^{4}+r^{5}+r^{7}+r^{8}+r^{10}+\cdots+r^{3n-1}.
\]
Show that, if $\left|r\right|<1,$ then, as $n\rightarrow\infty,$
\[
\mathrm{S}_{n}(r)\rightarrow\frac{1}{1-r}-\frac{1}{1-r^{3}}\,.
\]
If $\left|r\right|\neq1,$ find an expression for $\mathrm{T}_{n}(r),$
where 
\[
\mathrm{T}_{n}(r)=1+r^{2}+r^{3}+r^{4}+r^{6}+r^{8}+r^{9}+r^{10}+r^{12}+r^{14}+r^{15}+r^{16}+\cdots+r^{6n}.
\]
If $\left|r\right|<1,$ find the limit of $\mathrm{T}_{n}(r)$ as
$n\rightarrow\infty.$


What happens to $\mathrm{T}_{n}(r)$ as $n\rightarrow\infty$ in the
three cases $r>1,r=1$ and $r=-1$? In each case give reasons for
your answer. 
\end{question}

%%%%%%%%% Q3
\begin{question}
\begin{questionparts}
\item Find all the integer solutions with $1\leqslant p\leqslant q\leqslant r$
of the equation 
\[
\frac{1}{p}+\frac{1}{q}+\frac{1}{r}=1\,,
\]
showing that there are no others. 
\item The integer solutions with $1\leqslant p\leqslant q\leqslant r$
of 
\[
\frac{1}{p}+\frac{1}{q}+\frac{1}{r}>1\,,
\]
include $p=1$, $q=n,$ $r=m$ where $n$ and $m$ are any integers
satisfying $1\leqslant m\leqslant n.$ Find all the other solutions,
showing that you have found them all. 
\end{questionparts}
\end{question}

%%%%%% Q4 
\begin{question}
By making the change of variable $t=\pi-x$ in the integral 
\[
\int_{0}^{\pi}x\mathrm{f}(\sin x)\,\mathrm{d}x,
\]
or otherwise, show that, for any function $\mathrm{f},$ 
\[
\int_{0}^{\pi}x\mathrm{f}(\sin x)\,\mathrm{d}x=\frac{\pi}{2}\int_{0}^{\pi}\mathrm{f}(\sin x)\,\mathrm{d}x\,.
\]
Evaluate 
\[
\int_{0}^{\pi}\frac{x\sin x}{1+\cos^{2}x}\,\mathrm{d}x\quad\mbox{ and }\quad\int_{0}^{2\pi}\frac{x\sin x}{1+\cos^{2}x}\,\mathrm{d}x\,.
\]
	\end{question}

%%%%%%%%% Q5
\begin{question}
If $z=x+\mathrm{i}y$ where $x$ and $y$ are real, define $\left|z\right|$
in terms of $x$ and $y$. Show, using your definition, that if $z_{1},z_{2}\in\mathbb{C}$
then $\left|z_{1}z_{2}\right|=\left|z_{1}\right|\left|z_{2}\right|.$ 

Explain, by means of a diagram, or otherwise, why $\left|z_{1}+z_{2}\right|\leqslant\left|z_{1}\right|+\left|z_{2}\right|.$ 


Suppose that $a_{j}\in\mathbb{C}$ and $\left|a_{j}\right|\leqslant1$
for $j=1,2,\ldots,n.$ Show that, if $\left|z\right|\leqslant\frac{1}{2},$
then 
\[
\left|a_{n}z^{n}+a_{n-1}z^{n-1}+\cdots+a_{1}z\right|<1,
\]
and deduce that any root $w$ of the equation 
\[
a_{n}z^{n}+a_{n-1}z^{n-1}+\cdots+a_{1}z+1=0
\]
must satisfy $\left|x\right|>\frac{1}{2}.$ 
	\end{question}
	
	%%%%%%%%% Q6
	\begin{question}
Let $N=10^{100}.$ The graph of 
\[
\mathrm{f}(x)=\frac{x^{N}}{1+x^{N}}+2
\]
for $-3\leqslant x\leqslant3$ is sketched in the following diagram. 


\noindent \begin{center}
\psset{xunit=1.0cm,yunit=1.0cm,algebraic=true,dotstyle=o,dotsize=3pt 0,linewidth=0.5pt,arrowsize=3pt 2,arrowinset=0.25} \begin{pspicture*}(-4.15,-1.01)(4.08,4.09) \psaxes[labelFontSize=\scriptstyle,xAxis=true,yAxis=true,labels=none,Dx=1,Dy=1,ticksize=0pt 0,subticks=2]{->}(0,0)(-4.15,-1.01)(4.08,4.09)[$x$,140] [$y$,-40] \psline(-4,3)(-1,3) \psline(-1,3)(-1,2) \psline(-1,2)(1,2) \psline(1,2)(1,3) \psline(1,3)(4,3) \rput[tl](-1.39,-0.2){$-1$} \rput[tl](1,-0.2){$1$} \rput[tl](0.19,1.9){$2$} \rput[tl](0.19,3.18){$3$} \end{pspicture*}
\par\end{center}


Explain the main features of the sketch. 


Sketch the graphs for $-3\leqslant x\leqslant3$ of the two functions
\[
\mathrm{g}(x)=\frac{x^{N+1}}{1+x^{N}}
\]
and 
\[
\mathrm{h}(x)=10^{N}\sin(10^{-N}x).
\]
In each case explain briefly the main features of your sketch. 

	\end{question}
	
%%%%%%%%% Q7
\begin{question}
Sketch the curve 
\[
\mathrm{f}(x)=x^{3}+Ax^{2}+B
\]
first in the case $A>0$ and $B>0$, and then in the case $A<0$ and
$B>0.$


Show that the equation 
\[
x^{3}+ax^{2}+b=0,
\]
where $a$ and $b$ are real, will have three distinct real roots
if 
\[
27b^{2}+3a^{3}b<0,
\]
but will have fewer than three if 
\[
27b^{2}+4a^{3}b<0.
\]
\end{question}
		
%%%%%%%%% Q8
\begin{question}	
\begin{questionparts}
\item Prove that the intersection of the surface
of a sphere with a plane is always a circle, a point or the empty
set. Prove that the intersection of the surfaces of two spheres with
distinct centres is always a circle, a point or the empty set. 


{[}If you use coordinate geometry, a careful choice of origin and
axes may help.{]} 


\item The parish council of Little Fitton have just bought a modern
sculpture entitled `Truth, Love and Justice pouring forth their blessings
on Little Fitton.' It consists of three vertical poles $AD,BE$ and
$CF$ of heights 2 metres, 3 metres and 4 metres respectively. Show
that $\angle DEF=\cos^{-1}\frac{1}{5}.$


Vandals now shift the pole $AD$ so that $A$ is unchanged and the
pole is still straight but $D$ is vertically above $AB$ with $\angle BAD=\frac{1}{4}\pi$
(in radians). Find the new angle $\angle DEF$ in radians correct
to four figures. 
\end{questionparts}
\end{question}	
		
%%%%%%%%%% Q9
\begin{question}
In the manufacture of Grandma's Home Made Ice-cream, chemicals $A$
and $B$ pour at constant rates $a$ and $b-a$ litres per second
($0<a<b$) into a mixing vat which mixes the chamicals rapidly and
empties at a rate $b$ litres per second into a second mixing vat.
At time $t=0$ the first vat contains $K$ litres of chemical $B$
only. Show that the volume $V(t)$ (in litres) of the chemical $A$
in the first vat is governed by the differential equation 
\[
\dot{V}(t)=-\frac{bV(t)}{K}+a,
\]
and that 
\[
V(t)=\frac{aK}{b}(1-\mathrm{e}^{-bt/K})
\]
for $t\geqslant0.$


The second vat also mixes chemicals rapidly and empties at the rate
of $b$ litres per second. If at time $t=0$ it contains $L$ litres
of chemical $C$ only (where $L\neq K$), how many litres of chemical
$A$ will it contain at a later time $t$? 
\end{question}
			
		
			

		
	
\newpage
\section*{Section B: \ \ \ Mechanics}

%%%%%%%%%% Q10
\begin{question}
A small lamp of mass $m$ is at the end $A$ of a light rod $AB$
of length $2a$ attached at $B$ to a vertical wall in such a way
that the rod can rotate freely about $B$ in a vertical plane perpendicular
to the wall. A spring $CD$ of natural length $a$ and modulus of
elasticity $\lambda$ is joined to the rod at its mid-point $C$ and
to the wall at a point $D$ a distance $a$ vertically above $B$.
The arrangement is sketched below. 


\noindent \begin{center}
\psset{xunit=0.8cm,yunit=0.8cm,algebraic=true,dotstyle=o,dotsize=3pt 0,linewidth=0.5pt,arrowsize=3pt 2,arrowinset=0.25} \begin{pspicture*}(-1.55,-0.7)(4.5,5.27) \psline(0,5)(0,-1.16) \psline(0,0)(3.11,3.89) \pscoil[coilheight=1,coilwidth=0.2,coilarm=0.05](0,2.53)(1.47,1.84)
 \rput[tl](-0.56,0.41){$B$} \rput[tl](1.59,1.89){$C$} \rput[tl](3.41,4.28){$A$} \rput[tl](-0.56,2.84){$D$} \parametricplot{0.7583777142101807}{3.8999703677999737}{1*0.16*cos(t)+0*0.16*sin(t)+3.22|0*0.16*cos(t)+1*0.16*sin(t)+3.77} \psline(3.1,3.66)(3.33,3.88) \begin{scriptsize} \psdots[dotsize=5pt 0](0,0) \psdots[dotstyle=*](3.11,3.89) \psdots[dotstyle=*](0,2.53) \end{scriptsize} \end{pspicture*}
\par\end{center}


Show that if $\lambda>4mg$ the lamp can hang in equilibrium away
from the wall and calculate the angle $\angle DBA$. 
\end{question}
	
%%%%%%%%%% Q11
\begin{question}
A piece of uniform wire is bent into three sides of a square $ABCD$
so that the side $AD$ is missing. Show that if it is first hung up
by the point $A$ and then by the point $B$ then the angle between
the two directions of $BC$ is $\tan^{-1}18.$ 
	\end{question}
	
%%%%%%%%%% Q12
\begin{question}	
In a clay pigeon shoot the target is launched vertically from ground
level with speed $v$. At a time $T$ later the competitor fires a
rifle inclined at angle $\alpha$ to the horizontal. The competitor
is also at ground level and is a distance $l$ from the launcher.
The speed of the bullet leaving the rifle is $u$. Show that, if the
competitor scores a hit, then 
\[
l\sin\alpha-\left(vT-\tfrac{1}{2}gT^{2}\right)\cos\alpha=\frac{v-gT}{u}l.
\]
Suppose now that $T=0$. Show that if the competitor can hit the target
before it hits the ground then $v<u$ and 
\[
\frac{2v\sqrt{u^{2}-v^{2}}}{g}>l.
\]
\end{question}

%%%%%%%%%% Q13

\begin{question}
A train starts from a station. The tractive force exerted by the engine
is at first constant and equal to $F$. However, after the speed attains
the value $u$, the engine works at constant rate $P,$ where $P=Fu.$
The mass of the engine and the train together is $M.$ Forces opposing
motion may be neglected. Show that the engine will attain a speed
$v$, with $v\geqslant u,$ after a time 
\[
t=\frac{M}{2P}\left(u^{2}+v^{2}\right).
\]
Show also that it will have travelled a distance 
\[
\frac{M}{6P}(2v^{3}+u^{2})
\]
in this time. 




\end{question}
	

	
	\newpage
\section*{Section C: \ \ \ Probability and Statistics}

%%%%%%%%%% Q14
\begin{question}
When he sets out on a drive Mr Toad selects a speed $V$ kilometres
per minute where $V$ is a random variable with probability density
\[
\alpha v^{-2}\mathrm{e}^{-\alpha v^{-1}}
\]
 and $\alpha$ is a strictly
positive constant. He then drives at constant speed, regardless of
other drivers, road conditions and the Highway Code. The traffic lights
at the Wild Wood cross-roads change from red to green when Mr Toad
is exactly 1 kilometre away in his journey towards them. If the traffic
light is green for $g$ minutes, then red for $r$ minutes, then green
for $g$ minutes, and so on, show that the probability that he passes
them after $n(g+r)$ minutes but before $n(g+r)+g$ minutes, where
$n$ is a positive integer, is 
\[
\mathrm{e}^{-\alpha n(g+r)}-\mathrm{e}^{-\alpha\left(n(g+r)\right)+g}.
\]
Find the probability $\mathrm{P}(\alpha)$ that he passes the traffic
lights when they are green. 


Show that $\mathrm{P}(\alpha)\rightarrow1$ as $\alpha\rightarrow\infty$
and, by noting that $(\mathrm{e}^{x}-1)/x\rightarrow1$ as $x\rightarrow0$,
or otherwise, show that 
\[
\mathrm{P}(\alpha)\rightarrow\frac{g}{r+g}\quad\mbox{ as }\alpha\rightarrow0.
\]
{[}NB: the traffic light show only green and red - not amber.{]} 
\end{question}

%%%%%%%%%% Q15
\begin{question}
Captain Spalding is on a visit to the idyllic island of Gambriced.
The population of the island consists of the two lost tribes of Frodox
and the latest census shows that $11/16$ of the population belong
to the Ascii who tell the truth $3/4$ of the time and $5/16$ to
the Biscii who always lie. The answers of an Ascii to each question
(even if it is the same as one before) are independent. 


Show that the probability that an Ascii gives the same answer twice
in succession to the same question is $5/8$. Show that the probability
that an Ascii gives the same answer twice is telling the truth is
$9/10.$ 


Captain Spalding addresses one of the natives as follows. 


\hspace{1.5em} \textsl{Spalding: }My good man, I'm afraid I'm lost. Should I go left or right to reach the nearest town?\nolinebreak


\hspace{1.5em}\textsl{Native: }Left.


\hspace{1.5em}\textsl{Spalding: }I am a little deaf. Should I go left or right to
reach the nearest town?


\hspace{1.5em}\textsl{Native (patiently): }Left.


Show that, on the basis of this conversation, Captain Spalding should
go left to try and reach the nearest town and that there is a probability
$99/190$ that this is the correct direction. 


The conversation resumes as follows. 


\hspace{1.5em}\textsl{Spalding: }I'm sorry I didn't quite hear that. Should I go
left or right to reach the nearest town?


\hspace{1.5em}\textsl{Native (loudly and clearly): }Left.


Shouls Captain Spalding go left or right and why? Show that if he
follows your advice the probability that this is the correct direction
is $331/628$. 
\end{question}

%%%%%%%%%% Q16
\begin{question}
By making the substitution $y=\cos^{-1}t,$ or otherwise, show that
\[
\int_{0}^{1}\cos^{-1}t\,\mathrm{d}t=1.
\]
A pin of length $2a$ is thrown onto a floor ruled with parallel lines
equally spaced at a distance $2b$ apart. The distance $X$ of its
centre from the nearest line is a uniformly distributed random variable
taking values between $0$ and $b$ and the acute angle $Y$ the pin
makes with a direction perpendicular to the line is a uniformly distributed
random variable taking values between $0$ and $\pi/2$. $X$ and
$Y$ are independent. If $X=x$ what is the probability that the pin
crosses the line? 


If $a<b$ show that the probability that the pin crosses a line for
a general throw is $\dfrac{2a}{\pi b}.$
\end{question}
\end{document}
