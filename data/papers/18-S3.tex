\documentclass[a4, 11pt]{report}


\pagestyle{myheadings}
%\markboth{}{Paper III, 2008   final draft 
%\ \ \ \ \ 
%\today 
%}               


\usepackage{pstricks-add}
\usepackage{epsfig}

\RequirePackage{amssymb}
\RequirePackage{amsmath}
\RequirePackage{graphicx}
\RequirePackage{color}
\RequirePackage{xcolor}


\RequirePackage[flushleft]{paralist}[2013/06/09]



\RequirePackage{geometry}
\geometry{%
  a4paper,
  lmargin=2cm,
  rmargin=2.5cm,
  tmargin=3.5cm,
  bmargin=2.5cm,
  footskip=12pt,
  headheight=24pt}
\usepackage{verbatim}


\newcommand{\bct}[1]{{\color{blue}#1}}
%\renewcommand{\comment}[1]{}
\newcommand{\rct}[1]{{\color{red}#1}}

\setlength{\parskip}{10pt}
\setlength{\parindent}{0pt}

\newlength{\qspace}
\setlength{\qspace}{20pt}

\newcounter{qnumber}
\setcounter{qnumber}{0}

\newenvironment{question}%
 {\vspace{\qspace}
  \begin{enumerate}[\bfseries 1\quad][10]%
    \setcounter{enumi}{\value{qnumber}}%
    \item%
 }
{
  \end{enumerate}
  \filbreak
  \stepcounter{qnumber}
 }

\newenvironment{questionparts}[1][1]%
 {
  \begin{enumerate}[\bfseries (i)]%
    \setcounter{enumii}{#1}
    \addtocounter{enumii}{-1}
    \setlength{\itemsep}{3mm}
    \setlength{\parskip}{8pt}
 }
 {
  \end{enumerate}
 }


\DeclareMathOperator{\cosec}{cosec}
\DeclareMathOperator{\Var}{Var}

\def\d{{\rm d}}
\def\e{{\rm e}}
\def\g{{\rm g}}
\def\h{{\rm h}}
\def\f{{\rm f}}
\def\p{{\rm p}}
\def\q{{\rm q}}
\def\s{{\rm s}}
\def\t{{\rm t}}


\def\A{{\rm A}}
\def\B{{\rm B}}
\def\E{{\rm E}}
\def\F{{\rm F}}
\def\G{{\rm G}}
\def\H{{\rm H}}
\def\P{{\rm P}}


\def\bb {\mathbf b}
\def\bc {\mathbf c}
\def\bx {\mathbf x}
\def\bn {\mathbf n}

\makeatletter
\newcommand{\raisemath}[1]{\mathpalette{\raisem@th{#1}}}
\newcommand{\raisem@th}[3]{\raisebox{#1}{$#2#3$}}
\makeatother
%%%To raise suffices: e.g.  $\Pi_{\raisemath{2pt}{-}}$.


\def\le{\leqslant}
\def\ge{\geqslant}


\def\var{{\rm Var}\,}

\newcommand{\ds}{\displaystyle}
\newcommand{\ts}{\textstyle}


\begin{document}

\setcounter{page}{2}


 
\section*{Section A: \ \ \ Pure Mathematics}


%%%%%%%%%%%%%Q1
\begin{question}
\begin{questionparts}
\item 
The function $\f$  is given by
\[
\f(\beta)=\beta - \frac 1 \beta - \frac 1 {\beta^2}
\ \ \ \ \ \ \ \ (\beta\ne0) \,.
\]
Find the stationary point of 
the curve $y=\f(\beta)\,$ and sketch the curve.


Sketch also the curve $y=\g(\beta)\,$, where 
\[
\g(\beta) = \beta + \frac 3 \beta - \frac 1 {\beta^2}
\ \ \ \ \ \ \ \ (\beta\ne0)\,.
\]
\item
Let $u$ and $v$ be the roots of the equation
\[
x^2 +\alpha x +\beta = 0
\,,
\]
where $\beta\ne0\,$.
Obtain expressions in terms of $\alpha$ and $\beta$ for 
$\displaystyle u+v + \frac 1 {uv}$ and  
$ \displaystyle \frac 1 u + \frac 1 v + uv\,$.

\item  Given that 
$\displaystyle u+v + \frac 1 {uv} = -1\,$,   
and that 
$u$ and $v$ are real, 
show  that  $\displaystyle \frac 1 u+ \frac 1 v +  {uv} \le -1\;$.  


\item Given  instead that  
$\displaystyle u+v + \frac 1 {uv} = 3 \;$, and that $u$ and $v$ are real,
find the greatest value of   
$\displaystyle  \frac 1 u+ \frac 1v +  {uv}\,$. 


\end{questionparts}


\end{question}

%%%%%%%%%%%%%% Q2
\begin{question}
The sequence of functions 
$y_0$, $y_1$, $y_2$, $\ldots\,$ is defined by $y_0=1$ and, for $n\ge1\,$,
\[
 y_n = (-1)^n \frac {1}{z} \, \frac{\d^{\raisemath{1pt}{n}} z}{\d x^n}
\,,
\] 
where $z= \e^{-x^2}\!$.

\begin{questionparts}
\item Show that 
$\dfrac{\d y_n}{\hspace{-4.7pt}\d x} = 2x y_n -y_{n+1}\,$ for $n\ge1\,$.
\item Prove by induction that, for $n\ge1\,$,
\[
y_{n+1}  = 2x y_n  -2ny_{n-1}
\,.
\]
Deduce that, for $n\ge1\,$, 
\[
y_{n+1}^2  - {y\vphantom{^2}}_n {y\vphantom{^n}}_{n+2}   
= 2n
\big( y_n^2 - {y\vphantom{^n}}_{n-1}{y\vphantom{^2}}_{n+1}) 
+ 2 y_n^2
\,.
\]
\item Hence show that $y_{n}^2 
- {y\vphantom{^2}}_{n-1} {y\vphantom{^2}}_{n+1} >0$ for $n\ge1\,$.
\end{questionparts}
\end{question}


%%%%%%%%%%%%% Q3
\begin{question}
Show that
 the second-order differential equation
\[
x^2y''+(1-2p) x\, y' + (p^2-q^2) \, y= \f(x)
\,,
\]
where $p$ and $q$ are constants, can be written in the 
form
\[
x^a  
\big(x^b 
(x^cy)'\big)' = \f(x)
\,,
\tag{$*$}
\]
where $a$, $b$ and $c$ are constants. 

\begin{questionparts}
\item Use $(*)$ to derive the general solution of the equation
\[
x^{2}y''+(1-2p)xy'+(p^2-q^{2})y=0
\]
in the different cases that arise according to the values of $p$ and $q$.

\item Use $(*)$ to derive the general solution of the equation
\[
x^{2}y''+(1-2p)xy'+p^2y=x^{n}
\]
in the different cases that arise according to the values of $p$ and $n$.

\end{questionparts}

\end{question}

%%%%%%%%%%%%%% Q4
\begin{question}
The point $P(a\sec \theta, b\tan \theta )$ 
lies on the hyperbola 
\[
\dfrac{x^{2}}{a^{2}}-\dfrac{y^{2}}{b^{2}}=1\,,
\]
where $a>b>0\,$. 
Show that the equation of the 
tangent to the hyperbola at $P$ can be written as
\[
bx- ay \sin\theta  = 
ab \cos\theta
\,.
\]

\begin{questionparts}
\item This tangent  meets the 
lines
$\dfrac x a = \dfrac yb$ and $\dfrac x a =- \dfrac y b$
at
 $S$ and $T$,
 respectively. 

How is  the mid-point of $ST$
related to $P$?
\item
The point $Q(a\sec \phi, b\tan \phi)$ also lies on the
hyperbola and  the tangents to the 
hyperbola at $P$ and $Q$ 
 are perpendicular.               
These two  tangents intersect at $(x,y)$. 

Obtain expressions for $x^2$ and $y^2$ in terms of $a$, $\theta$ and $\phi$.

Hence, or otherwise,
show that
 $x^2+y^2 = a^2 -b^2$. 
\end{questionparts}
\end{question}



%%%%%%%%%%%%%%%%%% Q5

\begin{question}
The real numbers $a_1$, $a_2$, $a_3$, $\ldots$ are all positive. For each positive
integer $n$, $A_n$ and $G_n$ are defined by 
\[
A_n = \frac{a_1+a_2 + \cdots + a_n}n 
\ \ \ \ \ \text{and } \ \ \ \ \ 
G_n = \big( a_1a_2\cdots a_n\big) ^{1/n}
\,.
\]
\begin{questionparts}
\item Show that, for any given positive integer $k$, 
\[
(k+1) ( A_{k+1} - G_{k+1}) \ge  k (A_k-G_k)
\]
if and only if 
\[\lambda^{k+1}_k -(k+1)\lambda_{\raisemath{-2  pt}{k}} +k \ge 0\,,
\]
 where
$ \lambda_{\raisemath{-2pt}{k}} = \left(\dfrac{a_{k+1}}{G_{k}}\right)^{\frac1 {k+1}}\,$. 

\item
Let 
\[
 \f(x)=x^{k+1} -(k+1)x +k \,,
\]
 where
$x>0$ and $k$ is a positive integer. Show that $\f(x)\ge0$
and that $\f(x)=0$  
if and only if
 $x = 1\,$. 

\item
Deduce  that:
\begin{enumerate}
\item[(a)]
 $A_n \ge G_n$ for all $n$; 
\\
\item[(b)]
if $A_n=G_n$ for some $n$, then $a_1=a_2 = \cdots = a_n\,$.
\end{enumerate}
\end{questionparts}
\end{question}








%%%%%%%%%%%%%% Q6
\begin{question}
\begin{questionparts}
\item
The 
distinct
points $A$, $Q$ and $C$ lie on a straight line in the Argand diagram,
and 
represent the distinct
complex numbers $a$, $q$ and $c$, respectively. 
Show that
$\dfrac {q-a}{c-a}$
 is real and hence that 
$(c-a)(q^*-a^*) = (c^*-a^*)(q-a)\,$.

Given that $aa^* = cc^* = 1$, show further that 
\[
q+ ac q^* = a+c
\,.
\]

\item
The distinct points $A$, $B$, $C$ and $D$ lie, in anticlockwise order,
 on the circle of unit radius 
with centre at the origin  
(so that, for example, $aa^* =1$).
The lines $AC$ and $BD$ meet at $Q$.
Show that 
\[
(ac-bd)q^* = (a+c)-(b+d)
\,,
\]
where  
$b$ and $d$  are complex numbers represented by the 
points $B$ and $D$
  respectively,
and show further that
\[
(ac-bd)
 (q+q^*) = 
(a-b)(1+cd) +(c-d)(1+ab)
\,.
\]

\item
The lines $AB$ and $CD$
meet at $P$, which  represents the complex number
$p$. Given that~$p$ is real, show that $p(1+ab)=a+b\,$.
Given further that $ac-bd \ne 0\,$,
show that 
\[
p(q+q^*) =  2 
\,.
\]

\end{questionparts}
\end{question}

%%%%%%%%%%%%%% Q7
\begin{question}
\begin{questionparts}
\item Use De Moivre's theorem to show that,
if $\sin\theta\ne0$\,, then
\[
\frac{
\left(\cot \theta + \rm{i}\right)^{2n+1}
-\left(\cot \theta - \rm{i}\right)^{2n+1}}{2\rm{i}}
=
\frac{\sin \left(2n+1\right)\theta}
{\sin^{2n+1}\theta}
\,,
\]
for any positive integer $n$.

Deduce that the solutions of the equation
\[
\binom{2n+1}{1}x^{n}-\binom{2n+1}{3}x^{n-1}
+\cdots +
\left(-1\right)^{n}=0
\] 
are
\[x=\cot^{2}\left(\frac{m\pi}{2n+1}\right)
\]
where $ m=1$, $2$, $\ldots$ , $n\,$.

\item
Hence show that 
\[
\sum_{m=1}^n
\cot^{2}\left(\frac{m\pi}{2n+1}\right)
=\frac{n\left(2n-1\right)}{3}.
\]

\item
Given that 
$0<\sin \theta <\theta <\tan \theta$ 
for 
$0 < \theta < \frac{1}{2}\pi$, 
show that 
\[
\cot^{2}\theta<\frac{1}{\theta^{2}}<1+\cot^{2}\theta.
\] 
Hence show that 
\[
\sum^\infty_{m=1}
\frac{1}{m^2}=
\frac{\pi^2}{6}\,.\]

\end{questionparts}
 

\end{question}



%%%%%%%%%%%%%% Q8
\begin{question}
In this question, you should ignore issues of convergence.


\begin{questionparts}
\item
Let 
\[
I = \int_0^1 \frac{\f(x^{-1}) } {1+x} \, \d x
\,,
\] 
where $\f(x)$ is a function for which the integral exists.

Show that 
\[
I = \sum_{n=1}^\infty \int_n^{n+1} \frac{\f(y) } {y(1+y)}\, \d y
\]
and deduce that, if $\f(x) = \f(x+1)$ for all $x$, then 
\[
I= \int_0^1 \frac{\f(x)}  {1+x} \, \d x
\,.
\]
\item
The {\em fractional part}, $\{x\}$, of a real number 
$x$ is defined to be $x-\lfloor x\rfloor$ where 
$\lfloor x \rfloor$ is 
the largest integer less than or equal to $x$.
For example $\{3.2\} = 0.2$ and $\{3\}=0\,$.

Use the result of part (i) to evaluate
\vspace{3mm}
\[
  \displaystyle \int _0^1 \frac { \{x^{-1}\}}{1+x}\, \d x \text{ \ \ and \ \ }
  \displaystyle \int _0^1 \frac { \{2x^{-1}\}}{1+x}\, \d x \,. 
\]
%\item Use the same method to evaluate 
%\[
%\int_0^1 \frac {x\{x^{-1}\}}{1-x^2} \, \d x \,.
%\]
\end{questionparts}
\end{question}

\newpage

\section*{Section B: \ \ \ Mechanics}


%%%%%%%%%%%%%% Q9
\begin{question}
A particle $P$ of mass $m$ is projected 
with speed $u_0$
 along a smooth
horizontal floor
directly towards a wall.
 It collides with a particle $Q$ of mass
$km$ which is moving directly away from the wall
with speed
$v_0$. 
In the subsequent motion, $Q$ collides alternately
with the wall and with $P$. 
 The coefficient of restitution between $Q$ and $P$ is $e$, and 
the coefficient of restitution between~$Q$ and the wall is 1.

Let $u_n$ and $v_n$ be the velocities of  $P$ and $Q$, respectively,
towards the wall after the $n$th 
collision between $P$ and $Q$. 

\begin{questionparts}
\item
Show that, for $n\ge2$,
\[
(1+k)u_{n} - (1-k)(1+e)u_{n-1} + e(1+k)u_{n-2} =0\,.
\tag{$*$}
\]

\item
You are now given that $e=\frac12$ and  $k = \frac1{34}$, and that the 
solution of  
$(*)$  is of  the form  
\[
\phantom{(n\ge0)}
u_n=
A\left( \tfrac 7{10}\right)^n
+
B\left( \tfrac 5{7 }\right)^n
\ \ \ \ \ \ 
(n\ge0)
\,,
\]
where $A$ and $B$ are independent of $n$.
Find expressions for $A$ and $B$ in terms of $u_0$ and $v_0$.

Show that, if $0<6u_0<v_0$, then $u_n$ will be negative for 
large $n$.

\end{questionparts}
\end{question}

%%%%%%%%%%%%%% Q10
\begin{question}
A uniform
disc  with centre $O$ and radius $a$
is suspended from a  point $A$ on 
its circumference, so that it can swing freely about a horizontal 
axis $L$ through $A$. The plane of the disc is perpendicular to $L$.
A particle $P$ is attached to a point on the circumference of the 
disc. The mass of the disc is $M$ 
and the mass of the particle is $m$.

In equilibrium, the disc hangs with $OP$ horizontal, and  the angle
between $AO$ and the downward vertical through $A$ is $\beta$.
Find $\sin\beta$ in terms of $M$ and $m$ 
and show that
\[
\frac{AP}{a}  = \sqrt{\frac{2M}{M+m}} 
\,.
\]

The disc is rotated about $L$  and then released. 
 At later time~$t$, the angle between $OP$ and the 
horizontal is $\theta$; when $P$ is higher than $O$, $\theta$ is positive
and when $P$ is lower than $O$, $\theta$ is negative. Show that 
\[
\tfrac12 I \dot\theta^2 + (1-\sin\beta)ma^2 \dot \theta^2
+ (m+M)g a\cos\beta \, (1- \cos\theta)  
\] 
is constant during the motion, where $I$ is the moment of inertia
of the disc about $L$.

Given that $m= \frac 32 M$ and that $I= \frac32Ma^2$, 
show  that the period  of 
small oscillations  is 
\[
3\pi \sqrt{\frac {3a}{5g}}
\,.
\]

\end{question}

%%%%%%%%%%%%%% Q11

\begin{question}
A particle  
is attached to one end of a light inextensible string of length $ b$. 
The other end of the string is attached to a 
fixed point $O$.
Initially the particle hangs vertically below $O$.
The particle then receives a horizontal impulse.

The particle moves  in a circular arc with the string taut
until the acute angle between the string and the upward
vertical is $\alpha   $, at which time it becomes slack.
Express $V$, the speed of the particle when the string
becomes slack, in terms of $ b$, $g$ and $\alpha   $.
  
Show that 
the string becomes taut again a time
$T$ later, where
\[
gT = 4V \sin\alpha     
\,,\]
and that just before this time the trajectory of the  
particle makes an angle $\beta $ with 
the horizontal where $\tan\beta = 3\tan \alpha \,$.

When the string becomes taut, the momentum of the 
particle  in the direction of the string is destroyed.
Show that the particle comes instantaneously to rest 
at this time if and only if 
\[
\sin^2\alpha = \dfrac {1+\sqrt3}4 \,.
\]


\end{question}

\newpage
\section*{Section C: \ \ \ Probability and Statistics}

%%%%%%%%%%%%%% Q12
\begin{question}
A random process generates, independently, $n$
numbers each of which is drawn from a uniform (rectangular) 
distribution on  the interval 0 to 1. 
The random variable $Y_k$
is defined to be   the $k$th smallest number (so there
are $k-1$ smaller numbers).

\begin{questionparts}
\item Show  that, for $0\le y\le1\,$, 
\[
{\rm P}\big(Y_k\le y) =\sum^{n}_{m=k}\binom{n}{m}y^{m}\left(1-y\right)^{n-m}
.
\tag{$*$}
\] 
\item
Show that 
\[
m\binom n m = n \binom {n-1}{m-1}
\]
and obtain a similar expression for 
$\displaystyle (n-m) \, \binom n m\,$.

Starting from $(*)$,
show that
the  
probability density function of $Y_k$ is
\[
 n\binom{ n-1}{k-1}
y^{k-1}\left(1-y\right)^{ n-k}
\,.            
\]
Deduce an expression for
$
\displaystyle
\int_0^1 y^{k-1}(1-y)^{n-k} \, \d y
\,$.
\item
Find $\E(Y_k) $ in terms of $n$ and $k$.
\end{questionparts}
\end{question}
 
%%%%%%%%%%%%%% Q13
\begin{question}

The random variable 
$X$ takes only non-negative integer 
values and has  probability generating function 
$\G(t)$. 
Show that 
\[
\P(X = 0 \text{ or } 2 \text{ or } 4 \text { or }  6 \   \ldots ) 
= \frac{1}{2}\big(\G\left(1\right)+\G\left(-1\right)\big).
\]

You are now given that $X$ has a Poisson distribution 
with mean $\lambda$.  Show that
\[
\G(t) = \e^{-\lambda(1-t)}
\,.
\]
\begin{questionparts}
\item
The random variable $Y$ is defined by 
\[
\P(Y=r)=
\begin{cases}
k\P(X=r) &  \text{if $r=0, \ 2, \ 4, \ 6, \ \ldots$ \ }, \\[2mm]
 0& \text{otherwise}, 
\end{cases}
\]
where $k$ is an appropriate constant.

Show that
 the probability generating function of $Y$ is $\dfrac{\cosh\lambda t}{\cosh\lambda}\,$. 

Deduce that 
\mbox{$\E(Y)<\lambda$} 
for~$\lambda>0\,$. 

\item The random variable $Z$ is defined by 
\[\P(Z=r)=
\begin{cases}
        c \P(X=r) &  
\text{if $r = 0, \ 4, \ 8, \ 12, \ \ldots \ $}, \\[2mm]
 0& \text{otherwise,} 
\end{cases}
\]
where $c    $ is an appropriate constant.

Is 
$\E(Z)<\lambda$
 for all positive values of $\lambda\,$?  


\end{questionparts}
\end{question}


\end{document}


