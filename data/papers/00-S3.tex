\documentclass[a4, 11pt]{report}


\pagestyle{myheadings}
\markboth{}{Paper III, 2000
\ \ \ \ \ 
\today 
}               

\RequirePackage{amssymb}
\RequirePackage{amsmath}
\RequirePackage{graphicx}
\RequirePackage{color}
\RequirePackage[flushleft]{paralist}[2013/06/09]



\RequirePackage{geometry}
\geometry{%
  a4paper,
  lmargin=2cm,
  rmargin=2.5cm,
  tmargin=3.5cm,
  bmargin=2.5cm,
  footskip=12pt,
  headheight=24pt}


\newcommand{\comment}[1]{{\bf Comment} {\it #1}}
%\renewcommand{\comment}[1]{}

\newcommand{\bluecomment}[1]{{\color{blue}#1}}
%\renewcommand{\comment}[1]{}
\newcommand{\redcomment}[1]{{\color{red}#1}}



\usepackage{epsfig}
\usepackage{pstricks-add}
\usepackage{tgheros} %% changes sans-serif font to TeX Gyre Heros (tex-gyre)
\renewcommand{\familydefault}{\sfdefault} %% changes font to sans-serif
%\usepackage{sfmath}  %%%% this makes equation sans-serif
%\input RexFigs


\setlength{\parskip}{10pt}
\setlength{\parindent}{0pt}

\newlength{\qspace}
\setlength{\qspace}{20pt}


\newcounter{qnumber}
\setcounter{qnumber}{0}

\newenvironment{question}%
 {\vspace{\qspace}
  \begin{enumerate}[\bfseries 1\quad][10]%
    \setcounter{enumi}{\value{qnumber}}%
    \item%
 }
{
  \end{enumerate}
  \filbreak
  \stepcounter{qnumber}
 }


\newenvironment{questionparts}[1][1]%
 {
  \begin{enumerate}[\bfseries (i)]%
    \setcounter{enumii}{#1}
    \addtocounter{enumii}{-1}
    \setlength{\itemsep}{5mm}
    \setlength{\parskip}{8pt}
 }
 {
  \end{enumerate}
 }



\DeclareMathOperator{\cosec}{cosec}
\DeclareMathOperator{\Var}{Var}

\def\d{{\mathrm d}}
\def\e{{\mathrm e}}
\def\g{{\mathrm g}}
\def\h{{\mathrm h}}
\def\f{{\mathrm f}}
\def\p{{\mathrm p}}
\def\s{{\mathrm s}}
\def\t{{\mathrm t}}


\def\A{{\mathrm A}}
\def\B{{\mathrm B}}
\def\E{{\mathrm E}}
\def\F{{\mathrm F}}
\def\G{{\mathrm G}}
\def\H{{\mathrm H}}
\def\P{{\mathrm P}}


\def\bb{\mathbf b}
\def \bc{\mathbf c}
\def\bx {\mathbf x}
\def\bn {\mathbf n}

\newcommand{\low}{^{\vphantom{()}}}
%%%%% to lower suffices: $X\low_1$ etc


\newcommand{\subone}{ {\vphantom{\dot A}1}}
\newcommand{\subtwo}{ {\vphantom{\dot A}2}}




\def\le{\leqslant}
\def\ge{\geqslant}


\def\var{{\rm Var}\,}

\newcommand{\ds}{\displaystyle}
\newcommand{\ts}{\textstyle}
\def\half{{\textstyle \frac12}}




\begin{document}
\setcounter{page}{2}

 
\section*{Section A: \ \ \ Pure Mathematics}

%%%%%%%%%%Q1
\begin{question}
 Sketch on the same axes the two curves $C_1$ and $C_2$, given by 
\begin{center} 
\begin{tabular}{cccc} 
$C_1:$ & \ \ $ x y$ & $=$ & $1$ ,\\   
$C_2:$ & \ \ $x^2-y^2$ & $=$ &  $2$ . 
\end{tabular} 
\end{center} 
The curves 
intersect at $P$ and $Q$. Given that   
 the coordinates of $P$ are $(a,b)$ (which you need not evaluate), 
write down the coordinates of $Q$ in terms of $a$ and $b$. 
 
The tangent to $C_1$ through  $P$ meets the tangent to $C_2$ 
through $Q$ at the point $M$, and the tangent to $C_2$ through $P$ meets the  
tangent to $C_1$ through $Q$ at $N$. Show that the coordinates of $M$ are 
$(-b,a)$ 
and write down the coordinates of $N$. 
 
Show that $PMQN$ is a square. 
\end{question}

%%%%%%%%%%Q2
\begin{question}
Use the substitution   
$x = 2-\cos \theta $   
to evaluate the integral 
 
$$ 
\int_{3/2}^2 \left(x -  1 \over 3 - x\right)^{\!\frac12}\! \d x.  
$$  
 
Show that, for $a<b$, 
$$ 
\int_p^q \left( x -  a \over b - x\right)^{\!\frac12} \!\d x = 
\frac{(b-a)(\pi +3{\surd3}  -6)}{12},  
$$  
where $p=  {(3a+b)/4}$ and $q={(a+b)/2}$. 
\end{question}

%%%%%%%%% Q3
\begin{question}
Given that $\alpha = \e^{\mathrm{i} \pi/3}$ ,  
prove that  $1 + \alpha^2 = \alpha$. 
 
A triangle in the Argand plane has vertices $A$, $B$, and $C$ 
 represented by the complex numbers  
$p$, $q\alpha^2$ and $- r\alpha$   
respectively, where  $p$, $q$ and $r$  
are positive real numbers. Sketch the triangle~$ABC$. 
 
Three equilateral triangles $ABL$, $BCM$ and $CAN$ 
(each lettered clockwise) 
 are erected on sides $AB$, $BC$ and $CA$ respectively.  
Show that the complex number representing $N$ is  
\mbox{$( 1 - \alpha) p- \alpha^2 r$}   
and find similar expressions for the  
complex numbers representing $L$ and $M$. 
 
Show that lines $LC$, $MA$ and $NB$ all meet at the origin,  
and that these three line segments have the  
common length  $p+q+r$. 
\end{question}

%%%%%% Q4 
\begin{question}
The function $\f(x)$ is defined by     
$$ 
\f(x) = \frac{x( x - 2 )(x-a)}{  x^2 - 1}. 
$$ 
 
 
Prove algebraically that the line $y = x + c$  
intersects the curve  $y = \f ( x )$ 
if  $\vert a \vert \ge1$, but there are values of $c$ 
for which there are no points of intersection if $\vert a \vert <1$. 
 
Find the equation of the oblique asymptote of the curve $y=\f(x)$. 
Sketch the graph in the two cases 
 \ \ (i) \ $a<-1$ ; and \ \ (ii) \ $-1<a<-\frac12$.  
(You need not calculate the turning points.)

\end{question}

%%%%%%%%% Q5
\begin{question}
Given two non-zero vectors  
$\mathbf{a}=\begin{pmatrix}a_{1}\\
a_{2}
\end{pmatrix}$ and $\mathbf{b}=\begin{pmatrix}b_{1}\\
b_{2}
\end{pmatrix}$
\mbox{define  $\Delta\!\! \left( \bf a, \bf b \right)$ by 
 $\Delta\!\! \left( \bf a, \bf b \right) = a_1 b_2 - a_2 b_1$.} 
 
 
Let $A$, $B$ and $C$ be  points with position vectors $\bf a$, $\bf b$ and $\bf c$,  
respectively, no two of which are parallel. Let   $P$, $Q$ and $R$ be points with position 
vectors $\bf p$, $\bf q$ and $\bf r$, respectively, none of which are parallel. 
 
\begin{questionparts} 
\item
Show 
%, by considering first the case  
%$\displaystyle \bf a =  \pmatrix{\!a_1\!\cr 0 \cr}$,  
%or otherwise,  
that there exists a $2 \times 2$ matrix $\bf M$ such that 
$P$ and  $Q$ 
are the images of $A$ and  $B$   
under the transformation represented by $\bf M$. 
  
\item
Show that 
$ 
\Delta\!\! \left( \bf a, \bf b \right) \bf c  
+ \Delta\!\! \left( \bf c, \bf a \right) \bf b  
+ \Delta\!\! \left( \bf b, \bf c \right) \bf a = 0.  
$ 
 
Hence, or otherwise, prove that a necessary and sufficient condition for the points  
$P$, $Q$, and $R$ 
to be the images of points $A$, $B$ and  $C$  
under the transformation represented by some $2 \times 2$ matrix $\bf M$ is that 
\[ 
\Delta\!\! \left( \bf a, \bf b \right) :  
\Delta\!\! \left( \bf b, \bf c \right) :  
\Delta\!\! \left( \bf c, \bf a \right) =  
\Delta\!\! \left( \bf p, \bf q \right) :  
\Delta\!\! \left( \bf q, \bf r \right) :   
\Delta\!\! \left( \bf r, \bf p \right). 
\] 
\end{questionparts} 

	\end{question}
	
%%%%%%%%% Q6
\begin{question}
Given that 
\[ 
x^4 + p x^2 + q x + r = ( x^2 - a x + b ) ( x^2 + a x + c ) , 
\]  
express  $p$, $q$ and $r$ in terms of $a$, $b$ and $c$. 
 
Show also that $ a^2$ is a root of the cubic equation 
$$ 
u^3 + 2 p u^2 + ( p^2 - 4 r ) u - q^2 = 0 .  
$$ 
Explain why this equation  always has a non-negative root,  
and verify  that $u = 9$ is a root in the  
case $p = -1$, $q = -6$, $r = 15$ . 
 
Hence, or otherwise, express 
$$y^4 - 8 y^3 + 23 y^2 - 34 y + 39$$ 
as a product of two quadratic factors.  
\end{question}
	
%%%%%%%%% Q7
\begin{question}
Given that 
$$\e = 1 + {1 \over 1 !} + {1 \over 2 !} + {1 \over 3 !} + \cdots + {1 \over r !} + \cdots \; ,$$ 
use the binomial theorem to show that 
$$ 
{\left( 1 + {1 \over n} \right)}^{\!n} < \e 
$$  
for any positive integer  $n$. 
 
The product ${\rm P }( n )$  is defined, for any positive integer $n$, by 
$$ 
{\rm P} ( n ) = {3 \over 2} \cdot {5 \over 4}  
\cdot {9 \over 8} \cdot \ldots \cdot {2^n + 1 \over 2^n} . 
$$ 
Use the arithmetic-geometric mean inequality, 
$$ 
{a_1 + a_2 + \cdots + a_n \over n}  
\ge 
 \ {\left( a_1 \cdot a_2 \cdot \ldots \cdot a_n \right)}^{1 \over n}\,, 
$$  
to show that ${\rm P }( n ) < \e$  for all $n$ . 
 
 
Explain briefly why  
 ${\rm P} ( n )$  tends to a limit as $n\to\infty$. 
Show that this limit, $L$, satisfies $2<L\le\e$.
\end{question}
		
%%%%%%%%% Q8
\begin{question}	
The sequence $a_n$ is defined by $a_0 = 1$ , $a_1 = 1$ , and 
$$ 
a_n = {1 + a_{n - 1}^2 \over a_{n - 2} } \ \ \ \ \ \ \ \ \ \ \ \ \ \ \ ( n \ge 2 ) . 
$$ 
Prove by induction that 
$$ 
a_n = 3 a_{n - 1} - a_{n - 2} \ \ \ \ \ \ \ \ \ \ \ ( n \ge2 ) . 
$$ 
Hence show that  
$$  
a_n = {\alpha^{2 n - 1} + \alpha^{- ( 2 n - 1 )} \over \sqrt 5} 
\ \ \ \ \ \  (n\ge1), 
$$ 
where $\displaystyle{\alpha = {1 + \sqrt 5 \over 2}}$. 
\end{question}	
		

		
	
\newpage
\section*{Section B: \ \ \ Mechanics}


	
%%%%%%%%%% Q9
\begin{question}
Two small discs of masses $m$ and $\mu m$  
lie on a smooth horizontal surface.  
The disc of mass $\mu m$ is at rest,  
and the disc of mass $m$ is projected towards it with velocity $\mathbf{u}$.  
After the collision, the disc of mass $\mu m$ moves in the direction  
given by unit vector $\bn$. The collision is perfectly elastic. 
 
\begin{questionparts} 
\item
Show that  the speed of the disc of mass $\mu m$ after the collision  
is \ \ 
$ 
\dfrac {2\mathbf{u} \cdot \mathbf{n}}{1+\mu}. 
$ 
 
\item
 Given that  the two discs have equal kinetic energy  
after the collision, 
find an expression for the cosine of the angle between 
 $\bf n$ and $\bf u$ and show that  
$3-\sqrt8\le \mu \le 3+\sqrt8$.  
\end{questionparts} 
	\end{question}
	
%%%%%%%%%% Q10
\begin{question}	
A sphere of radius $a$ and weight $W$ rests  on horizontal ground.  
A thin uniform beam of weight $3\sqrt3\,W$ 
and length $2a$ is freely hinged to the ground at $X$,  
which is a distance ${\sqrt 3} \, a$  
from the point  of contact of the sphere  with the ground.  
The beam rests on the sphere, lying in the same vertical plane 
as the centre of the  sphere. 
The coefficients of friction between the beam and the sphere  
and between 
the sphere  and the ground are $\mu_1$ and $\mu_2$ respectively. 
 
Given that the sphere is on the point of slipping at its contacts  
with both the ground and the beam, 
find the values of $\mu_1$ and $\mu_2$. 
\end{question}

%%%%%%%%%% Q11

\begin{question}
A thin beam is fixed at a height $2a$ above a horizontal plane. 
A uniform straight rod $ACB$ of length $9a$ and mass $m$ 
is supported by the beam at $C$. Initially, the rod  is held  
so that it is horizontal and perpendicular to the beam. 
The distance $AC$ is $3a$, and 
the coefficient of friction between the beam and the rod is $\mu$. 
 
The rod is now released.  
Find the minimum value of $\mu$ for which $B$ 
strikes the horizontal plane  
before slipping takes place at $C$. 
\end{question}
	

	
	\newpage
\section*{Section C: \ \ \ Probability and Statistics}


%%%%%%%%%% Q12
\begin{question}
In a lottery, any one of $N$ numbers, where $N$ is large,  
is chosen at random and independently for each player by  
machine. Each week there are $2N$ players 
 and one winning number is drawn. Write down an  
exact expression for the probability that there are  
three or fewer winners in a week, given that you hold  
a winning ticket that week. Using the fact that  
$$ 
{\biggl( 1 - {a \over n} \biggr) ^n \approx \e^{-a}}$$ 
for $n$ much larger than $a$, or otherwise, show that this probability  
is approximately ${2 \over 3}$ . 
 
Discuss briefly whether  
 this probability would  increase or decrease  
if the numbers were chosen by the players. 
 
 
Show that the expected number of winners in a week,  
given that you hold a winning ticket that week, is 
$ 3-N^{-1}$. 

\end{question}

%%%%%%%%%% Q13
\begin{question}
A set of $n$ dice  is rolled repeatedly. 
For each die the probability  
of showing a six is $p$. Show that  
the probability that the first of the dice to show a six  
does so on the $r$th roll is 
$$q^{n  r } ( q^{-n} - 1 )$$ 
where $q = 1 - p$. 
 
Determine, and simplify,  
an expression for the probability generating function  
for this distribution, in terms of $q$ and $n$. 
The first of the dice to show a six does so on the $R$th roll. Find the  
expected value  of $R$ and show that, in the case 
 $n = 2$, $p=1/6$, this value  is $36/11$. 
 
Show that  the probability that the last of the dice to show a  
six does so on the $r$th roll 
is 
\[ 
\big(1-q^r\big)^n-\big(1-q^{r-1}\big)^n. 
\] 
Find, for the  
case $n = 2$, the probability generating function. The  
last of the dice to show a six does so on the $S$th roll. Find the expected value 
of $S$ and  evaluate this when $p=1/6$. 
\end{question}

%%%%%%%%%% Q14
\begin{question}
The random variable $X$ takes  
only the values $x_1$ and $x_2$  (where $ x_1 \not= x_2 $), 
 and the random variable $Y$ takes only  
the values $y_1$ and $y_2$  (where $y_1 \not= y_2$). 
Their joint distribution is given by  
$$ 
\P ( X = x_1 , Y = y_1 ) = a \ ; \ \  
\P ( X = x_1 , Y = y_2 ) = q - a \ ; \ \  
\P ( X = x_2 , Y = y_1 ) = p - a \ . 
$$  
Show that if $\E(X Y) = \E(X)\E(Y)$ then  
$$  
(a - p q ) ( x_1 - x_2 ) ( y_1 - y_2 ) = 0 . 
$$ 
Hence show that two random variables each taking  
only two distinct values are independent if  
$\E(X Y) = \E(X) \E(Y)$. 
 
Give a  joint distribution for two random variables  
$A$ and $B$, each taking the three values $- 1$, $0$ and $1$  
with probability ${1 \over 3}$,  
which have $\E(A B) = \E( A)\E (B)$, but which are not independent. 
\end{question}
	
\end{document}
