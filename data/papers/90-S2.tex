
\documentclass[a4, 11pt]{report}


\pagestyle{myheadings}
\markboth{}{Paper II, 1990
\ \ \ \ \ 
\today 
}               

\RequirePackage{amssymb}
\RequirePackage{amsmath}
\RequirePackage{graphicx}
\RequirePackage{color}
\RequirePackage[flushleft]{paralist}[2013/06/09]



\RequirePackage{geometry}
\geometry{%
  a4paper,
  lmargin=2cm,
  rmargin=2.5cm,
  tmargin=3.5cm,
  bmargin=2.5cm,
  footskip=12pt,
  headheight=24pt}


\newcommand{\comment}[1]{{\bf Comment} {\it #1}}
%\renewcommand{\comment}[1]{}

\newcommand{\bluecomment}[1]{{\color{blue}#1}}
%\renewcommand{\comment}[1]{}
\newcommand{\redcomment}[1]{{\color{red}#1}}



\usepackage{epsfig}
\usepackage{pstricks-add}
\usepackage{tgheros} %% changes sans-serif font to TeX Gyre Heros (tex-gyre)
\renewcommand{\familydefault}{\sfdefault} %% changes font to sans-serif
%\usepackage{sfmath}  %%%% this makes equation sans-serif
%\input RexFigs


\setlength{\parskip}{10pt}
\setlength{\parindent}{0pt}

\newlength{\qspace}
\setlength{\qspace}{20pt}


\newcounter{qnumber}
\setcounter{qnumber}{0}

\newenvironment{question}%
 {\vspace{\qspace}
  \begin{enumerate}[\bfseries 1\quad][10]%
    \setcounter{enumi}{\value{qnumber}}%
    \item%
 }
{
  \end{enumerate}
  \filbreak
  \stepcounter{qnumber}
 }


\newenvironment{questionparts}[1][1]%
 {
  \begin{enumerate}[\bfseries (i)]%
    \setcounter{enumii}{#1}
    \addtocounter{enumii}{-1}
    \setlength{\itemsep}{5mm}
    \setlength{\parskip}{8pt}
 }
 {
  \end{enumerate}
 }



\DeclareMathOperator{\cosec}{cosec}
\DeclareMathOperator{\Var}{Var}

\def\d{{\rm d}}
\def\e{{\rm e}}
\def\g{{\rm g}}
\def\h{{\rm h}}
\def\f{{\rm f}}
\def\p{{\rm p}}
\def\s{{\rm s}}
\def\t{{\rm t}}


\def\A{{\rm A}}
\def\B{{\rm B}}
\def\E{{\rm E}}
\def\F{{\rm F}}
\def\G{{\rm G}}
\def\H{{\rm H}}
\def\P{{\rm P}}


\def\bb{\mathbf b}
\def \bc{\mathbf c}
\def\bx {\mathbf x}
\def\bn {\mathbf n}

\newcommand{\low}{^{\vphantom{()}}}
%%%%% to lower suffices: $X\low_1$ etc


\newcommand{\subone}{ {\vphantom{\dot A}1}}
\newcommand{\subtwo}{ {\vphantom{\dot A}2}}




\def\le{\leqslant}
\def\ge{\geqslant}


\def\var{{\rm Var}\,}

\newcommand{\ds}{\displaystyle}
\newcommand{\ts}{\textstyle}




\begin{document}
\setcounter{page}{2}

 
\section*{Section A: \ \ \ Pure Mathematics}

%%%%%%%%%%Q1
\begin{question}
Prove that both $x^{4}-2x^{3}+x^{2}$ and $x^{2}-8x+17$ are non-negative
for all real $x$. By considering the intervals $x\leqslant0$, $0<x\leqslant2$
and $x>2$ separately, or otherwise, prove that the equation 
\[
x^{4}-2x^{3}+x^{2}-8x+17=0
\]
has no real roots. 


Prove that the equation $x^{4}-x^{3}+x^{2}-4x+4=0$ has no real roots.
\end{question}

%%%%%%%%%%Q2
\begin{question}
Prove that if $A+B+C+D=\pi,$ then 
\[
\sin\left(A+B\right)\sin\left(A+D\right)-\sin B\sin D=\sin A\sin C.
\]
The points $P,Q,R$ and $S$ lie, in that order, on a circle of centre
$O$. Prove that 
\[
PQ\times RS+QR\times PS=PR\times QS.
\]
\end{question}

%%%%%%%%% Q3
\begin{question}
Sketch the curves given by 
\[
y=x^{3}-2bx^{2}+c^{2}x,
\]
where $b$ and $c$ are non-negative, in the cases: 
\begin{alignat*}{6}
\sf{\bf{(i)\ \ }} & 2b<c\sqrt{3}, & \quad & \mathbf{(ii)\ \ } & 2b=c\sqrt{3}\neq0, & \quad & \mathbf{(iii)\ \ } & c\sqrt{3}<2b<2c, & \quad & \mathbf{(iv)\ \ } & b=c\neq0,\\
\mathbf{(v)\ \ } & b>c>0, &  & \mathbf{(vi)}\ \  & c=0,b\neq0, &  & \mathbf{(vii)\ \ } & c=b=0.
\end{alignat*}
Sketch also the curves given by $y^{2}=x^{3}-2bx^{2}+c^{2}x$ in the
cases \textbf{(i)}, \textbf{(v)} and \textbf{(vii)}. 


\end{question}

%%%%%% Q4 

\begin{question}
A plane contains $n$ distinct given lines, no two of which are parallel,
and no three of which intersect at a point. By first considering the
cases $n=1,2,3$ and $4$, provide and justify, by induction or otherwise,
a formula for the number of line segments (including the infinite
segments). 


Prove also that the plane is divided into $\frac{1}{2}(n^{2}+n+2)$
regions (including those extending to infinity). 
\end{question}


%%%%%%%%% Q5
\begin{question}
	The distinct points $L,M,P$ and $Q$ of the Argand diagram lie on
	a circle $S$ centred on the origin and the corresponding complex
	numbers are $l,m,p$ and $q$. By considering the perpendicular bisectors
	of the chords, or otherwise, prove that the chord $LM$ is perpendicular
	to the chord $PQ$ if and only if $lm+pq=0.$ 


	Let $A_{1},A_{2}$ and $A_{3}$ be three distinct points on $S$.
	For any given point $A_{1}'$ on $S$, the points $A_{2}',A_{3}'$
	and $A_{1}''$ are chosen on $S$ such that $A_{1}'A_{2}',A_{2}'A_{3}'$
	and $A_{3}'A_{1}''$ are perpendicular to $A_{1}A_{2},A_{2}A_{3}$
	and $A_{3}A_{1},$ respectively. Show that for exactly two positions
	of $A_{1}',$ the points $A_{1}'$ and $A_{1}''$ coincide. 


	If, instead, $A_{1},A_{2},A_{3}$ and $A_{4}$ are four given distinct
	points on $S$ and, for any given point $A_{1}',$ the points $A_{2}',A_{3}',A_{4}'$
	and $A_{1}''$ are chosen on $S$ such that $A_{1}'A_{2}',A_{2}'A_{3}',A_{3}'A_{4}'$
	and $A_{4}'A_{1}''$ are respectively perpendicular to $A_{1}A_{2},A_{2}A_{3},A_{3}A_{4}$
	and $A_{4}A_{1},$ show that $A_{1}'$ coincides with $A_{1}''.$ 


	Give the corresponding result for $n$ distinct points on $S$.
	\end{question}
	
	%%%%%%%%% Q6
	\begin{question}
Let $a,b,c,d,p$ and $q$ be positive integers. Prove that:

\begin{itemize}
\setlength{\itemsep}{3mm}
\item[\bf (i)] if $b>a$ and $c>1,$ then $bc\geqslant2c\geqslant2+c$; 
\item[\bf (ii)] if $a<b$ and $d<c$, then $bc-ad\geqslant a+c$; 
\item[\bf (iii)] if ${\displaystyle \frac{a}{b}<p<\frac{c}{d}}$, then $\left(bc-ad\right)p\geqslant a+c$; 
\item[\bf (iv)] if ${\displaystyle \frac{a}{b}<\frac{p}{q}<\frac{c}{d}},$ then ${\displaystyle p\geqslant\frac{a+c}{bc-ad}}$
and ${\displaystyle q\geqslant\frac{b+d}{bc-ad}}$. 
\end{itemize}

Hence find all fractions with denominators less than 20 which lie
between $8/9$ and $9/10$.
	 \end{question}
	 
	 %%%%%%%%% Q7
\begin{question}
	A damped system with feedback is modelled by the equation 
	\[
	\mathrm{f}'(t)+\mathrm{f}(t)-k\mathrm{f}(t-1)=0,\mbox{ }\tag{\ensuremath{\dagger}}
	\]
	where $k$ is a given non-zero constant. Show that (non-zero) solutions
	for $\mathrm{f}$ of the form \phantom{ } $\mathrm{f}(t)=A\mathrm{e}^{pt},$ where
	$A$ and $p$ are constants, are possible provided $p$ satisfies
	\[
	p+1=k\mathrm{e}^{-p}.\mbox{ }\tag{\ensuremath{*}}
	\]
	Show also, by means of a sketch, or otherwise, that equation $(*)$
	can have $0,1$ or $2$ real roots, depending on the value of $k$,
	and find the set of values of $k$ for which such solutions of $(\dagger)$
	exist. For what set of values of $k$ do such solutions tend to zero
	as $t\rightarrow+\infty$?
	\end{question}
	
	%%%%%%%%% Q8
	\begin{question}
The functions $\mbox{\ensuremath{\mathrm{x}} and \ensuremath{\mathrm{y}}}$
are related by 
\[
\mathrm{x}(t)=\int_{0}^{t}\mathrm{y}(u)\,\mathrm{d}u,
\]
so that $\mathrm{x}'(t)=\mathrm{y}(t)$. Show that 
\[
\int_{0}^{1}\mathrm{x}(t)\mathrm{y}(t)\,\mathrm{d}t=\tfrac{1}{2}\left[\mathrm{x}(1)\right]^{2}.
\]
In addition, it is given that $\mbox{y}(t)$ satisfies 
\[
\mathrm{y}''+\mbox{(\ensuremath{\mathrm{y}}}^{2}-1)\mathrm{y}'+\mathrm{y}=0,\mbox{ }(*)
\]
with $\mathrm{y}(0)=\mathrm{y}(1)$ and $\mathrm{y}'(0)=\mathrm{y}'(1)$.
By integrating $(*)$, prove that $\mathrm{x}(1)=0.$ 


By multiplying $(*)$ by $\mathrm{x}(t)$ and integrating by parts,
prove the relation 
\[
\int_{0}^{1}\left[\mathrm{y}(t)\right]^{2}\,\mathrm{d}t=\tfrac{1}{3}\int_{0}^{1}\left[\mathrm{y}(t)\right]^{4}\,\mathrm{d}t.
\]
Prove also the relation 
\[
\int_{0}^{1}\left[\mathrm{y}'(t)\right]^{2}\,\mathrm{d}t=\int_{0}^{1}\left[\mathrm{y}(t)\right]^{2}\,\mathrm{d}t.
\]

		\end{question}
		
		
%%%%%%%%% Q9
		\begin{question}
Show by means of a sketch that the parabola $r(1+\cos\theta)=1$ cuts
the interior of the cardioid $r=4(1+\cos\theta)$ into two parts. 


Show that the total length of the boundary of the part that includes
the point $r=1,\theta=0$ is $18\sqrt{3}+\ln(2+\sqrt{3}).$ 
		\end{question}
		
	
%%%%%%%%%% 10
\begin{question}
Two square matrices $\mathbf{A}$ and $\mathbf{B}$ satisfies $\mathbf{AB=0}.$
Show that either $\det\mathbf{A}=0$ or $\det\mathbf{B}=0$ or $\det\mathbf{A}=\det\mathbf{B}=0$.
If $\det\mathbf{B}\neq0$, what must $\mathbf{A}$ be? Give an example
to show that the condition $\det\mathbf{A}=\det\mathbf{B}=0$ is not
sufficient for the equation $\mathbf{AB=0}$ to hold. 


Find real numbers $p,q$ and $r$ such that 
\[
\mathbf{M}^{3}+2\mathbf{M}^{2}-5\mathbf{M}-6\mathbf{I}=(\mathbf{M}+p\mathbf{I})(\mathbf{M}+q\mathbf{I})(\mathbf{M}+r\mathbf{I}),
\]
where $\mathbf{M}$ is any square matrix and $\mathbf{I}$ is the
appropriate identity matrix. 


Hence, or otherwise, find all matrices $\mathbf{M}$ of the form $\begin{pmatrix}a & c\\
0 & b
\end{pmatrix}$ which satisfy the equation 
\[
\mathbf{M}^{3}+2\mathbf{M}^{2}-5\mathbf{M}-6\mathbf{I}=\mathbf{0}.
\]
			\end{question}
			
		
		
		
	
\newpage
\section*{Section B: \ \ \ Mechanics}


	
%%%%%%%%%% Q11
\begin{question}
A disc is free to rotate in a horizontal plane about a vertical axis
through its centre. The moment of inertia of the disc about this axis
is $mk^{2}.$ Along one diameter is a narrow groove in which a particle
of mass $m$ slides freely. At time $t=0,$ the disc is rotating with
angular speed $\Omega,$ and the particle is at a distance $a$ from
the axis and is moving towards the axis with speed $V$, where $k^{2}V^{2}=\Omega^{2}a^{2}(k^{2}+a^{2}).$
Show that, at a later time $t,$ while the particle is still moving
towards the axis, the angular speed $\omega$ of the disc and the
distance $r$ of the particle from the axis are related by 
\[
\omega=\frac{\Omega(k^{2}+a^{2})}{k^{2}+r^{2}}\qquad\mbox{ and }\qquad\frac{\mathrm{d}r}{\mathrm{d}t}=-\frac{\Omega r(k^{2}+a^{2})}{k(k^{2}+r^{2})^{\frac{1}{2}}}.
\]
Deduce that 
\[
k\frac{\mathrm{d}r}{\mathrm{d}\theta}=-r(k^{2}+r^{2})^{\frac{1}{2}},
\]
where $\theta$ is the angle through which the disc has turned at
time $t$. By making the substitution $u=1/r$, or otherwise, show
that $r\sinh(\theta+\alpha)=k,$ where $\sinh\alpha=k/a.$ Hence,
or otherwise, show that the particle never reaches the axis. 
	\end{question}
	
%%%%%%%%%% Q12
\begin{question}	
A straight staircase consists of $N$ smooth horizontal stairs each
of height $h$. A particle slides over the top stair at speed $U$,
with velocity perpendicular to the edge of the stair, and then falls
down the staircase, bouncing once on every stair. The coefficient
of restitution between the particle and each stair is $e$, where
$e<1$. Show that the horizontal distance $d_{n}$ travelled between
the $n$th and $(n+1)$th bounces is given by 
\[
d_{n}=U\left(\frac{2h}{g}\right)^{\frac{1}{2}}\left(e\alpha_{n}+\alpha_{n+1}\right),
\]
where ${\displaystyle \alpha_{n}=\left(\frac{1-e^{2n}}{1-e^{2}}\right)^{\frac{1}{2}}}$. 


If $N$ is very large, show that $U$ must satisfy
\[
U=\left(\frac{L^{2}g}{2h}\right)^{\frac{1}{2}}\left(\frac{1-e}{1+e}\right)^{\frac{1}{2}},
\]
where $L$ is the horizontal distance between the edges of successive
stairs. 
\end{question}

%%%%%%%%%% Q13

\begin{question}
A thin non-uniform rod $PQ$ of length $2a$ has its centre of gravity
a distance $a+d$ from $P$. It hangs (not vertically) in equilibrium
suspended from a small smooth peg $O$ by means of a light inextensible
string of length $2b$ which passes over the peg and is attached at
its ends to $P$ and $Q$. Express $OP$ and $OQ$ in terms of $a,b$
and $d$. By considering the angle $POQ$, or otherwise, show that
$d<a^{2}/b$. 

\end{question}
	
%%%%%%%%%% Q14
\begin{question}
The identical uniform smooth spherical marbles $A_{1},A_{2},\ldots,A_{n},$
where $n\geqslant3,$ each of mass $m,$ lie in that order in a smooth
straight trough, with each marble touching the next. The marble $A_{n+1},$
which is similar to $A_{n}$ but has mass $\lambda m,$ is placed
in the trough so that it touches $A_{n}.$ Another marble $A_{0},$
identical to $A_{n},$ slides along the trough with speed $u$ and
hits $A_{1}.$ It is given that kinetic energy is conserved throughout. 
\begin{questionparts}
\item Show that if $\lambda<1,$ there is a possible subsequent motion in
which only $A_{n}$ and $A_{n+1}$ move (and $A_{0}$ is reduced to
rest), but that if $\lambda>1,$ such a motion is not possible. 
\item If $\lambda>1,$ show that a subsequent motion in which only $A_{n-1},A_{n}$
and $A_{n+1}$ move is not possible. 
\item If $\lambda>1,$ find a possible subsequent motion in which only two
marbles move. 
\end{questionparts}
\end{question}
	
	\newpage
\section*{Section C: \ \ \ Probability and Statistics}


%%%%%%%%%% Q15
\begin{question}
A target consists of a disc of unit radius and centre $O$. A certain
marksman never misses the target, and the probability of any given
shot hitting the target within a distance $t$ from $O$ it $t^{2}$,
where $0\leqslant t\leqslant1$. The marksman fires $n$ shots independently.
The random variable $Y$ is the radius of the smallest circle, with
centre $O$, which encloses all the shots. Show that the probability
density function of $Y$ is $2ny^{2n-1}$ and find the expected area
of the circle. 


The shot which is furthest from $O$ is rejected. Show that the expected
area of the smallest circle, with centre $O$, which encloses the
remaining $(n-1)$ shots is 
\[
\left(\frac{n-1}{n+1}\right)\pi.
\]
\end{question}

%%%%%%%%%% Q16
\begin{question}
Each day, I choose at random between my brown trousers, my grey trousers
and my expensive but fashionable designer jeans. Also in my wardrobe,
I have a black silk tie, a rather smart brown and fawn polka-dot tie,
my regimental tie, and an elegant powder-blue cravat which I was given
for Christmas. With my brown or grey trousers, I choose ties (including
the cravat) at random, except of course that I don\textquoteright t
wear the cravat with the brown trousers or the polka-dot tie with
the grey trousers. With the jeans, the choice depends on whether it
is Sunday or one of the six weekdays: on weekdays, half the time I
wear a cream-coloured sweat-shirt with $E=mc{}^{2}$ on the front
and no tie; otherwise, and on Sundays (when naturally I always wear
a tie), I just pick at random from my four ties.


This morning, I received through the post a compromising photograph
of myself. I often receive such photographs and they are equally likely
to have been taken on any day of the week. However, in this particular
photograph, I am wearing my black silk tie. Show that, on the basis
of this information, the probability that the photograph was taken
on Sunday is $11/68$. 


I should have mentioned that on Mondays I lecture on calculus and
I therefore always wear my jeans (to make the lectures seem easier
to understand). Find, on the basis of the complete information, the
probability that the photograph was taken on Sunday.

[The phrase `at random' means `with equal probability'.]
\end{question}
\end{document}
