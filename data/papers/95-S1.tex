\documentclass[a4, 11pt]{report}


\pagestyle{myheadings}
\markboth{}{Paper I, 1995
\ \ \ \ \ 
\today 
}               

\RequirePackage{amssymb}
\RequirePackage{amsmath}
\RequirePackage{graphicx}
\RequirePackage{color}
\RequirePackage[flushleft]{paralist}[2013/06/09]



\RequirePackage{geometry}
\geometry{%
  a4paper,
  lmargin=2cm,
  rmargin=2.5cm,
  tmargin=3.5cm,
  bmargin=2.5cm,
  footskip=12pt,
  headheight=24pt}


\newcommand{\comment}[1]{{\bf Comment} {\it #1}}
%\renewcommand{\comment}[1]{}

\newcommand{\bluecomment}[1]{{\color{blue}#1}}
%\renewcommand{\comment}[1]{}
\newcommand{\redcomment}[1]{{\color{red}#1}}



\usepackage{epsfig}
\usepackage{pstricks-add}
\usepackage{tgheros} %% changes sans-serif font to TeX Gyre Heros (tex-gyre)
\renewcommand{\familydefault}{\sfdefault} %% changes font to sans-serif
%\usepackage{sfmath}  %%%% this makes equation sans-serif
%\input RexFigs


\setlength{\parskip}{10pt}
\setlength{\parindent}{0pt}

\newlength{\qspace}
\setlength{\qspace}{20pt}


\newcounter{qnumber}
\setcounter{qnumber}{0}

\newenvironment{question}%
 {\vspace{\qspace}
  \begin{enumerate}[\bfseries 1\quad][10]%
    \setcounter{enumi}{\value{qnumber}}%
    \item%
 }
{
  \end{enumerate}
  \filbreak
  \stepcounter{qnumber}
 }


\newenvironment{questionparts}[1][1]%
 {
  \begin{enumerate}[\bfseries (i)]%
    \setcounter{enumii}{#1}
    \addtocounter{enumii}{-1}
    \setlength{\itemsep}{5mm}
    \setlength{\parskip}{8pt}
 }
 {
  \end{enumerate}
 }



\DeclareMathOperator{\cosec}{cosec}
\DeclareMathOperator{\Var}{Var}

\def\d{{\rm d}}
\def\e{{\rm e}}
\def\g{{\rm g}}
\def\h{{\rm h}}
\def\f{{\rm f}}
\def\p{{\rm p}}
\def\s{{\rm s}}
\def\t{{\rm t}}


\def\A{{\rm A}}
\def\B{{\rm B}}
\def\E{{\rm E}}
\def\F{{\rm F}}
\def\G{{\rm G}}
\def\H{{\rm H}}
\def\P{{\rm P}}


\def\bb{\mathbf b}
\def \bc{\mathbf c}
\def\bx {\mathbf x}
\def\bn {\mathbf n}

\newcommand{\low}{^{\vphantom{()}}}
%%%%% to lower suffices: $X\low_1$ etc


\newcommand{\subone}{ {\vphantom{\dot A}1}}
\newcommand{\subtwo}{ {\vphantom{\dot A}2}}




\def\le{\leqslant}
\def\ge{\geqslant}


\def\var{{\rm Var}\,}

\newcommand{\ds}{\displaystyle}
\newcommand{\ts}{\textstyle}




\begin{document}
\setcounter{page}{2}

 
\section*{Section A: \ \ \ Pure Mathematics}

%%%%%%%%%%Q1
\begin{question}
\begin{questionparts} 
\item Find the real values of $x$ for which 
\[
x^{3}-4x^{2}-x+4\geqslant0.
\]
\item Find the three lines in the $(x,y)$ plane on which 
\[
x^{3}-4x^{2}y-xy^{2}+4y^{3}=0.
\]
\item On a sketch shade the regions of the $(x,y)$ plane for which
\[
x^{3}-4x^{2}y-xy^{2}+4y^{3}\geqslant0.
\]
\end{questionparts}
\end{question}

%%%%%%%%%%Q2
\begin{question}
\begin{questionparts} 
\item Suppose that 
\[
S=\int\frac{\cos x}{\cos x+\sin x}\,\mathrm{d}x\quad\mbox{ and }\quad T=\int\frac{\sin x}{\cos x+\sin x}\,\mathrm{d}x.
\]
By considering $S+T$ and $S-T$ determine $S$ and $T$. 
\item Evaluate ${\displaystyle \int_{\frac{1}{4}}^{\frac{1}{2}}(1-4x)\sqrt{\frac{1}{x}-1}\,\mathrm{d}x}$
by using the substitution $x=\sin^{2}t.$ 
\end{questionparts}
\end{question}

%%%%%%%%% Q3
\begin{question}
\begin{questionparts}
 \item If $\mathrm{f}(r)$ is a function defined
for $r=0,1,2,3,\ldots,$ show that 
\[
\sum_{r=1}^{n}\left\{ \mathrm{f}(r)-\mathrm{f}(r-1)\right\} =\mathrm{f}(n)-\mathrm{f}(0).
\]
\item If $\mathrm{f}(r)=r^{2}(r+1)^{2},$ evaluate $\mathrm{f}(r)-\mathrm{f}(r-1)$
and hence determine ${\displaystyle \sum_{r=1}^{n}r^{3}.}$
\item Find the sum of the series $1^{3}-2^{3}+3^{3}-4^{3}+\cdots+(2n+1)^{3}.$ 
\end{questionparts}
\end{question}

%%%%%% Q4 
\begin{question}
By applying de Moivre's theorem to $\cos5\theta+\mathrm{i}\sin5\theta,$
expanding the result using the binomial theorem, and then equating
imaginary parts, show that 
\[
\sin5\theta=\sin\theta\left(16\cos^{4}\theta-12\cos^{2}\theta+1\right).
\]
Use this identity to evaluate $\cos^{2}\frac{1}{5}\pi$, and deduce
that $\cos\frac{1}{5}\pi=\frac{1}{4}(1+\sqrt{5}).$ 
	\end{question}

%%%%%%%%% Q5
\begin{question}
If 
\[
\mathrm{f}(x)=nx-\binom{n}{2}\frac{x^{2}}{2}+\binom{n}{3}\frac{x^{3}}{3}-\cdots+(-1)^{r+1}\binom{n}{r}\frac{x^{r}}{r}+\cdots+(-1)^{n+1}\frac{x^{n}}{n}\,,
\]
show that 
\[
\mathrm{f}'(x)=\frac{1-(1-x)^{n}}{x}\,.
\]
Deduce that 
\[
\mathrm{f}(x)=\int_{1-x}^{1}\frac{1-y^{n}}{1-y}\,\mathrm{d}y.
\]
Hence show that 
\[
\mathrm{f}(1)=1+\frac{1}{2}+\frac{1}{3}+\cdots+\frac{1}{n}\,.
\]

	\end{question}
	
%%%%%%%%% Q6
\begin{question}
\begin{questionparts}
\item In the differential equation 
\[
\frac{1}{y^{2}}\frac{\mathrm{d}y}{\mathrm{d}x}+\frac{1}{y}=\mathrm{e}^{2x}
\]
make the substitution $u=1/y,$ and hence show that the general solution
of the original equation is 
\[
y=\frac{1}{A\mathrm{e}^{x}-\mathrm{e}^{2x}}\,.
\]
\item Use a similar method to solve the equation 
\[
\frac{1}{y^{3}}\frac{\mathrm{d}y}{\mathrm{d}x}+\frac{1}{y^{2}}=\mathrm{e}^{2x}.
\]
\end{questionparts}
\end{question}
	
%%%%%%%%% Q7
\begin{question}
Let $A,B,C$ be three non-collinear points in the plane. Explain briefly
why it is possible to choose an origin equidistant from the three
points. Let $O$ be such an origin, let $G$ be the centroid of the
triangle $ABC,$ let $Q$ be a point such that $\overrightarrow{GQ}=2\overrightarrow{OG},$
and let $N$ be the midpoint of $OQ.$ 

\begin{questionparts}
\item Show that $\overrightarrow{AQ}$ is perpendicular to $\overrightarrow{BC}$
and deduce that the three altitudes of $\triangle ABC$ are concurrent. 
\item Show that the midpoints of $AQ,BQ$ and $CQ$, and the midpoints of
the sides of $\triangle ABC$ are all equidistant from $N$. 
\end{questionparts}

{[}The \textit{centroid }of $\triangle ABC$ is the point $G$ such
that $\overrightarrow{OG}=\frac{1}{3}(\overrightarrow{OA}+\overrightarrow{OB}+\overrightarrow{OC}).$
The \textit{altitudes }of the triangle are the lines through the vertices
perpendicular to the opposite sides.{]} 
\end{question}
		
%%%%%%%%% Q8
\begin{question}	
Find functions $\mathrm{f,g}$ and $\mathrm{h}$ such that 
\[
\frac{\mathrm{d}^{2}y}{\mathrm{d}x^{2}}+\mathrm{f}(x)\frac{\mathrm{d}y}{\mathrm{d}x}+\mathrm{g}(x)y=\mathrm{h}(x)\tag{\ensuremath{*}}
\]
is satisfied by all three of the solutions $y=x,y=1$ and $y=x^{-1}$
for $0<x<1.$ 


If $\mathrm{f,g}$ and $\mathrm{h}$ are the functions you have found
in the first paragraph, what condition must the real numbers $a,b$
and $c$ satisfy in order that 
\[
y=ax+b+\frac{c}{x}
\]
should be a solution of $(*)$?
\end{question}	
		

		
	
\newpage
\section*{Section B: \ \ \ Mechanics}


	
%%%%%%%%%% Q9
\begin{question}
A particle is projected from a point $O$ with speed $\sqrt{2gh},$
where $g$ is the acceleration due to gravity. Show that it is impossible,
whatever the angle of projection, for the particle to reach a point
above the parabola 
\[
x^{2}=4h(h-y),
\]
where $x$ is the horizontal distance from $O$ and $y$ is the vertical
distance above $O$. State briefly the simplifying assumptions which
this solution requires. 
	\end{question}
	
%%%%%%%%%% Q10
\begin{question}	
A small ball of mass $m$ is suspended in equilibrium by a light elastic
string of natural length $l$ and modulus of elasticity $\lambda.$
Show that the total length of the string in equilibrium is $l(1+mg/\lambda).$ 


If the ball is now projected downwards from the equilibrium position
with speed $u_{0},$ show that the speed $v$ of the ball at distance
$x$ below the equilibrium position is given by 
\[
v^{2}+\frac{\lambda}{lm}x^{2}=u_{0}^{2}.
\]
At distance $h$, where $\lambda h^{2}<lmu_{0}^{2},$ below the equilibrium
position is a horizontal surface on which the ball bounces with a
coefficient of restitution $e$. Show that after one bounce the velocity
$u_{1}$ at $x=0$ is given by 
\[
u_{1}^{2}=e^{2}u_{0}^{2}+\frac{\lambda}{lm}h^{2}(1-e^{2}),
\]
and that after the second bounce the velocity $u_{2}$ at $x=0$ is
given by 
\[
u_{2}^{2}=e^{4}u_{0}^{2}+\frac{\lambda}{lm}h^{2}(1-e^{4}).
\]
\end{question}

%%%%%%%%%% Q11

\begin{question}
Two identical uniform cylinders, each of mass $m,$ lie in contact
with one another on a horizontal plane and a third identical cylinder
rests symmetrically on them in such a way that the axes of the three
cylinders are parallel. Assuming that all the surfaces in contact
are equally rough, show that the minimum possible coefficient of friction
is $2-\sqrt{3}.$ 
\end{question}
	

	
	\newpage
\section*{Section C: \ \ \ Probability and Statistics}


%%%%%%%%%% Q12
\begin{question}
A school has $n$ pupils, of whom $r$ play hocket, where $n\geqslant r\geqslant2.$
All $n$ pupils are arranged in a row at random. 

\begin{questionparts}
\item What is the probability that there is a hockey player at each end
of the row?
\item What is the probability that all the hockey players are standing together?
\item By considering the gaps between the non-hockey-players, find the probability
that no two hockey players are standing together, distinguishing between
cases when the probability is zero and when it is non-zero. 
\end{questionparts}
\end{question}

%%%%%%%%%% Q13
\begin{question}
A scientist is checking a sequence of microscope slides for cancerous
cells, marking each cancerous cell that she detects with a red dye.
The number of cancerous cells on a slide is random and has a Poisson
distribution with mean $\mu.$ The probability that the scientist
spots any one cancerous cell is $p$, and is independent of the probability
that she spots any other one. 
\begin{questionparts}
\item Show that the number of cancerous cells which she marks on a single
slide has a Poisson distribution of mean $p\mu.$ 
\item Show that the probability $Q$ that the second cancerous cell which
she marks is on the $k$th slide is given by 
\[
Q=\mathrm{e}^{-\mu p(k-1)}\left\{ (1+k\mu p)(1-\mathrm{e}^{-\mu p})-\mu p\right\} .
\]
\end{questionparts}
\end{question}

%%%%%%%%%% Q14
\begin{question}
\begin{questionparts}
\item Find the maximum value of $\sqrt{p(1-p)}$
as $p$ varies between $0$ and $1$. 
\item Suppose that a proportion $p$ of the population is female.
In order to estimate $p$ we pick a sample of $n$ people at random
and find the proportion of them who are female. Find the value of
$n$ which ensures that the chance of our estimate of $p$ being more
than $0.01$ in error is less than 1\%. 
\item Discuss how the required value of $n$ would be affected if
(a) $p$ were the proportion of people in the population who are left-handed;
(b) $p$ were the proportion of people in the population who are millionaires. 
\end{questionparts}
\end{question}
	
\end{document}
