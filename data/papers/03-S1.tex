\documentclass[a4, 11pt]{report}


\pagestyle{myheadings}
\markboth{}{Paper I, 2003
\ \ \ \ \ 
\today 
}               

\RequirePackage{amssymb}
\RequirePackage{amsmath}
\RequirePackage{graphicx}
\RequirePackage{color}
\RequirePackage[flushleft]{paralist}[2013/06/09]



\RequirePackage{geometry}
\geometry{%
  a4paper,
  lmargin=2cm,
  rmargin=2.5cm,
  tmargin=3.5cm,
  bmargin=2.5cm,
  footskip=12pt,
  headheight=24pt}


\newcommand{\comment}[1]{{\bf Comment} {\it #1}}
%\renewcommand{\comment}[1]{}

\newcommand{\bluecomment}[1]{{\color{blue}#1}}
%\renewcommand{\comment}[1]{}
\newcommand{\redcomment}[1]{{\color{red}#1}}



\usepackage{epsfig}
\usepackage{pstricks-add}
\usepackage{tgheros} %% changes sans-serif font to TeX Gyre Heros (tex-gyre)
\renewcommand{\familydefault}{\sfdefault} %% changes font to sans-serif
%\usepackage{sfmath}  %%%% this makes equation sans-serif
%\input RexFigs


\setlength{\parskip}{10pt}
\setlength{\parindent}{0pt}

\newlength{\qspace}
\setlength{\qspace}{20pt}


\newcounter{qnumber}
\setcounter{qnumber}{0}

\newenvironment{question}%
 {\vspace{\qspace}
  \begin{enumerate}[\bfseries 1\quad][10]%
    \setcounter{enumi}{\value{qnumber}}%
    \item%
 }
{
  \end{enumerate}
  \filbreak
  \stepcounter{qnumber}
 }


\newenvironment{questionparts}[1][1]%
 {
  \begin{enumerate}[\bfseries (i)]%
    \setcounter{enumii}{#1}
    \addtocounter{enumii}{-1}
    \setlength{\itemsep}{5mm}
    \setlength{\parskip}{8pt}
 }
 {
  \end{enumerate}
 }



\DeclareMathOperator{\cosec}{cosec}
\DeclareMathOperator{\Var}{Var}

\def\d{{\mathrm d}}
\def\e{{\mathrm e}}
\def\g{{\mathrm g}}
\def\h{{\mathrm h}}
\def\f{{\mathrm f}}
\def\p{{\mathrm p}}
\def\s{{\mathrm s}}
\def\t{{\mathrm t}}


\def\A{{\mathrm A}}
\def\B{{\mathrm B}}
\def\E{{\mathrm E}}
\def\F{{\mathrm F}}
\def\G{{\mathrm G}}
\def\H{{\mathrm H}}
\def\P{{\mathrm P}}


\def\bb{\mathbf b}
\def \bc{\mathbf c}
\def\bx {\mathbf x}
\def\bn {\mathbf n}

\newcommand{\low}{^{\vphantom{()}}}
%%%%% to lower suffices: $X\low_1$ etc


\newcommand{\subone}{ {\vphantom{\dot A}1}}
\newcommand{\subtwo}{ {\vphantom{\dot A}2}}




\def\le{\leqslant}
\def\ge{\geqslant}


\def\var{{\rm Var}\,}

\newcommand{\ds}{\displaystyle}
\newcommand{\ts}{\textstyle}
\def\half{{\textstyle \frac12}}
\def\l{\left(}
\def\r{\right)}



\begin{document}
\setcounter{page}{2}

 
\section*{Section A: \ \ \ Pure Mathematics}

%%%%%%%%%%Q1
\begin{question}
It is given that $\sum\limits_{r=-1}^ {n} r^2$  can be written  in the form 
$pn^3 +qn^2+rn+s\,$, where $p\,$, $q\,$, $r\,$ and $s$ are numbers. By setting $n=-1$, $0$, $1$  and $2$,
obtain four  equations that must be satisfied by $p\,$, $q\,$, $r\,$ and $s$ 
and hence show that
\[
{\ts \sum\limits_{r=0} ^n} r^2= {\textstyle \frac16} n(n+1)(2n+1)\;.
\]

Given that $\sum\limits_{r=-2}^ nr^3$  can be written  in the form 
$an^4 +bn^3+cn^2+dn +e\,$,
show similarly that
\[
{\ts \sum\limits_{r=0} ^n} r^3= {\textstyle \frac14} n^2(n+1)^2\;.
\]
\end{question}

%%%%%%%%%%Q2
\begin{question}
The first question on an examination paper is:

\hspace*{3cm}
Solve for $x$ the equation \ \ \
$\ds
\frac 1x = \frac 1 a + \frac 1b \;.
$

\noindent
where (in the question) $a$ and $b$ are given non-zero real numbers.
One candidate writes $x=a+b$ as the solution. Show that there
are no values of $a$ and $b$ for which this will give the correct answer.

The  next question on the  examination paper is:

\hspace*{3cm}
Solve for $x$ the equation \ \ \ 
$\ds
\frac 1x = \frac 1 a + \frac 1b  +\frac 1c \;.
$

\noindent
where (in the question) $a\,$, $b$ and $c$  are given non-zero numbers.
The  candidate uses the same technique, giving the answer as
$\ds
x = a + b  +c \;.
$
Show that the candidate's
answer will be correct if and only if  $a\,$, $b$ and $c$
satisfy at least one of the equations
$a+b=0\,$, $b+c=0$ or $c+a=0\,$.
\end{question}

%%%%%%%%% Q3
\begin{question}
\begin{questionparts}
\item
 Show that  $ 2\sin(\half\theta)=\sin \theta$ 
if and only if $\sin(\half\theta)=0\,$. 

\item Solve the equation $2\tan (\half\theta) = \tan\theta\,$.
 
\item
Show that $2\cos(\half \theta)=\cos \theta$ 
if and only if $\theta=(4n+2)\pi\pm 2\phi$ where $\phi$ is 
defined by
$\cos \phi=\half(\sqrt 3-1)\;$, $0\le \phi\le \frac{1}{2}\pi$, and 
$n$ is any integer. 
\end{questionparts}
\end{question}

%%%%%% Q4 
\begin{question}
Solve the inequality
$$\frac{\sin\theta+1}{\cos\theta}\le1\;$$
where $0\le\theta<2\pi\,$ and $\cos\theta\ne0\,$.
\end{question}

%%%%%%%%% Q5
\begin{question}
\begin{questionparts}
\item
In the binomial 
expansion of $(2x+1/x^{2})^{6}\;$ for $x\ne0$, show that the
 term which is independent of $x$ is $240$. 

Find the term which is independent of $x$ in the binomial 
expansion of $(ax^3+b/x^{2})^{5n}\,$.

\item Let $\f(x) =(x^6+3x^5)^{1/2}\,$. By considering the expansion 
 of
$(1+3/x)^{1/2}\,$ show that the term
which is independent of $x$ in the expansion 
of $\f(x)$ in powers of $1/x\,$,  for $ \vert x\vert>3\,$, is $27/16\,$. 

Show that there is no term independent of $x\,$ in the
expansion  of $\f(x)$ in powers of $x\,$,  for $ \vert x\vert<3\,$. 

\end{questionparts}
	\end{question}
	
%%%%%%%%% Q6
\begin{question}
 Evaluate the following integrals, in the different cases that
arise according to the value
 of the positive constant
$a\,$:
\begin{questionparts}
\item \hspace{1cm}
$\ds
\int_0^1  \frac 1 {x^2 + (a+2)x +2a} \; \d x  \;;
$
\\[3mm]

\item
\hspace{1cm}
$\ds
\int _{1}^2\frac 1 {u^2 +au +a-1} \; \d u  \;.
$
\end{questionparts}
\end{question}
	
%%%%%%%%% Q7
\begin{question}
Let $k$ be an integer satisfying $0\le k \le 9\,$.
Show that  $0\le 10k-k^2\le 25$ and that 
$k(k-1)(k+1)$ is 
divisible by $3\,$.

For each $3$-digit number $N$, where $N\ge100$, let $S$  be the sum of the
hundreds digit, the square of the {tens} digit and the
cube of the {units} digit. Find the numbers $N$ such that
$S=N$.

\noindent[Hint: write $N=100a+10b+c\,$ where $a\,$, $b\,$ and $c$ are the
digits of $N\,$.]
\end{question}
		
%%%%%%%%% Q8
\begin{question}	
A liquid of fixed volume $V$ is made up of two chemicals $A$ and $B\,$. 
A reaction takes place in which $A$ converts to $B\,$. The volume 
of $A$ at time $t$ is $xV$ 
and the volume of $B$ at time 
$t$ is $yV$ where $x$ and $y$ depend on $t$ and $x+y=1\,$. 
The rate at which $A$ converts into $B$ is given by $kVxy\,$,
 where $k$ is a  positive constant. Show that if both $x$ and $y$ are
 strictly positive at the start, then at time $t$
\[
y= \frac {D\e^{kt}}{1+D \e^{kt}} \;,
\]
where $D$ is a constant. 

Does 
$A$ ever completely convert to
 $B\,$? Justify your answer.
\end{question}	
		

		
	
\newpage
\section*{Section B: \ \ \ Mechanics}


	
%%%%%%%%%% Q9
\begin{question}
A particle is projected with speed  $V$ 
at an angle $\theta$ above  the horizontal. 
The particle passes through the  point $P$ which is a horizontal distance
$d$ and a vertical distance
$h$  from the point of projection. 
Show that 
\[
T^2 -2kT + \frac{2kh}{d}+1=0\;,
\]
where $T=\tan\theta$ and $\ds k= \frac{V^2}{gd}\,$.

%Derive an equation relating $\tan \theta$, $V$, $g$, $d$ and $h$. 
Show that, if 
$\displaystyle {kd > h + \sqrt {h^2 + d^2}}\;$,
 there are two distinct possible angles of projection.



Let these two angles be $\alpha$ and $\beta$. 
Show that 
$\displaystyle \alpha + \beta = \pi - \arctan ( {d/  h}) \,$.
	\end{question}
	
%%%%%%%%%% Q10 
\begin{question}	
$ABCD$ is a uniform rectangular lamina and $X$ is a point on $BC\,$.
The lengths of $AD$, $AB$ and $BX$ are  $p\,$, $q$ and $r$ respectively. 
The triangle $ABX$ is cut off the lamina.
Let $(a,b)$ be the position of the centre of gravity of the lamina,
where the axes are such that 
the coordinates of $A\,$, $D$ and $C$  are $(0,0)\,$,
$(p,0)$ and $(p,q)$ respectively. Derive equations for
$a$ and $b$ in terms of $p\,$, $q$ and $r\,$.

 
When the resulting trapezium is freely suspended from the point $A\,$, 
the side $AD$ is inclined at $45^\circ$ below the horizontal. 
Show that 
$\displaystyle r = q - \sqrt{q^2 - 3pq + 3p^2}\,$. You should justify 
carefully the choice of sign in front of the square root.
\end{question}

%%%%%%%%%% Q11

\begin{question}
A smooth plane is inclined at an angle $\alpha$ to the horizontal.
$A$ and $B$ are two points a distance~$d$ apart
on a line of greatest slope of the  plane, with $B$ higher than $A$. 
A particle is projected up the plane from $A$ towards $B$ with 
initial speed  $u$, and simultaneously another particle 
is released from rest at $B\,$. Show that they collide after 
a time $\displaystyle {d /u}\,$.

The coefficient of restitution between the two particles is $e$ and both
particles have mass $m\,$. 
Show that the loss of kinetic energy in the collision is 
$\frac14 {m u^2 \big( 1 - e^2 \big) }\,$.
\end{question}
	

	
	\newpage
\section*{Section C: \ \ \ Probability and Statistics}


%%%%%%%%%% Q12
\begin{question}
In a bag are $n$ balls numbered 1, 2, $\ldots\,$, $n\,$. When a ball is taken
out of the bag, each ball is equally likely to be taken. 
\begin{questionparts}
\item A ball is taken out of the bag. The number on the ball is noted
and the ball is replaced in the bag. The process is repeated once.
Explain why the expected value of the product of the numbers on the two balls
is 
\[
\frac 1 {n^2}  \sum_{r=1}^n\sum_{s=1}^n rs 
\]
and simplify this expression.

\item
 A ball is taken out of the bag. The number on the ball is noted
and the ball is {\sl not} replaced in the bag. Another ball is taken out of the 
bag and the number on this ball is noted.
Show that the expected value of the product of the two numbers is 
\[
\frac{(n+1)(3n+2)}{12}\;.
\]
\end{questionparts}

\noindent{\bf Note: } $\ds\sum_{r=1}^n r = \frac12 n(n+1)$ \ \ and \  \
 $\ds\sum_{r=1}^n r^2 = \frac16 n(n+1)(2n+1)\;$.

\end{question}

%%%%%%%%%% Q13
\begin{question}
If a football match ends in a draw, 
there may be a "penalty shoot-out". Initially the 
teams each take 5 shots at
goal. If one team scores more times than the other, 
then that team wins. If the scores are level, the
teams take shots alternately until 
one team scores and the other team does not score,
both teams having taken the same number of shots. The team that scores wins.

Two teams, Team A and Team B, take part in a penalty shoot-out.
Their probabilities of  scoring when they take a single 
shot are  $p_A$ and $p_B$ respectively.
Explain why the
probability $\alpha$  of neither side having won at the end of the 
initial $10$-shot  period is given by
$$\alpha =\sum_{i=0}^5\binom{5}{i}^2(1-p_A)^i(1-p_B)^i\,p_A^{5-i}p_B^{5-i}.$$


Show that the expected number of shots  taken is 
$\ds10+ \frac{2\alpha}\beta\;,$ 
where
$\beta=p_A+p_B-2p_Ap_B\,.$
\end{question}

%%%%%%%%%% Q14
\begin{question}
 Jane goes out with any of her friends 
who call, except that she never  goes out with more
than two friends in a day. The number of her friends who call
on a given day follows  a Poisson distribution with parameter $2$.
Show that the average number of friends she sees in a day is~$2-4\e^{-2}\,$.

Now Jane has a new friend who calls on any given day with 
probability $p$. Her old friends call as before, independently
of the new friend.  She never  goes out with more
than two friends in a day. Find the average  number of friends she now
sees in a day.
\end{question}
	
\end{document}
