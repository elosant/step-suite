\documentclass[a4, 11pt]{report}


\pagestyle{myheadings}
\markboth{}{Paper III, 2013
\ \ \ \ \ 
\today 
}               

\RequirePackage{amssymb}
\RequirePackage{amsmath}
\RequirePackage{graphicx}
\RequirePackage{color}
\RequirePackage[flushleft]{paralist}[2013/06/09]



\RequirePackage{geometry}
\geometry{%
  a4paper,
  lmargin=2cm,
  rmargin=2.5cm,
  tmargin=3.5cm,
  bmargin=2.5cm,
  footskip=12pt,
  headheight=24pt}


\newcommand{\comment}[1]{{\bf Comment} {\it #1}}
%\renewcommand{\comment}[1]{}

\newcommand{\bluecomment}[1]{{\color{blue}#1}}
%\renewcommand{\comment}[1]{}
\newcommand{\redcomment}[1]{{\color{red}#1}}



\usepackage{epsfig}
\usepackage{pstricks-add}
\usepackage{tgheros} %% changes sans-serif font to TeX Gyre Heros (tex-gyre)
\renewcommand{\familydefault}{\sfdefault} %% changes font to sans-serif
%\usepackage{sfmath}  %%%% this makes equation sans-serif
%\input RexFigs


\setlength{\parskip}{10pt}
\setlength{\parindent}{0pt}

\newlength{\qspace}
\setlength{\qspace}{20pt}


\newcounter{qnumber}
\setcounter{qnumber}{0}

\newenvironment{question}%
 {\vspace{\qspace}
  \begin{enumerate}[\bfseries 1\quad][10]%
    \setcounter{enumi}{\value{qnumber}}%
    \item%
 }
{
  \end{enumerate}
  \filbreak
  \stepcounter{qnumber}
 }


\newenvironment{questionparts}[1][1]%
 {
  \begin{enumerate}[\bfseries (i)]%
    \setcounter{enumii}{#1}
    \addtocounter{enumii}{-1}
    \setlength{\itemsep}{5mm}
    \setlength{\parskip}{8pt}
 }
 {
  \end{enumerate}
 }



\DeclareMathOperator{\cosec}{cosec}
\DeclareMathOperator{\Var}{Var}

\def\d{{\mathrm d}}
\def\e{{\mathrm e}}
\def\g{{\mathrm g}}
\def\h{{\mathrm h}}
\def\f{{\mathrm f}}
\def\p{{\mathrm p}}
\def\s{{\mathrm s}}
\def\t{{\mathrm t}}
\def\i{{\mathrm i}}

\def\A{{\mathrm A}}
\def\B{{\mathrm B}}
\def\E{{\mathrm E}}
\def\F{{\mathrm F}}
\def\G{{\mathrm G}}
\def\H{{\mathrm H}}
\def\P{{\mathrm P}}


\def\bb{\mathbf b}
\def \bc{\mathbf c}
\def\bx {\mathbf x}
\def\bn {\mathbf n}

\newcommand{\low}{^{\vphantom{()}}}
%%%%% to lower suffices: $X\low_1$ etc


\newcommand{\subone}{ {\vphantom{\dot A}1}}
\newcommand{\subtwo}{ {\vphantom{\dot A}2}}




\def\le{\leqslant}
\def\ge{\geqslant}
\def\arcosh{{\rm arcosh}\,}


\def\var{{\rm Var}\,}

\newcommand{\ds}{\displaystyle}
\newcommand{\ts}{\textstyle}
\def\half{{\textstyle \frac12}}
\def\l{\left(}
\def\r{\right)}



\begin{document}
\setcounter{page}{2}

 
\section*{Section A: \ \ \ Pure Mathematics}

%%%%%%%%%%Q1
\begin{question}
Given that $t= \tan \frac12 x$, show that 
$\dfrac {\d t}{\d x} = \frac12(1+t^2)$ and $ \sin x = \dfrac {2t}{1+t^2}\,$.

Hence show that
\[
\int_0^{\frac12\pi} \frac 1{1+a \sin x}\, \d x = 
\frac2 {\sqrt{1-a^2}} \arctan \frac{\sqrt{1-a}}{\sqrt{1+a}}\,
\ \ \ \ \ \ \ \ \ \ \ (0<a<1).
\]

Let 
\[
I_n = \int_0^{\frac12\pi} \frac{ \sin ^nx}{2+\sin x} \, \d  x
\ \ \ \ \ \ \ \ \ \ \ (n\ge0).
\]
By considering  
$I_{n+1}+2I_{n}\,$, or otherwise,  evaluate $I_3$.
\end{question}

%%%%%%%%%%Q2
\begin{question}
In this question, you may ignore questions of convergence.

Let $y= \dfrac {\arcsin x}{\sqrt{1-x^2}}\,$. Show that 
\[
(1-x^2)\frac {\d y}{\d x} -xy -1 =0 
\]
and prove that, for 
any positive integer $n$,
\[
(1-x^2) \frac{\d^{n+2}y}{\d x^{n+2}} - (2n+3)x \frac{\d ^{n+1}y}{\d x ^{n+1}}
-(n+1)^2 \frac{\d^ny}{\d x^n}=0\,
.
\]
Hence obtain  the                
Maclaurin series for $ \dfrac {\arcsin x}{\sqrt{1-x^2}}\,$, giving  the   
general term for  odd and for even powers of $x$. 

Evaluate the infinite sum
\[
1 + \frac 1 {3!} + \frac{2^2}{5!} + \frac {2^2\times 3^2}{7!}+\cdots + 
\frac {2^2\times 3^2\times \cdots \times n^2}{(2n+1)!} + \cdots\,.
\] 
\end{question}

%%%%%%%%% Q3
\begin{question}
The four vertices  $P_i$ ($i= 1, 2, 3, 4$) of a regular tetrahedron
lie on the surface of a sphere 
with centre at $O$ and of radius 1.  
The position vector of $P_i$ with respect to $O$ is ${\bf p}_i$
 ($i= 1, 2, 3, 4$). Use the fact             that
${\bf p}_1+ {\bf p}_2+{\bf p}_3+{\bf p}_4={\bf 0}\,$ 
to show that
 ${\bf p}_i \,.\, {\bf p}_j =-\frac13\,$
 for $i\ne j$. 

Let $X$ be any point on the surface of the sphere, and let
$XP_i$ denote the length of the line joining $X$ and $P_i$
 ($i= 1, 2, 3, 4$).

\begin{questionparts}
\item
By writing  $(XP_i) ^2$ as $({\bf p}_i- {\bf x)}\,.\,({\bf p}_i- {\bf x})$,
where ${\bf x}$ is the position vector of $X$ with respect to $O$,
 show that  
 \[
\sum_{i=1}^4(XP_i) ^2 =8\,.
\]

\item Given that $P_1$ has coordinates $(0,0,1)$ and that 
the coordinates of $P_2$ are of the form $(a,0,b)$, where $a>0$,
show that  $a=2\sqrt2/3$ and $b=-1/3$,
 and find the coordinates of $P_3$ and $P_4$. 
\item Show that 
\[
\sum_{i=1}^4 (XP_i)^4 = 4 \sum_{i=1}^4 (1- {\bf x}\,.\,{\bf p}_i)^2\,.
\]
By letting the coordinates of $X$ be $ (x,y,z)$, show further that $
\sum\limits_{i=1}^4 (XP_i)^4$ is independent of the position of $X$.

\end{questionparts}
\end{question}

%%%%%% Q4 
\begin{question}
Show that $(z-\e^{\i\theta})(z-\e^{-\i\theta})=z^2 -2z\cos\theta +1\,$.

Write down the $(2n)$th roots of $-1$ in the form $\e^{\i\theta}$, where
$-\pi <\theta \le \pi$, and deduce that
\[
z^{2n} +1 
= \prod_{k=1}^n \left(z^2-2z \cos\left( \tfrac{(2k-1)\pi}{2n}\right) +1\right)
\,.
\]
Here, $n$ is a positive integer, and the $\prod$ notation denotes the product.
 
\begin{questionparts}
\item By substituting $z=\i$ show that, when $n$ is even,
\[
\cos  \left(\tfrac \pi {2n}\right) 
\cos  \left(\tfrac {3\pi} {2n}\right) 
\cos  \left(\tfrac {5\pi} {2n}\right) 
\cdots
\cos  \left(\tfrac{(2n-1) \pi} {2n}\right) 
 = {(-1\vphantom{\dot A})}^{\frac12 n} 2^{1-n}
\,.
\]
\item 
Show that, when $n$ is odd,       
\[
\cos^2  \left(\tfrac \pi {2n}\right) 
\cos ^2 \left(\tfrac {3\pi} {2n}\right) 
\cos ^2 \left(\tfrac {5\pi} {2n}\right) 
\cdots
\cos ^2 \left(\tfrac{(n-2) \pi} {2n}\right) 
= n 2^{1-n} 
\,.
\]
 You may use without proof         
the fact that
$1+z^{2n}= (1+z^2)(1-z^2+z^4 - \cdots + z^{2n-2})\,$
 when $n$ is odd.

\end{questionparts}
\end{question}

%%%%%%%%% Q5
\begin{question}
In this question, you may assume that, if $a$, $b$ and $c$ are positive integers
such that $a$ and $b$ are coprime
and $a$ divides $bc$, then $a$ divides $c$. (Two positive integers are
said to be {\em coprime} if their highest common factor is 1.)

\begin{questionparts}
\item
Suppose that there are positive integers $p$, $q$, $n$ and $N$ such that $p$ and $q$
are coprime and  $q^nN=p^n$. Show that $N=kp^n$ for some positive integer $k$
and deduce the value of~$q$.

Hence prove that, for any positive integers $n$ and $N$, $\sqrt[n]N$
 is either a positive integer or irrational.

\item
Suppose that there are positive integers $a$, $b$, $c$ and $d$ such that 
$a$ and $b$ are coprime and $c$ and $d$ are coprime,
and $a^ad^b = b^a c^b \,$.
Prove that $d^b = b^a$ and deduce that, if $p$ is a prime factor of $d$, then
$p$ is also a prime factor of $b$.

If $p^m$ and $p^n$ are the highest powers of  the prime number $p$ that divide 
$d$ and $b$, respectively, express $b$ in terms of $a$, $m$ and $n$ and hence show that
$p^n\le n$. Deduce the value of $b$. (You may assume that
if $x>0$ and $y\ge2$ then 
 $y^x>x$.)

Hence prove that, 
if $r$ is a positive rational number such that $r^r$ is rational, then
$r$ is a positive integer.

\end{questionparts}
\end{question}
	
%%%%%%%%% Q6
\begin{question}
Let $z$ and $w$ be complex numbers. Use a diagram to show that
 $\vert z-w \vert \le \vert z\vert + \vert w \vert\,.$

For any complex numbers $z$ and $w$,  $E$ is defined by 
\[
E = zw^* + z^*w +2 \vert zw \vert\,.
\]

\begin{questionparts}
\item
Show that 
 $\vert z-w\vert^2 = \left( \vert z \vert + \vert w\vert\right)^2 -E\,$,
and deduce that $E$ is real and non-negative.

\item Show that 
 $\vert 1-zw^*\vert^2 = \left ( 1 +\vert zw \vert \right)^2 -E\,$.
\end{questionparts}

Hence show  that, if both $\vert z \vert >1$ and $\vert w \vert >1$, then
\[
 \frac {\vert z-w\vert} {\vert 1-zw^*\vert } 
\le \frac{\vert z \vert   +\vert w\vert  }{1+\vert z  w \vert}\,.
\]
Does this inequality also hold if  both $\vert z \vert <1$ and 
$\vert w \vert <1$?
\end{question}
	
%%%%%%%%% Q7
\begin{question}
\begin{questionparts}
\item Let $y(x)$ be a solution of the differential equation 
$  \dfrac {\d^2 y}{\d x^2}+y^3=0$ 
with $y = 1$ and $\dfrac{\d y}{\d x} =0$ at $x=0$, and let
 \[
{\rm E} (x)=
\left ( \frac {\d y}{\d x}\right)^{\!\!2} + \tfrac 12 y^4\,.
\]
Show by differentiation that ${\rm E}$
is constant and deduce that $ \vert y(x) \vert \le 1$ for all $x$.


\item Let $v(x)$ be a solution of the differential equation 
$ 
\dfrac{\d^2 v}{\d x^2} + x \dfrac {\d v}{\d x} +\sinh v =0
$  
with $v = \ln 3$ and $\dfrac{\d v}{\d x} =0$ at $x=0$, and let
 \[
{\rm E} (x)=
\left ( \frac {\d v}{\d x}\right)^{\!\!2} + 2 \cosh v\,.
\]
Show that $\dfrac{\d{\rm E}}{\d x}\le 0$ for $x\ge0$ 
and deduce that $\cosh v(x) \le \frac53$
 for $x\ge0$.



\item Let $w(x)$ be a solution of the differential equation 
\[
\frac{\d^2 w}{\d x^2} + (5\cosh x - 4 \sinh x -3) \frac{\d w}{\d x} + 
(w\cosh w + 2 \sinh w) =0
\]
with $\dfrac{\d w }{\d x}=\dfrac 1 { \sqrt 2 }$ and $w=0$ at $x=0$. 
Show that $\cosh w(x)  \le \frac54$ for $x\ge0$. 

\end{questionparts}
\end{question}
		
%%%%%%%%% Q8
\begin{question}
Evaluate $\displaystyle \sum_{r=0}^{n-1} \e^{2\i(\alpha + r\pi/n)}$
where $\alpha$ is a fixed angle and $n\ge2$. 

The fixed point $O$ is a distance $d$ from a fixed line $D$.
For any point $P$, let
$s$ be the distance from $P$ to $D$ and let $r$
 be the distance from $P$ to $O$. Write
down an expression for $s$ in terms of $d$, $r$ and the angle $\theta$, 
where $\theta$
 is as shown
in the diagram below.





%\vspace{-1cm}
\begin{center}
\psset{xunit=0.8cm,yunit=0.8cm,algebraic=true,dotstyle=o,dotsize=3pt 0,linewidth=0.3pt,arrowsize=3pt 2,arrowinset=0.25}
\begin{pspicture*}(-6.44,-2.23)(2.5,6.7)
\psbezier(-5.93, 5.21)(1.07, 4.02)(1.05, 0.57)(-1.07, -1.78)
\psline(2,6)(2,-2)
\psline(2,4)(-1.99,3.99)
\psline(-1.99,3.99)(-5,0)
\psline(-5,0)(2,0)
\psline{<->}(-5.02,-0.25)(2,-0.25)
\pscustom{\parametricplot[linewidth=0.1pt]{-0.0}{0.9245031027687439}{1*cos(t)+-5|1*sin(t)+0}\lineto(-5,0)\closepath}
\rput[tl](-3.7,2.48){$r$}
\rput[tl](-5.5,0.05){$O$}
\rput[tl](-4.5,0.5){$\theta$}
\rput[tl](-6.33,5.61){$E$}
\rput[tl](-1.7,-0.38){$d$}
\rput[tl](0.01,4.45){$s$}
\rput[tl](-2.15,4.55){$P$}
\rput[tl](1.84,6.66){$D$}
\end{pspicture*}     
\end{center}

 The curve $E$ shown in the diagram is 
such that, for any point $P$ on $E$, the 
relation
$ 
r = k s
$ holds, 
where $k$ is a fixed number with $0< k <1$. 


Each of the $n$ lines $L_1$, $\ldots\,$, $L_n$   
passes through $O$
and the angle between adjacent lines is~$\frac \pi n$. The line $L_j$ 
($j=1$,  $\ldots\,$, $n$) 
intersects $E$ in two points forming a chord of length $l_j$.   
Show that, for $n\ge2$, 
\[
\sum_{j=1}^n \frac 1 {l_j} = \frac {(2-k^2)n} {4kd}\,.
\]
\end{question}	
		

		
	
\newpage
\section*{Section B: \ \ \ Mechanics}


	
%%%%%%%%%% Q9
\begin{question}
A sphere of radius $R$ and uniform density $\rho_{\text{s}}$ is floating in 
a large tank of liquid of uniform density $\rho$. 
Given that the centre of the sphere is a distance 
$x$ above the level of the liquid, where $x<R$, show that the volume
of liquid displaced is 
\[
\frac \pi 3 (2R^3-3R^2x +x^3)\,. 
\]

The sphere is acted upon by two forces only: its weight and 
an upward force equal in magnitude to the weight of the liquid it has 
displaced. Show that
\[
4 R^3\rho_{\text{s}} (g+\ddot x) = (2R^3 -3R^2x +x^3)\rho g\,.
\]
Given that the sphere is in equilibrium when $x=\frac12 R$, find 
$\rho_{\text{s}}$ in terms of $\rho$. Find, in terms of $R$ and $g$,
 the period of small oscillations 
about this equilibrium position.
\end{question}
	
%%%%%%%%%% Q10 
\begin{question}	
A uniform rod $AB$ has mass $M$ and length $2a$. The point
$P$ lies on the rod a distance $a-x$ from~$A$. Show that the 
moment of inertia of the rod about an axis through $P$ and 
perpendicular to 
the rod is
\[
\tfrac13 M(a^2 +3x^2)\,.
\]

The rod is free 
to rotate, in a horizontal plane, about a fixed vertical axis through $P$. 
Initially the rod is at rest. The end $B$ is struck by a particle of
mass $m$ moving horizontally with speed $u$ in a direction
perpendicular to the rod.
The coefficient of restitution between the rod and the particle is $e$.
Show that the angular velocity of the rod immediately after impact
is 
\[
\frac{3mu(1+e)(a+x)}{M(a^2+3x^2) +3m(a+x)^2}\,.
\]

In the case $m=2M$, find the value of $x$ for which the angular velocity 
is greatest and show that this angular velocity is $u(1+e)/a\,$.
\end{question}

%%%%%%%%%% Q11

\begin{question}
An equilateral triangle, comprising three light rods each of
length $\sqrt3a$, has a particle of mass $m$ attached to each of 
its vertices. The triangle is suspended horizontally from a point vertically
above its centre by three identical springs, so that the springs and 
rods form a tetrahedron.
 Each spring has natural length $a$ and modulus
of elasticity $kmg$, and is light. 
Show that when the springs make an angle $\theta$
with the horizontal the tension in each spring is
\[
\frac{ kmg(1-\cos\theta)}{\cos\theta}\,.
\]
Given that the triangle is in equilibrium when $\theta = \frac16 \pi$, show
that $k=4\sqrt3 +6$.

The triangle is released from rest from the position 
at which $\theta=\frac13\pi$.
Show that when it passes through the equilibrium position its speed $V$ 
satisfies 
\[
V^2 = \frac{4ag}3(6+\sqrt3)\,.
\]
\end{question}
	

	
	\newpage
\section*{Section C: \ \ \ Probability and Statistics}


%%%%%%%%%% Q12
\begin{question}
A list consists only of letters $A$ and $B$ arranged in 
a row. In the list, there are 
$a$ letter $A$s and $b$ letter $B$s, where $a\ge2$ and $b\ge2$, and
  $a+b=n$.
Each possible ordering of the letters is equally probable.
The random variable $X_1$ is defined by
\[
X_1 = 
\begin{cases}
1 & \text{if the first letter in the row is $A$};\\
0 & \text{otherwise.}
\end{cases}
\]
The random variables $X_k$ ($2 \le k \le n$) are defined by
\[
X_k = 
\begin{cases}
1 & \text{if the $(k-1)$th  letter  is $B$ and the $k$th is $A$};\\
0 & \text{otherwise.}
\end{cases}
\]
The random variable $S$ is defined by $S = \sum\limits_ {i=1}^n X_i\,$.
\begin{questionparts}
\item Find expressions for $\E(X_i)$, 
distinguishing between the cases $i=1$ and $i\ne1$, and show
that $\E(S)= \dfrac{a(b+1)}n\,$.
\item Show that:
\begin{questionparts}
\item[\bf (a)] for $j\ge3$, \  $\E(X_1X_j) = \dfrac{a(a-1)b}{n(n-1)(n-2)}\,$;

\item[\bf (b)] $\ds \sum\limits_{i=2}^{n-2} \bigg( \sum\limits_{j=i+2}^n \E(X_iX_j)\bigg)
= \dfrac{a(a-1)b(b-1)}{2n(n-1)}\,$;

\item[\bf (c)] $\var(S) = \dfrac {a(a-1)b(b+1)}{n^2(n-1)}\,$.
\end{questionparts}
\end{questionparts}
\end{question}

%%%%%%%%%% Q13
\begin{question}
\begin{questionparts}
\item
The continuous random variable $X$ satisfies
$0\le X\le 1$, and has probability density function
$\f(x)$ and cumulative distribution
function $\F(x)$. The 
greatest value of $\f(x)$ is~$M$, so that $0\le \f(x) \le M$. 

\begin{questionparts}
\item[\bf (a)]
Show that $0\le \F(x) \le Mx$ for $0\le x\le1$.

\item[\bf (b)]
For any function $\g(x)$, show that
\[
\int_0^1 2 \g(x) \F(x) \f(x) \d x =  \g(1) - 
 \int_0^1 \g'(x) \big( \F(x)\big)^2 \d x
\,.
\]
\end{questionparts}
\item
The 
continuous random variable $Y$ 
satisfies
$0\le Y\le 1$, and has probability density function
$k \F(y) \f(y)$, where $\f$ and $\F$ are as above. 

\begin{questionparts}
\item[\bf (a)]
Determine the
value of the constant $k$.
\item[\bf (b)]
Show that
\[
1+ \frac{nM}{n+1}\mu_{n+1} - \frac{nM}{n+1}
\le \E(Y^n) \le 2M\mu_{n+1}\,,
\]
where $\mu_{n+1} = \E(X^{n+1})$ and $n\ge0$.
\item[\bf (c)]
Hence  show that, for $n\ge 1$,
\[
\mu _n \ge \frac{n}{(n+1)M} -\frac{n-1}{n+1}
\,.\]

\end{questionparts}
\end{questionparts}
\end{question}

\end{document}
