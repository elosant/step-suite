\documentclass[a4, 11pt]{report}


\pagestyle{myheadings}
\markboth{}{Paper I, 2010
\ \ \ \ \ 
\today 
}               

\RequirePackage{amssymb}
\RequirePackage{amsmath}
\RequirePackage{graphicx}
\RequirePackage{color}
\RequirePackage[flushleft]{paralist}[2013/06/09]



\RequirePackage{geometry}
\geometry{%
  a4paper,
  lmargin=2cm,
  rmargin=2.5cm,
  tmargin=3.5cm,
  bmargin=2.5cm,
  footskip=12pt,
  headheight=24pt}


\newcommand{\comment}[1]{{\bf Comment} {\it #1}}
%\renewcommand{\comment}[1]{}

\newcommand{\bluecomment}[1]{{\color{blue}#1}}
%\renewcommand{\comment}[1]{}
\newcommand{\redcomment}[1]{{\color{red}#1}}



\usepackage{epsfig}
\usepackage{pstricks-add}
\usepackage{tgheros} %% changes sans-serif font to TeX Gyre Heros (tex-gyre)
\renewcommand{\familydefault}{\sfdefault} %% changes font to sans-serif
%\usepackage{sfmath}  %%%% this makes equation sans-serif
%\input RexFigs


\setlength{\parskip}{10pt}
\setlength{\parindent}{0pt}

\newlength{\qspace}
\setlength{\qspace}{20pt}


\newcounter{qnumber}
\setcounter{qnumber}{0}

\newenvironment{question}%
 {\vspace{\qspace}
  \begin{enumerate}[\bfseries 1\quad][10]%
    \setcounter{enumi}{\value{qnumber}}%
    \item%
 }
{
  \end{enumerate}
  \filbreak
  \stepcounter{qnumber}
 }


\newenvironment{questionparts}[1][1]%
 {
  \begin{enumerate}[\bfseries (i)]%
    \setcounter{enumii}{#1}
    \addtocounter{enumii}{-1}
    \setlength{\itemsep}{5mm}
    \setlength{\parskip}{8pt}
 }
 {
  \end{enumerate}
 }



\DeclareMathOperator{\cosec}{cosec}
\DeclareMathOperator{\Var}{Var}

\def\d{{\mathrm d}}
\def\e{{\mathrm e}}
\def\g{{\mathrm g}}
\def\h{{\mathrm h}}
\def\f{{\mathrm f}}
\def\p{{\mathrm p}}
\def\s{{\mathrm s}}
\def\t{{\mathrm t}}


\def\A{{\mathrm A}}
\def\B{{\mathrm B}}
\def\E{{\mathrm E}}
\def\F{{\mathrm F}}
\def\G{{\mathrm G}}
\def\H{{\mathrm H}}
\def\P{{\mathrm P}}


\def\bb{\mathbf b}
\def \bc{\mathbf c}
\def\bx {\mathbf x}
\def\bn {\mathbf n}

\newcommand{\low}{^{\vphantom{()}}}
%%%%% to lower suffices: $X\low_1$ etc


\newcommand{\subone}{ {\vphantom{\dot A}1}}
\newcommand{\subtwo}{ {\vphantom{\dot A}2}}




\def\le{\leqslant}
\def\ge{\geqslant}
\def\arcosh{{\rm arcosh}\,}


\def\var{{\rm Var}\,}

\newcommand{\ds}{\displaystyle}
\newcommand{\ts}{\textstyle}
\def\half{{\textstyle \frac12}}
\def\l{\left(}
\def\r{\right)}



\begin{document}
\setcounter{page}{2}

 
\section*{Section A: \ \ \ Pure Mathematics}

%%%%%%%%%%Q1
\begin{question}
Given that 
\[
5x^{2}+2y^{2}-6xy+4x-4y\equiv
a\left(x-y+2\right)^{2}
+b\left(cx+y\right)^{2}+d\,,
\] 
find the values of the constants $a$, $b$, $c$
and $d$.

Solve the simultaneous equations 
\begin{align*}
5x^{2}+2y^{2}-6xy+4x-4y&=9\,,
\\
6x^{2}+3y^{2}-8xy+8x-8y&=14\,.
\end{align*}
\end{question}

%%%%%%%%%%Q2
\begin{question}
The 
curve 
$\displaystyle y=\Bigl(\frac{x-a}{x-b}\Bigr)\e^{x}$, 
where $a$ and $b$ are constants, 
has two stationary points. 
Show that 
\[
a-b<0 \ \  \ \text{or} \ \ \ a-b>4 \,.
\]

\begin{questionparts}
\item[(i)]  Show that, in the case $a=0$ and $b= \frac12$,
there is one stationary point  on either side of the 
curve's vertical asymptote, and  sketch the curve.
\item[(ii)] Sketch the curve in the case 
$  a=\tfrac{9}{2}$ and $b=0\,$. 
\end{questionparts}

\end{question}

%%%%%%%%% Q3
\begin{question}
Show that 
\[
\sin(x+y) -\sin(x-y) = 2 \cos x \, \sin y
\]
and deduce that  
\[
\sin A - \sin B = 2
\cos \tfrac12 (A+B)
\,
\sin\tfrac12 (A-B)
\,.
\]
Show also that
\[
\cos A - \cos B = 
-2
\sin \tfrac12(A+B) 
\,
\sin\tfrac12(A-B)\,.
\] 

The points $P$, $Q$, $R$ and $S$ have coordinates
$\left(a\cos p,b\sin p\right)$, 
$\left(a\cos q,b\sin q\right)$, 
$\left(a\cos r,b\sin r\right)$ and 
$\left(a\cos s,b\sin s\right)$ respectively,
where $0\le p<q<r<s<2\pi$, and $a$ and $b$ are positive.

Given that neither of the lines $PQ$ and $SR$ is vertical,
show that these lines are parallel if and only if 
\[
r+s-p-q = 2\pi\,.
\]
\end{question}

%%%%%% Q4 
\begin{question}
 Use the 
substitution 
$x=\dfrac{1}{t^{2}-1}\; $, where $t>1$,  
to show that, for $ x>0$,
\[
\int  \frac{1}{\sqrt{x\left(x+1\right) \; } \ }\; \d x
=2 \ln \left(\sqrt x+ \sqrt{x +1} \; \right)+c
\,.
\]
{\bf [Note} You may use without proof the result 
$\displaystyle
\int \! \frac{1}{t^2-a^2} \, \d t 
= \frac{1}{2a} \ln \left| \frac{t-a}{t+a}\right| + \rm {constant}$.
{\bf]} 

The section of the curve 
\[
y=\dfrac{1}{\sqrt{x}\; }-\dfrac{1}{\sqrt{x+1}\; }
\] 
between $x=\frac{1}{8}$ and $x=\frac{9}{16}$ 
is rotated through $360^{o}$ about the $x$-axis. 
Show that the  volume 
enclosed is $2\pi \ln \tfrac{5}{4}\,$. $\phantom{\dfrac AB}$
\end{question}

%%%%%%%%% Q5
\begin{question}
By considering the expansion 
of $\left(1+x\right)^{n}$ where $n$ is a positive integer, 
or otherwise, show that:
\begin{questionparts}
\item 
$\displaystyle \begin{pmatrix}n\\0\end{pmatrix}
+\begin{pmatrix}n\\1\end{pmatrix}+
\begin{pmatrix}n\\2\end{pmatrix}+ 
\cdots +\begin{pmatrix}n\\n\end{pmatrix}=2^{n}\,;
$
\item 
$\displaystyle
\begin{pmatrix}n\\1\end{pmatrix}+2\begin{pmatrix}
n\\2\end{pmatrix}+3\begin{pmatrix}n\\3\end{pmatrix}
+\cdots +n\begin{pmatrix}n\\n\end{pmatrix}=n2^{n-1}\,;
$
\item
$\displaystyle 
\begin{pmatrix}n\\0\end{pmatrix}
+\dfrac{1}{2}\begin{pmatrix}n\\1\end{pmatrix}
+\dfrac{1}{3}\begin{pmatrix}n \\2\end{pmatrix}+
\cdots +\dfrac{1}{n+1}\begin{pmatrix}n\\n\end{pmatrix}
=\dfrac{1}{n+1}\left(2^{n+1}-1\right)\,;$
\item
 $\displaystyle
\begin{pmatrix}n\\1\end{pmatrix}
+2^{2}\begin{pmatrix}n\\2\end{pmatrix}
+3^{2}\begin{pmatrix}n\\3\end{pmatrix}
+ \cdots
+n^{2}\begin{pmatrix}n\\n\end{pmatrix}
=n\left(n+1\right)2^{n-2}\,.$
\end{questionparts}
	\end{question}
	
%%%%%%%%% Q6
\begin{question}
Show that, if  $y=\e^x$, then 
\[
(x-1) \frac{\d^2 y}{\d x^2}  -x \frac{\d y}{\d x} 
+y=0\,.
\tag{$*$}
\]

In order to find other solutions of this differential equation, now
let $y=u\e^x$, where $u$ is a function of $x$. 
By substituting this into   $(*)$, 
show that 
\[
 (x-1) \frac{\d^2 u}{\d x^2} + (x-2) \frac{\d u}{\d x} 
=0\,.
\tag{$**$}
\]
By setting $ \dfrac {\d u}{\d x}= v$ in $(**)$ and solving the
resulting first order differential 
equation for $v$, find~$u$ in terms of $x$.
Hence  show that $y=Ax + B\e^x$ satisfies $(*)$, where $A$ and $B$
are any  constants.   
\end{question}
	
%%%%%%%%% Q7
\begin{question}
Relative to a fixed origin 
$O$, the points 
$A$ and $B$ have position vectors 
$\bf{a}$ and $\bf{b}$, respectively. (The points $O$, $A$ and $B$ are not
collinear.) 
The point $C$ has position vector $\bf c$ given by  
\[
{\bf c} =\alpha {\bf a}+ \beta {\bf b}\,,
\] 
where $\alpha$ and $\beta$ are 
positive constants with $\alpha+\beta<1\,$. 
The lines $OA$ 
and $BC$ meet at the point $P$ with position vector $\bf p$
and the lines $OB$ and $AC$ meet at the point $Q$ with position vector $\bf q$. 
Show that 
\[
{\bf p}  =\frac{\alpha {\bf a} }{1-\beta}\,,
\] 
and write down $\bf q$ in terms of 
    $\alpha,\ \beta$ and $\bf   {b}$.


Show  further that the point $R$ 
with position vector $\bf r$ given by 
\[
{\bf r} =\frac{\alpha {\bf a}
+ \beta {\bf b}}{\alpha + \beta}\,,
\] 
 lies on the lines $OC$ and $AB$.

The lines $OB$ and $PR$ intersect at the point $S$.  
Prove that 
$
\dfrac{OQ}{BQ} = \dfrac{OS}{BS}\,$.
\end{question}
		
%%%%%%%%% Q8
\begin{question}
\begin{questionparts}
\item Suppose that $a$, $b$ and $c$ are integers 
that satisfy the equation 
\[
a^{3}+3b^{3}=9c^{3}.
\]
Explain why $a$ must be divisible by 3,
and show further that both $b$ and $c$ must also  be divisible by 3. 
Hence show that the only integer solution is $a=b=c=0\,$.

\item Suppose that $p$, $q$ and $r$ are integers
that satisfy the equation
\[
 p^4 +2q^4 = 5r^4  
\,.\]
 By considering the possible final digit of each
term, or otherwise, show that 
$p$ and $q$ are divisible by 5. Hence
show that the only integer solution is $p=q=r=0\,$.

\end{questionparts}
\end{question}	
		

		
	
\newpage
\section*{Section B: \ \ \ Mechanics}


	
%%%%%%%%%% Q9
\begin{question}$\,$
\begin{center}
\psset{xunit=1.0cm,yunit=1.0cm,algebraic=true,dotstyle=o,dotsize=3pt 0,linewidth=0.5pt,arrowsize=3pt 2,arrowinset=0.25}
\begin{pspicture*}(0.19,0.2)(4.27,3.8)
\psline(0.73,3.8)(0.73,0.63)
\psline(0.28,0.62)(3.61,0.62)
\psline(0.73,2.13)(3.22,0.62)
\psline(3.22,0.62)(4.05,1.71)
\psline(4.05,1.71)(1.58,3.28)
\psline(0.73,2.13)(1.58,3.28)
\rput[tl](2.56,0.85){$\alpha$}
\rput[tl](1.52,1.5){$2a$}
\rput[tl](3.74,1.33){$2b$}
\pscustom{\parametricplot{2.5962489219025393}{3.13892917236946}{0.84*cos(t)+3.22|0.84*sin(t)+0.62}\lineto(3.22,0.62)\closepath}
\end{pspicture*}
\end{center}

The diagram shows
a uniform rectangular lamina with sides of lengths $2a$ and $2b$ leaning 
against a rough vertical wall, 
with one corner resting on a rough horizontal plane.
The plane of the lamina is vertical and perpendicular to
the wall, and one edge
makes  an angle of $\alpha$ 
with the horizontal plane. Show that the centre of mass
of the lamina is a distance $a\cos\alpha + b\sin\alpha$ from the
wall.

The coefficients of friction at 
the  two points of contact are each  $\mu$ and the friction
is limiting at both contacts. 
Show that
\[
a\cos(2\lambda +\alpha) = b\sin\alpha \,,
\]
where $\tan\lambda = \mu$.

Show also that if the lamina is square, then $\lambda = \frac{1}{4}\pi -\alpha$.
	\end{question}
	
%%%%%%%%%% Q10 
\begin{question}	
A particle $P$ moves so that, at time $t$,
 its displacement 
$ \bf r $
 from a fixed origin  is given by 
\[
{\bf r} =\left( \e^{t}\cos t \right) {\bf i}+
\left(\e^t \sin t\right) {\bf j}\,.\]
Show that the velocity of the particle always makes an 
angle of $\frac{\pi}{4}$ with the particle's displacement, 
and that the acceleration of the particle is always
perpendicular to its displacement. Sketch the path of 
the particle for $0\le t \le \pi$.

 A second particle $Q$ moves on the same path, passing through
each point on the path a fixed time $T$ after $P$ does.
Show that the distance between $P$ and $Q$ is 
proportional to $\e^{t}$. 
\end{question}

%%%%%%%%%% Q11

\begin{question}
Two particles  of masses $m$ and $M$, with $M>m$,
lie in  a smooth circular groove on a horizontal plane. The coefficient
of restitution between the particles is $e$. The particles are initially
projected round the groove
with the same speed $u$ but in opposite directions. Find the
speeds of the particles after they collide for the first time and show
that they will both change direction if $2em> M-m$. 

After a further $2n$ collisions, the speed of the particle of mass $m$ is $v$
and the speed of the particle of mass $M$ is $V$. Given
that  at each collision both particles change 
their directions of motion, explain why  
\[
mv-MV  = u(M-m),
\]
and find $v$ and $V$ in terms of $m$, $M$, $e$, $u$ and $n$.
\end{question}
	

	
	\newpage
\section*{Section C: \ \ \ Probability and Statistics}


%%%%%%%%%% Q12
\begin{question}
A discrete random variable $X$ takes only positive integer values.
 Define $\E(X)$ for this case, and 
show that 
\[\E(X)      =\sum^{\infty}_{n=1}\P\left(X\ge n \right).\]

I am  collecting toy penguins from cereal boxes. Each box 
contains either one daddy penguin or one mummy penguin. The 
probability that a given box contains a daddy penguin is $p$ 
and the probability that a given box contains a mummy penguin is
$q$,
where $p\ne0$, $q\ne0$ and $p+q=1\,$.

 Let $X$       be the number of boxes that I
need to open to get at least one of each kind of penguin. Show that 
$\P(X\ge 4)= p^{3}+q^{3}$,   and that 
\[
\E(X)=\frac{1}{pq}-1.\,
\] 
Hence show that $\E(X)\ge 3\,$.
\end{question}

%%%%%%%%%% Q13
\begin{question}
The number of texts that George receives on his 
mobile phone can be modelled by a Poisson random variable 
with mean $\lambda$ texts per hour. Given that the
probability George waits between 1 and 2 hours in 
the morning before he receives his first text is
$p$, show that
\[
p\e^{2\lambda}-\e^{\lambda}+1=0.
\]
Given that $4p<1$, show that there are two positive
values of $\lambda$ that satisfy this equation. 

The
number of texts that Mildred receives on each
of her two mobile phones 
can  be modelled by independent Poisson random variables
with different means $\lambda_{1}$ and $\lambda_{2}$ texts per hour.
Given that, for each phone, the probability that Mildred waits between
1 and 2 hours in the morning before she receives her first 
text is also $p$, find an expression for $\lambda_{1}+\lambda_{2}$
in terms of~$p$. 

Find  the probability, in terms of $p$,
that she waits between 1 and 2 hours in the morning 
to receive her first text.
\end{question}

\end{document}
