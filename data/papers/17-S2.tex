\documentclass[a4, 11pt]{report}


%\pagestyle{myheadings}
%\markboth{}{Paper II, 2007  final draft  
%\ \ \ \ \ 
%\today 
%}               
\pagestyle{empty}

\usepackage{pstricks-add}
\usepackage{epsfig}

\RequirePackage{amssymb}
\RequirePackage{amsmath}
\RequirePackage{graphicx}
\RequirePackage{color}
\RequirePackage{xcolor}


\RequirePackage[flushleft]{paralist}[2013/06/09]



\RequirePackage{geometry}
\geometry{%
  a4paper,
  lmargin=2cm,
  rmargin=2.5cm,
  tmargin=3.5cm,
  bmargin=2.5cm,
  footskip=12pt,
  headheight=24pt}


\newcommand{\comment}[1]{{\bf Comment} {\it #1}}
%\renewcommand{\comment}[1]{}

\newcommand{\bct}[1]{{\color{blue}#1}}
%\renewcommand{\comment}[1]{}
\newcommand{\rct}[1]{{\color{red}#1}}






\setlength{\parskip}{10pt}
\setlength{\parindent}{0pt}

\newlength{\qspace}
\setlength{\qspace}{20pt}


\newcounter{qnumber}
\setcounter{qnumber}{0}

\newenvironment{question}%
 {\vspace{\qspace}
  \begin{enumerate}[\bfseries 1\quad][10]%
    \setcounter{enumi}{\value{qnumber}}%
    \item%
 }
{
  \end{enumerate}
  \filbreak
  \stepcounter{qnumber}
 }


\newenvironment{questionparts}[1][1]%
 {
  \begin{enumerate}[\bfseries (i)]%
    \setcounter{enumii}{#1}
    \addtocounter{enumii}{-1}
    \setlength{\itemsep}{2mm}
    \setlength{\parskip}{5pt}
 }
 {
  \end{enumerate}
 }



\DeclareMathOperator{\cosec}{cosec}
\DeclareMathOperator{\Var}{Var}

\def\d{{\rm d}}
\def\e{{\rm e}}
\def\g{{\rm g}}
\def\h{{\rm h}}
\def\f{{\rm f}}
\def\p{{\rm p}}
\def\s{{\rm s}}
\def\t{{\rm t}}


\def\A{{\rm A}}
\def\B{{\rm B}}
\def\E{{\rm E}}
\def\F{{\rm F}}
\def\G{{\rm G}}
\def\H{{\rm H}}
\def\P{{\rm P}}


\def\bb{\mathbf b}
\def \bc{\mathbf c}
\def\bx {\mathbf x}
\def\bn {\mathbf n}

\makeatletter
\newcommand{\raisemath}[1]{\mathpalette{\raisem@th{#1}}}
\newcommand{\raisem@th}[3]{\raisebox{#1}{$#2#3$}}
\makeatother
%%%To raise suffices: e.g.  $\Pi_{\raisemath{2pt}{-}}$.




\def\le{\leqslant}
\def\ge{\geqslant}


\def\var{{\rm Var}\,}

\newcommand{\ds}{\displaystyle}
\newcommand{\ts}{\textstyle}



\begin{document}

\setcounter{page}{2}

 
\section*{Section A: \ \ \ Pure Mathematics}

%%%%%%%%%%%%%% Q1
\begin{question}
{\bf Note: } In this question
you may use without proof the result
$ \dfrac{\d \ }{\d x}\big(\!\arctan x \big) = \dfrac 1 {1+x^2}\,$.

\vspace{3mm}
Let
\[
I_n = \int_0^1 x^n \arctan x \, \d x \;,
\]
where $n=0$, 1, 2, 3, $\ldots$ . 

\begin{questionparts}  
\item
 Show that, for $n\ge0\,$, 
\[
(n+1)  I_n = \frac \pi 4  - 
\int _0^1 \frac {x^{n+1}}{1+x^2} \, \d x
\, 
\]
and evaluate $I_0$.
\item
Find an expression, in terms of $n$, 
for $(n+3)I_{n+2}+(n+1)I_{n}\,$.

Use this result to evaluate $I_4$.
\item
Prove by induction that, for $n\ge1$, 
\[
 (4n+1) I_{4n} 
=A   - \frac12 \sum_{r=1}^{2n} (-1)^r \frac 1 {r}
\,,
\]
where $A$ is a constant to be determined.

\end{questionparts}

 
\end{question}

%%%%%%%%Q2
\begin{question}
The sequence of numbers $x_0$, $x_1$, $x_2$, $\ldots$ satisfies 
\[
x_{n+1} = \frac{ax_n-1}{x_n+b}
\,.
\]
(You may assume that $a$, $b$ and $x_0$ are such that $x_n+b\ne0\,$.)
 
Find an expression for $x_{n+2}$ in terms of $a$, $b$  and $x_n$.

\begin{questionparts}
\item
 Show that $a+b=0$ is a necessary condition for 
the sequence to be periodic with period~2. 

{\bf Note: } The sequence is said to  be 
 periodic with period $k$ if $x_{n+k} = x_n$ for all $n$,
and there is no integer $m$ with $0<m<k$ such that   $x_{n+m} = x_n$
for all $n$.

\item Find  necessary and sufficient conditions for the sequence to have
period 4. 
\end{questionparts}
\end{question}

%%%%%%%%%%%%%%%%%%%Q3
\begin{question}
\begin{questionparts}
\item Sketch, on $x$-$y$ axes, the set of all points satisfying
$\sin y = \sin x$, for $-\pi \le x \le \pi$ and  $-\pi \le y \le \pi$.
You should give the equations of all the lines on your sketch.

\item Given that
\[
\sin y = \tfrac12 \sin x
\]
obtain an expression, in terms of $x$,  for $y'$  when
$0\le x \le \frac12 \pi$
and $0\le y \le \frac12 \pi$,
and show that
\[
y'' = - \frac {3\sin x}{(4-\sin^2 x)^{\frac32}}
\;.
\]
Use these results to sketch the
set of all points satisfying $\sin y = \tfrac12 \sin x$ for
$0 \le x \le \frac12 \pi$ and
$0 \le y \le \frac12 \pi$.


Hence sketch the set of all points satisfying 
$\sin y = \tfrac12 \sin x$ for
$-\pi\! \le \! x \! \le \! \pi$ and
\mbox{$ -\pi \, \le\, y\, \le\, \pi\,$}.

\item
Without further calculation,
sketch the set of all points satisfying
 $\cos y = \tfrac12 \sin x$ for $- \pi \le x \le \pi$ and
$ -\pi \le y \le \pi$.
\end{questionparts}
\end{question}




%%%%%%%%%%%%%% Q4
\begin{question}
The Schwarz inequality
is
\[
\left( \int_a^b \f(x)\, \g(x)\,\d x\right)^{\!\!2}
\le 
\left(
\int_a^b \big( \f(x)\big)^2 \d x 
\right)
\left(
\int_a^b \big( \g(x)\big)^2 \d x 
\right)
.
\tag{$*$}
\]
\begin{questionparts}
\item
By setting $ \f(x)=1$ in $(*)$, and choosing
$\g(x)$, 
$a$ and $b$ suitably, show that for~$t>  0\,$,
\[
\frac {\e^t -1}{\e^t+1} \le \frac t 2
\,.
\]
\item
By setting $ \f(x)= x  $  
in $(*)$, 
and choosing $ \g(x)$ suitably,
show that
\[
\int_0^1\e^{-\frac12 x^2}\d x \ge 12 \big(1-\e^{-\frac14})^2
\,.
\]

\item
Use $(*)$  to show that 
\[
\frac {64}{25\pi} \le \int_0^{\frac12\pi} 
\!\!
{\textstyle \sqrt{\, \sin x\, } }
\, \d x 
\le \sqrt{\frac \pi 2 }
\,.
\]
\end{questionparts}

\end{question}

%%%%%%%%%%%%%% Q5

\begin{question}
A curve $C$ is determined by the parametric equations
\[
x=at^2 \, , \; y = 2at\,,
\]
where $a>0$\,.
\begin{questionparts}
\item Show that the normal to 
$C$ at a point $P$, with non-zero parameter $p$, 
meets $C$
again at a point $N$, with parameter $n$, where
\[
n=  - \left( p + \frac{2}{p} \right).
\]

\item
Show that the distance $\left| PN \right|$ is given by
\[
\vert PN\vert^2 = 16a^2\frac{(p^2+1)^3}{p^4}
\]
and that this is minimised when $p^2=2\,$. 

\item The point  $Q$, with parameter $q$, 
is the point at which the circle with 
diameter $PN$ cuts~$C$ again. 
By considering the gradients of $QP$ and $QN$,
show that 
\[
2 = p^2-q^2 + \frac{2q}p
\,.
\]
Deduce that  $\left| PN \right|$ is at its minimum when 
$Q$ is at the origin.

\end{questionparts}
\end{question}


%%%%%%%%%%%%%% Q6
\begin{question}
Let
\[
S_n = \sum_{r=1}^n \frac 1 {\sqrt r \ }
\,,
\]
where $n$ is a positive integer.
\begin{questionparts}


\item 
Prove by induction that 
\[
 S_n  \le   2\sqrt n -1\,
.
\]

\item
Show that $(4k+1)\sqrt{k+1} > (4k+3)\sqrt k\,$ for $k\ge0\,$.

Determine the smallest number $C$ such that
\[
S_n \ge 2\sqrt n + \frac 1 {2\sqrt n} -C
\,.\]
 

\end{questionparts}

\end{question}

%%%%%Q7
\begin{question}
%In this question, 
%the definition of $a^b$ (for $a>0$) is 
%$ 
%a^b = \e^{b \ln a} \,.
%$ 
%\\
The functions $\f$ and $\g$ are defined, for $x>0$, by
\[
\f(x) = x^x\,, \ \ \ \ \ \g(x) = x^{\f(x)}\,.
\]

\begin{questionparts}
\item 
By taking logarithms, 
or otherwise, show that $\f(x)> x$ for $0<x<1\,$.
 Show further that $x<\g(x)<\f(x)$ for $0<x<1\,$. 

Write down the 
corresponding results for $x>1 \,$. 

\item
Find the value of 
$x$ for which $\f'(x)=0\,$. 

\item
Use the result $x\ln x \to 0$ as $x\to 0$ 
to
find  $\lim\limits_{x\to0}\f(x)$,  
and write down   $\lim\limits_{x\to0}\g(x)\,$. 

\item
Show that $  x^{-1}+\, \ln x \ge 1\,$ for $x>0$. 
\\[5pt]
Using this result, or otherwise, 
show 
that~$\g'(x) >0\,$.
\end{questionparts}

\vspace{3pt}
Sketch the graphs, for $x>0$,  of $y=x$, \ $y=\f(x)$ and $y=\g(x)$ 
on the same axes.
  

\end{question}




%%%%%%%%%%%%%% Q8
\begin{question}
All vectors in this question lie in the same plane.

 
The vertices of the non-right-angled triangle $ABC$ have position 
vectors $\bf a$, $\bf b$ and $\bf c$, respectively.
The non-zero vectors $\bf u$ and $\bf v$ are
perpendicular to $BC$ and $CA$, respectively.


Write down the vector equation of the line through $A$
perpendicular to $BC$, in terms of 
 $\bf u$,~$\bf a$  
and  a parameter $\lambda $.

The line through $A$ perpendicular
to $BC$ intersects  the line through $B$ perpendicular to $CA$ at $P$. 
Find
the position
vector  of $P$
 in terms of $\bf a$,~$\bf b$, $\bf c$ and $\bf u$.

Hence show that
 the line $CP$ is perpendicular to the line $AB$.



\end{question}

\section*{Section B: \ \ \ Mechanics}

%%%%%%%%%%%%%% Q9
\begin{question}


Two identical rough cylinders of radius 
$r$
and weight 
$W$
rest, 
not 
touching each other
but a 
negligible distance apart, 
on a horizontal floor. 
A thin flat rough plank  of width $2a$, where 
$a < r$,
and weight 
$kW$
rests symmetrically and horizontally
on the cylinders, with its length parallel
to the axes of the cylinders and its faces horizontal. 
A vertical cross-section is shown in the diagram below. 


\vspace{1.1cm}
\hspace{5.0cm}
\begin{pspicture}(9.3,-5.00 )          
\psset{xunit=1.0cm,yunit=1.0cm,algebraic=true,dotstyle=o,dotsize=3pt 0,linewidth=0.8pt,arrowsize=3pt 2,arrowinset=0.25}
\psline(-2,-3)(7,-3)
\pscircle(1,-1.5){1.50}
\pscircle(4.02,-1.5 ){1.50}
\psline[linewidth=3pt](1.45,-0.06)(3.58,-0.06)
\end{pspicture}
\vspace{-1.5cm}


The coefficient of friction at all four 
contacts is 
$\frac12$.
The system is in equilibrium.

\begin{questionparts}
\item
Let $F$ be the frictional force between one cylinder and the floor,
and let $R$ be the normal reaction between the plank
 and one cylinder.
Show that
\[
R\sin\theta = F(1+\cos\theta)\,,
\]
where $\theta$ is the acute angle between  the plank and the tangent to 
the cylinder at the point of contact.

Deduce that $2\sin\theta \le 1+\cos\theta\,$.


\item 
Show that
\[
N=  
\left( 1+\frac2 k\right)\left(\frac{1+\cos\theta}{\sin\theta} \right) F
\,,
\]
where $N$ is the normal reaction between the 
floor and one cylinder. 

Write down the condition 
that the cylinder does
not slip on the floor and 
show that it is satisfied with no extra restrictions  on $\theta$. 

\item
Show that $\sin\theta\le\frac45\,$ and hence that $r\le5a\,$.

\end{questionparts}
\end{question}


%%%%%%%%%%%%%% Q10
\begin{question}
A car of mass $m$ makes a journey of  distance $2d$ 
in  a straight line. 
It  experiences air resistance  and rolling resistance
so that the total resistance to motion when it is moving with 
speed $v$ is $Av^2 +R$, where $A$ and $R$ are constants.


The car starts from rest and 
moves with constant acceleration $a$ for a distance $d$.
Show that the work done by the engine 
for this half of the journey 
is
\[
\int_0^d  (ma+R+Av^2) \, \d x
\]
and that it can be written 
in the form
\[
\int_0^w \frac {(ma+R+Av^2)v}a\;  \d v
\,,
\]
where $w =\sqrt {2ad\,}\,$.

For the second half of the journey, the acceleration of 
the car is $-a$. 



\begin{questionparts}
\item
In the case $R>ma$, 
show that the work done by the  
engine for the whole  journey~is 
\[
2Aad^2 +  2Rd
\,.
\]

\item
In the case $ma-2Aad<   R<   ma$, show that  at a certain speed 
  the driving
force required to maintain the constant acceleration
falls to zero. 

Thereafter, the engine does no work
(and the driver applies the brakes to maintain
the constant acceleration).
Show that the  work done by the engine for the whole journey~is  
\[
 2Aad^2 + 2 Rd
 + \frac{(ma-R)^2}{4Aa} 
\,
.\]


\end{questionparts}
\end{question}



%%%%%%%%%%%%%% Q11
\begin{question}
Two thin vertical parallel
walls, each of height $2a$, stand a distance $a$ apart
on horizontal ground.
The projectiles in this
question move in a plane perpendicular to the walls.

\begin{questionparts}
\item
A particle is projected with speed $\sqrt{5ag}$
towards the two
walls 
 from a point $ A$ at ground level.
It just clears the first wall. By considering the energy
of the particle, find its 
speed when it passes over the first wall.

Given that it just clears the second wall, 
show that
the angle 
its trajectory makes with the horizontal when it 
passes over the first wall 
is $45^\circ\,$.

 
Find the distance of $A$ from the foot of the first wall.

\item 
A second particle is projected with  speed $\sqrt{5ag}$
 from a point $B$ at ground level towards the two
walls. It passes a distance $h$ above the first wall, where $h>0$. 
Show that it does not clear the second wall.  

\end{questionparts}
\end{question}

\newpage
\section*{Section C: \ \ \ Probability and Statistics}

%%%%%%%%%%%%%% Q12
\begin{question}

Adam 
and Eve are
catching fish. 
The number of fish, $X$,
that Adam catches in any time interval is Poisson distributed
with parameter $\lambda t$, where $\lambda$ is a constant and $t$ is the
length of the time interval.
The number of fish, $Y$,
that Eve catches in any time interval is Poisson distributed
with parameter $\mu t$, where $\mu$ is a constant and $t$ is the
length of the time interval


The two Poisson variables are independent.
You may assume that 
that expected time between 
Adam catching a fish and Adam catching his next fish is $\lambda^{-1}$,
and similarly for Eve.
\begin{questionparts}
\item
By considering $\P( X + Y = r)$, show that the total number of fish caught 
by Adam and Eve in time $T$ also has a Poisson distribution.
\item
Given that Adam and Eve catch a
total of $k$ fish in time $T$, 
where $k$ is fixed, show that the number caught by 
Adam has a binomial distribution.
\item
Given that 
Adam and Eve start fishing at the same time, find
the probability that the first fish is caught by Adam.
\item
Find the expected time from the moment Adam and Eve 
start fishing until they have each caught at least one fish.

\end{questionparts}
\noindent
[{\bf Note } 
This question has been redrafted to make the meaning clearer.]

\end{question}
 
%%%%%%%%%%%%%% Q13

\begin{question}
In a television game show, a contestant has to open a door
using a key. The contestant is given a 
bag containing $n$ keys, 
where $n\ge2$. Only one key in the bag will open the door. 
There are three versions
of the game. In each version, the contestant starts by choosing a 
key at random from the bag.
\begin{questionparts}
\item
In version 1, after each failed attempt at opening the door
 the key that has
been tried is put back into the bag and the contestant again
selects a key at random from the bag.
By considering the binomial expansion of $( 1 - q)^{-2}$,
or otherwise, find the expected number of attempts required to open 
the door.
\item
In 
version 2, after each failed attempt at opening the door
the key that has been tried is put aside and the contestant 
selects another key
at random
from the bag. Find the expected number of 
attempts required to open the door.
\item 
In version
3, after each failed attempt at opening the door the key that has
been~tried is put back into the bag and another incorrect key is added to 
the bag. 
The contestant then selects a key at random from the bag. 
Show that  the probability that the contestant draws the
correct key at the $k$th attempt is 
\[
\frac{n-1}{(n+k-1)(n+k-2)}  
\,.
\]
Show also, using partial fractions, that 
 the expected number of attempts required to open the door is infinite.

You may use without proof the result that 
$
\displaystyle
\sum_{m=1}^N \dfrac 1 m \to \infty
 \,
$
as $N\to \infty\,$.
\end{questionparts}

\end{question}
\end{document}


