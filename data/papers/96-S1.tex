\documentclass[a4, 11pt]{report}


\pagestyle{myheadings}
\markboth{}{Paper I, 1996
\ \ \ \ \ 
\today 
}               

\RequirePackage{amssymb}
\RequirePackage{amsmath}
\RequirePackage{graphicx}
\RequirePackage{color}
\RequirePackage[flushleft]{paralist}[2013/06/09]



\RequirePackage{geometry}
\geometry{%
  a4paper,
  lmargin=2cm,
  rmargin=2.5cm,
  tmargin=3.5cm,
  bmargin=2.5cm,
  footskip=12pt,
  headheight=24pt}


\newcommand{\comment}[1]{{\bf Comment} {\it #1}}
%\renewcommand{\comment}[1]{}

\newcommand{\bluecomment}[1]{{\color{blue}#1}}
%\renewcommand{\comment}[1]{}
\newcommand{\redcomment}[1]{{\color{red}#1}}



\usepackage{epsfig}
\usepackage{pstricks-add}
\usepackage{tgheros} %% changes sans-serif font to TeX Gyre Heros (tex-gyre)
\renewcommand{\familydefault}{\sfdefault} %% changes font to sans-serif
%\usepackage{sfmath}  %%%% this makes equation sans-serif
%\input RexFigs


\setlength{\parskip}{10pt}
\setlength{\parindent}{0pt}

\newlength{\qspace}
\setlength{\qspace}{20pt}


\newcounter{qnumber}
\setcounter{qnumber}{0}

\newenvironment{question}%
 {\vspace{\qspace}
  \begin{enumerate}[\bfseries 1\quad][10]%
    \setcounter{enumi}{\value{qnumber}}%
    \item%
 }
{
  \end{enumerate}
  \filbreak
  \stepcounter{qnumber}
 }


\newenvironment{questionparts}[1][1]%
 {
  \begin{enumerate}[\bfseries (i)]%
    \setcounter{enumii}{#1}
    \addtocounter{enumii}{-1}
    \setlength{\itemsep}{5mm}
    \setlength{\parskip}{8pt}
 }
 {
  \end{enumerate}
 }



\DeclareMathOperator{\cosec}{cosec}
\DeclareMathOperator{\Var}{Var}

\def\d{{\rm d}}
\def\e{{\rm e}}
\def\g{{\rm g}}
\def\h{{\rm h}}
\def\f{{\rm f}}
\def\p{{\rm p}}
\def\s{{\rm s}}
\def\t{{\rm t}}


\def\A{{\rm A}}
\def\B{{\rm B}}
\def\E{{\rm E}}
\def\F{{\rm F}}
\def\G{{\rm G}}
\def\H{{\rm H}}
\def\P{{\rm P}}


\def\bb{\mathbf b}
\def \bc{\mathbf c}
\def\bx {\mathbf x}
\def\bn {\mathbf n}

\newcommand{\low}{^{\vphantom{()}}}
%%%%% to lower suffices: $X\low_1$ etc


\newcommand{\subone}{ {\vphantom{\dot A}1}}
\newcommand{\subtwo}{ {\vphantom{\dot A}2}}




\def\le{\leqslant}
\def\ge{\geqslant}


\def\var{{\rm Var}\,}

\newcommand{\ds}{\displaystyle}
\newcommand{\ts}{\textstyle}




\begin{document}
\setcounter{page}{2}

 
\section*{Section A: \ \ \ Pure Mathematics}

%%%%%%%%%%Q1
\begin{question}
A cylindrical biscuit tin has volume $V$ and surface area $S$ (including
the ends). Show that the minimum possible surface area for a given
value of $V$ is $S=3(2\pi V^{2})^{1/3}.$ For this value of $S$
show that the volume of the largest sphere which can fit inside the
tin is $\frac{2}{3}V$, and find the volume of the smallest sphere
into which the tin fits. 
\end{question}

%%%%%%%%%%Q2
\begin{question}
 \begin{questionparts}
\item Show that 
\[
\int_{0}^{1}\left(1+(\alpha-1)x\right)^{n}\,\mathrm{d}x=\frac{\alpha^{n+1}-1}{(n+1)(\alpha-1)}
\]
when $\alpha\neq1$ and $n$ is a positive integer. 
\item Show that if $0\leqslant k\leqslant n$ then the coefficient
of $\alpha^{k}$ in the polynomial 
\[
\int_{0}^{1}\left(\alpha x+(1-x)\right)^{n}\,\mathrm{d}x
\]
is 
\[
\binom{n}{k}\int_{0}^{1}x^{k}(1-x)^{n-k}\,\mathrm{d}x\,.
\]
\item Hence, or otherwise, show that 
\[
\int_{0}^{1}x^{k}(1-x)^{n-k}\,\mathrm{d}x=\frac{k!(n-k)!}{(n+1)!}\,.
\]
\end{questionparts}
\end{question}

%%%%%%%%% Q3
\begin{question}
Let $n$ be a positive integer. 
\begin{questionparts}
\item Factorise $n^{5}-n^{3},$ and show that it
is divisible by 24. 
\item Prove that $2^{2n}-1$ is divisible by 3. 
\item If $n-1$ is divisible by 3, show that $n^{3}-1$ is divisible
by 9. 
\end{questionparts}
\end{question}

%%%%%% Q4 
\begin{question}
Show that 
\[
\int_{0}^{1}\frac{1}{x^{2}+2ax+1}\,\mathrm{d}x=\begin{cases}
\dfrac{1}{\sqrt{1-a^{2}}}\tan^{-1}\sqrt{\dfrac{1-a}{1+a}} & \text{ if }\left|a\right|<1,\\
\dfrac{1}{2\sqrt{a^{2}-1}}\ln\left|a+\sqrt{a^{2}-1}\right| & \text{ if }\left|a\right|>1.
\end{cases}
\]

\end{question}

%%%%%%%%% Q5
\begin{question}
\begin{questionparts}
\item Find all rational numbers $r$ and $s$ which satisfy 
\[
(r+s\sqrt{3})^{2}=4-2\sqrt{3}.
\]
\item Find all real numbers $p$ and $q$ which satisfy 
\[
(p+q\mathrm{i})^{2}=(3-2\sqrt{3})+2(1-\sqrt{3})\mathrm{i}.
\]
\item Solve the equation 
\[
(1+\mathrm{i})z^{2}-2z+2\sqrt{3}-2=0,
\]
writing your solutions in as simple a form as possible. 
\end{questionparts}

{[}No credit will be given to answers involving use of calculators.{]} 

	\end{question}
	
%%%%%%%%% Q6
\begin{question}
Let $\mathrm{f}(x)=\dfrac{\sin(n+\frac{1}{2})x}{\sin\frac{1}{2}x}$
for $0<x\leqslant\pi.$ 
\begin{questionparts}
\item Using the formula 
\[
2\sin\tfrac{1}{2}x\cos kx=\sin(k+\tfrac{1}{2})x-\sin(k-\tfrac{1}{2})x
\]
(which you may assume), or otherwise, show that 
\[
\mathrm{f}(x)=1+2\sum_{k=1}^{n}\cos kx\,.
\]
\item Find ${\displaystyle \int_{0}^{\pi}\mathrm{f}(x)\,\mathrm{d}x}$
and ${\displaystyle \int_{0}^{\pi}\mathrm{f}(x)\cos x\,\mathrm{d}x}.$ 
\end{questionparts}

\end{question}
	
%%%%%%%%% Q7
\begin{question}
\begin{questionparts}
\item At time $t=0$ a tank contains one unit of water. Water flows
out of the tank at a rate proportional to the amount of water in the
tank. The amount of water in the tank at time $t$ is $y$. Show that
there is a constant $b<1$ such that $y=b^{t}.$ 
\item Suppose instead that the tank contains one unit of water at
time $t=0,$ but that in addition to water flowing out as described,
water is added at a steady rate $a>0.$ Show that 
\[
\frac{\mathrm{d}y}{\mathrm{d}t}-y\ln b=a,
\]
and hence find $y$ in terms of $a,b$ and $t$. 
\end{questionparts}

\end{question}
		
%%%%%%%%% Q8
\begin{question}	
\begin{questionparts}
\item By using the formula for the sum of a geometric series, or
otherwise, express the number $0.38383838\ldots$ as a fraction in
its lowest terms. 
\item Let $x$ be a real number which has a recurring decimal expansion
\[
x=0\cdot a_{1}a_{2}a_{2}\cdots,
\]
so that there exists positive integers $N$ and $k$ such that $a_{n+k}=a_{n}$
for all $n>N.$ Show that 
\[
x=\frac{b}{10^{N}}+\frac{c}{10^{N}(10^{k}-1)}\,,
\]
where $b$ and $c$ are integers to be found. Deduce that $x$ is
rational. 
\end{questionparts}
\end{question}	
		

		
	
\newpage
\section*{Section B: \ \ \ Mechanics}


	
%%%%%%%%%% Q9
\begin{question}
A bungee-jumper of mass $m$ is attached by means of a light rope
of natural length $l$ and modulus of elasticity $mg/k,$ where $k$
is a constant, to a bridge over a ravine. She jumps from the bridge
and falls vertically towards the ground. If she only just avoids hitting
the ground, show that the height $h$ of the bridge above the floor
of the ravine satisfies 
\[
h^{2}-2hl(k+1)+l^{2}=0,
\]
and hence find $h.$ Show that the maximum speed $v$ which she attains
during her fall satisfies 
\[
v^{2}=(k+2)gl.
\]
	\end{question}
	
%%%%%%%%%% Q10
\begin{question}	
A spaceship of mass $M$ is at rest. It separates into two parts in
an explosion in which the total kinetic energy released is $E$. Immediately
after the explosion the two parts have masses $m_{1}$ and $m_{2}$
and speeds $v_{1}$ and $v_{2}$ respectively. Show that the minimum
possible relative speed $v_{1}+v_{2}$ of the two parts of the spaceship
after the explosion is $(8E/M)^{1/2}.$ 
\end{question}

%%%%%%%%%% Q11

\begin{question}
A particle is projected under the influence of gravity from a point
$O$ on a level plane in such a way that, when its horizontal distance
from $O$ is $c$, its height is $h$. It then lands on the plane
at a distance $c+d$ from $O$. Show that the angle of projection
$\alpha$ satisfies 
\[
\tan\alpha=\frac{h(c+d)}{cd}
\]
and that the speed of projection $v$ satisfies 
\[
v^{2}=\frac{g}{2}\left(\frac{cd}{h}+\frac{(c+d)^{2}h}{cd}\right)\,.
\]
\end{question}
	

	
	\newpage
\section*{Section C: \ \ \ Probability and Statistics}


%%%%%%%%%% Q12
\begin{question}
An examiner has to assign a mark between 1 and $m$ inclusive to each
of $n$ examination scripts ($n\leqslant m$). He does this randomly,
but never assigns the same mark twice. If $K$ is the highest mark
that he assigns, explain why 
\[
\mathrm{P}(K=k)=\left.\binom{k-1}{n-1}\right/\binom{m}{n}
\]
for $n\leqslant k\leqslant m,$ and deduce that 
\[
\sum_{k=n}^{m}\binom{k-1}{n-1}=\binom{m}{n}\,.
\]
Find the expected value of $K$. 
\end{question}

%%%%%%%%%% Q13
\begin{question}
I have a Penny Black stamp which I want to sell to my friend Jim,
but we cannot agree a price. So I put the stamp under one of two cups,
jumble them up, and let Jim guess which one it is under. If he guesses
correctly, I add a third cup, jumble them up, and let Jim guess correctly,
adding another cup each time. The price he pays for the stamp is $\pounds N,$
where $N$ is the number of cups present when Jim fails to guess correctly.
Find $\mathrm{P}(N=k)$. Show that $\mathrm{E}(N)=\mathrm{e}$ and
calculate $\mathrm{Var}(N).$ 
\end{question}

%%%%%%%%%% Q14
\begin{question}
A biased coin, with a probability $p$ of coming up heads and a probability
$q=1-p$ of coming up tails, is tossed repeatedly. Let $A$ be the
event that the first run of $r$ successive heads occurs before the
first run of $s$ successive tails. If $H$ is the even that on the
first toss the coin comes up heads and $T$ is the event that it comes
up tails, show that 
\begin{alignat*}{1}
\mathrm{P}(A|H) & =p^{\alpha}+(1-p^{\alpha})\mathrm{P}(A|T),\\
\mathrm{P}(A|T) & =(1-q^{\beta})\mathrm{P}(A|H),
\end{alignat*}
where $\alpha$ and $\beta$ are to be determined. Use these two equations
to find $\mathrm{P}(A|H),$ $\mathrm{P}(A|T),$ and hence $\mathrm{P}(A).$
\end{question}
	
\end{document}
