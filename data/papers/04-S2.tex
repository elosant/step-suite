\documentclass[a4, 11pt]{report}


\pagestyle{myheadings}
\markboth{}{Paper II, 2004
\ \ \ \ \ 
\today 
}               

\RequirePackage{amssymb}
\RequirePackage{amsmath}
\RequirePackage{graphicx}
\RequirePackage{color}
\RequirePackage[flushleft]{paralist}[2013/06/09]



\RequirePackage{geometry}
\geometry{%
  a4paper,
  lmargin=2cm,
  rmargin=2.5cm,
  tmargin=3.5cm,
  bmargin=2.5cm,
  footskip=12pt,
  headheight=24pt}


\newcommand{\comment}[1]{{\bf Comment} {\it #1}}
%\renewcommand{\comment}[1]{}

\newcommand{\bluecomment}[1]{{\color{blue}#1}}
%\renewcommand{\comment}[1]{}
\newcommand{\redcomment}[1]{{\color{red}#1}}



\usepackage{epsfig}
\usepackage{pstricks-add}
\usepackage{tgheros} %% changes sans-serif font to TeX Gyre Heros (tex-gyre)
\renewcommand{\familydefault}{\sfdefault} %% changes font to sans-serif
%\usepackage{sfmath}  %%%% this makes equation sans-serif
%\input RexFigs


\setlength{\parskip}{10pt}
\setlength{\parindent}{0pt}

\newlength{\qspace}
\setlength{\qspace}{20pt}


\newcounter{qnumber}
\setcounter{qnumber}{0}

\newenvironment{question}%
 {\vspace{\qspace}
  \begin{enumerate}[\bfseries 1\quad][10]%
    \setcounter{enumi}{\value{qnumber}}%
    \item%
 }
{
  \end{enumerate}
  \filbreak
  \stepcounter{qnumber}
 }


\newenvironment{questionparts}[1][1]%
 {
  \begin{enumerate}[\bfseries (i)]%
    \setcounter{enumii}{#1}
    \addtocounter{enumii}{-1}
    \setlength{\itemsep}{5mm}
    \setlength{\parskip}{8pt}
 }
 {
  \end{enumerate}
 }



\DeclareMathOperator{\cosec}{cosec}
\DeclareMathOperator{\Var}{Var}

\def\d{{\mathrm d}}
\def\e{{\mathrm e}}
\def\g{{\mathrm g}}
\def\h{{\mathrm h}}
\def\f{{\mathrm f}}
\def\p{{\mathrm p}}
\def\s{{\mathrm s}}
\def\t{{\mathrm t}}


\def\A{{\mathrm A}}
\def\B{{\mathrm B}}
\def\E{{\mathrm E}}
\def\F{{\mathrm F}}
\def\G{{\mathrm G}}
\def\H{{\mathrm H}}
\def\P{{\mathrm P}}


\def\bb{\mathbf b}
\def \bc{\mathbf c}
\def\bx {\mathbf x}
\def\bn {\mathbf n}

\newcommand{\low}{^{\vphantom{()}}}
%%%%% to lower suffices: $X\low_1$ etc


\newcommand{\subone}{ {\vphantom{\dot A}1}}
\newcommand{\subtwo}{ {\vphantom{\dot A}2}}




\def\le{\leqslant}
\def\ge{\geqslant}


\def\var{{\rm Var}\,}

\newcommand{\ds}{\displaystyle}
\newcommand{\ts}{\textstyle}
\def\half{{\textstyle \frac12}}
\def\l{\left(}
\def\r{\right)}



\begin{document}
\setcounter{page}{2}

 
\section*{Section A: \ \ \ Pure Mathematics}

%%%%%%%%%%Q1
\begin{question}
Find all real values of $x$ that satisfy:
\begin{questionparts}
\item $ \ds \sqrt{3x^2+1} + \sqrt{x} -2x-1=0 \;;$

\item  $ \ds \sqrt{3x^2+1} - 2\sqrt{x} +x-1=0 \;;$

\item  $ \ds \sqrt{3x^2+1} - 2\sqrt{x} -x+1=0 \;.$
\end{questionparts}
\end{question}

%%%%%%%%%%Q2
\begin{question}
Prove that, if $\vert \alpha\vert < 2\sqrt{2},$ then there is no
value of $x$ for which
\begin{equation}
x^2 -{\alpha}\vert x \vert + 2 < 0\;.
\tag{$*$}
\end{equation}
Find the solution set of $(*)$ for ${\alpha}=3\,$.

For ${\alpha} > 2\sqrt{2}\,$, 
the sum of the lengths of the intervals in which  $x$ satisfies  $(*)$ 
is denoted by $S\,$. Find $S$ in terms of ${\alpha}$  and deduce that
$S < 2{\alpha}\,$.

Sketch the graph of $S\,$ against $\alpha \,$.
\end{question}

%%%%%%%%% Q3
\begin{question}
The curve $C$ has equation
$$
y = x(x+1)(x-2)^4.
$$
Determine the coordinates of all the stationary points of $C$ and
the nature of each. \mbox{Sketch  $C$.}


In  separate diagrams draw  sketches of the curves whose equations are: 
\begin{questionparts}
\item $ y^2 = x(x+1)(x-2)^4\;$;
\item $y = x^2(x^2+1)(x^2-2)^4\,$. 
\end{questionparts}
\end{question}

%%%%%% Q4 
\begin{question}$\,$
\setlength{\unitlength}{1cm}
\begin{center}
\hspace{2cm}
\begin{picture}(6,3.5)
\put(-1.5,4.3){Figure 1}

\thicklines
%\put(1,3){\line(3,-2){2}}
%\put(1,3){\line(2,3){0.5}}
\put(0,3.75){\line(3,-2){3.5}}
%\put(3,1.67){\line(2,3){0.5}}


\put(-1,3.75){\line(1,0){4.5}}
\put(2,2.3){\line(0,-1){1.55}}
\put(-1,2.3){\line(1,0){3}}
\put(3.5,3.75){\line(0,-1){3}}


\put(1.8,2.7){$L$}

\thinlines
\put(-0.6,2.3){\line(0,1){1.45}}
\put(2,1){\line(1,0){1.5}}
\put(2.7,1.1){$b$}
\put(-0.86, 3){$a$}

\end{picture}
\hspace{0cm}
\begin{picture}(6,4.5)

\put(-1.5,4.3){Figure 2}

\thicklines
\put(1,3){\line(3,-2){2}}
\put(1,3){\line(2,3){0.5}}
\put(1.5,3.75){\line(3,-2){2}}
\put(3,1.67){\line(2,3){0.5}}


\put(-1,3.75){\line(1,0){4.5}}
\put(2,2.3){\line(0,-1){1.55}}
\put(-1,2.3){\line(1,0){3}}
\put(3.5,3.75){\line(0,-1){3}}

\put(1.25,3.15){$w$}
\put(2.6,3.1){$l$}

\thinlines
\put(-0.6,2.3){\line(0,1){1.45}}
\put(2,1){\line(1,0){1.5}}
\put(2.7,1.1){$b$}
\put(-0.86, 3){$a$}

\end{picture}
 \end{center}

\vspace*{-10mm}
\begin{questionparts}
\item An attempt is made to move a rod of length $L$ from a corridor 
of width $a$ into a corridor of width~$b$, where $a \ne b.$ The corridors
meet at right angles, as shown in Figure 1 and the rod remains horizontal.
Show that if the attempt is to be successful then 
$$
L \le a \cosec {\alpha} + b \sec {\alpha} \;,
$$ 
where ${\alpha}$ satisfies 
$$
\tan^3\alpha =\frac a b \;.
$$



\item
An attempt is made to move a rectangular table-top,  of width $w$ and length $l$,
from one corridor to the other, as shown in the Figure 2. 
The table-top remains horizontal.
Show that if the attempt is to be successful then 
$$
l\le a \cosec {\beta} + b \sec {\beta}  -2w \cosec 2{\beta},
$$ 
where ${\beta}$ satisfies 
$$
w=  \left(\frac {a -b \tan^3 \beta} {1 - \tan^2 \beta} \right)
\cos \beta \;.
$$

\end{questionparts}
\end{question}

%%%%%%%%% Q5
\begin{question}
Evaluate  $\int_0^{{\pi}} x \sin x\,\d x$
and $\int_0^{{\pi}} x \cos x\,\d x\;$.

The function $\f$ satisfies the equation
\begin{equation}
\f(t)=t + \int_0^{{\pi}} \f(x)\sin(x+t)\,\d x\;. 
\tag{$*$}
\end{equation}
Show that 
\[
\f(t)=t + A\sin t + B\cos t\;,        
\]
where $A= \int_0^{{\pi}}\,\f(x)\cos x\,\d x\;$ and
      $B= \int_0^{{\pi}}\,\f(x)\sin x\,\d x\;$. 

Find $A$ and $B$ by substituting for $\f(t)$ and $\f(x)$ in $(*)$
and equating coefficients of 
$\sin t$ and $\cos t\,$.
	\end{question}
	
%%%%%%%%% Q6
\begin{question}
The vectors ${\bf a}$ and ${\bf b}$ lie in the plane
$\Pi\,$.  Given that $\vert {\bf a} \vert= 1$  and 
${\bf a}.{\bf b} = 3,$ find, in terms of
${\bf a}$ and ${\bf b}\,$, a vector ${\bf p}$ parallel to
${\bf a}$  and a vector ${\bf q}$  perpendicular to ${\bf a}\,$, both 
lying in the plane $\Pi\,$, such that 
$${\bf p}+{\bf q}={\bf a}+{\bf b}\;.$$

The vector ${\bf c}$  is not parallel to the plane $\Pi$ and is such that
${\bf a}.{\bf c} = -2$ and ${\bf b}.{\bf c} = 2\,$. Given  that
$\vert {\bf b} \vert = 5\,$, find, in terms of
${\bf a}, {\bf b}$  and ${\bf c},$ vectors ${\bf P}$, ${\bf Q}$
and ${\bf R}$  such that ${\bf P}$ and ${\bf Q}$ are parallel to
${\bf p}$ and ${\bf q},$  respectively, ${\bf R}$
is perpendicular to the plane $\Pi$  and
$${\bf P} + {\bf Q} + {\bf R} =  {\bf a}+{\bf b}+{\bf c}\;.$$
\end{question}
	
%%%%%%%%% Q7
\begin{question}
The function f is defined by
$$\f(x) = 2\sin x - x\,.$$
Show graphically that the equation $\f(x)=0$
has exactly one root in the interval 
$[\half \pi ,\,{\pi}]\,$. This interval is denoted $I_0$.

In order to determine the root, a sequence of
intervals $I_1$, $I_2$, \, $\ldots$ is generated in the following way.
If the interval 
$I_n=[a_n,b_n]\,$, and $c_n=(a_n+b_n)/2\,$, then
\begin{equation*}
I_{n+1}=
\begin{cases}
[a_n,c_n] & \text{if $\; \f(a_n)\f(c_n)<0 \,$}; \\[5pt]
[c_n,b_n] & \text{if $\; \f(c_n)\f(b_n)<0 \,$}.
\end{cases}
\end{equation*}


By using the approximations $\ds \frac 1{\sqrt{2}} \approx 0.7$ and
${\pi} \approx \sqrt{10} \,$,
show that $I_2=[\half{\pi},\,\frac58{\pi}]$ and find $I_3\,$.
\end{question}
		
%%%%%%%%% Q8
\begin{question}	
Let $x$ satisfy the differential equation
$$
\frac {\d x}{\d t} = {\big( 1-x^n\big)\vphantom{\Big)}}^{\!1/n} 
$$
and   the condition  $x=0$ when $t=0 \,$.

\begin{questionparts}
\item
Solve the equation in the case $n=1$ and sketch the graph of the solution for $t>0 \,$.

\item
Prove that $1-x <  (1-x^2)^{1/2} $ for $0<x<1 \,$. 

Use this result to
sketch the graph of the solution in the case $n=2$ for $0<t<\frac12 \pi \,$, 
using the same axes as your previous sketch.

By setting $x=\sin y\,$, solve the equation in this case.

\item
Use the result (which you need not prove)   
\[
(1-x^2)^{1/2} <  (1-x^3)^{1/3} \text{ \ \ for \ \ } 0<x<1 \;,
\] 
to sketch, without solving the equation, the graph of the solution
of the equation in the case $n=3$  using the same axes as your previous sketches.
Use your sketch to show that $x=1$ at a value of $t$ less than $\frac12 \pi \,$. 
\end{questionparts}
\end{question}	
		

		
	
\newpage
\section*{Section B: \ \ \ Mechanics}


	
%%%%%%%%%% Q9
\begin{question}
The base of a non-uniform solid hemisphere,
of mass $M,$ has radius $r.$ The distance of the centre of gravity, $G$, 
of the hemisphere from the base is $p$ and from the centre of the base 
is $\sqrt{p^2 + q^2} \,$.  The hemisphere rests in equilibrium with  its curved surface
on a horizontal plane. 


A particle of mass $m,\,$ where $m$ is small, 
is attached to $A\,$, the lowest point of the circumference of the base.  
In the new position of equilibrium,
find the angle, $\alpha$, that the base makes with the horizontal.


The particle is removed and attached to the point $B$ of the base 
which is at the other end of the diameter through $A\,$. In the new position
of equilibrium the base  makes an angle ${\beta}$ 
with the horizontal.
Show that
$$
\tan(\alpha-\beta)= 
\frac{2mMrp}  {M^2\left(p^2+q^2\right)-m^2r^2}\;.
$$ 
	\end{question}
	
%%%%%%%%%% Q10 
\begin{question}	
In this question take $g = 10 ms^{-2}.$

The point $A$ lies on a fixed rough plane inclined at $30^{\circ}$
to the horizontal and $\ell$ is the line of greatest slope through $A$.
A particle $P$ is projected up $\ell$ from $A$ with initial 
speed $6$ms$^{-1}$.
A time $T$ seconds later, a particle $Q$ is projected from $A$
up $\ell$, also  with speed $6$ms$^{-1}$.  
The coefficient of friction between 
each particle and the plane is $1/(5\sqrt{3})\,$ and the mass of each particle is $4$kg.


\begin{questionparts}
\item Given that $T<1+\sqrt{3/2}$, show 
that the particles collide at a time $(3-\sqrt6)b+1$ seconds
after $P$ is projected.

\item In the case $T=1+\sqrt{2/3}\,$, 
determine the energy lost due to friction from the instant at
which $P$ is projected  to the time of the collision.
\end{questionparts}
\end{question}

%%%%%%%%%% Q11

\begin{question}
The maximum power that can be developed by the engine 
of train $A$, 
of mass $m$, when travelling at speed $v$ is $Pv^{3/2}\,$, where
$P$ is a constant. 
The maximum power that can be developed by the engine of train $B$, 
of mass $2m$,
when travelling at speed $v$ is  $2Pv^{3/2}.$   For both $A$ and $B$
 resistance to motion is equal to $kv$, where $k$ is a constant.

For $t\le0$, the engines are crawling along at very low equal speeds. 
At $t = 0\,$, both
drivers switch on full power and at time $t$ the speeds of 
$A$ and $B$ are $v_{\vphantom{\dot A}\!A}$ and $v_{\vphantom{\dot B}\hspace{-1pt}B},$ respectively.

\begin{questionparts}
\item
Show that 
\[
v_{\vphantom{\dot A}\!A} = \frac{P^2 \left(1-\e^{-kt/2m}\right)^2}{k^2}
\]
and write down the corresponding result for $v_{\vphantom{\dot B}B}$.

\item Find  $v_{\vphantom{\dot B}A}$ and  $v_{\vphantom{\dot B}B}$
when $9  v_{\vphantom{\dot B}A} =4v_{\vphantom{\dot B}B}\;$.

%Show that 
%$1 < v_{\vphantom{\dot B}\hspace{-1pt}B} /v_{\vphantom{\dot A}\!A} < 4$ for $t>0\,$.

\item
Both engines are switched off when  $9  v_{\vphantom{\dot B}A} =4v_{\vphantom{\dot B}B}\,$.
Show that thereafter $k^2 v_{\vphantom{\dot B}B}^2 = 4 P^2  v_{\vphantom{\dot B}A}\,$.
 
\end{questionparts}
\end{question}
	

	
	\newpage
\section*{Section C: \ \ \ Probability and Statistics}


%%%%%%%%%% Q12
\begin{question}
Sketch the graph, for $x \ge 0\,$, of 
$$ 
y = kx\e^{-ax^2} \;,
$$
where $a$ and $k$ are  positive constants.


The random variable $X$ has probability density function
$\f(x)$ given by
\begin{equation*}
\f(x)=
\begin{cases}
kx\e^{-ax^2} & \text{for $0 \le x \le 1$}\\[3pt]
            0 & \text{otherwise}. 
\end{cases}
\end{equation*}

Show that $\ds k=\frac{2a}{1-\e^{-a}}$ and find the mode $m$ in terms of $a\,$,
distinguishing between the cases $a < \frac12$ and $a > \frac12\,$.

Find the median $h$ in terms of $a$\,  and  show that
$h > m$ if
$a > -\ln\left(2\e^{-1/2} - 1\right).$

Show that, $-\ln\left(2\e^{-1/2}-1\right)> \frac12 \,$.
Show also that, 
if $a > -\ln\left(2\e^{-1/2} - 1\right) \,$,  then 
$$
P(X > m \;\vert\; X < h) = 
{{2\e^{-1/2}-\e^{-a}-1} \over 1-\e^{-a}}\;.
$$ 
\end{question}

%%%%%%%%%% Q13
\begin{question}
A bag contains $b$ balls, $r$ of them red and the rest white. 
In a game the player must remove balls one at a time from the bag (without replacement). 
She may remove  as many balls as she wishes, but if she removes any red
ball, she loses and gets no reward at all.
If she does not remove a red ball,
she is rewarded with \pounds 1 for each white ball she has removed.


If she removes $n$ white balls on her first $n$ draws, calculate her expected 
gain on the next draw and show that 
%her expected total reward would be the same as before 
it is zero 
if $\ds n = {b-r \over r+1}\,$.

Hence, or otherwise, show that she will maximise her expected total
reward if she aims to remove $n$ balls, where 
\[
n = \mbox{ the integer part of } \ds {b + 1 \over r + 1}\;.
\]

With this value of $n$, show that in
the case $r=1$ and $b$ even,
her expected total reward is $\pounds {1 \over 4}b\,$, and find her expected total reward in
the case $r=1$ and $b$ odd.
\end{question}

%%%%%%%%%% Q14
\begin{question}
Explain why, if $\mathrm{A, B}$ and  $\mathrm{C}$ are three events, 
\[
\mathrm{P(A \cup B \cup C) = P(A) + P(B) + P(C) - P(A \cap B) - P(B \cap C) - P(C \cap A) +P(A \cap B \cap C)},
\]
where $\mathrm{P(X)}$ denotes the probability of event $\mathrm{X}$.

A cook makes three plum puddings for Christmas. 
He stirs $r$ silver sixpences thoroughly into the pudding mixture before 
dividing it into three equal portions. Find an expression for the probability 
that each pudding contains at least one sixpence. 
Show that the cook must stir 6 or more sixpences into the mixture 
if there is to be less than ${1 \over 3}$ chance that 
at least one of the puddings contains no sixpence.

Given that the cook stirs 6 sixpences into the mixture and   that  
each pudding contains at least one sixpence, 
find  the probability that there are two  sixpences in each pudding.
\end{question}
	
\end{document}
