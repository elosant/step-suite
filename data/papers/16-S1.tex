\documentclass[a4, 11pt]{report}


%\pagestyle{myheadings}
%\markboth{}{Paper I, 2006  second vetter draft 
%\ \ \ \ \ 
%\today 
%}               


\usepackage{pstricks-add}
\usepackage{epsfig}

\RequirePackage{amssymb}
\RequirePackage{amsmath}
\RequirePackage{graphicx}
\RequirePackage{color}
\RequirePackage{xcolor}


\RequirePackage[flushleft]{paralist}[2013/06/09]



\RequirePackage{geometry}
\geometry{%
  a4paper,
  lmargin=2cm,
  rmargin=2.5cm,
  tmargin=3.5cm,
  bmargin=2.5cm,
  footskip=12pt,
  headheight=24pt}


\newcommand{\comment}[1]{{\bf Comment} {\it #1}}
%\renewcommand{\comment}[1]{}

\newcommand{\bluecomment}[1]{{\color{blue}#1}}
%\renewcommand{\comment}[1]{}
\newcommand{\redcomment}[1]{{\color{red}#1}}




%\input RexFigs


\setlength{\parskip}{10pt}
\setlength{\parindent}{0pt}

\newlength{\qspace}
\setlength{\qspace}{20pt}


\newcounter{qnumber}
\setcounter{qnumber}{0}

\newenvironment{question}%
 {\vspace{\qspace}
  \begin{enumerate}[\bfseries 1\quad][10]%
    \setcounter{enumi}{\value{qnumber}}%
    \item%
 }
{
  \end{enumerate}
  \filbreak
  \stepcounter{qnumber}
 }


\newenvironment{questionparts}[1][1]%
 {
  \begin{enumerate}[\bfseries (i)]%
    \setcounter{enumii}{#1}
    \addtocounter{enumii}{-1}
    \setlength{\itemsep}{5mm}
    \setlength{\parskip}{8pt}
 }
 {
  \end{enumerate}
 }



\DeclareMathOperator{\cosec}{cosec}
\DeclareMathOperator{\Var}{Var}

\def\d{{\rm d}}
\def\e{{\rm e}}
\def\g{{\rm g}}
\def\h{{\rm h}}
\def\f{{\rm f}}
\def\p{{\rm p}}
\def\q{{\rm q}}
\def\s{{\rm s}}
\def\t{{\rm t}}


\def\A{{\rm A}}
\def\B{{\rm B}}
\def\E{{\rm E}}
\def\F{{\rm F}}
\def\G{{\rm G}}
\def\H{{\rm H}}
\def\P{{\rm P}}


\def\bb{\mathbf b}
\def \bc{\mathbf c}
\def\bx {\mathbf x}
\def\bn {\mathbf n}

\makeatletter
\newcommand{\raisemath}[1]{\mathpalette{\raisem@th{#1}}}
\newcommand{\raisem@th}[3]{\raisebox{#1}{$#2#3$}}
\makeatother
%%%To raise suffices: e.g.  $\Pi_{\raisemath{2pt}{-}}$.




\def\le{\leqslant}
\def\ge{\geqslant}


\def\var{{\rm Var}\,}

\newcommand{\ds}{\displaystyle}
\newcommand{\ts}{\textstyle}




\begin{document}

\setcounter{page}{2}

 
\section*{Section A: \ \ \ Pure Mathematics}

%%%%%%%%%%%%%% Q1
\begin{question}
\begin{questionparts}
\item
For $n=1$, $2$, $3$ and $4$, the functions $\p_n$ and $\q_n$ are
defined by
\[
\p_n(x) = (x+1)^{2n} - (2n+1)x (x^2+x+1)^{n-1}
\]
and 
\[
\q_n(x) = \frac{x^{2n+1}+1}{x+1}
\ \ \ \ \ \ \ \ \ \ \ \ (x\ne -1)
\,.  \ \ \ \ \ \ \ \ \ \   
\]
Show that
 \ $\p_n(x)\equiv \q_n(x)$ \ 
(for $x\ne-1$)
 in the cases $n=1$, $n=2$ and $n=3$.

Show also that this does not hold in the case $n=4$.




\item Using  results from part (i):   

\begin{itemize}
\item[\bf (a)] 
express
$ \ \dfrac {300^3 +1}{301}\,$
 as the product of two factors (neither of which is 1);
\\

\item[\bf (b)] 
express 
$ \ \dfrac {7^{49}+1}{7^7+1}\,$ as the product of two factors (neither of 
which is 1), each 
written 
\\[1mm]
in terms of 
various powers of 7 which you should not attempt
to calculate explicitly. 
\end{itemize}

\end{questionparts}
\end{question} 


%%%%%%%%%%%%%% Q2

\begin{question}
Differentiate, with respect to $x$,
\[
(ax^2+bx+c)\,\ln  \big( x+\sqrt{1+x^2}\big) +\big(dx+e\big)\sqrt{1+x^2}
\,,
\]
where $a$, $b$, $c$, $d$ and  $e$ are constants. You should simplify your
answer as far as possible.

Hence integrate:
\begin{questionparts}
\item $ \ln \big( x+\sqrt{1+x^2}\,\big) \,;$
\item $\sqrt{1+x^2} \,; $
\item $ x\ln \big( x+\sqrt{1+x^2}\,\big) \,.$

\end{questionparts}
\end{question}

%%%%%%%%%%%%%% Q3


\begin{question}
In this question, $\lfloor x \rfloor$ denotes the greatest integer
that  is less than or equal to $x$, so that (for example)
$\lfloor 2.9 \rfloor = 2$,  $\lfloor 2\rfloor = 2$
and
 $\lfloor -1.5 \rfloor = -2$.

On separate diagrams draw the graphs, for $-\pi \le x \le \pi$, of:
\begin{center}
(i) \ \  $y = \lfloor x \rfloor$; \ \ \ \ \ 
(ii) \ \ $y=\sin\lfloor x \rfloor$; \ \ \ \ \ 
(iii) \ \ $y = \lfloor \sin x\rfloor$; \ \ \ \ \ 
(iv) \ \ $y= \lfloor 2\sin x\rfloor$. 
\end{center}
In each case,
you should indicate clearly the value of $y$ at points where the graph
is discontinuous.  

\end{question}
%%%%%%%%%%%%%% Q4

\begin{question}
\begin{questionparts}
\item Differentiate
$\displaystyle
\;
\frac z {(1+z^2)^{\frac12}}
\;$
with respect to $z$.


\item 
The {\em signed curvature} $\kappa$ of the  curve $y=\f(x)$ is 
defined by
\[
\kappa = \frac {\f''(x)}{\big({1+ (\f'(x))^2\big)^{\frac32}}} 
\,.\]
Use this definition to  
determine all curves for which the signed curvature
is a non-zero constant. For these
curves, what is the geometrical significance of $\kappa$?
\end{questionparts}

\end{question}


%%%%%%%%%%%%%% Q5
\begin{question}
\begin{questionparts}
\item 
\noindent\vspace{-4cm}
%%%%%%%The diagram requires scale of 1 unit = 15cm and 11 pt font. 
\begin{center}

\psset{xunit=15.0cm,yunit=15.0cm,algebraic=true,dimen=middle,dotstyle=o,dotsize=3pt 0,linewidth=0.8pt,arrowsize=3pt 2,arrowinset=0.25}
\begin{pspicture*}(0.579583766333,2.71182136869)(1.83739646118,3.59292873869)
\psline(0.630644718793,3.13692729767)(1.31015089163,3.14219478738)
\pscircle(0.785157750343,3.25807956104){1.75692695495}
\pscircle(1.00990397805,3.24754458162){1.58572438898}
\pscircle(0.900067062948,3.16853223594){0.425605184678}
\rput[tl](0.771128846767,3.26889831096){$A$}
\rput[tl](0.890844522038,3.17472197974){$B$}
\rput[tl](1.00098294329,3.26570589295){$C$}
\rput[tl](0.77272505577,3.12683570963){$P$}
\rput[tl](0.892440731042,3.12683570963){$Q$}
\rput[tl](1.01215640631,3.12523950063){$R$}
\end{pspicture*}


\end{center}

\vspace{-6cm}
The diagram shows three touching
circles $A$, $B$ and $C$, with a common tangent
$PQR$.
 The  radii of the circles are $a$, $b$ and $c$,
respectively.

Show that 
\[
\frac 1 {\sqrt b} = \frac 1 {\sqrt{a}} + \frac1{\sqrt{c}}
\tag{$*$}
\]
and deduce that
\[
2\left(\frac1{a^2} + \frac1 {b^2} + \frac1 {c^2}
\right)
=
\left(\frac1 a + \frac1 {b} + \frac1 {c}
\right)^{\!2}
.
\tag{$**$}
\]

\item
Instead, let $a$, $b$ and $c$ be  positive numbers,
with  $b<c<a$, which satisfy $(**)$. Show that they also satisfy $(*)$. 
\end{questionparts}


\end{question}
%%%%%%%%%%%%%% Q6
\begin{question}
The sides $OA$ and $CB$ of the quadrilateral $OABC$ are parallel. The point
$X$ lies on $OA$, between $O$ and $A$. The position vectors
of $A$, $B$, $C$ and $X$ relative to the origin $O$ are 
$\bf a$, $\bf b$, $\bf c$ and $\bf x$, respectively.
Explain why  $\bf c$ and $\bf x$ can be written in the form
\[
{\bf c} = k {\bf a} + {\bf b}
\text{ \ \ \ \ and \ \ \ \ }
{\bf x} = m {\bf a}\,,
\]
where $k$ and $m$ are scalars, and  
state the range of values
that each of $k$ and $m$ can take.

%\begin{questionparts}
%\item 
The lines $OB$ and $AC$ intersect at $D$, the lines $XD$ and $BC$
intersect at $Y$ and the lines $OY$ and $AB$ intersect at $Z$. Show 
that the position vector of $Z$ relative to $O$ can be written as
\[
\frac{ {\bf b} + mk {\bf a}}{mk+1}\,.
\]
%\end{questionparts}

The lines $DZ$ and $OA$ intersect at $T$. Show that 
\[
OT \times OA = OX\times TA
\text{ \ \ \ \ and \ \ \ \ }
\frac 1 {OT} = \frac 1 {OX} + \frac 1 {OA}
\,,
\]
where, for example, $OT$ denotes the length of the line joining
$O$ and $T$.


\end{question}

%%%%%%%%%%%%%% Q7
\begin{question}
The set $S$
% = \{1, 5, 9, 13, \,\ldots \}$ 
consists of all the 
positive integers that leave a remainder of 1 upon division by 4.
The set $T$
% = \{1, 5, 9, 13, \,\ldots \}$ 
consists of all the 
positive integers that leave a remainder of 3 upon division by 4.




\begin{questionparts}
\item 
Describe in words the sets 
$S \cup T$  and  $S \cap T$.


\item Prove that the product of any two integers in $S$ is also in $S$.
Determine whether the product of any two integers in $T$
is also in  $T$.

\item
 Given an integer in $T$ that is not a prime number, 
prove that at least one of its prime factors is in $T$.

\item  
For any set $X$ of positive integers,
 an integer in  $X$ (other than 1) is said to be 
{\em $X$-prime} if it cannot be expressed as the product of
two or more integers {\em all in $X$} (and all different from 1).

\begin{itemize}
\item[\bf (a)] Show that every integer  in $T$  is 
either  $T$-prime or  is the  product of an odd number of 
$T$-prime integers.

\item[\bf (b)] Find an 
example of an integer in $S$ that can be expressed as the product
of  \hbox{$S$-prime} integers in two distinct ways. [Note: 
 $s_1s_2$ and $s_2s_1$ are not counted as distinct ways of 
expressing the product of $s_1$ and $s_2$.]
\end{itemize}
\end{questionparts}

 
\end{question}
%%%%%%%%%%%%%% Q8
\begin{question}
Given an infinite sequence  of numbers
$u_0$, $u_1$, $u_2$, $\ldots\,$, we define the
{\em generating function}, $\f$, for the sequence by
\[
\f(x) =  u_0 + u_1x +u_2 x^2 +u_3 x^3 + \cdots \,.
\]
Issues of convergence can be ignored in this question.

\begin{questionparts}

\item 
Using the binomial series, show that 
the sequence given by $u_n=n\,$ has generating function
$x(1-x)^{-2}$,
and
find the sequence that has generating function $x(1-x)^{-3}$. 

Hence, or otherwise, find the generating function for the 
sequence $u_n =n^2$. You should simplify your answer.

\item 
\begin{itemize}
\item[\bf (a)]
The sequence $u_0$, $u_1$, $u_2$, $\ldots\,$ is 
 determined by
$u_{n} = ku_{n-1}$ ($n\ge1$),
where $k$ is independent of $n$,
and $u_0=a$. 
By summing 
the identity $u_{n}x^n \equiv ku_{n-1}x^n$, or otherwise,
show that
 the generating function, f, satisfies
\[
\f(x) = a + kx \f(x)
\,.
\]
Write down  an expression for $\f(x)$.

\vspace{3mm}
\item[\bf (b)]
The sequence 
 $u_0$, $u_1$, $u_2$, $\ldots\,$ is 
 determined by
$u_{n} = u_{n-1}+ u_{n-2}$ ($n\ge2$)
and $u_0=0$, $u_1=1$. 
Obtain the generating function.

\end{itemize}
\end{questionparts}

\end{question}



\newpage
\section*{Section B: \ \ \ Mechanics}
%%%%%%%%%%%%%% Q9
\begin{question}
A horizontal rail is fixed parallel
to a vertical wall and at
a distance $d$ from the wall.
A~uniform rod $AB$
of length $2a$
rests in equilibrium on the rail with the end $A$ in contact with 
the wall. The rod lies in 
 a vertical plane perpendicular to the wall. It 
is inclined at an angle~$\theta$ to the 
vertical (where $0<\theta<\frac12\pi$)
 and  
$a\sin\theta < d$,
as shown in the 
diagram. 

\begin{center}
\newrgbcolor{wwwwww}{0.4 0.4 0.4}
\psset{xunit=1.0cm,yunit=1.0cm}
\begin{pspicture*}(-2.02,-0.1)(6.98,4.98)
\psset{xunit=1.0cm,yunit=1.0cm,algebraic=true,dotstyle=o,dotsize=3pt 0,linewidth=0.8pt,arrowsize=3pt 2,arrowinset=0.25}

\psline[linewidth=6.2pt,linestyle=dashed,dash=11pt 1pt,linecolor=wwwwww](-0.12,5)(-0.12,0)

\psline[linewidth=2pt](0,1.86)(5.86,3.34)
%\psline[linestyle=dotted](0,1.82)(4.02,1.82)
\psline[linestyle=dotted](4,4.48)(4,0.58)
\rput[tl](0.15,2.4){$ \theta $}
\psline[linestyle=dotted](0,4  )(3.98,4  )
\rput[tl](1.86,4.40){$ d $}
\rput[tl](0.12,1.78){$ A $}
\rput[tl](5.66,3.14){$ B $}
\parametricplot{0.2473863631292545}{1.5708}{1*0.80*cos(t)+0*0.80*sin(t)+0|0*0.80*cos(t)+1*0.80*sin(t)+1.86}
\begin{scriptsize}
\psdots[dotsize=5pt 0,dotstyle=*](4,2.78)
\end{scriptsize}
\end{pspicture*}
\end{center}





The coefficient of friction between the rod and the 
wall  is $\mu$, and the coefficient of friction between the
rod and the rail is $\lambda$.


Show that in limiting equilibrium, with the rod on the point
of slipping at both the wall and the rail, the angle $\theta$
satisfies
\[
d\cosec^2\theta = a\big( (\lambda+\mu)\cos\theta + (1-\lambda \mu)\sin\theta
\big)
\,.
\]

Derive the corresponding result  
if, instead, $ a\sin\theta > d $.


\end{question}
%%%%%%%%%%%%%% Q10
\begin{question}
Four particles  $A$, $B$, $C$ and $D$ 
are initially at rest 
on a smooth horizontal table. They lie
equally spaced a small distance apart, in the order $ABCD$, 
in a straight line.
Their  masses are $\lambda m$, $m$, $m$ and $m$,
respectively, where $\lambda>1$.

Particles $A$ and $D$ are simultaneously  projected, both at speed $u$,
so that they collide with $B$ and $C$ 
(respectively). In the following collision between $B$ and $C$,
particle $B$ 
is brought to rest. The coefficient of restitution in each collision is $e$.

\begin{questionparts}
\item Show that $e = \dfrac {\lambda-1}{3\lambda+1}$ and 
deduce that $e<   \frac 13\,$. 
\item Given also that $C$ and $D$ move towards each other with the
same speed,  find the value of  $\lambda$ and of $e$.
\end{questionparts}


\end{question}
%%%%%%%%%%%%%% Q11
\begin{question}
The point $O$ is at the top of a vertical tower 
of height $h$ which stands in the middle of a large horizontal
	plain. A projectile $P$ is fired from $O$ 
at a fixed speed $u$ and at an angle $\alpha$
 above the horizontal.

Show that the distance $x$ from the base of the tower
when $P$ hits the plain satisfies
\[
\frac{gx^2}{u^2} =
h(1+\cos 2\alpha) + x \sin 2\alpha \,.
\] 

Show that the greatest value of $x$ as $\alpha$ varies
occurs when $x=h\tan2\alpha$ and 
find the corresponding value of $\cos 2\alpha$ in terms of $g$, $h$ and $u$.

Show further that the greatest achievable 
distance between $O$ and the landing point is $\dfrac {u^2}g +h\,$. 

\end{question}
\newpage
\section*{Section C: \ \ \ Probability and Statistics}
%%%%%%%%%%%%%% Q12
\begin{question}
\begin{questionparts}
\item Alice tosses a fair coin twice and Bob tosses a fair coin three times.
Calculate the probability that Bob gets more heads than Alice.
\item Alice tosses a fair coin three times
 and Bob tosses a fair coin four times.
Calculate the probability that Bob gets more heads than Alice.

\item 
Let $p_1$ be the probability that Bob gets the same number of heads as
Alice, and let~$p_2$ be the probability that Bob gets more
heads than Alice,  when 
 Alice and Bob each toss a fair coin $n$ times.

Alice tosses a fair coin $n$ times and Bob tosses a fair coin $n+1$ times.
Express the probability that Bob gets more heads than Alice
in terms of $p_1$ and $p_2$, and hence obtain a generalisation of the 
results of parts (i) and (ii).

\end{questionparts}

\end{question}

%%%%%%%%%%%%%% Q13


\begin{question}
 An internet tester  sends $n$ e-mails simultaneously at 
time $t=0$. Their
arrival times at their destinations 
are independent random variables each having probability density function
$\lambda \e^{-\lambda t}$ ($0\le t<\infty$, $ \lambda >0$).

\begin{questionparts}
\item 
The random variable $T$ is 
the time of arrival of the e-mail that arrives first at its destination.
Show that the probability density function of 
$T$ is
\[
n  \lambda  \e^{-n\lambda t}
\,,
\]
and find the expected value of $T$.

\item
 Write down the probability
that the second e-mail to arrive at its destination arrives later than  time $t$
and hence derive the density function for the time of arrival of the second
e-mail. Show that the expected time of arrival of the second e-mail is
  \[
\frac{1}{\lambda} \left( \frac1{n-1} + \frac 1 n \right)
.
\] 
\end{questionparts}

\end{question}



\end{document}

\nq
For $k>m>0$, express $\dfrac 1 {(k+m)k(k-m)}$ in the form $\dfrac{A}{k(k-m)} +
\dfrac{B} {k(k+m)} $, where $A$ and $B$ are independent of $k$.

By taking $m=1$,  show that 
\[
\frac 1 {1^3}+\frac 1 {2^3} + \frac 1 {3^3}  + \cdots 
\ < \frac 54
\, .
\]

Show also that
\[
\frac 1{1^3} + \frac 1 {2^3}+ \frac 1 {3^3}  + \cdots \ < \frac {119}{96} \,.
\]
\nq
For $k>m>0$, express $\dfrac 1 {(k+m)k(k-m)}$ in the form $\dfrac{A}{k(k-m)} +
\dfrac{B} {k(k+m)} $ where $A$ and $B$ are independent of $k$.

By taking $m=1$,  show that 
\[
\frac 1 {1^3}+\frac 1 {2^3} + \frac 1 {3^3}  + \cdots 
\ < \frac 54
\, .
\]

Show also that
\[
\frac 1{1^3} + \frac 1 {2^3}+ \frac 1 {3^3}  + \cdots \ < \frac {119}{96} \,.
\]

