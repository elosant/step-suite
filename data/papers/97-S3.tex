\documentclass[a4, 11pt]{report}


\pagestyle{myheadings}
\markboth{}{Paper III, 1997
\ \ \ \ \ 
\today 
}               

\RequirePackage{amssymb}
\RequirePackage{amsmath}
\RequirePackage{graphicx}
\RequirePackage{color}
\RequirePackage[flushleft]{paralist}[2013/06/09]



\RequirePackage{geometry}
\geometry{%
  a4paper,
  lmargin=2cm,
  rmargin=2.5cm,
  tmargin=3.5cm,
  bmargin=2.5cm,
  footskip=12pt,
  headheight=24pt}


\newcommand{\comment}[1]{{\bf Comment} {\it #1}}
%\renewcommand{\comment}[1]{}

\newcommand{\bluecomment}[1]{{\color{blue}#1}}
%\renewcommand{\comment}[1]{}
\newcommand{\redcomment}[1]{{\color{red}#1}}



\usepackage{epsfig}
\usepackage{pstricks-add}
\usepackage{tgheros} %% changes sans-serif font to TeX Gyre Heros (tex-gyre)
\renewcommand{\familydefault}{\sfdefault} %% changes font to sans-serif
%\usepackage{sfmath}  %%%% this makes equation sans-serif
%\input RexFigs


\setlength{\parskip}{10pt}
\setlength{\parindent}{0pt}

\newlength{\qspace}
\setlength{\qspace}{20pt}


\newcounter{qnumber}
\setcounter{qnumber}{0}

\newenvironment{question}%
 {\vspace{\qspace}
  \begin{enumerate}[\bfseries 1\quad][10]%
    \setcounter{enumi}{\value{qnumber}}%
    \item%
 }
{
  \end{enumerate}
  \filbreak
  \stepcounter{qnumber}
 }


\newenvironment{questionparts}[1][1]%
 {
  \begin{enumerate}[\bfseries (i)]%
    \setcounter{enumii}{#1}
    \addtocounter{enumii}{-1}
    \setlength{\itemsep}{5mm}
    \setlength{\parskip}{8pt}
 }
 {
  \end{enumerate}
 }



\DeclareMathOperator{\cosec}{cosec}
\DeclareMathOperator{\Var}{Var}

\def\d{{\rm d}}
\def\e{{\rm e}}
\def\g{{\rm g}}
\def\h{{\rm h}}
\def\f{{\rm f}}
\def\p{{\rm p}}
\def\s{{\rm s}}
\def\t{{\rm t}}


\def\A{{\rm A}}
\def\B{{\rm B}}
\def\E{{\rm E}}
\def\F{{\rm F}}
\def\G{{\rm G}}
\def\H{{\rm H}}
\def\P{{\rm P}}


\def\bb{\mathbf b}
\def \bc{\mathbf c}
\def\bx {\mathbf x}
\def\bn {\mathbf n}

\newcommand{\low}{^{\vphantom{()}}}
%%%%% to lower suffices: $X\low_1$ etc


\newcommand{\subone}{ {\vphantom{\dot A}1}}
\newcommand{\subtwo}{ {\vphantom{\dot A}2}}




\def\le{\leqslant}
\def\ge{\geqslant}


\def\var{{\rm Var}\,}

\newcommand{\ds}{\displaystyle}
\newcommand{\ts}{\textstyle}




\begin{document}
\setcounter{page}{2}

 
\section*{Section A: \ \ \ Pure Mathematics}

%%%%%%%%%%Q1
\begin{question}
\begin{questionparts}
\item By considering the series expansion of
$(x^2+5x+4){\rm \; e}^x$ show that
\[10{\rm\,  e}=4+\frac{3^2}{1!}+\frac{4^2}{2!}+\frac{5^2}{3!}+\cdots\;.\]

\item Show that
\[5{\rm\,  e}=1+\frac{2^2}{1!}+\frac{3^2}{2!}+\frac{4^2}{3!}+\cdots\;.\]

\item Evaluate
\[1+\frac{2^3}{1!}+\frac{3^3}{2!}+\frac{4^3}{3!}+\cdots\;.\]
\end{questionparts}
\end{question}

%%%%%%%%%%Q2
\begin{question}
Let \[\mathrm{f}(t)=\frac{\ln t}t\quad\text{ for }t>0.\]
Sketch the graph of $\mathrm{f}(t)$ and find its maximum
value. How many positive values of $t$ correspond to a
given value of $\mathrm f(t)$?

Find how many positive values of $y$ satisfy
\(x^y=y^x\) for a given positive value of $x$. Sketch the
set of points $(x,y)$ which satisfy \(x^y=y^x\) with $x,y>0$.
\end{question}

%%%%%%%%% Q3
\begin{question}
By considering the solutions of the equation $z^n-1=0$, or
otherwise,  show that
\[(z-\omega)(z-\omega^2)\dots(z-\omega^{n-1})=1+z+z^2+\dots+z^{n-1},\]
where $z$ is any complex number and
$\omega={\rm e}^{2\pi i/n}$.

Let $A_1,A_2,A_3,\dots,A_n$ be points equally
spaced around a circle of radius $r$ centred at
$O$ (so that they are the vertices of a regular $n$-sided
polygon).

Show that
\[\overrightarrow{OA_1}+\overrightarrow{OA_2}+\overrightarrow{OA_3}
+\dots+\overrightarrow{OA_n}=\mathbf0.\]
Deduce, or prove otherwise, that
\[\sum_{k=1}^n|A_1A_k|^2=2r^2n.\]
\end{question}

%%%%%% Q4 
\begin{question}
In this question,
you may assume that if $k_1,\dots,k_n$ are distinct positive real
numbers, then
\[\frac1n\sum_{r=1}^nk_r>\left({\prod\limits_{r=1}^n}
k_r\right )^{\!\! \frac1n},\]
i.e.\ their arithmetic mean is greater than their geometric mean.

Suppose that $a$, $b$, $c$ and $d$ are positive real numbers such
that the polynomial
\[{\rm f}(x)=x^3-4ax^2+6b^2x^2-4c^3x+d^4\]
has four distinct positive roots.

\begin{questionparts}
\item Show that $pqr,qrs,rsp$ and $spq$ are distinct, where $p,q,r$ and $s$ are the roots of the polynomial $\mathrm{f}$.
\item By considering the relationship between the coefficients of $\mathrm{f}$ and
its roots, \mbox{show that $c>d$.}
\item Explain why the polynomial $\mathrm{f}'(x)$ must have three distinct roots. 
\item By differentiating $\mathrm{f}$, show that $b>c$.

\item  Show that $a>b$.
\end{questionparts}
	\end{question}

%%%%%%%%% Q5
\begin{question}
Find the ratio, over one revolution, of the distance moved by a
wheel rolling on a flat surface to the distance traced out by a point
on its circumference.
	\end{question}
	
%%%%%%%%% Q6
\begin{question}
 Suppose that $y_n$ satisfies the equations
\[(1-x^2)\frac{{\rm d}^2y_n}{{\rm d}x^2}-x\frac{{\rm
d}y_n}{{\rm d}x}+n^2y_n=0,\]
\[y_n(1)=1,\quad y_n(x)=(-1)^ny_n(-x).\]
If 
$x=\cos\theta$, show that
\[\frac{{\rm d}^2y_n}{{\rm d}\theta^2}+n^2y_n=0,\] and hence
obtain
$y_n$ as a function of
$\theta$. Deduce that for $|x|\leqslant1$
\[y_0=1,\quad y_1=x,\]
\[y_{n+1}-2xy_n+y_{n-1}=0.\]
\end{question}
	
%%%%%%%%% Q7
\begin{question}
 For each positive integer $n$,
let
\begin{align*}
a_n&=\frac1{n+1}+\frac1{(n+1)(n+2)}+\frac1{(n+1)(n+2)(n+3)}+\cdots;\\
b_n&=\frac1{n+1}+\frac1{(n+1)^2}+\frac1{(n+1)^3}+\cdots.
\end{align*}

\begin{questionparts}
\item Evaluate $b_n$.

\item Show that $0<a_n<1/n$.

\item Deduce that $a_n=n!{\rm e}-[n!{\rm e}]$ (where $[x]$ is
the integer part of $x$).

\item Hence show that $\mathrm{e}$ is irrational.
\end{questionparts}
\end{question}
		
%%%%%%%%% Q8
\begin{question}	
Let $R_{\alpha}$ be
the $2\times2$ matrix that represents a rotation through the
angle $\alpha$ and let
$$A=\begin{pmatrix}a&b\\b&c\end{pmatrix}.$$

\begin{questionparts}
\item Find in
terms of $a$, $b$ and $c$ an angle $\alpha$ such that
$R_{-\alpha}AR_{\alpha}$ is a diagonal matrix (i.e.\ has the
value zero in top-right and bottom-left positions).

\item Find values of $a$, $b$ and $c$ such that the equation of the
ellipse
\[x^2+(y+2x\cot2\theta)^2=1\qquad(0<\theta<\tfrac{1}{4}\pi)\]
can be expressed in the form
\[\begin{pmatrix}x&y\end{pmatrix}A\begin{pmatrix}x\\y\end{pmatrix}=1.\]
Show that, for this $A$,  $R_{-\alpha}AR_{\alpha}$ is  diagonal if
 $\alpha=\theta$. Express the non--zero elements
of this matrix in terms of $\theta$.

\item Deduce, or show otherwise,  that the minimum and maximum distances from
the centre to the circumference of this ellipse
 are
$\tan\theta$ and $\cot\theta$.
\end{questionparts}
\end{question}	
		

		
	
\newpage
\section*{Section B: \ \ \ Mechanics}


	
%%%%%%%%%% Q9
\begin{question}
A uniform rigid rod $BC$ is suspended from a fixed point $A$
by light stretched springs $AB,AC$. The springs are of different
natural lengths but the ratio of tension to extension is the same
constant $\kappa$ for each. The rod is {\em not} hanging
vertically. Show that the ratio of the lengths of the stretched
springs is equal to the ratio of the natural lengths of the unstretched
springs.
	\end{question}
	
%%%%%%%%%% Q10
\begin{question}	
By pressing a finger down on it, a uniform spherical marble of 
radius $a$  is made to slide  along a 
horizontal table top with an initial linear velocity $v_0$
and an initial {\sl backward} angular velocity $\omega_0$ about the
horizontal axis perpendicular to $v_0$. The frictional
force  between the marble and the table is constant (independent of
speed).


For what value of $v_0/(a\omega_0)$ does the marble

\begin{questionparts}
\item slide to a complete stop,

\item  come to a stop and then roll back towards its initial position
with linear speed $v_0/7$.
\end{questionparts}
\end{question}

%%%%%%%%%% Q11

\begin{question} $\,$
\begin{center}
\psset{xunit=1.0cm,yunit=1.0cm,algebraic=true,dotstyle=o,dotsize=3pt 0,linewidth=0.5pt,arrowsize=3pt 2,arrowinset=0.25} \begin{pspicture*}(-1.14,-1.32)(3.52,3.04) \psline(-0.92,2.74)(-0.06,-0.6) \psline(-0.06,-0.6)(2.66,-0.6) \psline(2.66,-0.6)(3.38,2.74) \psline[linewidth=0.1pt,linestyle=dashed,dash=2pt 2pt](1.22,-0.6)(1.22,2.74) \psline(3.07,2.11)(1.22,-0.6)  \psline{<->}(1.37,-0.6)(3.17,1.99)  \psline{<->}(1.06,-0.6)(1.05,1.76)
\psdots[dotstyle=*](1.22,1.86)
 \rput[tl](0.88,2.2){$\mathrm{G}$} \rput[tl](0.73,0.6){$h$} \pscustom{\parametricplot{0.9714278016111888}{1.5707963267948966}{0.83*cos(t)+1.22|0.83*sin(t)+-0.6}\lineto(1.22,-0.6)\closepath} \rput[tl](2.36,0.54){$l$} \rput[tl](1.46,0.7){$\beta $} \rput[tl](1.15,-0.69){$O$} \begin{scriptsize} \psdots[dotsize=10pt 0,dotstyle=*](3.07,2.11) \end{scriptsize} \end{pspicture*}
\par\end{center}
A heavy symmetrical bell and clapper can both swung freely in a vertical
plane about a point $O$ on a horizontal beam at the apex of the bell.
The mass of the bell is $M$ and its moment of inertia about the beam
is $Mk^{2}$. Its centre of mass, $G$, is a distance $h$ from $O$.
The clapper may be regarded as a small heavy ball on a light rod of
length $l$. Initially the bell is held with its axis vertical and
its mouth above the beam. The clapper ball rests against the side
of the bell, with the rod making an angle $\beta$ with the axis.
The bell is then released. Show that, at the moment when the clapper
and bell separate, the clapper rod makes an angle $\alpha$ with the
upwards vertical, where 
\[
\cot\alpha=\cot\beta-\frac{k^{2}}{hl}\mathrm{cosec}\beta.
\]

\end{question}
	

	
	\newpage
\section*{Section C: \ \ \ Probability and Statistics}


%%%%%%%%%% Q12
\begin{question}
\begin{questionparts}
\item I toss a biased coin which has a probability
$p$ of landing heads and a probability $q=1-p$ of landing tails.
Let $K$ be the number of tosses required to obtain the first head
and let 
\[
\mathrm{G}(s)=\sum_{k=1}^{\infty}\mathrm{P}(K=k)s^{k}.
\]
Show that 
\[
\mathrm{G}(s)=\frac{ps}{1-qs}
\]
and hence find the expectation and variance of $K$. 
\item I sample cards at random with replacement from a normal
pack of $52$. Let $N$ be the total number of draws I make in order
to sample every card at least once. By expressing $N$ as a sum $N=N_{1}+N_{2}+\cdots+N_{52}$
of random variables, or otherwise, find the expectation of $N$. Estimate
the numerical value of this expectation, using the approximations
$\mathrm{e}\approx2.7$ and $1+\frac{1}{2}+\frac{1}{3}+\cdots+\frac{1}{n}\approx0.5+\ln n$
if $n$ is large.
\end{questionparts}
\end{question}

%%%%%%%%%% Q13
\begin{question}
Let $X$ and $Y$ be independent standard normal random
variables: the probability density function, $\f$, of each
is therefore given by 
\[
\f(x)=\left(2\pi\right)^{-\frac{1}{2}}\e^{-\frac{1}{2}x^{2}}.
\]


\begin{questionparts}
\item  Find the moment generating function $\mathrm{E}(\e^{\theta X})$
of $X$. 
\item  Find the moment generating function of $aX+bY$ and hence
obtain the condition on $a$ and $b$ which ensures that $aX+bY$
has the same distribution as $X$ and $Y$. 
\item  Let $Z=\e^{\mu+\sigma X}$. Show that 
\[
\mathrm{E}(Z^{\theta})=\e^{\mu\theta+\frac{1}{2}\sigma^{2}\theta^{2}},
\]
and hence find the expectation and variance of $Z$.
\end{questionparts}
\end{question}

%%%%%%%%%% Q14
\begin{question}
An industrial process produces rectangular plates of mean
length $\mu_{1}$ and mean breadth $\mu_{2}$. The length and breadth
vary independently with non-zero standard deviations $\sigma_{1}$
and $\sigma_{2}$ respectively. Find the means and standard deviations
of the perimeter and of the area of the plates. Show that the perimeter
and area are not independent.
\end{question}
	
\end{document}
