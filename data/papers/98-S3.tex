\documentclass[a4, 11pt]{report}


\pagestyle{myheadings}
\markboth{}{Paper III, 1998
\ \ \ \ \ 
\today 
}               

\RequirePackage{amssymb}
\RequirePackage{amsmath}
\RequirePackage{graphicx}
\RequirePackage{color}
\RequirePackage[flushleft]{paralist}[2013/06/09]



\RequirePackage{geometry}
\geometry{%
  a4paper,
  lmargin=2cm,
  rmargin=2.5cm,
  tmargin=3.5cm,
  bmargin=2.5cm,
  footskip=12pt,
  headheight=24pt}


\newcommand{\comment}[1]{{\bf Comment} {\it #1}}
%\renewcommand{\comment}[1]{}

\newcommand{\bluecomment}[1]{{\color{blue}#1}}
%\renewcommand{\comment}[1]{}
\newcommand{\redcomment}[1]{{\color{red}#1}}



\usepackage{epsfig}
\usepackage{pstricks-add}
\usepackage{tgheros} %% changes sans-serif font to TeX Gyre Heros (tex-gyre)
\renewcommand{\familydefault}{\sfdefault} %% changes font to sans-serif
%\usepackage{sfmath}  %%%% this makes equation sans-serif
%\input RexFigs


\setlength{\parskip}{10pt}
\setlength{\parindent}{0pt}

\newlength{\qspace}
\setlength{\qspace}{20pt}


\newcounter{qnumber}
\setcounter{qnumber}{0}

\newenvironment{question}%
 {\vspace{\qspace}
  \begin{enumerate}[\bfseries 1\quad][10]%
    \setcounter{enumi}{\value{qnumber}}%
    \item%
 }
{
  \end{enumerate}
  \filbreak
  \stepcounter{qnumber}
 }


\newenvironment{questionparts}[1][1]%
 {
  \begin{enumerate}[\bfseries (i)]%
    \setcounter{enumii}{#1}
    \addtocounter{enumii}{-1}
    \setlength{\itemsep}{5mm}
    \setlength{\parskip}{8pt}
 }
 {
  \end{enumerate}
 }



\DeclareMathOperator{\cosec}{cosec}
\DeclareMathOperator{\Var}{Var}

\def\d{{\rm d}}
\def\e{{\rm e}}
\def\g{{\rm g}}
\def\h{{\rm h}}
\def\f{{\rm f}}
\def\p{{\rm p}}
\def\s{{\rm s}}
\def\t{{\rm t}}


\def\A{{\rm A}}
\def\B{{\rm B}}
\def\E{{\rm E}}
\def\F{{\rm F}}
\def\G{{\rm G}}
\def\H{{\rm H}}
\def\P{{\rm P}}


\def\bb{\mathbf b}
\def \bc{\mathbf c}
\def\bx {\mathbf x}
\def\bn {\mathbf n}

\newcommand{\low}{^{\vphantom{()}}}
%%%%% to lower suffices: $X\low_1$ etc


\newcommand{\subone}{ {\vphantom{\dot A}1}}
\newcommand{\subtwo}{ {\vphantom{\dot A}2}}




\def\le{\leqslant}
\def\ge{\geqslant}


\def\var{{\rm Var}\,}

\newcommand{\ds}{\displaystyle}
\newcommand{\ts}{\textstyle}




\begin{document}
\setcounter{page}{2}

 
\section*{Section A: \ \ \ Pure Mathematics}

%%%%%%%%%%Q1
\begin{question}
Let
$$
{\rm f}(x)=\sin^2x + 2 \cos x + 1
$$
for $0 \le x \le 2\pi$. Sketch the curve $y={\rm f}(x)$, giving
the coordinates of the stationary points. Now let
$$
\hspace{0.6in}{\rm g}(x)={a{\rm f}(x)+b \over c{\rm f}(x)+d} 
\hspace{0.8in} 
ad\neq bc\,,\; d\neq -3c\,,\;
d\neq c\;.
$$
Show that the stationary points of $y={\rm g}(x)$ occur at the same
values of $x$ as those of $y={\rm f}(x)$, and find the corresponding 
values of ${\rm g}(x)$.

Explain why, if $d/c <-3$ or $d/c>1$, $|{\rm g}(x)|$ 
cannot be arbitrarily large.
\end{question}

%%%%%%%%%%Q2
\begin{question}
Let 
$$
{\rm I}(a,b) = \int_0^1 t^{a}(1-t)^{b} \, \d t \;
\qquad (a\ge0,\ b\ge0) .$$ 


\begin{questionparts}
\item Show that  ${\rm I}(a,b)={\rm I}(b,a)$,
\item Show that ${\rm I}(a,b)={\rm I}(a+1,b)+{\rm I}(a,b+1)$.
\item Show that $(a+1){\rm I}(a,b)=b{\rm I}(a+1,b-1)$ 
when $a$ and $b$ are positive  
and hence calculate ${\rm I}(a,b)$ when $a$ and $b$ are positive integers.
\end{questionparts}
\end{question}

%%%%%%%%% Q3
\begin{question}
The value $V_N$ of a bond after $N$ days is determined by the equation
$$
V_{N+1} = (1+c) V_{N} -d \qquad (c>0, \ d>0),
$$
where $c$ and $d$ are given  constants. 
By looking for
solutions of the form $V_T= A k^T + B$ for some constants $A,B$ and $k$,
or otherwise, find $V_N$ in terms of $V_0$. 

What is the solution for $c=0$? Show that this is the limit 
(for fixed $N$) as $c\rightarrow 0$ of your solution for $c>0$.
\end{question}

%%%%%% Q4 
\begin{question}
Show that the equation (in plane polar
coordinates) $r=\cos\theta$, for $-\frac{1}{2}\pi \le \theta \le
\frac{1}{2}\pi$, represents  a circle.

Sketch the curve $r=\cos2\theta$ for $0\le\theta\le 2\pi$, 
and describe the curves
$r=\cos2n\theta$, where $n$ is an integer. Show that the area
enclosed by such a curve is independent of $n$.

Sketch also the curve $r=\cos3\theta$ for $0\le\theta\le 2\pi$.
	\end{question}

%%%%%%%%% Q5
\begin{question}
The exponential of a square matrix ${\bf A}$ is defined to be
$$
\exp ({\bf A}) = \sum_{r=0}^\infty {1\over r!} {\bf A}^r \,,
$$
where ${\bf A}^0={\bf I}$  and  $\bf I$ is the identity matrix. 

Let 
$$
{\bf M}=\left(\begin{array}{cc} 0 & -1 \\ 1 & \phantom{-} 0
\end{array}
\right) \,.
$$
Show that ${\bf M}^2=-{\bf I}$ and hence
 express $\exp({\theta {\bf M}})$ as a single  $2\times 2$ matrix,
where $\theta$ is a real number.
Explain the geometrical significance of $\exp({\theta {\bf M}})$.

Let
 $$
{\bf N}=\left(\begin{array}{rr} 0 & 1 \\ 0 & 0 \end{array}\right) \,.
$$
Express
similarly  $\exp({s{\bf N}})$, where $s$ is  a real number, and
 explain the geometrical significance of $\exp({s{\bf N}})$.

For which values of $\theta$ does
$$
\exp({s{\bf N}})\; \exp({\theta {\bf M}})\,  = \,
\exp({\theta {\bf M}})\;\exp({s{\bf N}})
$$
for all $s$?
Interpret this fact geometrically.
	\end{question}
	
%%%%%%%%% Q6
\begin{question}
\begin{questionparts}
\item
Show that four vertices of a cube, no two of which are adjacent,
form the vertices of a regular tetrahedron. 
Hence, or otherwise, find the volume of a regular 
tetrahedron whose edges are of unit length.
\item
Find the volume of a regular octahedron whose edges are of unit length.
\item
Show that the centres of the faces of a cube form the vertices
of a regular octahedron. Show that its volume is half that of the 
tetrahedron whose vertices are the vertices of the cube.
\end{questionparts}

\noindent
[{\em A regular tetrahedron (octahedron) 
has four (eight) faces, all equilateral triangles.}]
\end{question}
	
%%%%%%%%% Q7
\begin{question}
Sketch the graph of ${\rm f}(s)={ \e}^s(s-3)+3$ for $0\le s<\infty$. Taking
${\e\approx 2.7}$, find the smallest positive integer, $m$, such that
${\rm f}(m) >0$.

Now let
$$
{\rm b}(x) = {x^3 \over \e^{x/T} -1} \,
$$
where $T$ is a positive constant. 
Show that ${\rm b}(x)$ has a single turning
point in $0 <x<\infty$. By considering the behaviour for small $x$
and for large $x$, sketch ${\rm b}(x)$ for $0\le x < \infty$.

Let 
$$
 \int_0^\infty {\rm b}(x)\,\d x =B,
$$
which may be assumed to be finite.
Show that $B = K  T^n$ where $K$ is a constant,  and $n$ is an
integer which you should determine.

Given that $\displaystyle{B \approx 2 \int_0^{Tm} {\rm b}(x) {\,\rm d }x}$,   
use your graph of ${\rm b}(x)$ to  find a rough estimate for $K$.
\end{question}
		
%%%%%%%%% Q8
\begin{question}	
\begin{questionparts}
%\item[(i)] Consider the sphere of radius $a$ and centre the origin.
%Show that the line through the point with position vector
%${\bf b}$ and parallel to a unit 
%vector ${\bf m}$ intersects the sphere at two points if
%$$
%a^2 > {\bf b}.{\bf b} -({\bf b}.{\bf m})^2 \,.
%$$
%What is the corresponding condition for there to be precisely one
%point of intersection?
%If this point has position vector ${\bf p}$, show that the line
%is perpendicular to ${\bf p}$.

\item Show that the line ${\bf r} ={\bf b} + \lambda {\bf m}$,
where $\bf m$ is a unit vector,
intersects the sphere ${\bf r}\cdot {\bf r} = a^2$ at two points if
$$
a^2 > {\bf b}\cdot{\bf b} -({\bf b}\cdot{\bf m})^2 \,.
$$
Write down  the corresponding condition for there to be precisely one
point of intersection.
If this point has position vector ${\bf p}$, show that ${\bf m}\cdot{\bf p}=0$.

\item
Now consider a second sphere of radius $a$
and a plane perpendicular to a unit vector~${\bf n}$.
The centre of the sphere 
has position vector ${\bf d}$
and the 
minimum distance from the origin to the plane is $l$. What is the
condition for the plane to be tangential to this second sphere?

\item
Show that the first and second spheres intersect at right angles 
({\em i.e.\ }the two radii to each point of 
intersection are perpendicular) if
$$
{\bf d}\cdot{\bf d} = 2 a^2 \,.
$$
\end{questionparts}
\end{question}	
		

		
	
\newpage
\section*{Section B: \ \ \ Mechanics}


	
%%%%%%%%%% Q9
\begin{question}
A uniform right circular cone of mass $m$ has 
base of radius $a$ and perpendicular
height $h$ from base to apex. 
Show that its moment of inertia about its axis is ${3\over 10} ma^2$, 
and calculate its moment of inertia about an axis through
its apex parallel to its base.
\newline[{\em Any theorems used should be stated clearly.}]

The cone is now suspended from its apex and allowed
to perform small oscillations. Show that their
period is
$$
2\pi\sqrt{ 4h^2 + a^2\over 5gh} \,.
$$
\newline[{\em You may assume that the centre of mass of the cone
is a distance ${3\over 4}h$ from its apex.}]
	\end{question}
	
%%%%%%%%%% Q10
\begin{question}	
Two identical spherical balls, moving on a horizontal, smooth table, collide
in such a way that both momentum and kinetic energy are conserved. 
Let ${\bf v}_1$ and ${\bf v}_2$ be the
 velocities of the balls before the collision 
and let ${\bf v}'_1$ and ${\bf v}'_2$ be 
the velocities of the balls after the collision, where
  ${\bf v}_1$, ${\bf v}_2$, ${\bf v}'_1$ and ${\bf v}'_2$ are two-dimensional 
vectors. Write down 
the equations for conservation of momentum and kinetic energy in terms of 
these vectors. Hence show that their relative speed is also conserved.

Show that, if one ball is initially at rest but after the collision
both balls are moving, their final velocities are perpendicular.

Now suppose that one ball is initially at rest, and the second is
moving with speed $V$. After a collision in which they lose
a proportion $k$ of their original kinetic energy ($0\le k\le 1$), 
the direction of motion of the second ball has changed by an angle 
$\theta$. Find a quadratic equation satisfied by the final speed of the 
second ball, with coefficients depending on $k$, $V$ and $\theta$. 
Hence show that $k\le \frac{1}{2}$.
\end{question}

%%%%%%%%%% Q11

\begin{question}
Consider a simple pendulum of length $l$ and angular displacement
$\theta$, which is {\bf not} assumed to be small. Show that
$$
{1\over 2}l \left({\d\theta\over \d t}\right)^2 = g(\cos\theta
-\cos\gamma)\,,
$$
where $\gamma$ is the maximum value of $\theta$. Show also that
the period $P$ is given by
$$
P= 2 \sqrt{l\over g} \int_0^\gamma \left(
\sin^2(\gamma/2)-\sin^2(\theta/2)
\right)^{-{1\over 2}} \,\d\theta \,.
$$
By using the substitution $\sin(\theta/2)=\sin(\gamma/2) \sin\phi$,
and then finding an approximate expression for the integrand using
the binomial expansion,
show that for small values of $\gamma$ the period is approximately
$$
2\pi \sqrt{l\over g} \left(1+{\gamma^2\over 16}\right) \,.
$$
\end{question}
	

	
	\newpage
\section*{Section C: \ \ \ Probability and Statistics}


%%%%%%%%%% Q12
\begin{question}
The mountain villages $A,B,C$ and $D$ lie at the vertices of a
tetrahedron, and each pair of villages is joined by a road. After
a snowfall the probability that any road is blocked is $p$, and 
is independent of the conditions of any other road. The 
probability that, after a snowfall,
 it is possible to travel from any village to
any other village by some route is $P$.  Show that 
$$
P =1- p^2(6p^3-12p^2+3p+4).
$$

%In the case $p={1\over 3}$ show that this probability is ${208 \over 243}$.

\end{question}

%%%%%%%%%% Q13
\begin{question}
Write down the probability of obtaining $k$ heads in $n$ tosses
of a fair coin. Now suppose that $k$ is known but $n$ is unknown.
A {\em maximum likelihood estimator} (MLE) of $n$ is defined
to be a value (which must be an integer) of $n$ which maximizes 
the probability of $k$ heads.

A friend has thrown a fair coin a number of times. She
tells you that she has observed one head. Show that in this case there
are {\em two} MLEs of the number of tosses she has made. 

She now tells you that in a repeat of the exercise she has 
observed $k$ heads.  Find the two MLEs of the number of tosses 
she has made. 

She next uses a coin biased with probability $p$ (known) of showing 
a head, and again tells you that she has observed $k$ heads.
Find the MLEs of the number of tosses made. What is the condition for 
the MLE to be unique?
\end{question}

%%%%%%%%%% Q14
\begin{question}
A hostile naval power possesses a large, unknown number $N$ of 
submarines. Interception of radio signals yields a small number $n$ 
of their identification numbers $X_i$ ($i=1,2,...,n$), which are taken
to be independent and uniformly distributed over the continuous range
from $0$ to $N$. Show that $Z_1$ and $Z_2$, defined by
$$
Z_1 = {n+1\over n} {\max}\{X_1,X_2,...,X_n\} 
\hspace{0.3in} {\rm and} \hspace{0.3in}
Z_2 = {2\over n} \sum_{i=1}^n X_i \;,
$$
both have means equal to $N$.

Calculate the variance of $Z_1$ and of $Z_2$. Which estimator
do you prefer, and why?
\end{question}
	
\end{document}
