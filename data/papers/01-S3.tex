\documentclass[a4, 11pt]{report}


\pagestyle{myheadings}
\markboth{}{Paper III, 2001
\ \ \ \ \ 
\today 
}               

\RequirePackage{amssymb}
\RequirePackage{amsmath}
\RequirePackage{graphicx}
\RequirePackage{color}
\RequirePackage[flushleft]{paralist}[2013/06/09]



\RequirePackage{geometry}
\geometry{%
  a4paper,
  lmargin=2cm,
  rmargin=2.5cm,
  tmargin=3.5cm,
  bmargin=2.5cm,
  footskip=12pt,
  headheight=24pt}


\newcommand{\comment}[1]{{\bf Comment} {\it #1}}
%\renewcommand{\comment}[1]{}

\newcommand{\bluecomment}[1]{{\color{blue}#1}}
%\renewcommand{\comment}[1]{}
\newcommand{\redcomment}[1]{{\color{red}#1}}



\usepackage{epsfig}
\usepackage{pstricks-add}
\usepackage{tgheros} %% changes sans-serif font to TeX Gyre Heros (tex-gyre)
\renewcommand{\familydefault}{\sfdefault} %% changes font to sans-serif
%\usepackage{sfmath}  %%%% this makes equation sans-serif
%\input RexFigs


\setlength{\parskip}{10pt}
\setlength{\parindent}{0pt}

\newlength{\qspace}
\setlength{\qspace}{20pt}


\newcounter{qnumber}
\setcounter{qnumber}{0}

\newenvironment{question}%
 {\vspace{\qspace}
  \begin{enumerate}[\bfseries 1\quad][10]%
    \setcounter{enumi}{\value{qnumber}}%
    \item%
 }
{
  \end{enumerate}
  \filbreak
  \stepcounter{qnumber}
 }


\newenvironment{questionparts}[1][1]%
 {
  \begin{enumerate}[\bfseries (i)]%
    \setcounter{enumii}{#1}
    \addtocounter{enumii}{-1}
    \setlength{\itemsep}{5mm}
    \setlength{\parskip}{8pt}
 }
 {
  \end{enumerate}
 }



\DeclareMathOperator{\cosec}{cosec}
\DeclareMathOperator{\Var}{Var}

\def\d{{\mathrm d}}
\def\e{{\mathrm e}}
\def\g{{\mathrm g}}
\def\h{{\mathrm h}}
\def\f{{\mathrm f}}
\def\p{{\mathrm p}}
\def\s{{\mathrm s}}
\def\t{{\mathrm t}}


\def\A{{\mathrm A}}
\def\B{{\mathrm B}}
\def\E{{\mathrm E}}
\def\F{{\mathrm F}}
\def\G{{\mathrm G}}
\def\H{{\mathrm H}}
\def\P{{\mathrm P}}


\def\bb{\mathbf b}
\def \bc{\mathbf c}
\def\bx {\mathbf x}
\def\bn {\mathbf n}

\newcommand{\low}{^{\vphantom{()}}}
%%%%% to lower suffices: $X\low_1$ etc


\newcommand{\subone}{ {\vphantom{\dot A}1}}
\newcommand{\subtwo}{ {\vphantom{\dot A}2}}




\def\le{\leqslant}
\def\ge{\geqslant}


\def\var{{\rm Var}\,}

\newcommand{\ds}{\displaystyle}
\newcommand{\ts}{\textstyle}
\def\half{{\textstyle \frac12}}
\def\l{\left(}
\def\r{\right)}



\begin{document}
\setcounter{page}{2}

 
\section*{Section A: \ \ \ Pure Mathematics}

%%%%%%%%%%Q1
\begin{question}
Given that $y = \ln ( x + \sqrt{x^2 + 1})$, show that 
$ \displaystyle \frac{\d y}{\d x} = \frac1 {\sqrt{x^2 + 1} }\;$.

Prove by induction that, for $n \ge 0\,$,
\[
\l x^2 + 1 \r y^{\l n + 2 \r} + \l 2n + 1 \r x y^{\l n + 1 \r} + n^2
y^{\l n \r} = 0\;,
\]
where $\displaystyle y^{\l n \r} = {\mathrm{d}^n y \over \mathrm{d}
x^n}$ and $y^{(0)} =y\,$.

Using this result in the case $x = 0\,$, or otherwise, show that the
 Maclaurin series for $y$ begins
\[
x - {x^3 \over 6} +{3 x^5 \over 40}
\]
and find the next non-zero term.
\end{question}

%%%%%%%%%%Q2
\begin{question}
Show that 
$
\cosh^{-1} x = \ln ( x + \sqrt{x^2-1}) \;.
$

Show that the area of the region defined by the inequalities $\displaystyle y^2 \ge x^2-8$ 
and $\displaystyle x^2\ge 25y^2 -16 $ is~$\displaystyle  (72/5) \ln 2$.
\end{question}

%%%%%%%%% Q3
\begin{question}
Consider the equation
\[
x^2 - b x + c = 0 \;,
\]
where  $b$ and $c$ are real numbers.

\begin{questionparts}
\item Show that the roots of the equation
are  real and positive if and only if $b>0$ and \phantom{} $b^2\ge4c>0$, and sketch the region of the 
$b\,$-$c$ plane in which these conditions hold.

\item  Sketch the region of 
the $b\,$-$c$ plane in which the roots of the equation are real and less
than $1$ in magnitude.
\end{questionparts}
\end{question}

%%%%%% Q4 
\begin{question}
In this question, the  function $\sin^{-1}$ is defined to have domain $ -1\le x \le 1$ 
and range \linebreak $ - \frac{1}{2}\pi \le x \le \frac{1}{2}\pi$ and the function $\tan^{-1}$ is defined to have the real numbers as its
domain and range $ - \frac{1}{2}\pi <x < \frac{1}{2}\pi$.

\begin{questionparts}
\item
Let 
$$
\g(x) = \displaystyle {2x \over 1 + x^2}\;,  \ \ \ \ \ \ \ \ \ \ -\infty <x<\infty\;.
$$
Sketch the graph of $\g(x)$ and state the range of $\g$.
\item
Let 
\[
\displaystyle \f \l x \r = \sin^{-1} \l {2x \over 1 + x^2} \r \;,\ 
\ \ \ \ \ \ \ \ -\infty<x<\infty\;.
\] 
Show that
$
\f(x ) =  
2 \tan^{-1} x$   for $ -1 \le x \le 1\,$ and $\f(x) =
\pi - 2 \tan^{-1} x $ for $x\ge1\,$. 

Sketch the graph of $\f(x)$.
\end{questionparts}
\end{question}

%%%%%%%%% Q5
\begin{question}
Show that the equation $x^3 + px + q=0$ has exactly one real
solution if  $p \ge 0\,$.

A parabola $C$ is given parametrically by
\[
x = at^2, \: \ \ y = 2at \: \: \: \ \ \  \ \ \  \l a > 0 \r \;.
\]

Find an equation which must be satisfied by $t$ at points on $C$ at
which the normal passes through the point $\l h , \; k \r\,$. Hence
show that, if $h \le 2a \,$, exactly one normal to $C$ will pass through
 $\l h , \; k \r \, $.


Find, in Cartesian form, the equation of the locus of the  points
from which exactly two normals can be drawn to $C\,$. Sketch the locus.
	\end{question}
	
%%%%%%%%% Q6
\begin{question}
The plane
\[
{x \over a} + {y \over b} +{z \over c} = 1
\]
meets the co-ordinate axes at the points $A$, $B$ and $C\,$. The point $M$ has
 coordinates $\left( \frac12 a, \frac12 b, \frac 12 c \right)$
%$\displaystyle \l {a \over 2} \, , \; {b \over 2} \, , \; {c \over 2}\r$
and $O$ is the origin.

Show that $OM$ meets the plane at the centroid  
$\left( \frac13 a, \frac13 b, \frac 13 c \right)$
%$\displaystyle \l {a \over 3} \, , \; {b \over 3} \, , \; {c \over 3}\r$
of triangle~$ABC$.
Show also that the perpendiculars to the plane from $O$ and from $M$ meet the
plane at the orthocentre and at the circumcentre of triangle $ABC$ respectively.


Hence prove that the centroid of a triangle lies on the line segment
joining its orthocentre and circumcentre, and that it divides this
line segment in the ratio $2 : 1\,$.

\noindent[The {\sl orthocentre} of a triangle is the point at which
the three altitudes intersect; the {\sl circumcentre}
of a triangle is the point equidistant from the three vertices.]

\end{question}
	
%%%%%%%%% Q7
\begin{question}
Sketch the graph of the function $\ln x - {1 \over 2} x^2$.

Show that the differential equation 
\[
{\mathrm{d} y \over \mathrm{d} x} = {2xy \over x^2 - 1}
\]
describes a family of parabolas  each of which
passes  through the points $\l 1 \, , \; 0 \r$
and $\l -1 \, , \; 0 \r$ and has its vertex on the $y$--axis.

Hence find the equation of the curve that passes through the point
$\l 1 \, , \, 1 \r$ and intersects   each of the above parabolas orthogonally.
Sketch this curve.

\noindent[Two curves intersect {\sl orthogonally} if their tangents at the point of 
intersection are perpendicular.]
\end{question}
		
%%%%%%%%% Q8
\begin{question}	
\begin{questionparts}
\item Prove that the equations
$$
\left|z - (1 + \mathrm{i}) \right|^2 = 2
\eqno(*)
$$
and
$$
\qquad \quad \ \left|z - (1 - \mathrm{i}) \right|^2 = 2 \left|z - 1 \right|^2 
$$
describe the same locus in the complex $z$--plane. Sketch this
locus.

\item Prove that the equation
$$
\arg \l {z - 2 \over z} \r = {\pi \over 4} \eqno(**)
$$
describes part of this same locus, and show on your sketch which part.

\item The complex number $w$ is related to $z$ by
\[
w = {2 \over z}\;.
\]
Determine the locus produced in the complex $w$--plane if $z$
satisfies $(*)$. Sketch this locus and
indicate the part of this locus that corresponds to $(**)$.

\end{questionparts}
\end{question}	
		

		
	
\newpage
\section*{Section B: \ \ \ Mechanics}


	
%%%%%%%%%% Q9
\begin{question}
$B_1$ and $B_2$ are  parallel, thin, horizontal fixed beams.
$B_1$ is a vertical distance $d \sin \alpha $ above $B_2$, and a horizontal distance
$d\cos\alpha $ from $B_2\,$,
where $0<\alpha<\pi/2\,$. A long heavy
plank  is held so that it rests on the two beams, perpendicular to each, 
with its centre of gravity at  $B_1\,$.
The coefficients of friction between
the plank and $B_1$ and $B_2$ are $\mu_1$ and $\mu_2\,$, respectively, where $\mu_1<\mu_2$
and $\mu_1+\mu_2=2\tan\alpha\,$.

The plank is released and slips over the beams  
experiencing a force of resistance from each beam equal to the limiting
frictional force (i.e. the product of the appropriate coefficient of friction and the 
normal reaction). Show that it will come to rest with its centre of gravity over $B_2$
in a time 
\[
\pi \left(\frac{d}{g(\mu_2-\mu_1)\cos\alpha }\right)^{\!\frac12}\;.
\]
	\end{question}
	
%%%%%%%%%% Q10
\begin{question}	
Three ships $A$,  $B$ and $C$ move with velocities ${\bf v}_1$,
${\bf v}_2$ and $\bf u$ respectively. The velocities of $A$ and $B$ 
relative to $C$ are equal in magnitude and
perpendicular. Write down conditions that $\bf u$, ${\bf v}_1$ and ${\bf v}_2$ must satisfy
and show that
\[
\left| {\bf u} -{\textstyle\frac12} \l {\bf v}_1 + {\bf v}_2 \r \right|^2 =
\left|{\textstyle\frac12}  \l {\bf v}_1 - {\bf v}_2 \r \right|^2
\]
and
\[
\l {\bf u} -{\textstyle\frac12} \l {\bf v}_1 + {\bf v}_2 \r \r \cdot \l {\bf v}_1 -
{\bf v}_2 \r = 0 \;.
\]
Explain why these equations determine, for given ${\bf v}_1$ and ${\bf v}_2$, 
two possible velocities for  $C\,$, provided ${\bf v}_1 \ne {\bf v}_2 \,$.

If ${\bf v}_1$ and ${\bf v}_2$ are equal in magnitude and perpendicular,
show that if ${\bf u} \ne {\bf 0}$ then ${\bf u} = {\bf v}_1 + {\bf v}_2\,$.
\end{question}

%%%%%%%%%% Q11

\begin{question}
A uniform cylinder of radius $a$ rotates freely about its axis,
which is fixed and horizontal. The moment of inertia of the cylinder about 
its axis is $I\,$. A light string is wrapped around the
cylinder and supports a mass $m$ which hangs freely. A particle of mass
$M$ is fixed to the surface of the cylinder. The system is held at
rest with the particle  vertically below the axis of the cylinder, and
then released. Find, in terms of $I$, $a$, $M$, $m$,  $g$ and $\theta$, 
the angular velocity of
the cylinder when it has rotated through angle $\theta\,$.


Show that the cylinder will 
 rotate without coming to a halt  if
$m/M>\sin\alpha\,$, where $\alpha$ satisifes $\alpha=\tan \frac12\alpha$ and 
$0<\alpha<\pi\,$. 
\end{question}
	

	
	\newpage
\section*{Section C: \ \ \ Probability and Statistics}


%%%%%%%%%% Q12
\begin{question}
A bag contains $b$ black balls and $w$ white balls. Balls are drawn
at random from the bag and when a white ball is drawn it is put aside.
\begin{questionparts}
\item If the black balls drawn are also put aside, find an
expression for the expected number of black balls that have been drawn
when the last white ball is removed.
\item  If instead the black balls drawn are put back  into the bag,
prove that the expected number of times a black ball has been drawn when
the first  white ball is removed is $b/w\,$. Hence write down, in the form of
a sum,  an expression for the expected number of times a black ball has been drawn when
the last  white ball is removed.

\end{questionparts}
\end{question}

%%%%%%%%%% Q13
\begin{question}
In a  game for two players,  a fair coin is tossed repeatedly. Each player 
is assigned a sequence of heads and tails and the player whose sequence appears first wins.
Four players, $A$, $B$, $C$ and $D$ take turns to play the game. 
Each time they play, $A$ is assigned the sequence
TTH (i.e.~Tail then Tail then Head),  $B$ is  assigned  THH,
$C$ is assigned HHT and $D$ is assigned~HTT.

\begin{questionparts}
\item
$A$ and $B$ play the   game. 
Let 
$p_{\mathstrut\mbox{\tiny HH}}$, 
$p_{\mathstrut\mbox{\tiny HT}}$, 
$p_{\mathstrut\mbox{\tiny TH}}$ and 
$p_{\mathstrut\mbox{\tiny TT}}$ 
be the probabilities of $A$ winning the
game  given that the first two tosses of the coin show HH,  HT, TH and TT, respectively.
Explain why 
$p_{\mathstrut\mbox{\tiny TT}} = 1\,$, 
and why 
$p_{\mathstrut\mbox{\tiny HT}} = {1 \over 2} \,
p_{\mathstrut\mbox{\tiny TH}} + 
{1\over 2} \,
p_{\mathstrut\mbox{\tiny TT}}\,$.
Show that 
$p_{\mathstrut\mbox{\tiny HH}} = 
p_{\mathstrut\mbox{\tiny HT}} 
= {2 \over 3}$ 
and that 
$p_{\mathstrut\mbox{\tiny TH}} = {1\over 3}\,$. 
Deduce that the probability that A wins the game is ${2\over 3}\,$.

\item $B$ and $C$ play the game.
Find the probability that  $B$ wins.

\item Show that if $C$ plays $D$, then  $C$ is more likely to win than $D$, 
but that if $D$ plays $A$, then $D$ is more likely to win than $A$.

\end{questionparts}
\end{question}

%%%%%%%%%% Q14
\begin{question}
A random variable $X$ is distributed uniformly on $[\, 0\, , \, a\,]$. 
Show that the variance of $X$ is\phantom{ }${1 \over 12} a^2$.

A sample, $X_1$ and $X_2$, of two independent values of the random
variable is drawn, and the variance $V$ of the sample is determined. Show that
$V = {1 \over 4} \l X_1 -X_2 \r ^2$, and hence prove that $2 V$ 
is an unbiased estimator of the variance of X.

Find an exact expression for the probability that the value of $V$ is less
than ${1 \over 12} a^2$ and estimate the value of this probability
correct to one significant figure.
\end{question}
	
\end{document}
