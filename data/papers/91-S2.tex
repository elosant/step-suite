\documentclass[a4, 11pt]{report}


\pagestyle{myheadings}
\markboth{}{Paper II, 1991
\ \ \ \ \ 
\today 
}               

\RequirePackage{amssymb}
\RequirePackage{amsmath}
\RequirePackage{graphicx}
\RequirePackage{color}
\RequirePackage[flushleft]{paralist}[2013/06/09]



\RequirePackage{geometry}
\geometry{%
  a4paper,
  lmargin=2cm,
  rmargin=2.5cm,
  tmargin=3.5cm,
  bmargin=2.5cm,
  footskip=12pt,
  headheight=24pt}


\newcommand{\comment}[1]{{\bf Comment} {\it #1}}
%\renewcommand{\comment}[1]{}

\newcommand{\bluecomment}[1]{{\color{blue}#1}}
%\renewcommand{\comment}[1]{}
\newcommand{\redcomment}[1]{{\color{red}#1}}



\usepackage{epsfig}
\usepackage{pstricks-add}
\usepackage{tgheros} %% changes sans-serif font to TeX Gyre Heros (tex-gyre)
\renewcommand{\familydefault}{\sfdefault} %% changes font to sans-serif
%\usepackage{sfmath}  %%%% this makes equation sans-serif
%\input RexFigs


\setlength{\parskip}{10pt}
\setlength{\parindent}{0pt}

\newlength{\qspace}
\setlength{\qspace}{20pt}


\newcounter{qnumber}
\setcounter{qnumber}{0}

\newenvironment{question}%
 {\vspace{\qspace}
  \begin{enumerate}[\bfseries 1\quad][10]%
    \setcounter{enumi}{\value{qnumber}}%
    \item%
 }
{
  \end{enumerate}
  \filbreak
  \stepcounter{qnumber}
 }


\newenvironment{questionparts}[1][1]%
 {
  \begin{enumerate}[\bfseries (i)]%
    \setcounter{enumii}{#1}
    \addtocounter{enumii}{-1}
    \setlength{\itemsep}{5mm}
    \setlength{\parskip}{8pt}
 }
 {
  \end{enumerate}
 }



\DeclareMathOperator{\cosec}{cosec}
\DeclareMathOperator{\Var}{Var}

\def\d{{\rm d}}
\def\e{{\rm e}}
\def\g{{\rm g}}
\def\h{{\rm h}}
\def\f{{\rm f}}
\def\p{{\rm p}}
\def\s{{\rm s}}
\def\t{{\rm t}}


\def\A{{\rm A}}
\def\B{{\rm B}}
\def\E{{\rm E}}
\def\F{{\rm F}}
\def\G{{\rm G}}
\def\H{{\rm H}}
\def\P{{\rm P}}


\def\bb{\mathbf b}
\def \bc{\mathbf c}
\def\bx {\mathbf x}
\def\bn {\mathbf n}

\newcommand{\low}{^{\vphantom{()}}}
%%%%% to lower suffices: $X\low_1$ etc


\newcommand{\subone}{ {\vphantom{\dot A}1}}
\newcommand{\subtwo}{ {\vphantom{\dot A}2}}




\def\le{\leqslant}
\def\ge{\geqslant}


\def\var{{\rm Var}\,}

\newcommand{\ds}{\displaystyle}
\newcommand{\ts}{\textstyle}




\begin{document}
\setcounter{page}{2}

 
\section*{Section A: \ \ \ Pure Mathematics}

%%%%%%%%%%Q1
\begin{question}
Let $\mathrm{h}(x)=ax^{2}+bx+c,$ where $a,b$ and $c$ are constants,
and $a\neq0$. Give a condition which $a,b$ and $c$ must satisfy
in order that $\mathrm{h}(x)$ can be written in the form 
\[
a(x+k)^{2},\tag{\ensuremath{*}}
\]
where $k$ is a constant. 


If $\mathrm{f}(x)=3x^{2}+4x$ and $\mathrm{g}(x)=x^{2}-2$, find the
two constant values of $\lambda$ such that $\mathrm{f}(x)+\lambda\mathrm{g}(x)$
can be written in the form $(*)$. Hence, or otherwise, find constants
$A,B,C,D,m$ and $n$ such that 
\begin{alignat*}{1}
\mathrm{f}(x) & =A(x+m)^{2}+B(x+n)^{2}\\
\mathrm{g}(x) & =C(x+m)^{2}+D(x+n)^{2}.
\end{alignat*}
If $\mathrm{f}(x)=3x^{2}+4x$ and $\mathrm{g}(x)=x^{2}+\alpha$ and
it is given by that there is only one value of $\lambda$ for which
$\mathrm{f}(x)+\lambda\mathrm{g}(x)$ can be written in the form $(*)$,
find $\alpha$. 
\end{question}

%%%%%%%%%%Q2
\begin{question}
	The equation of a hyperbola (with respect to axes which are displaced
	and rotated with respect to the standard axes) is 
	\[
	3y^{2}-10xy+3x^{2}+16y-16x+15=0.\tag{\ensuremath{\dagger}}
	\]
	By differentiating $(\dagger)$, or otherwise, show that the equation
	of the tangent through the point $(s,t)$ on the curve is 
	\[
	y=\left(\frac{5t-3s+8}{3t-5s+8}\right)x-\left(\frac{8t-8s+15}{3t-5s+8}\right).
	\]
	Show that the equations of the asymptote (the limiting tangents as
	$s\rightarrow\infty$) are 
	\[
	y=3x-4\qquad\mbox{ and }\qquad3y=x-4.
	\]
	{[}\textbf{Hint}:\textbf{ }You will need to find a relationship between
	$s$ and $t$ which is valid in the limit as $s\rightarrow\infty.${]}


	Show that the angle between one asymptote and the $x$-axis is the
	same as the angle between the other asymptote and the $y$-axis. Deduce
	the slopes of the lines that bisect the angles between the asymptotes
	and find the equations of the axes of the hyperbola.
\end{question}

%%%%%%%%% Q3
\begin{question}
	It is given that $x,y$ and $z$ are distinct and non-zero, and that
	they satisfy 
	\[
	x+\frac{1}{y}=y+\frac{1}{z}=z+\frac{1}{x}.
	\]
	Show that $x^{2}y^{2}z^{2}=1$ and that the value of $x+\dfrac{1}{y}$
	is either $+1$ or $-1$.
\end{question}

%%%%%% Q4 
\begin{question}
	Let $y=\cos\phi+\cos2\phi$, where $\phi=\dfrac{2\pi}{5}.$ Verify
	by direct substitution that $y$ satisfies the quadratic equation
	$2y^{2}=3y+2$ and deduce that the value of $y$ is $-\frac{1}{2}.$


	Let $\theta=\dfrac{2\pi}{17}.$ Show that 
	\[
	\sum_{k=0}^{16}\cos k\theta=0.
	\]
	If $z=\cos\theta+\cos2\theta+\cos4\theta+\cos8\theta,$ show that
	the value of $z$ is $-(1-\sqrt{17})/4$.
	\end{question}

%%%%%%%%% Q5
\begin{question}
	Give a rough sketch of the function $\tan^{k}\theta$ for $0\leqslant\theta\leqslant\frac{1}{4}\pi$
	in the two cases $k=1$ and $k\gg1$ (i.e. $k$ is much greater than
	1). 


	Show that for any positive integer $n$ 
	\[
	\int_{0}^{\frac{1}{4}\pi}\tan^{2n+1}\theta\,\mathrm{d}\theta=(-1)^{n}\left(\tfrac{1}{2}\ln2+\sum_{m=1}^{n}\frac{(-1)^{m}}{2m}\right),
	\]
	and deduce that 
	\[
	\sum_{m=1}^{\infty}\frac{(-1)^{m-1}}{2m}=\tfrac{1}{2}\ln2.
	\]
	Show similarly that 
	\[
	\sum_{m=1}^{\infty}\frac{(-1)^{m-1}}{2m-1}=\frac{\pi}{4}.
	\]
	\end{question}
	
	%%%%%%%%% Q6
	\begin{question}
	Show by means of a sketch, or otherwise, that if $0\leqslant\mathrm{f}(y)\leqslant\mathrm{g}(y)$
	for $0\leqslant y\leqslant x$ then 
	\[
	0\leqslant\int_{0}^{x}\mathrm{f}(y)\,\mathrm{d}y\leqslant\int_{0}^{x}\mathrm{g}(y)\,\mathrm{d}y.
	\]
	Starting from the inequality $0\leqslant\cos y\leqslant1,$ or otherwise,
	prove that if $0\leqslant x\leqslant\frac{1}{2}\pi$ then $0\leqslant\sin x\leqslant x$
	and $\cos x\geqslant1-\frac{1}{2}x^{2}.$ Deduce that 
	\[
	\frac{1}{1800}\leqslant\int_{0}^{\frac{1}{10}}\frac{x}{(2+\cos x)^{2}}\,\mathrm{d}x\leqslant\frac{1}{1797}.
	\]
	Show further that if $0\leqslant x\leqslant\frac{1}{2}\pi$ then $\sin x\geqslant x-\frac{1}{6}x^{3}.$
	Hence prove that 
	\[
	\frac{1}{3000}\leqslant\int_{0}^{\frac{1}{10}}\frac{x^{2}}{(1-x+\sin x)^{2}}\,\mathrm{d}x\leqslant\frac{2}{5999}.
	\]
	\end{question}
	
	%%%%%%%%% Q7
	\begin{question}
		The function $\mathrm{g}$ satisfies, for all positive $x$ and $y$,
		\[
		\mathrm{g}(x)+\mathrm{g}(y)=\mathrm{g}(z),\tag{\ensuremath{*}}
		\]
		where $z=xy/(x+y+1).$ By treating $y$ as a constant, show that 
		\[
		\mathrm{g}'(x)=\frac{y^{2}+y}{(x+y+1)^{2}}\mathrm{g}'(z)=\frac{z(z+1)}{x(x+1)}\mathrm{g}'(z),
		\]
		and deduce that $2\mathrm{g}'(1)=(u^{2}+u)\mathrm{g}'(u)$ for all
		$u$ satisfying $0<u<1.$ Now by treating $u$ as a variable, show
		that 
		\[
		\mathrm{g}(u)=A\ln\left(\frac{u}{u+1}\right)+B,
		\]
		where $A$ and $B$ are constants. Verify that $\mathrm{g}$ satisfies
		$(*)$ for a suitable value of $B$. Can $A$ be determined from $(*)$? 


		The function $\mathrm{f}$ satisfies, for all positive $x$ and $y$,
		\[
		\mathrm{f}(x)+\mathrm{f}(y)=\mathrm{f}(z)
		\]
		where $z=xy.$ Show that $\mathrm{f}(x)=C\ln x$ where $C$ is a constant.
		\end{question}
		
%%%%%%%%% Q8
	\begin{question}	
		Solve the quadratic equation $u^{2}+2u\sinh x-1=0$, giving $u$ in
		terms of $x$. 


		Find the solution of the differential equation 
		\[
		\left(\frac{\mathrm{d}y}{\mathrm{d}x}\right)^{2}+2\frac{\mathrm{d}y}{\mathrm{d}x}\sinh x-1=0
		\]
		which satisfies $y=0$ and $y'>0$ at $x=0$. 


		Find the solution of the differential equation 
		\[
		\sinh x\left(\frac{\mathrm{d}y}{\mathrm{d}x}\right)^{2}+2\frac{\mathrm{d}y}{\mathrm{d}x}-\sinh x=0
		\]
		which satisfies $y=0$ at $x=0$.
		\end{question}	
		
		%%%%%%%%%% Q9
		\begin{question}
			Let $G$ be the set of all matrices of the form 
			\[
			\begin{pmatrix}a & b\\
			0 & c
			\end{pmatrix},
			\]
			where $a,b$ and $c$ are integeres modulo 5, and $a\neq0\neq c$.
			Show that $G$ forms a group under matrix multiplication (which may
			be assumed to be associative). What is the order of $G$? Determine
			whether or not $G$ is commutative. 


			Determine whether or not the set consisting of all elements in $G$
			of order $1$ or $2$ is a subgroup of $G$.
			\end{question}
			
		
			
%%%%%%%%%% Q10
		\begin{question} A straight stick of length $h$ stands vertically. On a sunny day,
the stick casts a shadow on flat horizontal ground. In cartesian axes
based on the centre of the Earth, the position of the Sun may be taken
to be $R(\cos\theta,\sin\theta,0)$ where $\theta$ varies but $R$
is constant. The positions of the base and tip of the stick are $a(0,\cos\phi,\sin\phi)$
and $b(0,\cos\phi,\sin\phi)$, respectively, where $b-a=h$. Show
that the displacement vector from the base of the stick to the tip
of the shadow is 
\[
Rh(R\cos\phi\sin\theta-b)^{-1}\begin{pmatrix}-\cos\theta\\
-\sin^{2}\phi\sin\theta\\
\cos\phi\sin\phi\sin\theta
\end{pmatrix}.
\]
{[}`Stands vertically' means that the centre of the Earth, the base
of the stick and the tip of the stick are collinear, `horizontal'
means perpendicular to the stick. 
\end{question}
		
	
\newpage
\section*{Section B: \ \ \ Mechanics}


	
%%%%%%%%%% Q11
\begin{question}
	The Ruritanian army is supplied with shells which may explode at any
	time in flight but not before the shell reaches its maximum height.
	The effect of the explosion on any observer depends only on the distance
	between the exploding shell and the observer (and decreases with distance).
	Ruritanian guns fire the shells with fixed muzzle speed, and it is
	the policy of the gunners to fire the shell at an angle of elevation
	which minimises the possible damages to themselves (assuming the ground
	is level) - i.e. they aim so that the point on the descending trajectory
	that is nearest to them is as far away as possible. With that intention,
	they choose the angle of elevation that minimises the damage to themselves
	if the shell explodes at its maximum height. What angle do they choose? 


	Does the shell then get any nearer to the gunners during its descent?
	\end{question}
	
%%%%%%%%%% Q12
\begin{question}	
A particle is attached to one end $B$ of a light elastic string of
unstretched length $a$. Initially the other end $A$ is at rest and
the particle hangs at rest at a distance $a+c$ vertically below $A$.
At time $t=0$, the end $A$ is forced to oscillate vertically, its
downwards displacement at time $t$ being $b\sin pt$. Let $x(t)$
be the downwards displacement of the particle at time $t$ from its
initial equilibrium position. Show that, while the string remains
taut, $x(t)$ satisfies 
\[
\frac{\mathrm{d}^{2}x}{\mathrm{d}t^{2}}=-n^{2}(x-b\sin pt),
\]
where $n^{2}=g/c$, and that if $0<p<n$, $x(t)$ is given by 
\[
x(t)=\frac{bn}{n^{2}-p^{2}}(n\sin pt-p\sin nt).
\]
Write down a necessary and sufficient condition that the string remains
taut throughout the subsequent motion, and show that it is satisfied
if $pb<(n-p)c.$ 
\end{question}

%%%%%%%%%% Q13

\begin{question}
A non-uniform rod $AB$ of mass $m$ is pivoted at one end $A$ so
that it can swing freely in a vertical plane. Its centre of mass is
a distance $d$ from $A$ and its moment of inertia about any axis
perpendicular to the rod through $A$ is $mk^{2}.$ A small ring of
mass $\alpha m$ is free to slide along the rod and the coefficient
of friction between the ring and rod is $\mu.$ The rod is initially
held in a horizontal position with the ring a distance $x$ from $A$.
If $k^{2}>xd$, show that when the rod is released, the ring will
start to slide when the rod makes an angle $\theta$ with the downward
vertical, where 
\[
\mu\tan\theta=\frac{3\alpha x^{2}+k^{2}+2xd}{k^{2}-xd}.
\]
Explain what will happen if (i) $k^{2}=xd$ and (ii) $k^{2}<xd$.
\end{question}
	
%%%%%%%%%% Q14
\begin{question}
	The current in a straight river of constant width $h$ flows at uniform
	speed $\alpha v$ parallel to the river banks, where $0<\alpha<1$.
	A boat has to cross from a point $A$ on one bank to a point $B$
	on the other bank directly opposite to $A$. The boat moves at constant
	speed $v$ relative to the water. When the position of the boat is
	$(x,y)$, where $x$ is the perpendicular distance from the opposite
	bank and $y$ is the distance downstream from $AB$, the boat is pointing
	in a direction which makes an angle $\theta$ with $AB$. Determine
	the velocity vector of the boat in terms of $v,\theta$ and $\alpha.$


	The pilot of the boat steers in such a way that the boat always points
	exactly towards $B$. Show that the velocity vector of the boat is
	\[
	\begin{pmatrix}\dfrac{\mathrm{d}x}{\mathrm{d}t}\\
	\tan\theta\dfrac{\mathrm{d}x}{\mathrm{d}t}+x\sec^{2}\theta\dfrac{\mathrm{d}\theta}{\mathrm{d}t}
	\end{pmatrix}.
	\]
	By comparing this with your previous expression deduce that 
	\[
	\alpha\frac{\mathrm{d}x}{\mathrm{d}\theta}=-x\sec\theta
	\]
	and hence show that
	\[
	(x/h)^{\alpha}=(\sec\theta+\tan\theta)^{-1}.
	\]
	Let $s(t)$ be a new variable defined by $\tan\theta=\sinh(\alpha s).$
	Show that $x=h\mathrm{e}^{-s},$ and that 
	\[
	h\mathrm{e}^{-s}\cosh(\alpha s)\frac{\mathrm{d}s}{\mathrm{d}t}=v.
	\]
	Hence show that the time of crossing is $hv^{-1}(1-\alpha^{2})^{-1}.$
\end{question}
	
	\newpage
\section*{Section C: \ \ \ Probability and Statistics}


%%%%%%%%%% Q15
\begin{question}
Integers $n_{1},n_{2},\ldots,n_{r}$ (possibly the same) are chosen
independently at random from the integers $1,2,3,\ldots,m$. Show
that the probability that $\left|n_{1}-n_{2}\right|=k$, where $1\leqslant k\leqslant m-1$,
is $2(m-k)/m^{2}$ and show that the expectation of $\left|n_{1}-n_{2}\right|$
is $(m^{2}-1)/(3m)$. Verify, for the case $m=2$, the result that
the expection of $\left|n_{1}-n_{2}\right|+\left|n_{2}-n_{3}\right|$
is $2(m^{2}-1)/(3m).$ Write down the expectation, for general $m$,
of 
\[
\left|n_{1}-n_{2}\right|+\left|n_{2}-n_{3}\right|+\cdots+\left|n_{r-1}-n_{r}\right|.
\]
Desks in an examination hall are placed a distance $d$ apart in straight
lines. Each invigilator looks after one line of $m$ desks. When called
by a candidate, the invigilator walks to that candidate's desk, and
stays there until called again. He or she is equally likely to be
called by any of the $m$ candidates in the line but candidates never
call simultaneously or while the invigilator is attending to another
call. At the beginning of the examination the invigilator stands by
the first desk. Show that the expected distance walked by the invigilator
in dealing with $N+1$ calls is 
\[
\frac{d(m-1)}{6m}[2N(m+1)+3m].
\]
\end{question}

%%%%%%%%%% Q16
\begin{question}
	Each time it rains over the Cabbibo dam, a volume $V$ of water is
	deposited, almost instanetaneously, in the reservoir. Each day (midnight
	to midnight) water flows from the reservoir at a constant rate $u$
	units of volume per day. An engineer, if present, may choose to alter
	the value of $u$ at any midnight. 

	\begin{questionparts}
	\item Suppose that it rains at most once in any day, that there is a probability
	$p$ that it will rain on any given day and that, if it does, the
	rain is equally likely to fall at any time in the 24 hours (i.e. the
	time at which the rain falls is a random variable uniform on the interval
	$[0,24]$). The engineers decides to take two days' holiday starting
	at midnight. If at this time the volume of water in the reservoir
	is $V$ below the top of the dam, find an expression for $u$ such
	that the probability of overflow in the two days is $Q$, where $Q<p^{2}.$


	For the engineer's summer holidays, which last 18 days, the reservoir
	is drained to a volume $kV$ below the top of the dam and the rate
	of outflow $u$ is set to zero. The engineer wants to drain off as
	little as possible, consistent with the requirement that the probability
	that the dam will overflow is less than $\frac{1}{10}.$ In the case
	$p=\frac{1}{3},$ find by means of a suitable approximation the required
	value of $k$. 

	\item Suppose instead that it may rain at most once before noon and at most
	once after noon each day, that the probability of rain in any given
	half-day is $\frac{1}{6}$ and that it is equally likely to rain at
	any time in each half-day. Is the required value of $k$ lower or
	higher? \end{questionparts}
	\end{question}
\end{document}
