\documentclass[a4, 11pt]{report}


\pagestyle{myheadings}
\markboth{}{Paper II, 2010
\ \ \ \ \ 
\today 
}               

\RequirePackage{amssymb}
\RequirePackage{amsmath}
\RequirePackage{graphicx}
\RequirePackage{color}
\RequirePackage[flushleft]{paralist}[2013/06/09]



\RequirePackage{geometry}
\geometry{%
  a4paper,
  lmargin=2cm,
  rmargin=2.5cm,
  tmargin=3.5cm,
  bmargin=2.5cm,
  footskip=12pt,
  headheight=24pt}


\newcommand{\comment}[1]{{\bf Comment} {\it #1}}
%\renewcommand{\comment}[1]{}

\newcommand{\bluecomment}[1]{{\color{blue}#1}}
%\renewcommand{\comment}[1]{}
\newcommand{\redcomment}[1]{{\color{red}#1}}



\usepackage{epsfig}
\usepackage{pstricks-add}
\usepackage{tgheros} %% changes sans-serif font to TeX Gyre Heros (tex-gyre)
\renewcommand{\familydefault}{\sfdefault} %% changes font to sans-serif
%\usepackage{sfmath}  %%%% this makes equation sans-serif
%\input RexFigs


\setlength{\parskip}{10pt}
\setlength{\parindent}{0pt}

\newlength{\qspace}
\setlength{\qspace}{20pt}


\newcounter{qnumber}
\setcounter{qnumber}{0}

\newenvironment{question}%
 {\vspace{\qspace}
  \begin{enumerate}[\bfseries 1\quad][10]%
    \setcounter{enumi}{\value{qnumber}}%
    \item%
 }
{
  \end{enumerate}
  \filbreak
  \stepcounter{qnumber}
 }


\newenvironment{questionparts}[1][1]%
 {
  \begin{enumerate}[\bfseries (i)]%
    \setcounter{enumii}{#1}
    \addtocounter{enumii}{-1}
    \setlength{\itemsep}{5mm}
    \setlength{\parskip}{8pt}
 }
 {
  \end{enumerate}
 }



\DeclareMathOperator{\cosec}{cosec}
\DeclareMathOperator{\Var}{Var}

\def\d{{\mathrm d}}
\def\e{{\mathrm e}}
\def\g{{\mathrm g}}
\def\h{{\mathrm h}}
\def\f{{\mathrm f}}
\def\p{{\mathrm p}}
\def\s{{\mathrm s}}
\def\t{{\mathrm t}}


\def\A{{\mathrm A}}
\def\B{{\mathrm B}}
\def\E{{\mathrm E}}
\def\F{{\mathrm F}}
\def\G{{\mathrm G}}
\def\H{{\mathrm H}}
\def\P{{\mathrm P}}


\def\bb{\mathbf b}
\def \bc{\mathbf c}
\def\bx {\mathbf x}
\def\bn {\mathbf n}

\newcommand{\low}{^{\vphantom{()}}}
%%%%% to lower suffices: $X\low_1$ etc


\newcommand{\subone}{ {\vphantom{\dot A}1}}
\newcommand{\subtwo}{ {\vphantom{\dot A}2}}




\def\le{\leqslant}
\def\ge{\geqslant}
\def\arcosh{{\rm arcosh}\,}


\def\var{{\rm Var}\,}

\newcommand{\ds}{\displaystyle}
\newcommand{\ts}{\textstyle}
\def\half{{\textstyle \frac12}}
\def\l{\left(}
\def\r{\right)}



\begin{document}
\setcounter{page}{2}

 
\section*{Section A: \ \ \ Pure Mathematics}

%%%%%%%%%%Q1
\begin{question}
Let $P$ be a given point on a given curve $C$. The {\em osculating 
circle} to $C$ at $P$
is defined to be the circle that satisfies the following two
conditions at $P$: it touches $C$; and  the rate of change of its gradient 
 is equal to the rate of change of the 
gradient of $C$.

Find the centre and radius of the osculating circle to the curve $y=1-x+\tan x$
at the point on the curve with $x$-coordinate $\frac14 \pi$.
\end{question}

%%%%%%%%%%Q2
\begin{question}
Prove that 
\[
\cos 3x = 4 \cos^3 x - 3 \cos x \,.
\]
Find  and prove a similar result for $\sin 3x$ in terms of $\sin x$.

\begin{questionparts}
\item Let 
\[
{\rm I}(\alpha) = \int_0^\alpha \big(7\sin x - 8 \sin^3 x\big) \d x\,.
\]
Show that 
\[
{\rm I}(\alpha) = -\tfrac 8 3 c^3 + c +\tfrac5 3\,,
\]
where $c = \cos \alpha$.
Write down one value of $c$ for which ${\rm I}(\alpha) =0$.

\item Useless Eustace believes that 
\[
\int \sin^n x \, \d x =\dfrac {\sin^{n+1}x}{n+1}\,
\]
 for $n=1, \ 2, \ 3,  \ldots\, $. 
Show that 
Eustace would obtain the correct value of ${\rm I}(\beta)\,$, where
$\cos \beta= -\frac16$.

Find 
all values of $\alpha$ for which he
would obtain the correct value of ${\rm I}(\alpha)$.
\end{questionparts}

\end{question}

%%%%%%%%% Q3
\begin{question}
The first four terms of a  sequence 
are given by $F_0=0$, $F_1=1$, $F_2=1$ and $F_3=2$. The general term
is given by
\[
F_n= a\lambda^n+b\mu^n\,,
\tag{$*$}
\]
where $a$, $b$, $\lambda$ and $\mu$ are independent of $n$, and $a$ is
positive.

\begin{questionparts}
\item Show that 
$\lambda^2 +\lambda\mu+ \mu^2 = 2$, and find the values of
$\lambda$, $\mu$, $a$ and $b$.
\item Use $(*)$ to evaluate $F_6$.
\item Evaluate
$\ds \sum_{n=0}^\infty \frac{F_n}{2^{n+1}}\,.$
\end{questionparts}
\end{question}

%%%%%% Q4 
\begin{question}
\begin{questionparts} 
\item Let  
\[
I=\int_0^a \frac {\f(x)}{\f(x)+\f(a-x)} \, \d x\,.
\]
Use a substitution to show that 
\[
I =
\int_0^a \frac {\f(a-x)}{\f(x)+\f(a-x)} \, \d x\,
\]
and hence evaluate  $I$ in terms of $a$.

Use this result to evaluate the integrals
\[
\int_0^1 \frac{\ln (x+1)}{\ln (2+x-x^2)}\, \d x
\text{ \ \ \ \ \ \ and \ \ \ \ \ }
\int_0^{\frac\pi 2} \frac{\sin x } {\sin(x+\frac \pi 4 )} \, \d x
\,.
\]
\item
Evaluate
\[
\int_{\frac12}^2 \frac {\sin x}{x \big(\sin x + \sin \frac 1 x\big)} 
\, \d x\,.
\]
\end{questionparts}
\end{question}

%%%%%%%%% Q5
\begin{question}
The points $A$ and $B$ have position vectors
$\bf i +j+k$
and $5{\bf i} - {\bf j} -{\bf k}$, respectively, relative to the origin $O$.
Find $\cos2\alpha$, where $2\alpha$ is the angle $\angle AOB$.
 
\begin{questionparts}
\item The line $L   _1$ has equation 
${\bf r} =\lambda(m{\bf i}+n {\bf j} + p{\bf k})$.
Given that $L   _1$ is inclined equally to $OA$ and to $OB$,
determine a relationship between $m$, $n$ and~$p$.
Find 
also values of $m$, $n$ and~$p$ for which $L   _1$ is the 
angle bisector of $\angle AOB$.

\item   The line $L   _2$ has equation 
${\bf r} =\mu(u{\bf i}+v {\bf j} + w{\bf k})$.
Given that $  L _2$ is inclined at an angle $\alpha$ to $OA$,
where $2\alpha = \angle AOB$,  determine a relationship between
$u$, $v$ and $w$.

Hence describe the surface with Cartesian equation 
$x^2+y^2+z^2 =2(yz+zx+xy)$.
\end{questionparts}
	\end{question}
	
%%%%%%%%% Q6
\begin{question}
Each edge of the tetrahedron $ABCD$ has unit length. The face
$ABC$ is horizontal, and 
$P$ is the point in  $ABC$ that is vertically below $D$.
\begin{questionparts}
\item Find the length of $PD$.
\item Show that the cosine of the angle between adjacent faces of the 
tetrahedron is
$1/3$.
\item Find the radius of the largest sphere that can fit inside
the tetrahedron.
\end{questionparts}
\end{question}
	
%%%%%%%%% Q7
\begin{question}
\begin{questionparts}
\item 
By considering the positions of its turning points, show that  the curve
with equation
\[
y=x^3-3qx-q(1+q)\,,
\]
where $q>0$ and $q\ne1$, crosses the $x$-axis once only.

\item 
Given that $x$ satisfies the cubic equation
\[
x^3-3qx-q(1+q)=0\,,
\]
and that
\[
x=u+q/u\,,
\]
obtain  a quadratic equation
satisfied by $u^3$.
Hence find the real root of the cubic equation in the case $q>0$, $q\ne1$.

\item The quadratic equation 
\[
t^2 -pt +q =0\,
\]
 has roots $\alpha $ and $\beta$. Show that
\[
\alpha^3+\beta^3 = p^3 -3qp\,.
\]

It is given that  one of these roots is the square of the other.
By considering the expression $(\alpha^2 -\beta)(\beta^2-\alpha)$,
find a relationship between                    $p$ and $q$.
Given further that  $q>0$, $q\ne1$ and $p$ is real,
determine  the value of $p$ in terms of $q$.
\end{questionparts}  
\end{question}
		
%%%%%%%%% Q8
\begin{question}
The curves $C_1$ and $C_2$ are defined by 
\[
y= \e^{-x} \ \ \ (x>0)  \text{ \ \ \ and \ \ \ }
y= \e^{-x}\sin x \ \ \ (x>0),
\]
respectively. Sketch roughly $C_1$ and $C_2$ on the same diagram.

Let $x_n$ denote the $x$-coordinate of the $n$th point of contact
between the two curves, where $0<x_1<x_2< \cdots$, and let
$A_n$ denote the area of the region enclosed by the two 
curves between $x_n$ and $x_{n+1}$. Show that
\[
A_n = \tfrac12(\e^{2\pi}-1) \e^{-(4n+1)\pi/2}
\]
and hence find $\displaystyle \sum_{n=1}^\infty A_n$.
\end{question}	
		

		
	
\newpage
\section*{Section B: \ \ \ Mechanics}


	
%%%%%%%%%% Q9
\begin{question}
Two points $A$ and $B$ lie on  horizontal ground.
  A particle $P_1$ is projected from $A$ towards $B$
at an acute angle of elevation $\alpha$  
and simultaneously
a particle $P_2$ is projected from $B$ towards $A$ 
at an acute angle of elevation $\beta$.
Given that the  two particles collide in the air
a horizontal distance $b$ from $B$,
and that the collision occurs after $P_1$
has attained its maximum height $h$, show that
\[
2h \cot\beta < b < 4h \cot\beta \hphantom{\,,}
\]
and
\[
2h \cot\alpha < a < 4h \cot\alpha
\,,
\]
where $a$ is the horizontal distance from $A$
to the point of collision.
	\end{question}
	
%%%%%%%%%% Q10 
\begin{question}	
\begin{questionparts}
\item
In an experiment,
a particle $A$ of mass $m$ is at rest on a smooth horizontal table.
A~particle $B$ of mass $bm$, where $b>1$, is projected along the 
table directly towards $A$ with speed $u$. The collision is perfectly 
elastic. 


Find an expression for the speed of $A$ after the collision
  in terms of $b$ and $u$, and 
show that, irrespective of the relative masses of the particles, 
$A$ cannot be made to move at twice the initial speed of $B$. 

\item 
In a second experiment, a particle $B_1$ 
 is  projected along the table directly towards $A$ with speed $u$. This time, 
particles $B_2$, $B_3$, $\ldots\,$, $B_n$ are at rest in order 
on the line between $B_1$ and $A$.
The mass of $B_i$ ($i=1$, $2$, $\ldots\,$, $n$)
is $\lambda^{n+1-i}m$, where $\lambda>1$.
All collisions are perfectly elastic. Show that, by choosing $n$ sufficiently
large, there is no upper limit on the speed at which $A$ can be made to move.

In the case $\lambda=4$,
determine the least value of 
$n$ for which  $A$ moves at more than $20u$. You may use the 
approximation
 $\log_{10}2 \approx 0.30103$.

\end{questionparts}
\end{question}

%%%%%%%%%% Q11

\begin{question}
A uniform rod $AB$ of length $4L   $ and
weight $W$ is inclined at an angle $\theta$ to the 
horizontal. Its lower end $A$ rests on a fixed support and the rod is held
in equilibrium by a string 
attached to the rod at a point $C$ which is $3L   $ from $A$.
The reaction of the support on the rod acts in a direction $\alpha$ to 
$AC$  and the string is inclined at an angle $\beta$ to $CA$.
Show that
\[
\cot\alpha = 3\tan \theta + 2 \cot \beta\,.
\]

Given that 
$\theta =30^\circ$ and $\beta = 45^\circ$,
show that $\alpha= 15^\circ$.
\end{question}
	

	
	\newpage
\section*{Section C: \ \ \ Probability and Statistics}


%%%%%%%%%% Q12
\begin{question}
The continuous random variable $X$ has probability density 
function $\f(x)$, where
\[
\f(x) = 
\begin{cases}
a & \text {for } 0\le x < k \\
b & \text{for } k \le x \le 1\\
0 & \text{otherwise},
\end{cases}
\]
where $a>b>0$ and $0<k<1$. Show that $a>1$ and $b<1$. 
\begin{questionparts}
\item Show that 
\[
\E(X) = \frac{1-2b+ab}{2(a-b)}\,.
\]
\item
Show that the median, $M$, of $X$ is given by 
$\displaystyle M=\frac 1 {2a}$ if
$a+b\ge 2ab$ and obtain an expression for the median if
$a+b\le 2ab$.
\item Show that $M<   \E(X)\,$.
\end{questionparts}
\end{question}

%%%%%%%%%% Q13
\begin{question}
Rosalind wants to join the Stepney Chess Club. In order to 
be accepted, she must play a challenge match consisting of several
games against 
Pardeep (the Club champion)
and Quentin (the Club secretary), in which she must win at least one
game against each of Pardeep and Quentin.
From past experience, she knows that
the probability of her winning a single game against 
Pardeep is $p$ and the probability of her 
winning a single game against Quentin is $q$,
where $0<p<q<1$.

\begin{questionparts}
\item
The challenge match consists of three games.
Before the match begins, Rosalind must choose either to 
play Pardeep twice and Quentin once or to play Quentin twice and
Pardeep once.
Show that she should choose  to play Pardeep twice. 

\item
In order to ease the entry requirements,
it is decided instead that the  challenge match will consist of 
four games. 
Now, before the match begins,
Rosalind must choose  whether to play Pardeep three times and 
Quentin once 
(strategy 1), or
to play Pardeep twice and Quentin twice (strategy 2) 
or to play  Pardeep once and Quentin three times (strategy~3).

Show that, if $q-p>\frac 12$, Rosalind should choose strategy 1.

If $q-p<\frac12$, give examples of values of $p$ and $q$ to show
that strategy 2 can be better or worse than strategy 1.
\end{questionparts}
\end{question}

\end{document}
