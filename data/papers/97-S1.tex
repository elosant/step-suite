\documentclass[a4, 11pt]{report}


\pagestyle{myheadings}
\markboth{}{Paper I, 1997
\ \ \ \ \ 
\today 
}               

\RequirePackage{amssymb}
\RequirePackage{amsmath}
\RequirePackage{graphicx}
\RequirePackage{color}
\RequirePackage[flushleft]{paralist}[2013/06/09]



\RequirePackage{geometry}
\geometry{%
  a4paper,
  lmargin=2cm,
  rmargin=2.5cm,
  tmargin=3.5cm,
  bmargin=2.5cm,
  footskip=12pt,
  headheight=24pt}


\newcommand{\comment}[1]{{\bf Comment} {\it #1}}
%\renewcommand{\comment}[1]{}

\newcommand{\bluecomment}[1]{{\color{blue}#1}}
%\renewcommand{\comment}[1]{}
\newcommand{\redcomment}[1]{{\color{red}#1}}



\usepackage{epsfig}
\usepackage{pstricks-add}
\usepackage{tgheros} %% changes sans-serif font to TeX Gyre Heros (tex-gyre)
\renewcommand{\familydefault}{\sfdefault} %% changes font to sans-serif
%\usepackage{sfmath}  %%%% this makes equation sans-serif
%\input RexFigs


\setlength{\parskip}{10pt}
\setlength{\parindent}{0pt}

\newlength{\qspace}
\setlength{\qspace}{20pt}


\newcounter{qnumber}
\setcounter{qnumber}{0}

\newenvironment{question}%
 {\vspace{\qspace}
  \begin{enumerate}[\bfseries 1\quad][10]%
    \setcounter{enumi}{\value{qnumber}}%
    \item%
 }
{
  \end{enumerate}
  \filbreak
  \stepcounter{qnumber}
 }


\newenvironment{questionparts}[1][1]%
 {
  \begin{enumerate}[\bfseries (i)]%
    \setcounter{enumii}{#1}
    \addtocounter{enumii}{-1}
    \setlength{\itemsep}{5mm}
    \setlength{\parskip}{8pt}
 }
 {
  \end{enumerate}
 }



\DeclareMathOperator{\cosec}{cosec}
\DeclareMathOperator{\Var}{Var}

\def\d{{\rm d}}
\def\e{{\rm e}}
\def\g{{\rm g}}
\def\h{{\rm h}}
\def\f{{\rm f}}
\def\p{{\rm p}}
\def\s{{\rm s}}
\def\t{{\rm t}}


\def\A{{\rm A}}
\def\B{{\rm B}}
\def\E{{\rm E}}
\def\F{{\rm F}}
\def\G{{\rm G}}
\def\H{{\rm H}}
\def\P{{\rm P}}


\def\bb{\mathbf b}
\def \bc{\mathbf c}
\def\bx {\mathbf x}
\def\bn {\mathbf n}

\newcommand{\low}{^{\vphantom{()}}}
%%%%% to lower suffices: $X\low_1$ etc


\newcommand{\subone}{ {\vphantom{\dot A}1}}
\newcommand{\subtwo}{ {\vphantom{\dot A}2}}




\def\le{\leqslant}
\def\ge{\geqslant}


\def\var{{\rm Var}\,}

\newcommand{\ds}{\displaystyle}
\newcommand{\ts}{\textstyle}




\begin{document}
\setcounter{page}{2}

 
\section*{Section A: \ \ \ Pure Mathematics}

%%%%%%%%%%Q1
\begin{question}
Show that you can make up 10 pence in eleven ways
using 10p, 5p, 2p and 1p coins.

In how many ways can you make up 20 pence using 20p,
10p, 5p, 2p and 1p coins?

\noindent[You are reminded that
no credit will be given for unexplained answers.]
\end{question}

%%%%%%%%%%Q2
\begin{question}
\begin{questionparts}
\item If
\[{\mathrm f}(x)=\tan^{-1}x+\tan^{-1}\left(\frac{1-x}{1+x}\right),\]
find ${\mathrm f}'(x)$. Hence, or otherwise, find a simple expression for 
${\mathrm f}(x)$.
\item Suppose that $y$ is a function of $x$ 
with $0<y<(\pi/2)^{1/2}$ and
\[x=y\sin y^{2}\]
for $0<x<(\pi/2)^{1/2}$. Show that
(for this range of $x$)
\[\frac{{\mathrm d}y}{{\mathrm d}x}=
\frac{y}{x+2y^2\sqrt{y^{2}-x^{2}}}.\]
\end{questionparts}
\end{question}

%%%%%%%%% Q3
\begin{question}
Let $a_{1}=3$, $a_{n+1}=a_{n}^{3}$ for $n\geqslant 1$.
(Thus $a_{2}=3^{3}$, $a_{3}=(3^{3})^{3}$ and so on.)
\begin{questionparts}
\item What digit appears in the unit place of $a_{7}$?
\item Show that $a_{7}\geqslant 10^{100}$.
\item What is $\dfrac{a_{7}+1}{2a_{7}}$ correct to
two places of decimals? Justify your answer.
\end{questionparts}
\end{question}

%%%%%% Q4 
\begin{question}
Find all the solutions of the equation
\[|x+1|-|x|+3|x-1|-2|x-2|=x+2.\]
	\end{question}

%%%%%%%%% Q5
\begin{question}
Four rigid rods $AB$, $BC$, $CD$ and $DA$
are freely jointed together to form a quadrilateral
in the plane.
Show
that if $P$, $Q$, $R$, $S$ are the mid-points of the sides $AB$,
$BC$, $CD$, $DA$, respectively, then
\[|AB|^{2}+|CD|^{2}+2|PR|^{2}=|AD|^{2}+|BC|^{2}+2|QS|^{2}.\]
Deduce that $|PR|^{2}-|QS|^{2}$
remains constant however the vertices move. (Here $|PR|$
denotes the length of $PR$.)
\end{question}
	
%%%%%%%%% Q6
\begin{question}
Find constants $a_{0}$, $a_{1}$, $a_{2}$, $a_{3}$, $a_{4}$, $a_{5}$, 
$a_{6}$  and $b$ such that
\[x^{4}(1-x)^{4}=(a_{6}x^{6}+a_{5}x^{5}+a_{4}x^{4}+a_{3}x^{3}+
a_{2}x^{2}+a_{1}x+a_{0})(x^{2}+1)+b.\]
Hence, or otherwise, prove that
\[\int_{0}^{1}\frac{x^{4}(1-x)^{4}}{1+x^{2}}{\rm d}x
=\frac{22}{7}-\pi.\]
Evaluate $\displaystyle{\int_{0}^{1}x^{4}(1-x)^{4}{\rm d}x}$
and deduce that
\[\frac{22}{7}>\pi>\frac{22}{7}-\frac{1}{630}.\]
\end{question}
	
%%%%%%%%% Q7
\begin{question}
Find constants $a_{1}$, $a_{2}$, $u_{1}$ and $u_{2}$
such that,
whenever ${\mathrm P}$ is a cubic polynomial,
\[\int_{-1}^{1}{\mathrm P}(t)\,{\mathrm d}t
=a_{1}{\mathrm P}(u_{1})+a_{2}{\mathrm P}(u_{2}).\]
\end{question}
		
%%%%%%%%% Q8
\begin{question}	
By considering the maximum of $\ln x-x\ln a$, or otherwise,
show that the equation
$x=a^{x}$ has no real roots if $a>e^{1/e}$.

How many real roots does the equation have if
$0<a< 1$? Justify your answer.
\end{question}	
		

		
	
\newpage
\section*{Section B: \ \ \ Mechanics}


	
%%%%%%%%%% Q9
\begin{question}
A single stream of cars, each of width $a$
and exactly in line, is passing along a straight road
of breadth $b$ with speed $V$. The distance between the
successive cars is $c$.
\begin{center}
\psset{xunit=0.9cm,yunit=0.9cm,algebraic=true,dotstyle=o,dotsize=3pt 0,linewidth=0.5pt,arrowsize=3pt 2,arrowinset=0.25}
\begin{pspicture*}(-5.32,-1.36)(10.5,2.44)
\psline(-5,2)(10,2)
\psline(-5,-1)(10,-1)
\psline(-4,1)(-4,0)
\psline(-4,1)(-1,1)
\psline(-1,1)(-1,0)
\psline(-4,0)(-1,0)
\psline(1,1)(1,0)
\psline(1,1)(4,1)
\psline(4,1)(4,0)
\psline(4,0)(1,0)
\psline(6,1)(6,0)
\psline(6,1)(9,1)
\psline(9,1)(9,0)
\psline(9,0)(6,0)
\psline{->}(-1,0.5)(1,0.5)
\psline{->}(1,0.5)(-1,0.5)
\psline{->}(4,0.5)(6,0.5)
\psline{->}(6,0.5)(4,0.5)
\psline{->}(-4.6,1)(-4.6,0)
\psline{->}(-4.6,0)(-4.6,1)
\psline{->}(10,2)(10,-1)
\psline{->}(10,-1)(10,2)
\rput[tl](-5,0.6){$a$}
\rput[tl](-0.14,0.96){$c$}
\rput[tl](4.86,0.98){$c$}
\rput[tl](10.16,0.68){$b$}
\end{pspicture*}
\end{center}
A chicken crosses the road in safety
at a constant speed $u$ in a straight line  making
an angle $\theta$ with the direction of traffic. 
Show that
\[u\geqslant
\frac{Va}{c\sin\theta+a\cos\theta}.\]

Show also that if the chicken chooses $\theta$ and $u$
so that it crosses the road
at the least possible uniform speed, it crosses
in time
\[\frac{b}{V}\left(\frac{c}{a}+\frac{a}{c}\right)
.
\]
	\end{question}
	
%%%%%%%%%% Q10
\begin{question}	
The point $A$ is vertically above 
the point $B$. A light inextensible
string, with a smooth ring $P$ 
of mass $m$ threaded onto it, has its ends
attached at $A$ and $B$. The plane $APB$ rotates
about $AB$ with constant angular velocity $\omega$
so that $P$ describes a horizontal circle of radius $r$
and the string is taut. The angle $BAP$ has value
$\theta$ and the angle $ABP$ has value $\phi$.
Show that
\[\tan\frac{\phi-\theta}{2}=\frac{g}{r\omega^{2}}.\]

Find the tension in the string
in terms of $m$, $g$, $r$, $\omega$
and $\sin\frac{1}{2}(\theta+\phi)$.
\end{question}

%%%%%%%%%% Q11

\begin{question}
A particle of unit mass is projected vertically upwards
in a medium whose resistance is $k$ times the square of the velocity of the
particle. If the initial velocity is $u$, prove that the velocity
$v$ after rising through a distance $s$ satisfies
\begin{equation*}
v^{2}=u^{2}\e^{-2ks}+\frac{g}{k}(\e^{-2ks}-1). \tag{\ensuremath{*}}
\end{equation*}

Find an expression for the maximum height of the particle above the
point of projection.

Does equation $(*)$ still hold on the downward path?
Justify your answer.
\end{question}
	

	
	\newpage
\section*{Section C: \ \ \ Probability and Statistics}


%%%%%%%%%% Q12
\begin{question}
An experiment produces a random number
$T$ uniformly distributed on $[0,1]$. 
Let $X$ be the larger root of the equation
\[x^{2}+2x+T=0.\]
What is the probability that $X>-1/3$? Find $\mathrm{E}(X)$
and show that $\mathrm{Var}(X)=1/18$.

The experiment is repeated independently 800 times generating
the larger roots $X_{1}$, $X_{2}$, \dots, $X_{800}$. If
\[Y=X_{1}+X_{2}+\dots+X_{800}.\]
find an approximate value for  $K$ such that
\[\mathrm{P}(Y\leqslant K)=0.08.\]
\end{question}

%%%%%%%%%% Q13
\begin{question}
Mr Blond returns to his flat to find
it in complete darkness. He knows that this
means that one of four assassins Mr 1, Mr 2,
Mr 3 or Mr 4 has set a trap for him.
His trained instinct tells him that the
probability that Mr $i$ has set the trap
is $i/10$. His knowledge of their habits
tells him that Mr $i$ uses 
a deadly trained silent anaconda
with probability
$(i+1)/10$,
a bomb with probability
$i/10$ and a vicious attack canary with probability
$(9-2i)/10$ $[i=1,2,3,4]$.
 
He now listens carefully and, hearing no singing,
concludes
correctly that no canary is involved. If he switches
on the light and the trap is a bomb he has probability
$1/2$  of being killed
but if the trap is an anaconda
he has probability $2/3$ of survival.
If he does not switch on the light
and the trap is a bomb he is certain to survive
but, if the trap is an anaconda, he has a probability $1/2$
of being killed.
His professional pride means that he must enter the flat.
Advise Mr Blond, giving reasons for your advice.
\end{question}

%%%%%%%%%% Q14
\begin{question}
The maximum height $X$ of flood water 
each year on a certain
river is a random variable with density function
\begin{equation*}
{\mathrm f}(x)=
\begin{cases}
\exp(-x)&\text{if $x\geqslant 0$,}\\
0&\text{otherwise}.
\end{cases}
\end{equation*}
It costs $y$ megadollars each year
to prepare for flood water
of height $y$ or less. If $X\leqslant y$ 
no further costs are incurred
but if $X\geqslant y$ the cost of flood damage 
is $r+s(X-y)$ megadollars where $r,s>0$. 
The total cost $T$ megadollars is thus
given by
\begin{equation*}
T=
\begin{cases}
y&\text{if $X\leqslant y$},\\
y+r+s(X-y)&\text{if $X>y$}.
\end{cases}
\end{equation*}
Show that we can minimise the expected total cost
by taking 
\[y=\ln(r+s).\]
\end{question}
	
\end{document}
