
\documentclass[a4, 11pt]{report}


\pagestyle{myheadings}
\markboth{}{Paper III, 1992
\ \ \ \ \ 
\today 
}               

\RequirePackage{amssymb}
\RequirePackage{amsmath}
\RequirePackage{graphicx}
\RequirePackage{color}
\RequirePackage[flushleft]{paralist}[2013/06/09]



\RequirePackage{geometry}
\geometry{%
  a4paper,
  lmargin=2cm,
  rmargin=2.5cm,
  tmargin=3.5cm,
  bmargin=2.5cm,
  footskip=12pt,
  headheight=24pt}


\newcommand{\comment}[1]{{\bf Comment} {\it #1}}
%\renewcommand{\comment}[1]{}

\newcommand{\bluecomment}[1]{{\color{blue}#1}}
%\renewcommand{\comment}[1]{}
\newcommand{\redcomment}[1]{{\color{red}#1}}



\usepackage{epsfig}
\usepackage{pstricks-add}
\usepackage{tgheros} %% changes sans-serif font to TeX Gyre Heros (tex-gyre)
\renewcommand{\familydefault}{\sfdefault} %% changes font to sans-serif
%\usepackage{sfmath}  %%%% this makes equation sans-serif
%\input RexFigs


\setlength{\parskip}{10pt}
\setlength{\parindent}{0pt}

\newlength{\qspace}
\setlength{\qspace}{20pt}


\newcounter{qnumber}
\setcounter{qnumber}{0}

\newenvironment{question}%
 {\vspace{\qspace}
  \begin{enumerate}[\bfseries 1\quad][10]%
    \setcounter{enumi}{\value{qnumber}}%
    \item%
 }
{
  \end{enumerate}
  \filbreak
  \stepcounter{qnumber}
 }


\newenvironment{questionparts}[1][1]%
 {
  \begin{enumerate}[\bfseries (i)]%
    \setcounter{enumii}{#1}
    \addtocounter{enumii}{-1}
    \setlength{\itemsep}{5mm}
    \setlength{\parskip}{8pt}
 }
 {
  \end{enumerate}
 }



\DeclareMathOperator{\cosec}{cosec}
\DeclareMathOperator{\Var}{Var}

\def\d{{\rm d}}
\def\e{{\rm e}}
\def\g{{\rm g}}
\def\h{{\rm h}}
\def\f{{\rm f}}
\def\p{{\rm p}}
\def\s{{\rm s}}
\def\t{{\rm t}}


\def\A{{\rm A}}
\def\B{{\rm B}}
\def\E{{\rm E}}
\def\F{{\rm F}}
\def\G{{\rm G}}
\def\H{{\rm H}}
\def\P{{\rm P}}


\def\bb{\mathbf b}
\def \bc{\mathbf c}
\def\bx {\mathbf x}
\def\bn {\mathbf n}

\newcommand{\low}{^{\vphantom{()}}}
%%%%% to lower suffices: $X\low_1$ etc


\newcommand{\subone}{ {\vphantom{\dot A}1}}
\newcommand{\subtwo}{ {\vphantom{\dot A}2}}




\def\le{\leqslant}
\def\ge{\geqslant}


\def\var{{\rm Var}\,}

\newcommand{\ds}{\displaystyle}
\newcommand{\ts}{\textstyle}




\begin{document}
\setcounter{page}{2}

 
\section*{Section A: \ \ \ Pure Mathematics}

%%%%%%%%%%Q1
\begin{question}
\begin{questionparts}


\item Given that 
\[
\mathrm{f}(x)=\ln(1+\mathrm{e}^{x}),
\]
prove that $\ln[\mathrm{f}'(x)]=x-\mathrm{f}(x)$ and that $\mathrm{f}''(x)=\mathrm{f}'(x)-[\mathrm{f}'(x)]^{2}.$
Hence, or otherwise, expand $\mathrm{f}(x)$ as a series in powers
of $x$ up to the term in $x^{4}.$ 


\item Given that 
\[
\mathrm{g}(x)=\frac{1}{\sinh x\cosh2x},
\]
explain why $\mathrm{g}(x)$ can not be expanded as a series of non-negative
powers of $x$ but that $x\mathrm{g}(x)$ can be so expanded. Explain
also why this latter expansion will consist of even powers of $x$
only. Expand $x\mathrm{g}(x)$ as a series as far as the term in $x^{4}.$
\end{questionparts}

\end{question}

%%%%%%%%%%Q2
\begin{question}
The matrices $\mathbf{I}$ and $\mathbf{J}$ are 
\[
\mathbf{I}=\begin{pmatrix}1 & 0\\
0 & 1
\end{pmatrix}\quad\mbox{ and }\quad\mathbf{J}=\begin{pmatrix}1 & 1\\
1 & 1
\end{pmatrix}
\]
respectively and $\mathbf{A}=\mathbf{I}+a\mathbf{J},$ where $a$
is a non-zero real constant. Prove that 
\[
\mathbf{A}^{2}=\mathbf{I}+\tfrac{1}{2}[(1+2a)^{2}-1]\mathbf{J}\quad\mbox{ and }\quad\mathbf{A}^{3}=\mathbf{I}+\tfrac{1}{2}[(1+2a)^{3}-1]\mathbf{J}
\]
and obtain a similar form for $\mathbf{A}^{4}.$


If $\mathbf{A}^{k}=\mathbf{I}+p_{k}\mathbf{J},$ suggest a suitable
form for $p_{k}$ and prove that it is correct by induction, or otherwise. 
\end{question}

%%%%%%%%% Q3
\begin{question}
Sketch the curve $C_{1}$ whose parametric equations are $x=t^{2},$
$y=t^{3}.$


The circle $C_{2}$ passes through the origin $O$. The points $R$
and $S$ with real non-zero parameters $r$ and $s$ respectively
are other intersections of $C_{1}$ and $C_{2}.$ Show that $r$ and
$s$ are roots of an equation of the form 
\[
t^{4}+t^{2}+at+b=0,
\]
where $a$ and $b$ are real constants. 


By obtaining a quadratic equation, with coefficients expressed in
terms of $r$ and $s$, whose roots would be the parameters of any
further intersections of $C_{1}$ and $C_{2},$ or otherwise, show
that $O$, $R$ and $S$ are the only real intersections of $C_{1}$
and $C_{2}.$  
\end{question}


%%%%%% Q4 

\begin{question}
A set of curves $S_{1}$ is defined by the equation 
\[
y=\frac{x}{x-a},
\]
where $a$ is a constant which is different for different members
of $S_{1}.$ Sketch on the same axes the curves for which $a=-2,-1,1$
and $2$. 


A second of curves $S_{2}$ is such that at each intersection between
a member of $S_{2}$ and a member of $S_{1}$ the tangents of the
intersecting curves are perpendicular. On the same axes as the already
sketched members of $S_{1},$ sketch the member of $S_{2}$ that passes
through the point $(1,-1)$. 


Obtain the first order differential equation for $y$ satisfied at
all points on all members of $S_{1}$ (i.e. an equation connecting
$x,y$ and $\mathrm{d}y/\mathrm{d}x$ which does not involve $a$). 


State the relationship between the values of $\mathrm{d}y/\mathrm{d}x$
on two intersecting curves, one from $S_{1}$ and one from $S_{2},$
at their intersection. Hence show that the differential equation for
the curves of $S_{2}$ is 
\[
x=y(y-1)\dfrac{\mathrm{d}y}{\mathrm{d}x}.
\]
Find an equation for the member of $S_{2}$ that you have sketched.
\end{question}


%%%%%%%%% Q5
\begin{question}
The tetrahedron $ABCD$ has $A$ at the point $(0,4,-2)$. It is symmetrical
about the plane $y+z=2,$ which passes through $A$ and $D$. The
mid-point of $BC$ is $N$. The centre, $Y$, of the sphere $ABCD$
is at the point $(3,-2,4)$ and lies on $AN$ such that $\overrightarrow{AY}=3\overrightarrow{YN}.$
Show that $BN=6\sqrt{2}$ and find the coordinates of $B$ and $C$. 


The angle $AYD$ is $\cos^{-1}\frac{1}{3}.$ Find the coordinates
of $D$. {[}There are two alternative answers for each point.{]}  
	\end{question}
	
	%%%%%%%%% Q6
	\begin{question}
Given that ${\displaystyle I_{n}=\int_{0}^{\pi}\frac{x\sin^{2}(nx)}{\sin^{2}x}\,\mathrm{d}x,}$
where $n$ is a positive integer, show that $I_{n}-I_{n-1}=J_{n},$
where 
\[
J_{n}=\int_{0}^{\pi}\frac{x\sin(2n-1)x}{\sin x}\,\mathrm{d}x.
\]
Obtain also a reduction formula for $J_{n}.$


The curve $C$ is given by the cartesian equation 
\[
y=\dfrac{x\sin^{2}(nx)}{\sin^{2}x},
\]
 where $n$ is a positive integer and $0\leqslant x\leqslant\pi.$
Show that the area under the curve $C$ is $\frac{1}{2}n\pi^{2}.$ 
	 \end{question}
	 
	 %%%%%%%%% Q7
\begin{question}
The points $P$ and $R$ lie on the sides $AB$ and $AD,$ respectively,
of the parallelogram $ABCD.$ The point $Q$ is the fourth vertex
of the parallelogram $APQR.$ Prove that $BR,CQ$ and $DP$ meet in
a point. 
	\end{question}
	
	%%%%%%%%% Q8
	\begin{question}
Show that 
\[
\sin(2n+1)\theta=\sin^{2n+1}\theta\sum_{r=0}^{n}(-1)^{n-r}\binom{2n+1}{2r}\cot^{2r}\theta,
\]
where $n$ is a positive integer. Deduce that the equation 
\[
\sum_{r=0}^{n}(-1)^{r}\binom{2n+1}{2r}x^{r}=0
\]
has roots $\cot^{2}(k\pi/(2n+1))$ for $k=1,2,\ldots,n$. 


Show that 


\begin{itemize}
\setlength{\itemsep}{3mm}
\item[\bf (i)]  ${\displaystyle \sum_{k=1}^{n}\cot^{2}\left(\frac{k\pi}{2n+1}\right)=\frac{n(2n-1)}{3}},$ 


\item[\bf (ii)] ${\displaystyle \sum_{k=1}^{n}\tan^{2}\left(\frac{k\pi}{2n+1}\right)=n(2n+1)},$ 


\item[\bf (iii)] ${\displaystyle \sum_{k=1}^{n}\mathrm{cosec}^{2}\left(\frac{k\pi}{2n+1}\right)=\frac{2n(n+1)}{3}}.$
\end{itemize}

		\end{question}
		
		
%%%%%%%%% Q9
		\begin{question}
The straight line $OSA,$ where $O$ is the origin, bisects the angle
between the positive $x$ and $y$ axes. The ellipse $E$ has $S$
as focus. In polar coordinates with $S$ as pole and $SA$ as the
initial line, $E$ has equation $\ell=r(1+e\cos\theta).$ Show that,
at the point on $E$ given by $\theta=\alpha,$ the gradient of the
tangent to the ellipse is given by 
\[
\frac{\mathrm{d}y}{\mathrm{d}x}=\frac{\sin\alpha-\cos\alpha-e}{\sin\alpha+\cos\alpha+e}.
\]
The points on $E$ given by $\theta=\alpha$ and $\theta=\beta$ are
the ends of a diameter of $E$. Show that 
\[
\tan(\alpha/2)\tan(\beta/2)=-\frac{1+e}{1-e}.
\]
{[}\textbf{Hint. }A diameter of an ellipse is a chord through its
centre.{]} 
		\end{question}
		
	
%%%%%%%%%% 10
\begin{question}
Sketch the curve $C$ whose polar equation is 
\[
r=4a\cos2\theta\qquad\mbox{ for }-\tfrac{1}{4}\pi<\theta<\tfrac{1}{4}\theta.
\]
The ellipse $E$ has parametric equations 
\[
x=2a\cos\phi,\qquad y=a\sin\phi.
\]
Show, without evaluating the integrals, that the perimeters of $C$
and $E$ are equal. 


Show also that the areas of the regions enclosed by $C$ and $E$
are equal.  
\end{question}
			
		
		
		
	
\newpage
\section*{Section B: \ \ \ Mechanics}


	
%%%%%%%%%% Q11
\begin{question}$\,$
\begin{center}
\psset{xunit=1.0cm,yunit=1.0cm,algebraic=true,dimen=middle,dotstyle=o,dotsize=3pt 0,linewidth=0.3pt,arrowsize=3pt 2,arrowinset=0.25}
\begin{pspicture*}(-1,0.3)(5.98,5.3)
\psline(0,5)(0,1)
\psline{->}(0,3)(2,3)
\psline(5,5)(5,1)
\rput[tl](-0.54,5.26){$A$}
\rput[tl](-0.56,3.16){$O$}
\rput[tl](-0.58,1.06){$B$}
\rput[tl](1.64,3.62){$V$}
\rput[tl](5.22,5.26){$X$}
\rput[tl](5.28,1.34){$Y$}
\end{pspicture*}
\end{center}
$AOB$ represents a smooth vertical wall and $XY$ represents a parallel
smooth vertical barrier, both standing on a smooth horizontal table.
A particle $P$ is projected along the table from $O$ with speed
$V$ in a direction perpendicular to the wall. At the time of projection,
the distance between the wall and the barrier is $(75/32)VT$, where
$T$ is a constant. The barrier moves directly towards the wall, remaining
parallel to the wall, with initial speed $4V$ and with constant acceleration
$4V/T$ directly away from the wall. The particle strikes the barrier
$XY$ and rebounds. Show that this impact takes place at time $5T/8$. 


The barrier is sufficiently massive for its motion to be unaffected
by the impact. Given that the coefficient of restitution is $1/2$,
find the speed of $P$ immediately after impact. 


$P$ strikes $AB$ and rebounds. Given that the coefficient of restitution
for this collision is also $1/2,$ show that the next collision of
$P$ with the barrier is at time $9T/8$ from the start of the motion.  
	\end{question}
	
%%%%%%%%%% Q12
\begin{question}$\,$
\begin{center}
\psset{xunit=1.0cm,yunit=1.0cm,algebraic=true,dimen=middle,dotstyle=o,dotsize=3pt 0,linewidth=0.3pt,arrowsize=3pt 2,arrowinset=0.25}
\begin{pspicture*}(-2.31,-0.3)(7.31,3.5)
\parametricplot{3.141592653589793}{6.283185307179586}{1*2.5*cos(t)+0*2.5*sin(t)+2.5|0*2.5*cos(t)+1*2.5*sin(t)+3}
\psline(-2,0)(7,0)
\psline(0.65,1.31)(-0.98,0.16)
\psline(-0.98,0.16)(5.96,0.16)
\psline(5.96,0.16)(4.41,1.39)
\psline[linestyle=dashed,dash=1pt 1pt](5,3)(2.36,3.01)
\psline[linestyle=dashed,dash=1pt 1pt](2.36,3.01)(3.63,0.77)
\rput[tl](3.85,0.77){$P$}
\parametricplot{-1.0566941425542427}{-0.004136610689158271}{0.82*cos(t)+2.36|0.82*sin(t)+3.01}
\rput[tl](2.74,2.9){$\theta$}
\rput[tl](1.84,3.4){$O$}
\begin{scriptsize}
\psdots[dotstyle=*](2.36,3.01)
\psdots[dotstyle=*](3.63,0.77)
\end{scriptsize}
\end{pspicture*}
\end{center}
A smooth hemispherical bowl of mass $2m$ is rigidly mounted on a
light carriage which slides freely on a horizontal table as shown
in the diagram. The rim of the bowl is horizontal and has centre $O$.
A particle $P$ of mass $m$ is free to slide on the inner surface
of the bowl. Initially, $P$ is in contact with the rim of the bowl
and the system is at rest. The system is released and when $OP$ makes
an angle $\theta$ with the horizontal the velocity of the bowl is
$v$? Show that 
\[3v=a\dot{\theta}\sin\theta
\]
and that 
\[
v^{2}=\frac{2ga\sin^{3}\theta}{3(3-\sin^{2}\theta)},
\]
where $a$ is the interior radius of the bowl. 


Find, in terms of $m,g$ and $\theta,$ the reaction between the bowl
and the particle. 
\end{question}

%%%%%%%%%% Q13

\begin{question}$\,$
\begin{center}
\psset{xunit=1.0cm,yunit=1.0cm,algebraic=true,dimen=middle,dotstyle=o,dotsize=3pt 0,linewidth=0.3pt,arrowsize=3pt 2,arrowinset=0.25}
\begin{pspicture*}(-2.8,-2.34)(3.26,4.36)
\pspolygon[linewidth=0.4pt](0.22,-0.18)(0.4,0.04)(0.18,0.22)(0,0)
\pscircle(0,0){2}
\psline(2,0)(2,4)
\rput[tl](2.24,2.64){$\pi b$}
\rput[tl](2.22,0.02){$C$}
\rput[tl](2.24,4.3){$A$}
\rput[tl](1.32,2.1){$Q$}
\rput[tl](-0.42,0.08){$O$}
\rput[tl](1.6,-1.38){$P$}
\psline[linestyle=dashed,dash=3pt 3pt](1.27,1.54)(0,0)
\psline[linestyle=dashed,dash=3pt 3pt](0,0)(1.54,-1.27)
\rput[tl](0.06,1.1){$2b$}
\begin{scriptsize}
\psdots[dotsize=2pt 0,dotstyle=*](2,4)
\end{scriptsize}
\end{pspicture*}
\end{center}
A uniform circular disc of radius $2b,$ mass $m$ and centre $O$
is free to turn about a fixed horizontal axis through $O$ perpendicular
to the plane of the disc. A light elastic string of modulus $kmg$,
where $k>4/\pi,$ has one end attached to a fixed point $A$ and the
other end to the rim of the disc at $P$. The string is in contact
with the rim of the disc along the arc $PC,$ and $OC$ is horizontal.
The natural length of the string and the length of the line $AC$
are each $\pi b$ and $AC$ is vertical. A particle $Q$ of mass $m$
is attached to the rim of the disc and $\angle POQ=90^{\circ}$ as
shown in the diagram. The system is released from rest with $OP$
vertical and $P$ below $O$. Show that $P$ reaches $C$ and that
then the upward vertical component of the reaction on the axis is
$mg(10-\pi k)/3$. 
\end{question}

%%%%%%%%%% Q14
\begin{question}$\,$
\begin{center}
\psset{xunit=1.0cm,yunit=1.0cm,algebraic=true,dimen=middle,dotstyle=o,dotsize=3pt 0,linewidth=0.3pt,arrowsize=3pt 2,arrowinset=0.25}
\begin{pspicture*}(-2.26,-2.36)(6,5.7)
\pscircle(0,0){2}
\psline(-1.52,1.3)(1.38,4.08)
\psline{->}(0,0)(0,5)
\psline{->}(0,0)(5,0)
\psline(0,0)(-1.52,1.3)
\psline(0,2)(4,2)
\parametricplot{0.0}{2.4340509797353143}{0.6*cos(t)+0|0.6*sin(t)+0}
\rput[tl](1.58,4.34){$P$}
\rput[tl](4.22,2.14){$B$}
\rput[tl](0.44,0.92){$\theta$}
\rput[tl](-2,1.75){$Q$}
\rput[tl](-0.26,-0.06){$O$}
\rput[tl](5.14,0.12){$x$}
\rput[tl](-0.08,5.4){$y$}
\begin{scriptsize}
\psdots[dotstyle=*](1.38,4.08)
\psdots[dotstyle=*](4,2)
\end{scriptsize}
\end{pspicture*}
\end{center}
A horizontal circular disc of radius $a$ and centre $O$ lies on
a horizontal table and is fixed to it so that it cannot rotate. A
light inextensible string of negligible thickness is wrapped round
the disc and attached at its free end to a particle $P$ of mass $m$.
When the string is all in contact with the disc, $P$ is at $A$.
The string is unwound so that the part not in contact with the disc
is taut and parallel to $OA$. $P$ is then at $B$. The particle
is projected along the table from $B$ with speed $V$ perpendicular
to and away from $OA$. In the general position, the string is tangential
to the disc at $Q$ and $\angle AOQ=\theta.$ Show that, in the general
position, the $x$-coordinate of $P$ with respect to the axes shown
in the figure is $a\cos\theta+a\theta\sin\theta,$ and find $y$-coordinate
of $P$. Hence, or otherwise, show that the acceleration of $P$ has
components $a\theta\dot{\theta}^{2}$ and $a\dot{\theta}^{2}+a\theta\ddot{\theta}$
along and perpendicular to $PQ,$ respectively. 


The friction force between $P$ and the table is $2\lambda mv^{2}/a,$
where $v$ is the speed of $P$ and $\lambda$ is a constant. Show
that 
\[
\frac{\ddot{\theta}}{\dot{\theta}}=-\left(\frac{1}{\theta}+2\lambda\theta\right)\dot{\theta}
\]
and find $\dot{\theta}$ in terms of $\theta,\lambda$ and $a$. Find
also the tension in the string when $\theta=\pi.$ 
\end{question}

	
	\newpage
\section*{Section C: \ \ \ Probability and Statistics}


%%%%%%%%%% Q15
\begin{question}
A goat $G$ lies in a square field $OABC$ of side $a$. It wanders
randomly round its field, so that at any time the probability of its
being in any given region is proportional to the area of this region.
Write down the probability that its distance, $R$, from $O$ is less
than $r$ if $0<r\leqslant a,$ and show that if $r\geqslant a$ the
probability is 
\[
\left(\frac{r^{2}}{a^{2}}-1\right)^{\frac{1}{2}}+\frac{\pi r^{2}}{4a^{2}}-\frac{r^{2}}{a^{2}}\cos^{-1}\left(\frac{a}{r}\right).
\]
Find the median of $R$ and probability density function of $R$. 


The goat is then tethered to the corner $O$ by a chain of length
$a$. Find the conditional probability that its distance from the
fence $OC$ is more than $a/2$.  
\end{question}

%%%%%%%%%% Q16
\begin{question}
The probability that there are exactly $n$ misprints in an issue
of a newspaper is $\mathrm{e}^{-\lambda}\lambda^{n}/n!$ where $\lambda$
is a positive constant. The probability that I spot a particular misprint
is $p$, independent of what happens for other misprints, and $0<p<1.$


\begin{questionparts}
	\item If there are exactly $m+n$ misprints, what
is the probability that I spot exactly $m$ of them?


\item Show that, if I spot exactly $m$ misprints, the probability
that I have failed to spot exactly $n$ misprints is 
\[
\frac{(1-p)^{n}\lambda^{n}}{n!}\mathrm{e}^{-(1-p)\lambda}.
\]
\end{questionparts}

\end{question}
\end{document}
