\documentclass[a4, 11pt]{report}


\pagestyle{myheadings}
\markboth{}{Paper III, 1995
\ \ \ \ \ 
\today 
}               

\RequirePackage{amssymb}
\RequirePackage{amsmath}
\RequirePackage{graphicx}
\RequirePackage{color}
\RequirePackage[flushleft]{paralist}[2013/06/09]



\RequirePackage{geometry}
\geometry{%
  a4paper,
  lmargin=2cm,
  rmargin=2.5cm,
  tmargin=3.5cm,
  bmargin=2.5cm,
  footskip=12pt,
  headheight=24pt}


\newcommand{\comment}[1]{{\bf Comment} {\it #1}}
%\renewcommand{\comment}[1]{}

\newcommand{\bluecomment}[1]{{\color{blue}#1}}
%\renewcommand{\comment}[1]{}
\newcommand{\redcomment}[1]{{\color{red}#1}}



\usepackage{epsfig}
\usepackage{pstricks-add}
\usepackage{tgheros} %% changes sans-serif font to TeX Gyre Heros (tex-gyre)
\renewcommand{\familydefault}{\sfdefault} %% changes font to sans-serif
%\usepackage{sfmath}  %%%% this makes equation sans-serif
%\input RexFigs


\setlength{\parskip}{10pt}
\setlength{\parindent}{0pt}

\newlength{\qspace}
\setlength{\qspace}{20pt}


\newcounter{qnumber}
\setcounter{qnumber}{0}

\newenvironment{question}%
 {\vspace{\qspace}
  \begin{enumerate}[\bfseries 1\quad][10]%
    \setcounter{enumi}{\value{qnumber}}%
    \item%
 }
{
  \end{enumerate}
  \filbreak
  \stepcounter{qnumber}
 }


\newenvironment{questionparts}[1][1]%
 {
  \begin{enumerate}[\bfseries (i)]%
    \setcounter{enumii}{#1}
    \addtocounter{enumii}{-1}
    \setlength{\itemsep}{5mm}
    \setlength{\parskip}{8pt}
 }
 {
  \end{enumerate}
 }



\DeclareMathOperator{\cosec}{cosec}
\DeclareMathOperator{\Var}{Var}

\def\d{{\rm d}}
\def\e{{\rm e}}
\def\g{{\rm g}}
\def\h{{\rm h}}
\def\f{{\rm f}}
\def\p{{\rm p}}
\def\s{{\rm s}}
\def\t{{\rm t}}


\def\A{{\rm A}}
\def\B{{\rm B}}
\def\E{{\rm E}}
\def\F{{\rm F}}
\def\G{{\rm G}}
\def\H{{\rm H}}
\def\P{{\rm P}}


\def\bb{\mathbf b}
\def \bc{\mathbf c}
\def\bx {\mathbf x}
\def\bn {\mathbf n}

\newcommand{\low}{^{\vphantom{()}}}
%%%%% to lower suffices: $X\low_1$ etc


\newcommand{\subone}{ {\vphantom{\dot A}1}}
\newcommand{\subtwo}{ {\vphantom{\dot A}2}}




\def\le{\leqslant}
\def\ge{\geqslant}


\def\var{{\rm Var}\,}

\newcommand{\ds}{\displaystyle}
\newcommand{\ts}{\textstyle}




\begin{document}
\setcounter{page}{2}

 
\section*{Section A: \ \ \ Pure Mathematics}

%%%%%%%%%%Q1
\begin{question}
Find the simultaneous solutions of the three linear equations
\begin{alignat*}{1}
a^{2}x+ay+z & =a^{2}\\
ax+y+bz & =1\\
a^{2}bx+y+bz & =b
\end{alignat*}
for all possible real values of $a$ and $b$. 
\end{question}

%%%%%%%%%%Q2
\begin{question}
If 
\[
\mathrm{I}_{n}=\int_{0}^{a}x^{n+\frac{1}{2}}(a-x)^{\frac{1}{2}}\,\mathrm{d}x,
\]
show that $\mathrm{I}_{0}=\pi a^{2}/8.$


Show that $(2n+4)\mathrm{I}_{n}=(2n+1)a\mathrm{I}_{n-1}$ and hence
evaluate $\mathrm{I}_{n}$. 
\end{question}

%%%%%%%%% Q3
\begin{question}
What is the general solution of the differential equation 
\[
\frac{\mathrm{d}^{2}x}{\mathrm{d}t^{2}}+2k\frac{\mathrm{d}x}{\mathrm{d}t}+x=0
\]
for each of the cases: (i) $k>1;$ (ii) $k=1$; (iii) $0<k<1$? 


In case (iii) the equation represents damped simple harmonic motion
with damping factor $k$. Let $x(0)=0$ and let $x_{1},x_{2},\ldots,x_{n},\ldots$
be the sequence of successive maxima and minima, so that if $x_{n}$
is a maximum then $x_{n+1}$ is the next minimum. Show that $\left|x_{n+1}/x_{n}\right|$
takes a value $\alpha$ which is independent of $n$, and that 
\[
k^{2}=\frac{(\ln\alpha)^{2}}{\pi^{2}+(\ln\alpha)^{2}}.
\]
\end{question}

%%%%%% Q4 
\begin{question}
Let 
\[
\mathrm{C}_{n}(\theta)=\sum_{k=0}^{n}\cos k\theta
\]
and let 
\[
\mathrm{S}_{n}(\theta)=\sum_{k=0}^{n}\sin k\theta,
\]
where $n$ is a positive integer and $0<\theta<2\pi.$ Show that 
\[
\mathrm{C}_{n}(\theta)=\frac{\cos(\tfrac{1}{2}n\theta)\sin\left(\frac{1}{2}(n+1)\theta\right)}{\sin(\frac{1}{2}\theta)},
\]
and obtain the corresponding expression for $\mathrm{S}_{n}(\theta)$. 


Hence, or otherwise, show that for $0<\theta<2\pi,$ 
\[
\left|\mathrm{C}_{n}(\theta)-\frac{1}{2}\right|\leqslant\frac{1}{2\sin(\frac{1}{2}\theta)}.
\]
	\end{question}

%%%%%%%%% Q5
\begin{question}
Show that $y=\sin^{2}(m\sin^{-1}x)$ satisfies the differential equation
\[
(1-x^{2})y^{(2)}=xy^{(1)}+2m^{2}(1-2y),
\]
and deduce that, for all $n\geqslant1,$ 
\[
(1-x^{2})y^{(n+2)}=(2n+1)xy^{(n+1)}+(n^{2}-4m^{2})y^{(n)},
\]
where $y^{(n)}$ denotes the $n$th derivative of $y$. 


Derive the Maclaurin series for $y$, making it clear what the general
term is. 
	\end{question}
	
%%%%%%%%% Q6
\begin{question}
The variable non-zero complex number $z$ is such that 
\[
\left|z-\mathrm{i}\right|=1.
\]
Find the modulus of $z$ when its argument is $\theta.$ Find also
the modulus and argument of $1/z$ in terms of $\theta$ and show
in an Argand diagram the loci of points which represent $z$ and $1/z$. 


Find the locus $C$ in the Argand diagram such that $w\in C$ if,
and only if, the real part of $(1/w)$ is $-1$. 
\end{question}
	
%%%%%%%%% Q7
\begin{question}
Consider the following sets with the usual definition of multiplication
appropriate to each. In each case you may assume that the multiplication
is associative. In each case state, giving adequate reasons, whether
or not the set is a group. 

\begin{itemize}
\setlength{\itemsep}{3mm}

\item[\bf (i)] the complex numbers of unit modulus; 
\item[\bf (ii)] the integers modulo 4; 
\item[\bf (iii)] the matrices 
\[
\mathrm{M}(\theta)=\begin{pmatrix}\cos\theta & -\sin\theta\\
\sin\theta & \cos\theta
\end{pmatrix},
\]
where $0\leqslant\theta<2\pi$; 
\item[\bf (iv)] the integers $1,3,5,7$ modulo 8; 
\item[\bf (v)] the $2\times2$ matrices all of whose entries are integers; 
\item[\bf (vi)] the integers $1,2,3,4$ modulo 5. 
\end{itemize}

In the case of each pair of groups above state, with reasons, whether
or not they are isomorphic. 

\end{question}
		
%%%%%%%%% Q8
\begin{question}	
A plane $\pi$ in 3-dimensional space is given by the vector equation
$\mathbf{r}\cdot\mathbf{n}=p,$ where $\mathbf{n}$ is a unit vector
and $p$ is a non-negative real number. If $\mathbf{x}$ is the position
vector of a general point $X$, find the equation of the normal to
$\pi$ through $X$ and the perpendicular distance of $X$ from $\pi$. 


The unit circles $C_{i},$ $i=1,2,$ with centres $\mathbf{r}_{i},$
lie in the planes $\pi_{i}$ given by $\mathbf{r}\cdot\mathbf{n}_{i}=p_{i},$
where the $\mathbf{n}_{i}$ are unit vectors, and $p_{i}$ are non-negative
real numbers. Prove that there is a sphere whose surface contains
both circles only if there is a real number $\lambda$ such that 
\[
\mathbf{r}_{1}+\lambda\mathbf{n}_{1}=\mathbf{r}_{2}\pm\lambda\mathbf{n}_{2}.
\]
Hence, or otherwise, deduce the necessary conditions that 
\[
(\mathbf{r}_{1}-\mathbf{r}_{2})\cdot(\mathbf{n}_{1}\times\mathbf{n}_{2})=0
\]
and that 
\[
(p_{1}-\mathbf{n}_{1}\cdot\mathbf{r}_{2})^{2}=(p_{2}-\mathbf{n}_{2}\cdot\mathbf{r}_{1})^{2}.
\]
Interpret each of these two conditions geometrically. 
\end{question}	
		

		
	
\newpage
\section*{Section B: \ \ \ Mechanics}


	
%%%%%%%%%% Q9
\begin{question}
 A thin circular disc of mass $m$, radius $r$ and with its centre
of mass at its centre $C$ can rotate freely in a vertical plane about
a fixed horizontal axis through a point $O$ of its circumference.
A particle $P$, also of mass $m,$ is attached to the circumference
of the disc so that the angle $OCP$ is $2\alpha,$ where $\alpha\leqslant\pi/2$. 

\begin{questionparts}
\item In the position of stable equilibrium $OC$ makes an angle $\beta$
with the vertical. Prove that 
\[
\tan\beta=\frac{\sin2\alpha}{2-\cos2\alpha}.
\]

\item The density of the disc at a point distant $x$ from $C$ is $\rho x/r.$
Show that its moment of inertia about the horizontal axis through
$O$ is $8mr^{2}/5$. 
\item The mid-point of $CP$ is $Q$. The disc is held at rest with $OQ$
horizontal and $C$ lower than $P$ and it is then released. Show
that the speed $v$ with which $C$ is moving when $P$ passes vertically
below $O$ is given by 
\[
v^{2}=\frac{15gr\sin\alpha}{2(2+5\sin^{2}\alpha)}.
\]
Find the maximum value of $v^{2}$ as $\alpha$ is varied. 
\end{questionparts}
	\end{question}
	
%%%%%%%%%% Q10
\begin{question}	
A cannon is situated at the bottom of a plane inclined at angle $\beta$
to the horizontal. A (small) cannon ball is fired from the cannon
at an initial speed $u.$ Ignoring air resistance, find the angle
of firing which will maximise the distance up the plane travelled
by the cannon ball and show that in this case the ball will land at
a distance 
\[
\frac{u^{2}}{g(1+\sin\beta)}
\]
from the cannon.
\end{question}

%%%%%%%%%% Q11

\begin{question}
A ship is sailing due west at $V$ knots while a plane, with an airspeed
of $kV$ knots, where $k>\sqrt{2},$ patrols so that it is always
to the north west of the ship. If the wind in the area is blowing
from north to south at $V$ knots and the pilot is instructed to return
to the ship every thirty minutes, how long will her outward flight
last? 


Assume that the maximum distance of the plane from the ship during
the above patrol was $d_{w}$ miles. If the air now becomes dead calm,
and the pilot's orders are maintained, show that the ratio $d_{w}/d_{c}$
of $d_{w}$ to the new maximum distance, $d_{c}$ miles, of the plane
from the ship is 
\[
\frac{k^{2}-2}{2k(k^{2}-1)}\sqrt{4k^{2}-2}.
\]
\end{question}
	

	
	\newpage
\section*{Section C: \ \ \ Probability and Statistics}


%%%%%%%%%% Q12
\begin{question}
The random variables $X$ and $Y$ are independently normally distributed
with means 0 and variances 1. Show that the joint probability density
function for $(X,Y)$ is 
\[
\mathrm{f}(x,y)=\frac{1}{2\pi}\mathrm{e}^{-\frac{1}{2}(x^{2}+y^{2})}\qquad-\infty<x<\infty,-\infty<y<\infty.
\]
If $(x,y)$ are the coordinates, referred to rectangular axes, of
a point in the plane, explain what is meant by saying that this density
is radially symmetrical. 


The random variables $U$ and $V$ have a joint probability density
function which is radially symmetrical (in the above sense). By considering
the straight line with equation $U=kV,$ or otherwise, show that 
\[
\mathrm{P}\left(\frac{U}{V}<k\right)=2\mathrm{P}(U<kV,V>0).
\]
Hence, or otherwise, show that the probability density function of
$U/V$ is 
\[
\mathrm{g}(k)=\frac{1}{\pi(1+k^{2})}\qquad-\infty<k<\infty.
\]
\end{question}

%%%%%%%%%% Q13
\begin{question}
A message of $10^{k}$ binary digits is sent along a fibre optic cable
with high probabilities $p_{0}$ and $p_{1}$ that the digits 0 and
1, respectively, are received correctly. If the probability of a digit
in the original message being a 1 is $\alpha,$ find the probability
that the entire message is received correctly. 


Find the probability $\beta$ that a randomly chosen digit in the
message is received as a 1 and show that $\beta=\alpha$ if, and only
if 
\[
\alpha=\frac{q_{0}}{q_{1}+q_{0}},
\]
where $q_{0}=1-p_{0}$ and $q_{1}=1-p_{1}.$ If this condition is
satisfied and the received message consists entirely of zeros, what
is the probability that it is correct? 


If now $q_{0}=q_{1}=q$ and $\alpha=\frac{1}{2},$ find the approximate
value of $q$ which will ensure that a message of one million binary
digits has a fifty-fifty chance of being received entirely correctly. 


The probability of error $q$ is proportional to the square of the
length of the cable. Initially the length is such that the probability
of a message of one million binary bits, among which 0 and 1 are equally
likely, being received correctly is $\frac{1}{2}.$ What wold this
probability become if a booster station were installed at its mid-point,
assuming that the booster station re-transmits the received version
of the message, and assuming that terms of order $q^{2}$ may be ignored? 
\end{question}

%%%%%%%%%% Q14
\begin{question}
A candidate finishes examination questions in time $T$, where $T$
has probability density function 
\[
\mathrm{f}(t)=t\mathrm{e}^{-t}\qquad t\geqslant0,
\]
the probabilities for the various questions being independent. Find
the moment generating function of $T$ and hence find the moment generating
function for the total time $U$ taken to finish two such questions.
Show that the probability density function for $U$ is 
\[
\mathrm{g}(u)=\frac{1}{6}u^{3}\mathrm{e}^{-u}\qquad u\geqslant0.
\]
Find the probability density function for the total time taken to
answer $n$ such questions.
\end{question}
	
\end{document}
