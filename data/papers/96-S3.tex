\documentclass[a4, 11pt]{report}


\pagestyle{myheadings}
\markboth{}{Paper III, 1996
\ \ \ \ \ 
\today 
}               

\RequirePackage{amssymb}
\RequirePackage{amsmath}
\RequirePackage{graphicx}
\RequirePackage{color}
\RequirePackage[flushleft]{paralist}[2013/06/09]



\RequirePackage{geometry}
\geometry{%
  a4paper,
  lmargin=2cm,
  rmargin=2.5cm,
  tmargin=3.5cm,
  bmargin=2.5cm,
  footskip=12pt,
  headheight=24pt}


\newcommand{\comment}[1]{{\bf Comment} {\it #1}}
%\renewcommand{\comment}[1]{}

\newcommand{\bluecomment}[1]{{\color{blue}#1}}
%\renewcommand{\comment}[1]{}
\newcommand{\redcomment}[1]{{\color{red}#1}}



\usepackage{epsfig}
\usepackage{pstricks-add}
\usepackage{tgheros} %% changes sans-serif font to TeX Gyre Heros (tex-gyre)
\renewcommand{\familydefault}{\sfdefault} %% changes font to sans-serif
%\usepackage{sfmath}  %%%% this makes equation sans-serif
%\input RexFigs


\setlength{\parskip}{10pt}
\setlength{\parindent}{0pt}

\newlength{\qspace}
\setlength{\qspace}{20pt}


\newcounter{qnumber}
\setcounter{qnumber}{0}

\newenvironment{question}%
 {\vspace{\qspace}
  \begin{enumerate}[\bfseries 1\quad][10]%
    \setcounter{enumi}{\value{qnumber}}%
    \item%
 }
{
  \end{enumerate}
  \filbreak
  \stepcounter{qnumber}
 }


\newenvironment{questionparts}[1][1]%
 {
  \begin{enumerate}[\bfseries (i)]%
    \setcounter{enumii}{#1}
    \addtocounter{enumii}{-1}
    \setlength{\itemsep}{5mm}
    \setlength{\parskip}{8pt}
 }
 {
  \end{enumerate}
 }



\DeclareMathOperator{\cosec}{cosec}
\DeclareMathOperator{\Var}{Var}

\def\d{{\rm d}}
\def\e{{\rm e}}
\def\g{{\rm g}}
\def\h{{\rm h}}
\def\f{{\rm f}}
\def\p{{\rm p}}
\def\s{{\rm s}}
\def\t{{\rm t}}


\def\A{{\rm A}}
\def\B{{\rm B}}
\def\E{{\rm E}}
\def\F{{\rm F}}
\def\G{{\rm G}}
\def\H{{\rm H}}
\def\P{{\rm P}}


\def\bb{\mathbf b}
\def \bc{\mathbf c}
\def\bx {\mathbf x}
\def\bn {\mathbf n}

\newcommand{\low}{^{\vphantom{()}}}
%%%%% to lower suffices: $X\low_1$ etc


\newcommand{\subone}{ {\vphantom{\dot A}1}}
\newcommand{\subtwo}{ {\vphantom{\dot A}2}}




\def\le{\leqslant}
\def\ge{\geqslant}


\def\var{{\rm Var}\,}

\newcommand{\ds}{\displaystyle}
\newcommand{\ts}{\textstyle}




\begin{document}
\setcounter{page}{2}

 
\section*{Section A: \ \ \ Pure Mathematics}

%%%%%%%%%%Q1
\begin{question}
Define $\cosh x$ and $\sinh x$ in terms of exponentials and prove,
from your definitions, that 
\[
\cosh^{4}x-\sinh^{4}x=\cosh2x
\]
and 
\[
\cosh^{4}x+\sinh^{4}x=\tfrac{1}{4}\cosh4x+\tfrac{3}{4}.
\]
Find $a_{0},a_{1},\ldots,a_{n}$ in terms of $n$ such that 
\[
\cosh^{n}x=a_{0}+a_{1}\cosh x+a_{2}\cosh2x+\cdots+a_{n}\cosh nx.
\]
Hence, or otherwise, find expressions for $\cosh^{2m}x-\sinh^{2m}x$
and $\cosh^{2m}x+\sinh^{2m}x,$ in terms of $\cosh kx,$ where $k=0,\ldots,2m.$ 
\end{question}

%%%%%%%%%%Q2
\begin{question}
For all values of $a$ and $b,$ either solve the simultaneous equations
\begin{alignat*}{1}
x+y+az & =2\\
x+ay+z & =2\\
2x+y+z & =2b
\end{alignat*}
or prove that they have no solution. 
\end{question}

%%%%%%%%% Q3
\begin{question}
Find 
\[
\int_{0}^{\theta}\frac{1}{1-a\cos x}\,\mathrm{d}x\,,
\]
where $0<\theta<\pi$ and $-1<a<1.$


Hence show that 
\[
\int_{0}^{\frac{1}{2}\pi}\frac{1}{2-a\cos x}\,\mathrm{d}x=\frac{2}{\sqrt{4-a^{2}}}\tan^{-1}\sqrt{\frac{2+a}{2-a}}\,,
\]
and also that 
\[
\int_{0}^{\frac{3}{4}\pi}\frac{1}{\sqrt{2}+\cos x}\,\mathrm{d}x=\frac{\pi}{2}\,.
\]
\end{question}

%%%%%% Q4 
\begin{question}
Find the integers $k$ satisfying the inequality $k\leqslant2(k-2).$


Given that $N$ is a strictly positive integer consider the problem
of finding strictly positive integers whose sum is $N$ and whose
product is as large as possible. Call this largest possible product
$P(N).$ Show that $P(5)=2\times3,P(6)=3^{2},P(7)=2^{2}\times3,P(8)=2\times3^{2}$ and
$P(9)=3^{3}.$


Find $P(1000)$ explaining your reasoning carefully. 
	\end{question}

%%%%%%%%% Q5
\begin{question}
Show, using de Moivre's theorem, or otherwise, that 
\[
\tan7\theta=\frac{t(t^{6}-21t^{4}+35t^{2}-7)}{7t^{6}-35t^{4}+21t^{2}-1}\,,
\]
where $t=\tan\theta.$

\begin{questionparts}
\item By considering the equation $\tan7\theta=0,$ or otherwise, obtain
a cubic equation with integer coefficients whose roots are 
\[
\tan^{2}\left(\frac{\pi}{7}\right),\ \tan^{2}\left(\frac{2\pi}{7}\right)\ \mbox{ and }\tan^{2}\left(\frac{3\pi}{7}\right)
\]
and deduce the value of 
\[
\tan\left(\frac{\pi}{7}\right)\tan\left(\frac{2\pi}{7}\right)\tan\left(\frac{3\pi}{7}\right)\,.
\]

\item Find, without using a calculator, the value of 
\[
\tan^{2}\left(\frac{\pi}{14}\right)+\tan^{2}\left(\frac{3\pi}{14}\right)+\tan^{2}\left(\frac{5\pi}{14}\right)\,.
\]

\end{questionparts}
	\end{question}
	
%%%%%%%%% Q6
\begin{question}
\begin{questionparts}  \item Let $S$ be the set of matrices of the form
\[
\begin{pmatrix}a & a\\
a & a
\end{pmatrix},
\]
where $a$ is any real non-zero number. Show that $S$ is closed under
matrix multiplication and, further, that $S$ is a group under matrix
multiplication. 


\item Let $G$ be a set of $n\times n$ matrices which is a group
under matrix multiplication, with identity element $\mathbf{E}.$
By considering equations of the form $\mathbf{BC=D}$ for suitable
elements $\mathbf{B},$ $\mathbf{C}$ and $\mathbf{D}$ of $G$, show
that if a given element $\mathbf{A}$ of $G$ is a singular matrix
(i.e. $\det\mathbf{A}=0$), then all elements of $G$ are singular.
Give, with justification, an example of such a group of singular matrices
in the case $n=3.$ \end{questionparts}


\end{question}
	
%%%%%%%%% Q7
\begin{question}
\begin{questionparts}
	 \item If $x+y+z=\alpha,$ $xy+yz+zx=\beta$ and
$xyz=\gamma,$ find numbers $A,B$ and $C$ such that 
\[
x^{3}+y^{3}+z^{3}=A\alpha^{3}+B\alpha\beta+C.
\]
Solve the equations 
\begin{alignat*}{1}
x+y+z & =1\\
x^{2}+y^{2}+z^{2} & =3\\
x^{3}+y^{3}+z^{3} & =4.
\end{alignat*}



\item The area of a triangle whose sides are $a,b$ and $c$ is given
by the formula 
\[
\mathrm{area}=\sqrt{s(s-a)(s-b)(s-c)}
\]
where $s$ is the semi-perimeter $\frac{1}{2}(a+b+c).$ If $a,b$
and $c$ are the roots of the equation 
\[
x^{3}-16x^{2}+81x-128=0,
\]
find the area of the triangle. 
\end{questionparts}
\end{question}
		
%%%%%%%%% Q8
\begin{question}	
A transformation $T$ of the real numbers is defined by 
\[
y=T(x)=\frac{ax-b}{cx-d}\,,
\]
where $a,b,c$, $d$ are real numbers such that $ad\neq bc$. Find
all numbers $x$ such that $T(x)=x.$ Show that the inverse operation,
$x=T^{-1}(y)$ expressing $x$ in terms of $y$ is of the same form
as $T$ and find corresponding numbers $a',b',c'$,$d'$. 


Let $S_{r}$ denote the set of all real numbers excluding $r$. Show
that, if $c\neq0,$ there is a value of $r$ such that $T$ is defined
for all $x\in S_{r}$ and find the image $T(S_{r}).$ What is the
corresponding result if $c=0$? 


If $T_{1},$ given by numbers $a_{1},b_{1},c_{1},d_{1},$ and $T_{2},$
given by numbers $a_{2},b_{2},c_{2},d_{2}$ are two such transformations,
show that their composition $T_{3},$ defined by $T_{3}(x)=T_{2}(T_{1}(x)),$
is of the same form. 


Find necessary and sufficient conditions on the numbers $a,b,c,d$
for $T^{2}$, the composition of $T$ with itself, to be the identity.
Hence, or otherwise, find transformations $T_{1},T_{2}$ and their
composition $T_{3}$ such that $T_{1}^{2}$ and $T_{2}^{2}$ are each
the identity but $T_{3}^{2}$ is not. 
\end{question}	
		

		
	
\newpage
\section*{Section B: \ \ \ Mechanics}


	
%%%%%%%%%% Q9
\begin{question}
A particle of mass $m$ is at rest on top of a smooth fixed sphere of radius
$a$. Show that, if the particle is given a small displacement, it reaches
the horizontal plane through the centre of the sphere at a distance
% at least
$$a(5\sqrt5+4\sqrt23)/27$$
from the centre of the sphere.

[Air resistance should be neglected.]
	\end{question}
	
%%%%%%%%%% Q10
\begin{question}	
Two rough solid circular cylinders, of equal radius and length and of uniform
density, lie side by side on a rough plane inclined at an angle $\alpha$
to the horizontal, where $0<\alpha<\pi/2$. Their axes are horizontal and
they touch along their entire length. The weight of the upper cylinder is
$W_1$ and the coefficient of friction between it and the plane
is $\mu_1$. The corresponding qunatities for the lower cylinder are $W_2$
and $\mu_2$ respectively and the coefficient of friction between
the two cylinders is $\mu$. Show that for equilibrium to be possible:

\begin{questionparts}
\item $W_1\ge W_2$;

\item $\mu\geqslant\dfrac{W_1+W_2}{W_1-W_2}$;

\item $\mu_{1}\geqslant\left(\dfrac{2W_{1}\cot\alpha}{W_{1}+W_{2}}-1\right)^{-1}\,.$
\end{questionparts}

Find the similar inequality to \textbf{ (iii)} for $\mu_2$.
\end{question}

%%%%%%%%%% Q11

\begin{question}
A smooth circular wire of radius $a$
is held fixed in a vertical plane with light elastic
strings of natural length $a$ and modulus $\lambda$ attached to the upper and
lower extremities, $A$ and $C$ respectively, of the vertical diameter.
The other ends of the two strings are attached to a small ring $B$ which
is free to slide on the wire. Show that, while both strings remain taut,
the equation for the motion of the ring is
$$2ma \ddot\theta=\lambda(\cos\theta-\sin\theta)-mg\sin\theta,$$
where $\theta$ is the angle $ \angle{CAB}$.

Initially the system is at rest in equilibrium with
$\sin\theta=\frac{3}{5}$. Deduce that $5\lambda=24mg$.

The ring is now displaced slightly. Show that, in the ensuing motion, it will
oscillate with period approximately
$$10\pi\sqrt{a\over91g}\,.$$
\end{question}
	

	
	\newpage
\section*{Section C: \ \ \ Probability and Statistics}


%%%%%%%%%% Q12
\begin{question}

It has been observed that Professor Ecks proves three types of theorems:
1, those that are correct and new; 2, those that are correct, but
already known; 3, those that are false. It has also been observed
that, if a certain of her theorems is of type $i$, then her next
theorem is of type $j$ with probability $p\low_{ij},$ where $p\low_{ij}$
is the entry in the $i$th row and $j$th column of the following
array: 
\[
\begin{pmatrix}0.3 & 0.3 & 0.4\\
0.2 & 0.4 & 0.4\\
0.1 & 0.3 & 0.6
\end{pmatrix}\,.
\]
Let $a_{i},$ $i=1,2,3$, be the probability that a given theorem
is of type $i$, and let $b_{j}$ be the consequent probability that
the next theorem is of type $j$. 

\begin{questionparts}
\item Explain why $b_{j}=a\low_{1}p\low_{1j}+a\low_{2}p\low_{2j}+a\low_{3}p\low_{3j}\,.$
\item Find values of $a\low_{1},a\low_{2}$ and $a\low_{3}$ such that $b_{i}=a_{i}$
for $i=1,2,3.$
\item For these values of the $a_{i}$ find the probabilities $q\low_{ij}$
that, if a particular theorem is of type $j$, then the \textit{preceding
}theorem was of type $i$. 
\end{questionparts}
\end{question}

%%%%%%%%%% Q13
\begin{question}
Let $X$ be a random variable which takes only the finite number of
different possible real values $x_{1},x_{2},\ldots,x_{n}.$ Define
the expectation $\mathrm{E}(X)$ and the variance $\mathrm{var}(X)$
of $X$. Show that , if $a$ and $b$ are real numbers, then $\mathrm{E}(aX+b)=a\mathrm{E}(X)+b$
and express $\mathrm{var}(aX+b)$ similarly in terms of $\mathrm{var}(X)$. 


Let $\lambda$ be a positive real number. By considering the contribution
to $\mathrm{var}(X)$ of those $x_{i}$ for which $\left|x_{i}-\mathrm{E}(X)\right|\geqslant\lambda,$
or otherwise, show that 
\[
\mathrm{P}\left(\left|X-\mathrm{E}(X)\right|\geqslant\lambda\right)\leqslant\frac{\mathrm{var}(X)}{\lambda^{2}}\,.
\]
Let $k$ be a real number satisfying $k\geqslant\lambda.$ If $\left|x_{i}-\mathrm{E}(X)\right|\leqslant k$
for all $i$, show that 
\[
\mathrm{P}\left(\left|X-\mathrm{E}(X)\right|\geqslant\lambda\right)\geqslant\frac{\mathrm{var}(X)-\lambda^{2}}{k^{2}-\lambda^{2}}\,.
\]
\end{question}

%%%%%%%%%% Q14
\begin{question}
Whenever I go cycling I start with my bike in good working order.
However if all is well at time $t$, the probability that I get a
puncture in the small interval $(t,t+\delta t)$ is $\alpha\,\delta t.$
How many punctures can I expect to get on a journey during which my
total cycling time is $T$? 


When I get a puncture I stop immediately to repair it and the probability
that, if I am repairing it at time $t$, the repair will be completed
in time $(t,t+\delta t)$ is $\beta\,\delta t.$ If $p(t)$ is the
probability that I am repairing a puncture at time $t$, write down
an equation relating $p(t)$ to $p(t+\delta t)$, and derive from
this a differential equation relating $p'(t)$ and $p(t).$ Show that
\[
p(t)=\frac{\alpha}{\alpha+\beta}(1-\mathrm{e}^{-(\alpha+\beta)t})
\]
satisfies this differential equation with the appropriate initial
condition. 


Find an expression, involving $\alpha,\beta$ and $T$, for the time
expected to be spent mending punctures during a journey of total time
$T$. Hence, or otherwise, show that, the fraction of the journey
expected to be spent mending punctures is given approximately by 
\[
\quad\frac{\alpha T}{2}\quad\ \mbox{ if }(\alpha+\beta)T\text{ is small, }
\]
and by 
\[
\frac{\alpha}{\alpha+\beta}\quad\mbox{ if }(\alpha+\beta)T\text{ is large.}
\]

\end{question}
	
\end{document}
