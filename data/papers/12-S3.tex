\documentclass[a4, 11pt]{report}


\pagestyle{myheadings}
\markboth{}{Paper III, 2012
\ \ \ \ \ 
\today 
}               

\RequirePackage{amssymb}
\RequirePackage{amsmath}
\RequirePackage{graphicx}
\RequirePackage{color}
\RequirePackage[flushleft]{paralist}[2013/06/09]



\RequirePackage{geometry}
\geometry{%
  a4paper,
  lmargin=2cm,
  rmargin=2.5cm,
  tmargin=3.5cm,
  bmargin=2.5cm,
  footskip=12pt,
  headheight=24pt}


\newcommand{\comment}[1]{{\bf Comment} {\it #1}}
%\renewcommand{\comment}[1]{}

\newcommand{\bluecomment}[1]{{\color{blue}#1}}
%\renewcommand{\comment}[1]{}
\newcommand{\redcomment}[1]{{\color{red}#1}}



\usepackage{epsfig}
\usepackage{pstricks-add}
\usepackage{tgheros} %% changes sans-serif font to TeX Gyre Heros (tex-gyre)
\renewcommand{\familydefault}{\sfdefault} %% changes font to sans-serif
%\usepackage{sfmath}  %%%% this makes equation sans-serif
%\input RexFigs


\setlength{\parskip}{10pt}
\setlength{\parindent}{0pt}

\newlength{\qspace}
\setlength{\qspace}{20pt}


\newcounter{qnumber}
\setcounter{qnumber}{0}

\newenvironment{question}%
 {\vspace{\qspace}
  \begin{enumerate}[\bfseries 1\quad][10]%
    \setcounter{enumi}{\value{qnumber}}%
    \item%
 }
{
  \end{enumerate}
  \filbreak
  \stepcounter{qnumber}
 }


\newenvironment{questionparts}[1][1]%
 {
  \begin{enumerate}[\bfseries (i)]%
    \setcounter{enumii}{#1}
    \addtocounter{enumii}{-1}
    \setlength{\itemsep}{5mm}
    \setlength{\parskip}{8pt}
 }
 {
  \end{enumerate}
 }



\DeclareMathOperator{\cosec}{cosec}
\DeclareMathOperator{\Var}{Var}

\def\d{{\mathrm d}}
\def\e{{\mathrm e}}
\def\g{{\mathrm g}}
\def\h{{\mathrm h}}
\def\f{{\mathrm f}}
\def\p{{\mathrm p}}
\def\q{{\mathrm q}}
\def\s{{\mathrm s}}
\def\t{{\mathrm t}}


\def\A{{\mathrm A}}
\def\B{{\mathrm B}}
\def\E{{\mathrm E}}
\def\F{{\mathrm F}}
\def\G{{\mathrm G}}
\def\H{{\mathrm H}}
\def\P{{\mathrm P}}


\def\bb{\mathbf b}
\def \bc{\mathbf c}
\def\bx {\mathbf x}
\def\bn {\mathbf n}

\newcommand{\low}{^{\vphantom{()}}}
%%%%% to lower suffices: $X\low_1$ etc


\newcommand{\subone}{ {\vphantom{\dot A}1}}
\newcommand{\subtwo}{ {\vphantom{\dot A}2}}




\def\le{\leqslant}
\def\ge{\geqslant}
\def\arcosh{{\rm arcosh}\,}


\def\var{{\rm Var}\,}

\newcommand{\ds}{\displaystyle}
\newcommand{\ts}{\textstyle}
\def\half{{\textstyle \frac12}}
\def\l{\left(}
\def\r{\right)}



\begin{document}
\setcounter{page}{2}

 
\section*{Section A: \ \ \ Pure Mathematics}

%%%%%%%%%%Q1
\begin{question}
Given that $\displaystyle z = y^n \left( \frac{\d y}{\d x}\right)^{\!2}$, show that
\[
\frac{\d z}{\d x} = 
y^{n-1} \frac{\d y}{\d x} \left( n \left(\frac{\d y}{\d x}\right)^{\!2} + 2y \frac{\d^2y}{\d x^2}\right)
.
\]
\begin{questionparts}
\item
Use the above result to show that the solution to the equation
\[
\left(\frac{\d y}{\d x}\right)^{\!2}
+ 2y \frac{\d^2y}{\d x^2} = \sqrt y \ \ \ \ \ \ \ \ \ \ (y>0)
\]
that satisfies $y=1$ and $\dfrac{\d y}{\d x}=0$ when $x=0$ is 
$y= \big  ( \frac38 x^2+1\big)^{\frac23}$.
\item
Find the solution to the equation
\[
\left(\frac{\d y}{\d x}\right)^{\!2}
 -y \frac{\d^2y}{\d x^2} + y^2=0 
\]
that satisfies $y=1$ and $\dfrac{\d y}{\d x}=0$ when $x=0$.


\end{questionparts}
\end{question}
\vspace{-0.5cm}

%%%%%%%%%%Q2
\begin{question}
In this question, $\vert x \vert <1$ and you may ignore issues of convergence.

\begin{questionparts}

\item
Simplify 
\[
 (1-x)(1+x)(1+x^2)(1+x^4) \cdots (1+x^{2^n})\,,
\]
where $n$ is a positive integer,
and deduce that
\[
\frac1{1-x} 
= (1+x)(1+x^2)(1+x^4) \cdots (1+x^{2^n}) + \frac {x^{2^{n+1}}}{1-x}\,.
\]
 Deduce further that
\[
\ln(1-x) = - \sum_{r=0}^\infty \ln \left  (1+ x ^{2^r}\right)
\,,
\]
and hence that 
\[
\frac1 {1-x} = \frac 1 {1+x} + \frac {2x}{1+x^2} + \frac {4x^3}{1+x^4}
+\cdots\,.
\]

\item
Show that

\[
\frac{1+2x}{1+x+x^2} = \frac{1-2x}{1-x+x^2} + \frac{2x-4x^3}{1-x^2+x^4}
+ \frac {4x^3-8x^7}{1-x^4+x^8} + \cdots\,.
\]

\end{questionparts}
\end{question}

%%%%%%%%% Q3
\begin{question}
It is given that the two curves
\[
y=4-x^2
\text{ \ \ \ and \ \ \ }
m  x = k-y^2\,,
\]
where $m>0$, 
touch exactly once. 

\begin{questionparts}
\item
In each of the following four cases, sketch the two curves on a single
diagram, noting the coordinates of any intersections with the axes:

\textbf{(a)} $k <0\, $;

\textbf{(b)} $0<k<16$, $k/m < 2\,$;

\textbf{(c)} $k>16$, $k/m >2\,$;

\textbf{(d)} $k>16$, $k/m <2\,$.


 

\item
Now set $m=12$.\\
Show that the $x$-coordinate of any point at which
the two curves meet satisfies
\[
x^4-8x^2 +12x +16-k=0\,.
\]

 Let $a$ be the value of $x$ at the point where the
curves touch. Show that $a$ satisfies
\[
a^3 -4a +3 =0
\]
and hence find the three possible values of $a$.

Derive also the equation
\[
k= -4a^2 +9a +16\,.
\]
Which of the four sketches in  
part (i) arise?

\end{questionparts}
\end{question}

%%%%%% Q4 
\begin{question}
\begin{questionparts}
\item
 Show that
\[
\sum_{n=1} ^\infty 
\frac{n+1}{n!} 
=  2\e - 1 
\]
and  
\[
\sum _{n=1}^\infty 
\frac {(n+1)^2}{n!} = 5\e-1\,.
\]
Sum the series   \ $\displaystyle 
\sum _{n=1}^\infty 
\frac {(2n-1)^3}{n!} 
 \,.$ 
\item
Sum the series
$\displaystyle 
\sum_{n=0}^\infty \frac{(n^2+1)2^{-n}}{(n+1)(n+2)}\,$,
giving your answer in terms of natural logarithms.
\end{questionparts}
\end{question}

%%%%%%%%% Q5
\begin{question}
\begin{questionparts}
\item The point with coordinates $(a, b)$, 
where $a$ and $b$  are rational numbers,
  is called:
\newline
\hspace*{1cm} 
an {\em integer rational point} if both $a$ and $b$ are integers;
\newline\hspace*{1cm}
a {\em non-integer rational point} if neither $a$ nor $b$ is an integer.





\begin{itemize}

\item[\bf (a)]
Write down an integer rational point and a non-integer rational point
on the circle $x^2+y^2 =1$.

\item [\bf (b)]
 Write down an integer rational point on the circle
$x^2+y^2=2$.
Simplify
\[
(\cos\theta + \sqrt m \sin\theta)^2 +
(\sin\theta - \sqrt m \cos\theta)^2 \,
\]
and hence obtain a non-integer rational point on the circle $x^2+y^2=2\,$.
\end{itemize}

\item
The point with coordinates $(p+\sqrt 2 \, q\,,\, r+\sqrt 2 \, s)$, where
 $p$, $q$, $r$ and $s$ 
are rational numbers, is called:
\newline\hspace*{1cm}
an {\em integer $2$-rational point} if all of 
 $p$, $q$, $r$ and $s$ 
 are integers;
\newline\hspace*{1cm}
a {\em non-integer $2$-rational point} if none of  
 $p$, $q$, $r$ and $s$ 
 is an integer.


\begin{itemize}
\item[\bf (a)]
Write down an  integer $2$-rational point, and 
obtain a non-integer $2$-rational point, on  the circle $x^2+y^2=3\,$.

\item [\bf(b)]
Obtain a non-integer $2$-rational point 
on the circle $x^2+y^2=11\,$.

\item [\bf(c)]Obtain a  non-integer $2$-rational point 
on the hyperbola $x^2-y^2 =7 $.

\end{itemize}

\end{questionparts}
	\end{question}
	
%%%%%%%%% Q6
\begin{question}
Let  $x+{\rm i} y$ be a root of 
the quadratic equation $z^2 + pz +1=0$, where 
$p$ is a real number. Show that $x^2-y^2 +px+1=0$ and $(2x+p)y=0$.
Show further that 
\[
\text{\ either \ } p=-2x \text{ \ \ or \ \ } p=-(x^2+1)/x \text{ \ with  \ }
x\ne0 \,.
\]

Hence show that the set of  points in the Argand diagram 
that can (as $p$ varies)
represent roots 
of the quadratic equation consists of the real axis with
one point missing and a circle.
This set of points is called
the {\em root locus} of the 
quadratic equation.

Obtain and sketch in the Argand diagram the root locus of the
equation
\[
pz^2 +z+1=0\,
\]
and the root locus of the equation
\[
pz^2 + p^2z +2=0\,.
\]
\end{question}
	
%%%%%%%%% Q7
\begin{question}
A pain-killing drug is injected into the bloodstream. It 
then diffuses into the brain, where it is absorbed. 
The quantities at time $t$ of the drug in the blood
and the brain respectively are $y(t)$
and $z(t)$.  These satisfy 
\[
\dot y = - 2(y-z)\,, 
\ \ \ \ \ \ \ 
\dot z = - \dot y -3z\, ,
\]
where the dot denotes differentiation with respect to $t$.

Obtain a second order differential equation for $y$ and hence derive
the solution
\[
y= A\e^{-t} + B\e ^{-6t}\,,
\ \ \ \ \ \ \
z= \tfrac12 A \e^{-t} - 2 B \e^{-6t}\,,
\] 
where $A$ and $B$ are arbitrary constants.


\begin{questionparts}
\item Obtain the solution that satisfies 
$z(0)=0$ and $y(0)=  5$. The quantity
of the drug in the brain for this solution is
denoted by $z_1(t)$.

\item Obtain the  solution that satisfies
$ 
z(0)=z(1)= c$,
where $c$ is a given constant.
%\[
%C=2(1-\e^{-1})^{-1} - 2(1-\e^{-6})^{-1}\,.
%\]
 The quantity
of the drug in the brain for this solution is
denoted by $z_2(t)$.

\item Show that for $0\le t \le 1$, 
\[
z_2(t) = \sum _{n=-\infty}^{0} z_1(t-n)\,,
\]
provided $c$ takes a particular value that you should find.

 
\end {questionparts}
\end{question}
		
%%%%%%%%% Q8
\begin{question}
The sequence $F_0$, $F_1$, $F_2$, $\ldots\,$ is defined
by $F_0=0$, $F_1=1$ and, for $n\ge0$,
\[
F_{n+2} = F_{n+1} + F_n
\,.
\]

\begin{questionparts}
\item Show that $F_0F_3-F_1F_2 = F_2F_5- F_3F_4\,$.
\item
Find the values of $F_nF_{n+3} - F_{n+1}F_{n+2}$
in the two cases that arise.
\item 
Prove that, for $r=1$, $2$, $3$, $\ldots\,$, 
\[
\arctan \left( \frac 1{F_{2r}}\right)
=\arctan \left( \frac 1{F_{2r+1}}\right)+
\arctan \left( \frac 1{F_{2r+2}}\right)
\]
and hence evaluate the following sum (which you may assume converges): 
\[
\sum_{r=1}^\infty \arctan \left( \frac 1{F_{2r+1}}\right)
\,.
\]
\end{questionparts}
\end{question}	
		

		
	
\newpage
\section*{Section B: \ \ \ Mechanics}


	
%%%%%%%%%% Q9
\begin{question}
A pulley consists of a disc of radius $r$
with centre $O$
and a light thin axle through $O$ perpendicular
to the plane of the disc. The disc is non-uniform,
its  mass  is $M$  and its
centre of mass is at $O$. The
 axle is fixed and horizontal.


Two particles, of masses $m_1$ and $m_2$ where $m_1>m_2$,
are connected by a light inextensible string which passes over
the pulley. 
The contact between the string 
and the pulley is rough enough to prevent the string sliding. 
The pulley turns and  the vertical force on the axle is
 found, by measurement, to be~$P+Mg$. 
\begin{questionparts}
\item
The moment of inertia of the pulley about its axle is calculated 
assuming that the pulley rotates without friction about its axle. 
Show that the calculated value is
\[
\frac{((m_1 + m_2)P - 4m_1m_2g)r^2}
{(m_1 + m_2)g - P}\,.
\tag{$*$}\]
 \item
Instead, the moment of inertia of the pulley about its axle
is calculated
assuming that a couple of magnitude $C$ due to      
friction acts on the axle of the pulley. 
Determine whether
this calculated value  is greater or 
smaller than $(*)$.

Show that $C<(m_1-m_2)rg$.
\end{questionparts}
	\end{question}
	
%%%%%%%%%% Q10 
\begin{question}	
A small ring of mass $m$ 
is free to slide without friction on  a hoop of radius $a$. 
The hoop is fixed in a vertical plane. 
The ring is connected by a light elastic string of natural length $a$ 
to the highest point of the hoop. 
The ring is initially at rest at the lowest 
point of the hoop and is then slightly displaced. 
In the subsequent motion the  
angle of the string to the downward vertical is $\phi$.
Given that the ring first comes to rest just as the string becomes slack,
find an expression for the modulus of elasticity of the string in terms
of $m$ and  $g$.
 
Show that, throughout the motion, the magnitude $R$  of the reaction
between the ring and the hoop is given by
\[
R  = ( 12\cos^2\phi -15\cos\phi +5) mg
\]
and 
that $R$ is non-zero throughout the motion.
\end{question}

%%%%%%%%%% Q11

\begin{question}
One end of a thin heavy uniform inextensible perfectly flexible rope of length $2L$
and mass $2M$
is attached to a fixed point $P$. A particle of mass $m$ is 
attached to the other end. Initially, the particle is held at
$P$ and the rope hangs vertically in a loop below $P$. The particle is then released
so that it and a section of the rope (of decreasing length)
fall vertically as shown in the diagram. 

\begin{center}
\psset{xunit=1.0cm,yunit=0.9cm,algebraic=true,dimen=middle,dotstyle=o,dotsize=3pt 0,linewidth=0.3pt,arrowsize=3pt 2,arrowinset=0.25}
\begin{pspicture*}(0.13,-0.26)(3.26,5.51)
\psline(1,5)(3,5)
\psline[linewidth=0.1pt,linestyle=dashed,dash=2pt 2pt]{<->}(1.52,0)(1.52,5)
\psline[linewidth=0.1pt,linestyle=dashed,dash=2pt 2pt]{<->}(2.53,3.2)(2.53,5)
\psline(2.1,3.18)(2.06,0.25)
\psline(2,5)(2.02,0.26)
\psline(2.02,0.26)(2.03,0)
\psline(2.03,0)(2.06,0.25)
\rput[tl](1.94,5.45){$P$}
\rput[tl](2.6,4.25){$x$}
\rput[tl](0.2,2.85){$L+\frac{1}{2}x$}
\begin{scriptsize}
\psdots[dotsize=4pt 0,dotstyle=*](2.1,3.18)
\end{scriptsize}
\end{pspicture*}
\end{center}

You may assume that
each point on the moving section of the rope falls at the same speed as the 
particle. Given that energy is conserved, show 
that, when  the particle has fallen a distance $x$ (where $x<   2L$),
its speed $v$ is given by
\[
v^2 = \frac { 2g x \big( mL +ML - \frac14 Mx)}{mL +ML - \frac12 Mx}\,.
\]  

Hence show that the acceleration of the particle is 
\[
 g +
 \frac{ Mgx\big(mL+ML- \frac14 Mx\big)}{2\big(mL +ML -\frac12 Mx\big)^2}\,
\,.\] 
Deduce that the acceleration of the particle after it is 
released is greater than $g$.
\end{question}
	

	
	\newpage
\section*{Section C: \ \ \ Probability and Statistics}


%%%%%%%%%% Q12
\begin{question}
\begin{questionparts}
\item A point $P$ lies in an equilateral 
triangle $ABC$ of height 1. The perpendicular
distances from $P$ to the sides $AB$, $BC$ and $CA$ are
$x_1$, $x_2$ and $x_3$, respectively. By considering the
areas of triangles with one vertex at $P$, show
that  $x_1+x_2+x_3=1$.

Suppose now that $P$ is placed at random in the equilateral triangle
(so that the probability of it lying in any given region of the triangle is
proportional to the area of that region).  The perpendicular
distances from $P$ to the sides $AB$, $BC$ and $CA$ are
random variables $X_1$, $X_2$ and $X_3$, respectively.
In the case $X_1= \min(X_1,X_2,X_3)$, give a sketch showing
the region of the triangle in which $P$ lies. 

Let $X= \min(X_1,X_2,X_3)$. Show that
the probability density function for $X$ is 
given by
\[
\f(x) = 
\begin{cases}
6(1-3x) & 0 \le x \le \frac13\,, \\
0 & \text{otherwise}\,.
\end{cases}
\]
Find the expected value of $X$.

\item
A point is chosen at random in a regular tetrahedron of height 1.
Find the expected value of the distance from the point to the 
closest face.
\newline
[The volume of a tetrahedron is 
$\frac13 \times \text{area of base}\times\text{height}$ and its centroid
is a distance $\frac14\times \text{height}$ from the base.]

\end{questionparts}
\end{question}

%%%%%%%%%% Q13
\begin{question}
\begin{questionparts}
\item The random variable $Z$ has a 
Normal distribution with mean $0$ and variance $1$. 
Show that the expectation of $Z$ given that $a < Z < b$ is
\[
 \frac{\exp(- \frac12 a^2) - \exp(- \frac12 b^2) }
{\sqrt{2\pi\,} \,\big(\Phi(b) - \Phi(a)\big)},
\]
where $\Phi$ denotes the cumulative distribution function for $Z$.

\item The random variable $X$ 
has a
 Normal distribution with mean $\mu$ and variance $\sigma^2$. 
Show that 
\[
\E(X \,\vert\, X>0) = \mu + \sigma \E(Z \,\vert\,Z > -\mu/\sigma).
\]
Hence, or otherwise, 
show that the expectation, $m$, of $\vert X\vert $ is given by			
\[
m= 
\mu \big(1 - 2 \Phi(- \mu / \sigma)\big)
+
 \sigma \sqrt{2 / \pi}\; \exp(- \tfrac12 \mu^2 / \sigma^2) 
\,.
\]
Obtain an expression for the variance 
of $\vert X \vert$ in terms of $\mu $, $\sigma $  and $m$.
\end{questionparts} 
\end{question}

\end{document}
