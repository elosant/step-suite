\documentclass[a4, 11pt]{report}


\pagestyle{myheadings}
\markboth{}{Paper III, 1999
\ \ \ \ \ 
\today 
}               

\RequirePackage{amssymb}
\RequirePackage{amsmath}
\RequirePackage{graphicx}
\RequirePackage{color}
\RequirePackage[flushleft]{paralist}[2013/06/09]



\RequirePackage{geometry}
\geometry{%
  a4paper,
  lmargin=2cm,
  rmargin=2.5cm,
  tmargin=3.5cm,
  bmargin=2.5cm,
  footskip=12pt,
  headheight=24pt}


\newcommand{\comment}[1]{{\bf Comment} {\it #1}}
%\renewcommand{\comment}[1]{}

\newcommand{\bluecomment}[1]{{\color{blue}#1}}
%\renewcommand{\comment}[1]{}
\newcommand{\redcomment}[1]{{\color{red}#1}}



\usepackage{epsfig}
\usepackage{pstricks-add}
\usepackage{tgheros} %% changes sans-serif font to TeX Gyre Heros (tex-gyre)
\renewcommand{\familydefault}{\sfdefault} %% changes font to sans-serif
%\usepackage{sfmath}  %%%% this makes equation sans-serif
%\input RexFigs


\setlength{\parskip}{10pt}
\setlength{\parindent}{0pt}

\newlength{\qspace}
\setlength{\qspace}{20pt}


\newcounter{qnumber}
\setcounter{qnumber}{0}

\newenvironment{question}%
 {\vspace{\qspace}
  \begin{enumerate}[\bfseries 1\quad][10]%
    \setcounter{enumi}{\value{qnumber}}%
    \item%
 }
{
  \end{enumerate}
  \filbreak
  \stepcounter{qnumber}
 }


\newenvironment{questionparts}[1][1]%
 {
  \begin{enumerate}[\bfseries (i)]%
    \setcounter{enumii}{#1}
    \addtocounter{enumii}{-1}
    \setlength{\itemsep}{5mm}
    \setlength{\parskip}{8pt}
 }
 {
  \end{enumerate}
 }



\DeclareMathOperator{\cosec}{cosec}
\DeclareMathOperator{\Var}{Var}

\def\d{{\mathrm d}}
\def\e{{\mathrm e}}
\def\g{{\mathrm g}}
\def\h{{\mathrm h}}
\def\f{{\mathrm f}}
\def\p{{\mathrm p}}
\def\s{{\mathrm s}}
\def\t{{\mathrm t}}


\def\A{{\mathrm A}}
\def\B{{\mathrm B}}
\def\E{{\mathrm E}}
\def\F{{\mathrm F}}
\def\G{{\mathrm G}}
\def\H{{\mathrm H}}
\def\P{{\mathrm P}}


\def\bb{\mathbf b}
\def \bc{\mathbf c}
\def\bx {\mathbf x}
\def\bn {\mathbf n}

\newcommand{\low}{^{\vphantom{()}}}
%%%%% to lower suffices: $X\low_1$ etc


\newcommand{\subone}{ {\vphantom{\dot A}1}}
\newcommand{\subtwo}{ {\vphantom{\dot A}2}}




\def\le{\leqslant}
\def\ge{\geqslant}


\def\var{{\rm Var}\,}

\newcommand{\ds}{\displaystyle}
\newcommand{\ts}{\textstyle}
\def\half{{\textstyle \frac12}}




\begin{document}
\setcounter{page}{2}

 
\section*{Section A: \ \ \ Pure Mathematics}

%%%%%%%%%%Q1
\begin{question}
Consider the cubic equation
\[
x^3-px^2+qx-r=0\;,
\]
where $p\ne0$ and $r\ne 0$.

\begin{questionparts}

\item If the three roots    
can be written in the form $ak^{-1}$, $a$ and $ak$ for some 
constants $a$ and $k$, show that 
one root is $q/p$ and that
$
q^3 -rp^3=0\;.
$

\item
If $
r=q^3/p^3\;$, show that $q/p$ is a root and that the product of the 
other two roots is $(q/p)^2$. Deduce that the roots are 
in geometric progression.

\item
Find a necessary and sufficient condition involving $p$, $q$ and $r$ 
for the  roots 
 to be  in  arithmetic progression.
\end{questionparts}
\end{question}

%%%%%%%%%%Q2
\begin{question}
\begin{questionparts}

\item
Let $\f(x)=(1+x^2)\e^x$. Show that $\f'(x)\ge 0$ and sketch the graph
of $\f(x)$. 
Hence, or otherwise, show that
the equation
\[
 (1+x^2)\e^x = k,
\]
where $k$ is a constant,
has exactly one real root if  $k>0$ and no real roots if  $k\le 0$.


\item
Determine 
the number of real roots of the equation
$$
(\e^x-1) - k \tan^{-1} x=0
$$ 
in the cases (a) $0<k\le 2/\pi$ and (b) $2/\pi<k<1$.
\end{questionparts}

\end{question}

%%%%%%%%% Q3
\begin{question}
Justify, by means of a sketch, the formula
$$
\lim_{n\rightarrow\infty}\left\{{1\over n}\sum_{m=1}^n
\f(1+m/n)\right\} = \int_1^2 \f(x)\,\d x \,.
$$
Show that 
$$
\lim_{n\rightarrow\infty}\left\{{1\over n+1} + {1\over n+2} + \cdots
+ {1\over n+n}\right\} = \ln 2 \,.
$$
Evaluate
$$
\lim_{n\rightarrow\infty}\left\{{n\over n^2+1} + {n\over n^2+4}
 + \cdots + {n\over n^2+n^2}\right\}\,.
$$
\end{question}

%%%%%% Q4 
\begin{question}
A polyhedron is a solid bounded by $F$ plane faces, which meet in
$E$ edges and $V$ vertices. You may assume {\em Euler's
formula}, that $V-E+F=2$.

In a regular polyhedron the faces are equal regular $m$-sided
polygons, $n$ of which meet at each vertex. Show that
$$
F={4n\over h}  \,,
$$
where $h=4-(n-2)(m-2)$.

By considering the possible values of $h$, 
or otherwise, prove that there
are only five regular polyhedra, and find $V$, $E$ and $F$ for each.
	\end{question}

%%%%%%%%% Q5
\begin{question}
The sequence $u_0$,  $u_1$, $u_2$, ... is defined by 
$$
u_0=1,\hspace{0.2in} u_1=1,\hspace{0.3in} u_{n+1}=u_n+u_{n-1}
\hspace{0.2in}{\rm for}\hspace{0.1in}n \ge 1\,.
$$
Prove that
$$
u^2_{n+2} + u^2_{n-1} = 2( u^2_{n+1} + u^2_n ) \,.
$$
Using induction, or otherwise, prove the following result:
\[
u_{2n}  =  u^2_n + u^2_{n-1} 
\mbox{ \ \ \ and \ \ \ }
 u_{2n+1} =  u^2_{n+1} - u^2_{n-1} 
\]
for any positive integer $n$.

	\end{question}
	
%%%%%%%%% Q6
\begin{question}
A closed curve is given by the equation
$$
x^{2/n} + y^{2/n} = a^{2/n} \eqno(*)
$$
where $n$ is an odd integer and $a$ is a positive constant.
Find a parametrization $x=x(t)$, $y=y(t)$ which
describes the curve  anticlockwise as $t$ ranges from $0$ to $2\pi$. 

Sketch the curve in the case $n=3$, justifying the main features 
of your sketch.

The area $A$ enclosed by such a curve
is given by the formula
$$
A= {1\over 2} \int_0^{2\pi} \left[ x(t) {\d y(t)\over \d t} -
y(t) {\d x(t)\over \d t} \right] \,\d t \,.
$$
Use this result to find the area enclosed by ($*$) for $n=3$.
\end{question}
	
%%%%%%%%% Q7
\begin{question}
Let $a$ be a non-zero real number and define a binary operation
on the set of real numbers by
$$
x*y = x+y+axy \,.
$$
Show that the operation $*$ is associative.

Show that $(G,*)$ is a group, where
$G$ is the set of all real numbers except for one number which you should 
identify.

Find a subgroup of $(G,*)$ which has exactly 2 elements.
\end{question}
		
%%%%%%%%% Q8
\begin{question}	
The function $y(x)$ is  defined for $x\ge0$ and satisfies the conditions
 \[
y=0
\mbox{ \ \  and \ \ }
\frac{\d y}{\d x}=1
\mbox{ \  \  at $x=0$}.
\]
When $x$ is in the range $2(n-1)\pi< x <2n\pi$, where $n$ is a positive 
integer, $y(t)$ satisfies the differential
equation 
$$
{\d^2y \over \d x^2} + n^2 y=0. 
$$
Both $y$ and $\displaystyle \frac{\d y}{\d x} $ are continuous at $x=2n\pi$ for 
$n=0,\; 1,\;2,\; \ldots\;$. 

\begin{questionparts}
\item Find $y(x)$ for $0\le x \le 2\pi$. 

\item Show that $y(x) = \frac12 \sin 2x $ for 
$2\pi\le x\le 4\pi$, and find $y(x)$ for all $x\ge0$.

\item Show that 
$$
\int_0^\infty y^2 \,\d x = \pi \sum_{n=1}^\infty {1\over n^2} \,.
$$
\end{questionparts}
\end{question}	
		

		
	
\newpage
\section*{Section B: \ \ \ Mechanics}


	
%%%%%%%%%% Q9
\begin{question}
The gravitational force between two point 
particles of masses $m$ and $m'$ is mutually attractive and has magnitude
$$
{G m m' \over r^2}\,,
$$
where $G$ is a constant and $r$ is the distance between them.

A particle of unit mass 
lies on the axis of a thin uniform circular ring of radius
$r$ and mass $m$, at a distance $x$ from its centre. Explain why
the net force on the particle is directed towards the centre of
the ring and show that its magnitude is
$$
{G m x \over (x^2 + r^2)^{3/2}} \,.
$$

The particle now lies inside a thin hollow spherical
shell of uniform density, mass $M$ and radius $a$, at a distance
$b$ from its centre. Show that the particle
experiences no gravitational force due to the shell.

%Explain without calculation the effect on this result if
%the shell has finite thickness $x$. 
	\end{question}
	
%%%%%%%%%% Q10
\begin{question}	
A chain of mass $m$ and length $l$ is composed of $n$ small smooth  links.
It is suspended vertically over a horizontal table with
its end just touching the table, and released so that it
collapses inelastically onto the table.
Calculate the change in momentum of the $(k+1)$th link from the bottom
of the chain as it
falls onto the table.

Write down an expression for the total impulse sustained by the table 
in this way from  the whole chain. By approximating the
sum by an integral, show that 
this  
total impulse  is approximately 
\[
{\textstyle \frac23} m \surd(2gl)
\]
when $n$ is large.


\end{question}

%%%%%%%%%% Q11

\begin{question}
Calculate the moment of inertia of a 
uniform thin
 circular hoop of mass $m$ and radius $a$ about an axis perpendicular to the 
plane of the hoop through a point on its circumference.

The hoop, which is rough,  
rolls with speed $v$ on a rough  horizontal
table straight towards the edge and rolls over the edge without initially 
losing contact with the edge. Show that the hoop will lose contact with
the edge when it has rotated about the edge of the table
through an angle $\theta$, where
\[
\cos\theta = \frac 12 +\frac {v^2}{2ag}.
\]

%Give the corresponding result for a smooth hoop and table.
\end{question}
	

	
	\newpage
\section*{Section C: \ \ \ Probability and Statistics}


%%%%%%%%%% Q12
\begin{question}
In the game of endless cricket the scores
$X$ and $Y$ of the two sides are such that 
\[
\P (X=j,\ Y=k)=\e^{-1}\frac{(j+k)\lambda^{j+k}}{j!k!},\]
for some positive constant $\lambda$, where $j,k = 0$, $1$, $2$, $\ldots$\ .
\begin{questionparts}
\item[\bf(i)] Find $\P(X+Y=n)$ for each $n>0$.

\item[\bf(ii)] Show that $2\lambda \e^{2\lambda-1}=1$.

\item[\bf(iii)] Show that $2x \e^{2x-1}$ is an increasing function
of $x$ for $x>0$ and deduce that the equation in (ii) has
at most one solution and hence determine $\lambda$.

\item[\bf(iv)] Calculate the expectation $\E(2^{X+Y})$.
\end{questionparts}

\end{question}

%%%%%%%%%% Q13
\begin{question}
The cakes in our canteen each contain
exactly four currants, each currant being   randomly placed in the cake.
I take a proportion $X$ of a cake where $X$ is a random 
variable with density function
\[{\mathrm f}(x)=Ax\]
for $0\leqslant x\leqslant 1$ where $A$ is a constant.

\begin{questionparts}
\item[\bf(i)] What is the expected number of currants in my
portion?

\item[\bf(ii)] If I find all  four currants in my portion,
what is the probability that I took more than
half the cake?
\end{questionparts}

\end{question}

%%%%%%%%%% Q14
\begin{question}
In the basic version of Horizons (H1) the player has 
a maximum of $n$ turns, where $n \ge 1$.
At each turn, she has a probability $p$ 
of success, where $0 < p < 1$. If her first success is at the 
$r$th turn, where $1 \le r \le n$, she collects $r$ pounds 
and then withdraws from the game. Otherwise, her winnings are nil.
Show that  in H1, her expected winnings are
$$
p^{-1}\left[1+nq^{n+1}-(n+1)q^n\right]\quad\hbox{pounds},
$$
where $q=1-p$.

The rules of H2 are the same as those of H1, except that
$n$ is randomly selected from a Poisson distribution 
with parameter $\lambda$. If $n=0$ her winnings are nil. 
Otherwise she plays H1 with the selected $n$.
Show that in H2, her expected winnings are
$$ 
{1 \over p}{\left(1-{\e^{-{\lambda}p}}\right)}
    -{{\lambda}q}{\e^{-{\lambda}p}}
\quad\hbox{pounds}.
$$

\end{question}
	
\end{document}
