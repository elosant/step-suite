\documentclass[a4, 11pt]{report}


\pagestyle{myheadings}
\markboth{}{Paper II, 1999
\ \ \ \ \ 
\today 
}               

\RequirePackage{amssymb}
\RequirePackage{amsmath}
\RequirePackage{graphicx}
\RequirePackage{color}
\RequirePackage[flushleft]{paralist}[2013/06/09]



\RequirePackage{geometry}
\geometry{%
  a4paper,
  lmargin=2cm,
  rmargin=2.5cm,
  tmargin=3.5cm,
  bmargin=2.5cm,
  footskip=12pt,
  headheight=24pt}


\newcommand{\comment}[1]{{\bf Comment} {\it #1}}
%\renewcommand{\comment}[1]{}

\newcommand{\bluecomment}[1]{{\color{blue}#1}}
%\renewcommand{\comment}[1]{}
\newcommand{\redcomment}[1]{{\color{red}#1}}



\usepackage{epsfig}
\usepackage{pstricks-add}
\usepackage{tgheros} %% changes sans-serif font to TeX Gyre Heros (tex-gyre)
\renewcommand{\familydefault}{\sfdefault} %% changes font to sans-serif
%\usepackage{sfmath}  %%%% this makes equation sans-serif
%\input RexFigs


\setlength{\parskip}{10pt}
\setlength{\parindent}{0pt}

\newlength{\qspace}
\setlength{\qspace}{20pt}


\newcounter{qnumber}
\setcounter{qnumber}{0}

\newenvironment{question}%
 {\vspace{\qspace}
  \begin{enumerate}[\bfseries 1\quad][10]%
    \setcounter{enumi}{\value{qnumber}}%
    \item%
 }
{
  \end{enumerate}
  \filbreak
  \stepcounter{qnumber}
 }


\newenvironment{questionparts}[1][1]%
 {
  \begin{enumerate}[\bfseries (i)]%
    \setcounter{enumii}{#1}
    \addtocounter{enumii}{-1}
    \setlength{\itemsep}{5mm}
    \setlength{\parskip}{8pt}
 }
 {
  \end{enumerate}
 }



\DeclareMathOperator{\cosec}{cosec}
\DeclareMathOperator{\Var}{Var}

\def\d{{\mathrm d}}
\def\e{{\mathrm e}}
\def\g{{\mathrm g}}
\def\h{{\mathrm h}}
\def\f{{\mathrm f}}
\def\p{{\mathrm p}}
\def\s{{\mathrm s}}
\def\t{{\mathrm t}}


\def\A{{\mathrm A}}
\def\B{{\mathrm B}}
\def\E{{\mathrm E}}
\def\F{{\mathrm F}}
\def\G{{\mathrm G}}
\def\H{{\mathrm H}}
\def\P{{\mathrm P}}


\def\bb{\mathbf b}
\def \bc{\mathbf c}
\def\bx {\mathbf x}
\def\bn {\mathbf n}

\newcommand{\low}{^{\vphantom{()}}}
%%%%% to lower suffices: $X\low_1$ etc


\newcommand{\subone}{ {\vphantom{\dot A}1}}
\newcommand{\subtwo}{ {\vphantom{\dot A}2}}




\def\le{\leqslant}
\def\ge{\geqslant}


\def\var{{\rm Var}\,}

\newcommand{\ds}{\displaystyle}
\newcommand{\ts}{\textstyle}
\def\half{{\textstyle \frac12}}




\begin{document}
\setcounter{page}{2}

 
\section*{Section A: \ \ \ Pure Mathematics}

%%%%%%%%%%Q1
\begin{question}
Let
$x=10^{100}$, 
$y=10^{x}$, 
$z=10^{y}$, 
and let
$$ 
a_1=x!, \quad a_2=x^y,\quad a_3=y^x,\quad a_4=z^x,\quad 
   a_5=\e^{xyz},\quad a_6=z^{1/y},\quad a_7 = y^{z/x}.
$$
\begin{questionparts}
\item Use Stirling's approximation 
$n! \approx \sqrt{2 \pi}\, {n^{n+{1\over2}}\e^{-n}}$, which is valid for 
large $n$, to show that 
$\log_{10}\left(\log_{10} a_1 \right) 
\approx 102$.
\item Arrange the seven numbers $a_1$, $\ldots$ , $a_7$ in ascending 
order of magnitude, justifying  your result.
\end{questionparts}
\end{question}

%%%%%%%%%%Q2
\begin{question}
Consider the quadratic equation
$$ 
nx^2+2x  \sqrt{pn^2+q} + rn + s = 0, 
\eqno (*)
$$ 
where $p>0$, $p\neq r$ and $n=1$, $2$, $3$, $\ldots$ .
\begin{questionparts}
\item For the case where $p=3$, $q=50$, $r=2$, $s=15$, 
find the set 
of values of $n$ for which  equation $(*)$ has no real roots.
\item Prove that if $p<r$ and $4q(p-r)>s^2$, then $(*)$
has no real roots for any value of $n$.
\item If $n=1$, $p-r=1$  and $q={s^2}/8$, 
show that $(*)$ has real roots if, and only if, 
$s \le 4-2\sqrt{2}\ $ or $\ s \ge 4+2\sqrt{2}$\,.
\end{questionparts}
\end{question}

%%%%%%%%% Q3
\begin{question}
Let
$$ 
{\rm S}_n(x)=\mathrm{e}^{x^3}{{\d^n}\over{\d x^n}}{(\mathrm{e}^{-x^3})}.
$$
Show that ${\rm S}_2(x)=9x^4-6x$ and find ${\rm S}_3(x)$.

Prove by induction on $n$ that    ${\rm S}_n(x)$ is a polynomial. By means
 of your induction argument,
 determine the order of this polynomial and  
the coefficient  of the highest power of~$x$.

Show also that 
if \  $\displaystyle \frac{\d S_n}{\d x}=0$ \ for  some value $a$ of $x$,
then $ \ S_n(a)S_{n+1}(a)\le0$.
\end{question}

%%%%%% Q4 
\begin{question}
By considering the expansions in powers of $x$
of both sides of the identity
$$ 
{(1+x)^n}{(1+x)^n}\equiv{(1+x)^{2n}},
$$
show that
$$ 
\sum_{s=0}^n {n\choose s}^2 = {2n\choose n},
$$
where $\displaystyle  {n\choose s}= \frac{n!}{s!\,(n-s)!}$.

By considering similar identities, or otherwise, show also that: 
\begin{questionparts}

\item if $n$ is an even integer, then
\\[4mm]
$\displaystyle
\sum_{s=0}^n {{(-1)}^s}{n \choose s}^2=
   (-1)^{n/2}{n \choose n/2};
$
\item 
$
\displaystyle 
\sum\limits_{t=1}^ n 2t { n \choose t}^2 
= n {2n\choose n} .
$ 
\end{questionparts}
	\end{question}

%%%%%%%%% Q5
\begin{question}
Show that if  $\alpha$ is a solution    of the equation
$$ 
5{\cos x} + 12{\sin x} = 7,
$$
then either
$$ 
{\cos  }{\alpha} = \frac{35 -12\sqrt{120}}{169}
$$
or $\cos \alpha$ has one other value which you should find.
 
Prove carefully that if 
$\frac{1}{2}\pi< \alpha < \pi$, then $\alpha < \frac{3}{4}\pi$. 

	\end{question}
	
%%%%%%%%% Q6
\begin{question}
Find $\displaystyle \ \frac{\d y}{\d x} \ $ if 
$$
y = \frac{ax+b}{cx+d}.
\eqno(*)
$$


By using  changes of variable of the form $(*)$, or otherwise,
show that
\[
\int_0^1 \frac{1}{(x+3)^2} \; \ln  \left(\frac{x+1}{x+3}\right)\d x
= {\frac16} \ln3 - {\frac14}\ln 2 - \frac 1{12},
\]
and evaluate the integrals
\[
\int_0^1 \frac{1}{(x+3)^2} \; \ln \left(\frac{x^2+3x+2}{(x+3)^2}\right)\d x
\mbox{ \ \ and \ \ }
\int_0^1 \frac{1}{(x+3)^2} \; \ln\left(\frac{x+1}{x+2}\right)\d x
.
\]


%By changing to the variable $y$ defined by
%$$ 
%y=\frac{2x-3}{x+1},
%$$
% evaluate the integral 
%$$ 
%\int_2^4 \frac{2x-3}{(x+1)^3}\;  
%\ln\!\left(\frac{2x-3}{x+1}\right)\d x.
%$$


%Evaluate the integral
%$$ 
%\int_9^{25} {\big({2z^{-3/2} -5z^{-2}}\big)}
%\ln{\big(2-5z^{-1/2}\big)}\; \d z.
%$$ 
\end{question}
	
%%%%%%%%% Q7
\begin{question}
The curve $C$ has equation
$$ 
y = \frac x {\sqrt{x^2-2x+a}}\; ,
$$
where the square root is positive.
Show that, if $a>1$,  then $C$ has exactly one stationary point. 

Sketch $C$ when \textbf{(i)} $a=2$ and \textbf{(ii)} $a=1$.
\end{question}
		
%%%%%%%%% Q8
\begin{question}	
Prove that
$$
\sum_{k=0}^n \sin k\theta = \frac
{ \cos \half\theta -  \cos (n+ \half) \theta}
{2\sin \half\theta}\;.
\eqno(*)
$$

\begin{questionparts}
\item Deduce that, when $n$ is large,
\[
\sum_{k=0}^n \sin \left(\frac{k\pi}{n}\right) \approx \frac{2n}\pi\;.
\]
\item By differentiating $(*)$ with respect to $\theta$, or otherwise,
show that, when $n$ is large, 
\[
\sum_{k=0}^n k \sin^2 \left(\frac{k\pi}{2n}\right)
 \approx \left(\frac{1}4 +\frac{1}{\pi^2}
\right)n^2\;.
\]
\end{questionparts}

\noindent
[The approximations, valid for small $\theta$, $\sin\theta \approx \theta$
and $\cos\theta \approx 1-{\textstyle\frac12}\,\theta^2$ may be assumed.]
\end{question}	
		

		
	
\newpage
\section*{Section B: \ \ \ Mechanics}


	
%%%%%%%%%% Q9
\begin{question}
In the $Z$--universe, a star of mass $M$
suddenly blows up, and the fragments, with various initial speeds, 
start to move away from the centre of mass $G$ which may be
regarded as a fixed point. In the  subsequent motion the 
acceleration of each fragment is directed towards $G$.
Moreover, in accordance with the laws of physics of the $Z$--universe, 
there  are positive constants 
$k_1$, $k_2$ and $R$ such that when a fragment is at a distance $x$
from $G$, the magnitude of its acceleration  is $k_1x^3$ if
$x<R$ and  is $k_2x^{-4}$ if $x \ge R$.
The initial speed of a fragment is denoted by $u$.

\begin{questionparts}
\item
For $x<R$, write down a differential equation for the speed $v$,
and hence determine $v$ in terms of $u$, $k_1$ and $x$ for $ x<R$.  
\item
 Show that if $u < a$, where $2a^2=k_1 R^4$, 
then the fragment does not reach a distance $R$ from $G$.
\item Show that if $u \ge b$, 
where 
$
6b^2= 3k_1R^4  + 4k_2 /R^3,
$ 
then from the moment
of the explosion the fragment is always moving away from $G$.
\item If $a<u<b$, determine in terms of 
$k_2$, $b$ and $u$ 
the maximum distance from $G$ attained by the fragment.
\end{questionparts}
	\end{question}
	
%%%%%%%%%% Q10
\begin{question}	
$N$ particles $P_1$, $P_2$, $P_3$, $\ldots$, $P_N$ with masses
$m$, $qm$, $q^2m$, $\ldots$ , ${q^{N-1}}m$, respectively,
are at rest 
at distinct points along a straight line in gravity-free space.
The particle $P_1$ is set in motion towards $P_2$ with velocity
$V$ and in every subsequent impact the coefficient of restitution
is $e$, where $0<e<1$. Show that after the first impact the
velocities of $P_1$ and $P_2$ are 
$$
{\left({{1-eq}\over{1+q}}\right)}V 
\mbox{ \ \ \ and \ \ \ } 
{\left({{1+e}\over{1+q}}\right)}V,
$$ respectively.  

Show that  if $q \le e$, then there are exactly $N-1$ impacts
and that if $q=e$, then the total loss of kinetic energy after all
impacts have occurred is equal to 
$$ 
{1\over 2}{me}{\left(1-e^{N-1}\right)}{V^2}.
$$ 

\end{question}

%%%%%%%%%% Q11

\begin{question}
An automated mobile dummy target for gunnery practice
is moving anti-clockwise around the circumference of a large circle
of radius $R$ in a horizontal plane at a constant angular speed~$\omega$.  
A shell is fired from $O$, the centre of this circle, with
initial speed $V$ and angle of elevation~$\alpha$.  
Show that if 
$V^2<gR$, then no matter what the value of~$\alpha$, or what 
vertical plane the shell is fired in, the shell cannot hit the target.

Assume now that $V^2>gR$ and that the shell hits the target, and let 
$\beta$ be the angle through which the target rotates between the 
time at which the  shell is fired and the time  of impact. Show that
$\beta$ satisfies the   equation
$$
g^2{{\beta}^4} - 4{{\omega}^2}{V^2}{{\beta}^2}
+4{R^2}{{\omega}^4}=0.
$$
Deduce that  there are exactly two possible 
values of~$\beta$.

Let $\beta_1$ and $\beta_2$ be the possible values of $\beta$
and let $P_1$ and $P_2$ be the corresponding points of impact.
By considering the quantities $(\beta_1^2 +\beta_2^2) $
and $\beta_1^2\beta_2^2\,$, or otherwise, 
show that the linear distance between $P_1$ and $P_2$ is
\[
2R \sin\Big( \frac\omega g \sqrt{V^2-Rg}\Big) 
\;.
\] 
\end{question}
	

	
	\newpage
\section*{Section C: \ \ \ Probability and Statistics}


%%%%%%%%%% Q12
\begin{question}
It is known  that there are
three manufacturers $A, B, C,$ who can  produce
micro chip MB666. The probability that a randomly selected MB666
is produced by $A$ is $2p$, and the corresponding probabilities for
$B$ and $C$ are $p$ and $1 - 3p$, respectively, where 
${{0} \le p \le {1 \over 3}}.$ It is also known that $70\%$ of 
MB666 micro chips from $A$ are sound  and that the corresponding
percentages for $B$ and $C$ are  $80\%$ and $90\%$, respectively. 

Find in terms of $p$, the conditional probability, $\P(A {\vert} S)$,
that if a randomly selected 
MB666 chip is found to be sound then it came from $A$, and also the 
conditional probability, $\P(C {\vert} S)$, that  
if it is sound then it came from $C$.

A quality inspector took a random sample 
of one MB666 micro chip and found it to be sound. She then traced
its place of manufacture to be $A$, and 
so estimated $p$ by calculating the value of $p$ 
that corresponds to the  greatest value 
of  $\P(A {\vert} S)$. A second quality 
inspector also a took random sample of one MB666 chip and 
found it to be sound. Later he traced its place of manufacture
to be $C$ and so estimated $p$ by applying the procedure of his
colleague to $\P(C {\vert} S)$.

Determine the values of the two estimates and comment 
briefly on the results obtained.

\end{question}

%%%%%%%%%% Q13
\begin{question}
A stick is broken at a point, chosen at random, along its length.
Find the probability that the ratio, $R$, of the length of the shorter
piece to the length of the longer piece is less than $r$.

Find the probability density  function for $R$, and calculate the  mean and 
variance of $R$.
\end{question}

%%%%%%%%%% Q14
\begin{question}
You play the following game. You throw a six-sided fair die repeatedly.
You may choose to stop after any throw, except that 
you must stop if you throw a 1. Your score is the
number obtained on your last throw.
Determine  the strategy that you should adopt in order 
 to maximize your expected score, explaining your reasoning carefully.
\end{question}
	
\end{document}
