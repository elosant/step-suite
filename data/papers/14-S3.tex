
\documentclass[a4, 11pt]{report}


\pagestyle{myheadings}
\markboth{}{Paper III, 2014
\ \ \ \ \ 
\today 
}               

\RequirePackage{amssymb}
\RequirePackage{amsmath}
\RequirePackage{graphicx}
\RequirePackage{color}
\RequirePackage[flushleft]{paralist}[2013/06/09]


\RequirePackage{geometry}
\geometry{%
  a4paper,
  lmargin=2cm,
  rmargin=2.5cm,
  tmargin=3.5cm,
  bmargin=2.5cm,
  footskip=12pt,
  headheight=24pt}
 

\newcommand{\comment}[1]{{\bf Comment} {\it #1}}
%\renewcommand{\comment}[1]{}

\newcommand{\bluecomment}[1]{{\color{blue}#1}}
%\renewcommand{\comment}[1]{}
\newcommand{\redcomment}[1]{{\color{red}#1}}



\usepackage{epsfig}
\usepackage{pstricks-add}
\usepackage{tgheros} %% changes sans-serif font to TeX Gyre Heros (tex-gyre)
\renewcommand{\familydefault}{\sfdefault} %% changes font to sans-serif
%\usepackage{sfmath}  %%%% this makes equation sans-serif
%\input RexFigs


\setlength{\parskip}{10pt}
\setlength{\parindent}{0pt}

\newlength{\qspace}
\setlength{\qspace}{20pt}


\newcounter{qnumber}
\setcounter{qnumber}{0}

\newenvironment{question}%
 {\vspace{\qspace}
  \begin{enumerate}[\bfseries 1\quad][10]%
    \setcounter{enumi}{\value{qnumber}}%
    \item%
 }
{
  \end{enumerate}
  \filbreak
  \stepcounter{qnumber}
 }


\newenvironment{questionparts}[1][1]%
 {
  \begin{enumerate}[\bfseries (i)]%
    \setcounter{enumii}{#1}
    \addtocounter{enumii}{-1}
    \setlength{\itemsep}{5mm}
    \setlength{\parskip}{8pt}
 }
 {
  \end{enumerate}
 }



\DeclareMathOperator{\cosec}{cosec}
\DeclareMathOperator{\Var}{Var}

\def\d{{\mathrm d}}
\def\e{{\mathrm e}}
\def\g{{\mathrm g}}
\def\h{{\mathrm h}}
\def\f{{\mathrm f}}
\def\p{{\mathrm p}}
\def\s{{\mathrm s}}
\def\t{{\mathrm t}}
\def\i{{\mathrm i}}

\def\A{{\mathrm A}}
\def\B{{\mathrm B}}
\def\E{{\mathrm E}}
\def\F{{\mathrm F}}
\def\G{{\mathrm G}}
\def\H{{\mathrm H}}
\def\P{{\mathrm P}}


\def\bb{\mathbf b}
\def \bc{\mathbf c}
\def\bx {\mathbf x}
\def\bn {\mathbf n}

\newcommand{\low}{^{\vphantom{()}}}
%%%%% to lower suffices: $X\low_1$ etc


\newcommand{\subone}{ {\vphantom{\dot A}1}}
\newcommand{\subtwo}{ {\vphantom{\dot A}2}}
\DeclareMathOperator{\sech}{sech}


\def\le{\leqslant}
\def\ge{\geqslant}
\def\arcosh{{\rm arcosh}\,}


\def\var{{\rm Var}\,}

\newcommand{\ds}{\displaystyle}
\newcommand{\ts}{\textstyle}
\def\half{{\textstyle \frac12}}
\def\l{\left(}
\def\r{\right)}
\renewcommand{\.}[1]{\ensuremath{\mathrm{#1}}}
\newcommand{\+}[1]{\ensuremath{\mathbf{#1}}}
\newcommand{\ud}{\mathop{}\!\mathrm{d}}



\begin{document}
\setcounter{page}{2}

 
\section*{Section A: \ \ \ Pure Mathematics}

%%%%%%%%%%Q1
\begin{question}
Let $a$, $b$ and $c$ be real numbers such that $a+b+c=0$ and let
\[(1+ax)(1+bx)(1+cx) = 1+qx^2 +rx^3\,\]
for all real $x$. Show that $q = bc+ca+ab$ and $r= abc$.
\begin{questionparts}
\item Show that the coefficient of $x^n$ in the series expansion
(in ascending powers of $x$)   
 of $\ln (1+qx^2+rx^3)$ is $(-1)^{n+1} S_n$ where
\[S_n = \frac{a^n+b^n+c^n}{n} \,,  \ \ \ \ \ \ \ \ (n\ge1).\]
\item Find, in terms of $q$ and $r$, the coefficients of $x^2$,
$x^3$ and~$x^5$ in the series expansion (in ascending powers of $x$)
of $\ln (1+qx^2+rx^3)$ and hence show that $S_2S_3 =S_5$\,.
\item Show that $S_2S_5 =S_7$\,.
\item Give a proof of, or find a counterexample to, the claim that $S_2S_7=S_9$\,.
\end{questionparts}
\end{question}

%%%%%%%%%%Q2
\begin{question}
  \begin{questionparts}
  \item Show, by means of the  substitution $u=\cosh x\,$, that
    \[
    \int \frac{\sinh x}{\cosh 2x} \ud x 
= \frac 1{2\sqrt2} 
\ln \left\vert \frac{\sqrt2 \cosh x - 1}{\sqrt2 \cosh x + 1 } \right\vert
+ C
    \,.\]

  \item Use a similar substitution to find an expression for     
    \[
    \int \frac{\cosh x}{\cosh 2x} \ud x
    \,.\]

  \item Using parts (i) and (ii) above, show that
    \[
    \int_0^1 \frac 1{1+u^4} \ud u = \frac{\pi + 2\ln(\sqrt2 +1)}{4\sqrt2}\,.
    \]
  \end{questionparts}
\end{question}

%%%%%%%%% Q3
\begin{question}
  \begin{questionparts}
  \item The line $L$ has equation $y=mx+c$, where $m>0$ and $c>0$.
    Show that, in the case  $mc>a>0$, 
the shortest distance between $L$ and the parabola
    $y^2=4ax$ is 
    \[ \frac{mc-a}{m\sqrt{m^2+1}}\,.\]

    What is the shortest distance in the case that $mc\le a$?
  \item Find the shortest distance between the point $(p,0)$, where
    $p>0$, and the parabola $y^2=4ax$, where $a>0$, in the different
    cases that arise according to the value of $p/a$.  [\textit{You
      may wish to use the parametric coordinates $(at^2, 2at)$ of
      points on the parabola.}]

    Hence find the shortest distance between the circle $(x-p)^2 + y^2
    =b^2$, where $p>0$ and $b>0$, and the parabola $y^2=4ax$, where $a>0$,
 in the
    different cases that arise according to the values of $p$, $a$
    and~$b$.
  \end{questionparts}
\end{question}

%%%%%% Q4 
\begin{question}
  \begin{questionparts}
  \item Let
    \[
    I = \int_0^1 \bigl((y')^2 -y^2\bigr)\ud x \qquad\text{and}\qquad
    I_1=\int_0^1 (y'+y\tan x)^2 \ud x \,,
    \]
    where $y$ is a given function of $x$ satisfying $y=0$ at 
    $x=1$.  Show that $I-I_1=0$ and deduce that $I\ge0$. Show further
    that $I=0$ only if $y=0$ for all $x$ ($0\le x \le 1$).

  \item 
Let 
\[
J  = \int_0^1 \bigl((y')^2 -a^2y^2\bigr)\ud x 
\,,
\]
where $a$ is a given positive constant and  $y$ is a given function of $x$,
not identically zero, satisfying $y=0$ at 
    $x=1$. 
By considering an integral of the form
\[
\int_0^1 (y'+ay\tan bx)^2 \ud x \,,
\]
where  $b$ is suitably chosen, show that
$J\ge0$. You should state the range of values of~$a$, in the form
 $a<k$, for which your proof is valid.

In the case $a=k$, find a function $y$ (not everywhere zero)
such that $J=0$.

  \end{questionparts} 
\end{question}

%%%%%%%%% Q5
\begin{question}
  A quadrilateral drawn in the complex plane has vertices $A$,
  $B$, $C$ and~$D$, labelled anticlockwise.  These vertices are
  represented, respectively, by the complex numbers $a$, $b$, $c$
  and~$d$.  Show that $ABCD$ is a parallelogram (defined as a quadrilateral
  in which  opposite sides are parallel and equal in length)
  if and only if
  $a+c =b+d\,$.  Show further that, in this case,
  $ABCD$ is a square if and only if ${\rm i}(a-c)=b-d$.

  Let $PQRS$ be a quadrilateral in the complex plane, with vertices
  labelled anticlockwise, the internal angles of which are all less
  than~$180^\circ$.  Squares with centres $X$, $Y$, $Z$ and~$T$ are
  constructed externally to the quadrilateral on the sides $PQ$, $QR$,
  $RS$ and~$SP$, respectively.

  \begin{questionparts}
  \item If $P$ and $Q$ are represented by the complex numbers $p$
    and~$q$, respectively, show that $X$ can be represented by
    \[
    \tfrac 12 \big( p(1+{\rm i} ) + q (1-{\rm i})\big) \,.
    \]
  \item Show that $XY\!ZT$ is a square if and only if $PQRS$ is a
    parallelogram.
  \end{questionparts}
\end{question}
	
%%%%%%%%% Q6
\begin{question}
  Starting from the result that
  \[
  \.h(t) >0\ \mathrm{for}\ 0< t < x \Longrightarrow \int_0^x \.h(t)\ud t
  > 0 \,,
  \]
  show that, if $\.f''(t)>0$ for $0<t<x_0$ and $\.f(0)=\.f'(0) =0$,
  then $\.f(t)>0$ for $0<t<x_0$.

  \begin{questionparts}
  \item Show that,
for $0<x < \frac12\pi$,
 \[
\cos x \cosh x <1 
\,.
\] 

  \item Show that, for $0<x < \frac12\pi$,
    \[
    \frac 1 {\cosh x} < \frac {\sin x} x < \frac x {\sinh x} \,.
    \]

%  \item Show that, for $0<x<\frac12\pi$, $\tanh x < \tan x$.
  \end{questionparts}
\end{question}
	
%%%%%%%%% Q7
\begin{question}
The four distinct points $P_i$ ($i=1$, $2$, $3$, $4$) are the
    vertices, labelled anticlockwise, of a cyclic quadrilateral.  The
    lines $P_1P_3$ and $P_2P_4$ intersect at $Q$. 

  \begin{questionparts}
  \item 
By considering the
    triangles $P_1QP_4$ and $P_2QP_3$ show that
    $(P_1Q)( QP_3) = (P_2Q) (QP_4)\,$.
  \item Let $\+p_i$ be the position vector of the point $P_i$
 ($i=1$, $2$, $3$, $4$). Show that there
    exist numbers $a_i$, not all zero, such
    that
    \begin{equation}
      \sum\limits_{i=1}^4 a_i =0 \qquad\text{and}\qquad
      \sum\limits_{i=1}^4 a_i \+p_i ={\bf 0} \,. \tag{$*$}
    \end{equation}
  \item Let $a_i$ ($i=1$,~$2$, $3$,~$4$) be any numbers, not all zero,
   that  satisfy~$(*)$.  Show that $a_1+a_3\ne 0$ and that the lines
    $P_1P_3$ and $P_2P_4$ intersect at the point with position vector
    \[
    \frac{a_1 \+p_1 + a_3 \+p_3}{a_1+a_3} \,.
    \]
    Deduce that $a_1a_3 (P_1P_3)^2 = a_2a_4 (P_2P_4)^2\,$.
  \end{questionparts}
\end{question}
		
%%%%%%%%% Q8
\begin{question}
  The numbers $\.f(r)$ satisfy $\.f(r)>\.f(r+1)$ for $r=1$, $2$,
  \dots.  Show that, for any non-negative integer $n$,
  \[
  k^n(k-1) \, \.f(k^{n+1}) \le \sum_{r=k^n}^{k^{n+1}-1}\.f(r) \le k^n(k-1)\,
  \.f(k^n)\,
  \]
where $k$ is an integer greater than 1.

  \begin{questionparts}
  \item By taking $\.f(r) = 1/r$, show that
    \[
    \frac{N+1}2 \le \sum_{r=1}^{2^{N+1}-1} \frac1r \le N+1 \,.
    \]

    Deduce that the sum $\sum\limits_{r=1}^\infty \frac1r$ does not
    converge.
  \item By taking $\.f(r)= 1/r^3$, show that
    \[
    \sum_{r=1}^\infty \frac1 {r^3} \le  1 \tfrac 13 \,.
    \]

  \item Let $S(n)$ be the set of positive integers less than $n$ which
    do not have a $2$ in their decimal representation and let
    $\sigma(n)$ be the sum of the reciprocals of the numbers in
    $S(n)$, so for example $\sigma(5) = 1+\frac13+\frac14$. Show that
    $S(1000)$ contains $9^3-1$ distinct numbers.

    Show  that $\sigma (n) < 80$ for all $n$.
  \end{questionparts}
\end{question}	
		

		
	
\newpage
\section*{Section B: \ \ \ Mechanics}


	
%%%%%%%%%% Q9
\begin{question}
  A particle of mass $m$ is projected with velocity $\+ u$. It is
  acted upon by the force $m\+g$ due to gravity and by a resistive
  force $-mk \+v$, where $\+v$ is its velocity and $k$ is a positive
  constant.

  Given that, at time $t$ after projection, its position $\+r$
  relative to the point of projection is given by
  \[
  \+r = \frac{kt -1 +\.e^{-kt}} {k^2} \, \+g + \frac{ 1-\.e^{-kt}}{k}
  \, \+u \,,
  \]
  find an expression for $\+v$ in terms of $k$, $t$, $\+g$ and
  $\+u$. Verify that the equation of motion and the initial conditions
  are satisfied.

  Let $\+u = u\cos\alpha \, \+i + u \sin\alpha \, \+j$ and $\+g = -g\,
  \+j$, where $0<\alpha<90^\circ$, and let $T$ be the time after projection at
  which $\+r \,.\, \+j =0$. Show that
  \[
  uk \sin\alpha = \left(\frac{kT}{1-\.e^{-kT}} -1\right)g\,.
  \]

  Let $\beta$ be the acute angle between $\+v$ and $\+i$ at time
  $T$. Show that
  \[
  \tan\beta = \frac{(\.e^{kT}-1)g}{uk\cos\alpha}-\tan\alpha \,.
  \]

  Show further that $\tan\beta >\tan\alpha$ (you may assume that
 $\sinh kT >kT$) and deduce that~$\beta >\alpha$.
	\end{question}
	
%%%%%%%%%% Q10 
\begin{question}	
Two particles $X$ and $Y$, of equal mass $m$, lie on a
smooth horizontal table and are connected by a 
light elastic spring of natural
length $a$ and modulus of elasticity $\lambda$.  Two more springs,
identical to the first, 
connect 
$X$ to a point $P$ on the table and 
$Y$  
to a point $Q$ on the table. The distance between $P$ and $Q$ is $3a$.


  Initially, the particles are held so that $XP=a$,     
$YQ= \frac12 a\,$, and $PXYQ$ is a straight line. 
   The particles are then released.

  At time $t$, the particle $X$ is a distance $a+x$ from $P$ and the
  particle $Y$ is a distance $a+y$ from $Q$.  Show that
  \[
  m \frac{\.d ^2 x}{\.d t^2} = -\frac\lambda a (2x+y)
  \]
  and find a similar expression involving $\dfrac{\.d^2 y}{\.d t^2}$.
  Deduce that
  \[
  x-y = A\cos \omega t +B \sin\omega t
  \]
  where $A$ and $B$ are constants to be determined and
  $ma\omega^2=\lambda$.  Find a similar expression for $x+y$.

  Show that $Y$ will never return to its initial position.
\end{question}

%%%%%%%%%% Q11

\begin{question}
  A particle $P$ of mass $m$ is connected by two light inextensible
  strings to two fixed points $A$ and $B$, with $A$ vertically above
  $B$. The string $AP$ has length $x$.  The particle is rotating about
  the vertical through $A$ and $B$ with angular velocity $\omega$, and
  both strings are taut.  Angles $PAB$ and $PBA$ are $\alpha$ and
  $\beta$, respectively.

  Find the tensions $T_A$ and $T_B$ in the strings $AP$ and $BP$
  (respectively), and hence show that $\omega^2 x\cos\alpha \ge g$.

  Consider now the case that $\omega^2 x\cos\alpha = g$.  Given that
  $AB=h$ and $BP=d$, where $h>d$,  show that
$h\cos\alpha \ge \sqrt{h^2-d^2}$. Show further that 
  \[
  mg < T_A \le \frac{mgh}{\sqrt{h^2-d^2}\,}\,.
  \]
  Describe the geometry of the strings when $T_A$ 
attains its upper bound.
\end{question}
	

	
	\newpage
\section*{Section C: \ \ \ Probability and Statistics}


%%%%%%%%%% Q12
\begin{question}
  The random variable $X$ has probability density function 
$\.f(x)$ (which you may assume is differentiable)
  and cumulative distribution function $\.F(x)$ where $-\infty < x <
  \infty$.  The random variable $Y$ is defined by $Y= \.e^X$.
You may assume throughout this question that $X$ and $Y$ have unique modes.
  \begin{questionparts}
  \item Find  the median value $y_m$ of $Y$ in terms of the
    median value $x_m$ of $X$.


  \item Show that the probability density function of $Y$ is $\.f(\ln
    y)/y$, and deduce that the mode~$\lambda$ of $Y$ satisfies 
$\.f'(\ln \lambda)
    = \.f(\ln \lambda)$.

  \item Suppose now that $X \sim {\rm N} (\mu,\sigma^2)$, so that
    \[
    \.f(x) = \frac{1}{\sigma \sqrt{2\pi}\,} \.e^{-(x-\mu)^2/(2\sigma^2)}
    \,.
    \]
    Explain why 
\[
\frac{1}{\sigma \sqrt{2\pi}\,}
    \int_{-\infty}^{\infty}\.e^{-(x-\mu-\sigma^2)^2/(2\sigma^2)} \ud x = 1
\] and hence show
    that $ \.E(Y) = \.e ^{\mu+\frac12\sigma^2}$.

  \item Show that, when $X \sim {\rm N} (\mu,\sigma^2)$,
    \[
   \lambda <y_m< \.E(Y)\,.
    \]

\end{questionparts}
\end{question}

%%%%%%%%%% Q13
\begin{question}
  I play a game which has repeated rounds. Before the
  first round, my score is $0$. Each round can have three outcomes:
  \begin{compactenum}[1.]
  \item my score is unchanged and the game ends;
  \item my score is unchanged and I continue to the next round;
  \item my score is increased by one and I continue to the next
    round.
  \end{compactenum}
   The probabilities of these outcomes  are $a$, $b$
  and~$c$, respectively (the same in each round), 
where $a+b+c=1$ and $abc\ne0$.  The random
  variable $N$ represents my score at the end of a randomly chosen
game.  

  Let $\.G(t)$ be the probability generating function of
    $N$. 
  \begin{questionparts}
  \item 
Suppose in the first round, the game ends. 
Show that the probability generating
function conditional on this happening is 1.

\item
Suppose in the first round, 
the game continues to the next round with no change in
score. 
Show that the probability generating function conditional on this happening
is~$\.G(t)$.

\item By comparing the coefficients of $t^n$, show that 
$   
    \.G(t) = a + b\.G(t) + ct\.G(t)\,.
 $  
      Deduce that,
for $n\ge0$,
    \[
P(N=n)  = \frac{ac^n}{(1-b)^{n+1}}\,.
    \]
  \item Show further that, for $n\ge0$,
    \[
 P(N=n)  = \frac{\mu^n}{(1+\mu)^{n+1}}\,,
    \]
where  $\mu=\.E(N)$.
  \end{questionparts}
\end{question}
\end{document}
